\documentclass{report}

\usepackage{a4}

\usepackage[english]{babel}
\usepackage[utf8]{inputenc}
\usepackage{amsmath, amssymb, mathtools}
\usepackage{graphicx}
\usepackage{color}
\usepackage{datetime}
\usepackage[hidelinks]{hyperref}
\usepackage{enumitem}
\usepackage{stmaryrd}

\setlength{\parindent}{0em} 

\usepackage{amsthm}
\newtheoremstyle{tommy}% ⟨name ⟩
{15pt}% ⟨Space above ⟩1
{15pt}% ⟨Space below ⟩1
{\normalfont}% ⟨Body font ⟩
{}% ⟨Indent amount ⟩2
{\bfseries}% ⟨Theorem head font⟩
{}% ⟨Punctuation after theorem head ⟩
{0.7em}% ⟨Space after theorem head ⟩3
{}% ⟨Theorem head spec (can be left empty, meaning ‘normal’)⟩

\theoremstyle{tommy}
\newtheorem{defn}{Definition}
\newtheorem{thm}[defn]{Theorem}
\newtheorem{lem}[defn]{Lemma}
\newtheorem{nota}[defn]{Notation}
\newtheorem{cor}[defn]{Corrolary}
\newtheorem{prop}[defn]{Proposition}
\newtheorem{eg}[defn]{Example}
\newtheorem{rem}[defn]{Remark}
\newtheorem{ex}[defn]{Exercise}

% Counter
\usepackage{chngcntr}
\counterwithin{defn}{chapter}

\renewcommand\div{\operatorname{div}}
\renewcommand\qedsymbol{\(\blacksquare\)}

\title{Partial Differerential Equations \\ Thanh Nam Phan \\ Winter Semester 2020/2021}
\author{Lecture notes \TeX{} ed by Thomas Eingartner}
\date{\today, \currenttime}

\begin{document}

\maketitle
\tableofcontents
\newpage



\chapter{Introduction}

A differential equation is an equation of a function and its derivatives. 

\begin{eg} (Linear ODE)
  Let \(f: \mathbb{R} \to \mathbb{R}\),
  \begin{align*}
    \begin{cases}
      f(t) = a f(t) \text{ for all } t \ge 0, a \in \mathbb{R} \\
      f(0) = a_0
    \end{cases}
  \end{align*}
  is a linear ODE (Ordinary differential equation). The solution is: \(f(t) = a_0 e^{at}\) for all \(t \ge 0\).
\end{eg}

\begin{eg} (Non-Linear ODE) \(f: \mathbb{R} \to \mathbb{R}\)
  \begin{align*}
    \begin{cases}
      f'(t) = 1 + f^2(t) \\
      f(0) = 1
    \end{cases}
  \end{align*}
  Lets consider \(f(t) = \tan(t) = \frac{\sin(t)}{\cos(t)}\). Then we have \[f'(t) = \frac{1}{\cos(t)} = 1 + \tan^2(t) = 1 + f^2(t),\] but this solution only is \emph{good} in \((- \pi, \pi)\). It's a problem to extend this to \(\mathbb{R} \to \mathbb{R}\).
\end{eg}

A PDE (Partial Differential Equation) is an equation of a function of 2 or more variables and its derivatives.

\begin{rem}
  Recall for \( \Omega \subseteq \mathbb{R}^d\) open and \(f: \Omega \to \{\mathbb{R}, \mathbb{C}\}\) the notation of partial derivatives:
  \begin{itemize}
    \item \(\partial_{x_i} f(x) = \lim_{h \to 0} \frac{f(x + he_i) - f(x)}{h}, \text{where } e_i = (0, 0, \dots, 1, \dots, 0, 0) \in \mathbb{R}^d\)
    \item \(D^\alpha f(x) = \partial_{x_1}^{\alpha_1} \cdots \partial_{x_d}^{\alpha_d} f(x), \text{where } \alpha \in \mathbb{N}^d\)
    \item \(Df = \nabla f = (\partial{x_1}, \ldots, \partial_{x_d})\)
    \item \(\Delta f = \partial_{x_1}^2 + \cdots + \partial^2_{x_d} f\)
    \item \(D^k f = (D^\alpha f)_{|\alpha| = k},  \text{where } |\alpha| = \sum_{i=1}^d |\alpha_i|\)
    \item \(D^2 f = (\partial_{x_i} \partial_{x_j} f)_{1 \le i, j \le d}\)
  \end{itemize}
\end{rem}

\begin{defn}
  Given a function \( F \). Then the equation of the form 
  \begin{align*}
    F(D^k u(x), D^{k-1} u(x), \dots, Du(x), u(x), x) = 0 
  \end{align*}
  with the unknown function \(u:\ \Omega \subseteq \mathbb{R}^d \longrightarrow \mathbb{R}\) is called a \emph{PDE of order \(k\)}.
  \begin{itemize}
    \item Equations \(\sum_d a_\alpha(x) D^\alpha u(x) = 0\), where \(a_\alpha\) and \(u\) are unknown functions are called \emph{Linear PDEs}. 
    \item Equations \(\sum_{|\alpha| = k} a_\alpha(x) D^\alpha u(x) + F(D^{k-1}u, D^{k-2}u, \dots, Du, u, x) = 0\) are called \emph{semi-linear PDEs}.
  \end{itemize}
\end{defn}

Goals: For \emph{solving a PDE} we want to
\begin{itemize}
  \item Find an explizit solution! This is in many cases impossible.
  \item Prove a \emph{well-posted theory} (existence of solutions, uniqueness of solutions, continuous dependence of solutions on the data)
\end{itemize}

We have two notations of solutions:
\begin{enumerate}
  \item Classical solution: The solution is continuous differentiable (e.g. \( \Delta u = f  \leadsto u \in C^2  \))
  \item Weak Solutions: The solution is not smooth/continuous
\end{enumerate}

\begin{defn} (Spaces of continous and differentiable functions)
  Let \(\Omega \subseteq R^d\) be open
  \begin{align*}
    C(\Omega) &= \left\{ f: \ \Omega \to \mathbb{R} \mid f \text{ continuous} \right\} \\
    C^k(\Omega) &= \left\{ f: \ \Omega \to \mathbb{R} \mid D^\alpha f \text{ is continuous for all } |\alpha| \le k \right\}
  \end{align*}
\end{defn}

Classical solution of a PDE of order \(k \leadsto C^k\) solutions!
\begin{align*}
  L^p(\Omega) = \left\{ f: \ \Omega \to \mathbb{R} \text{ lebesgue measurable} \mid \int_\Omega |f|^p d\lambda < \infty, 1 \le p < \infty \right\}
\end{align*}

Sobolev Space:
\begin{align*}
  W^{k,p}(\Omega)= \left \{ f \in L^p(\Omega) \mid \forall \alpha \in \mathbb{N}^n \text{ with } |\alpha| \leq k: D^{\alpha}f \in L^p(\Omega) \text{exists}  \right \}
\end{align*}

In this course we will investigate
\begin{itemize}
  \item Laplace / Poisson Equation: \(-\Delta u = f\)
  \item Heat Equation: \(\partial_t u - \Delta u = f\)
  \item Wave Equation: \(\partial_t^2 - \Delta u = f\)
  \item Schrödinger Equation: \(i \partial_t u - \Delta u = f\)
\end{itemize}


\chapter{Laplace / Poisson Equation}
\(- \Delta u = 0\) (Laplace) or \(-\Delta u = f(x)\) (Poisson).

\begin{defn} (Harmonic Function)
  Let \(Omega\) be an open set in \(\mathbb{R}^d\). If \(u \in C^2(\Omega)\) and \(\Delta u = 0\) in \(\Omega\), then \(u\) is a harmonic function in \(\Omega\).
\end{defn}

\begin{thm} (Gauss-Green Theorem)
  \[ \int_{\partial V} F \vec{u}\ dS(x) = \int_V \div(F)\ dx \]
\end{thm}
Thus
\begin{align*}
  0 = \int_{\partial V} \nabla u \vec{n}\ dS(x)
  = \int_V \div(\nabla u) \ dx
  = \int_V \nabla u(x) \ dx
\end{align*}
for any \(V \subseteq \Omega \) open.

\begin{ex}
Let \(\Omega \subseteq \mathbb{R}^d\) open, let \(f: \Omega \to \mathbb{R}\) be continuous. Prove that if \( \int_B f(x) \ dx = 0 \), then \( u \equiv 0 \) in \(\Omega\).
\end{ex}

\begin{thm} (Fundamential Lemma of Calculus of Variations)
  Let \(\Omega \subseteq \mathbb{R}^d\) open, let \(f \in L^1(\Omega)\). If 
  \(\int_B f(x) \ dx = 0\) for all \(x \in B_r(x) \subseteq \Omega\), then \(f(x) = 0\) a.e. (almost everywhere) \(x \in \Omega\).
\end{thm}

\begin{rem} (Solving Laplace Equation)
  \(\Delta u = 0\) in \(\mathbb{R}^d\). Consider the case when \(u\) is radial, i.e. \(u(x) = v(|x|)\), \(v: \mathbb{R} \to \mathbb{R}\). Denote \(r = |x|\), then 
  \[
    \frac{\partial r}{\partial x} 
    = \frac{\partial}{\partial x_i}  \left(\sqrt{x_1^2 + \dots + x_d^2}\right) \\
    = \frac{2 x_i}{2{\sqrt{x_1^2 + \dots + x_d^2}}} \\
    = \frac{x_i}{r}
  \]
  Then
  \begin{align*}
    \partial_{x_i} u &= \partial_{x_i} v = (\partial_r v) \frac{\partial r}{\partial {x_i}} 
    = v'(r) \frac{x_i}{r} \\
    \partial^2_{x_i} u 
    &= \partial_{x_i} \left(v(r)' \frac{x_i}{r}\right) 
    = \partial_{x_i}(v(r)') \frac{x_i}{r} + v'(r) \partial_{x_i} \left(\frac{x_i}{r}\right) \\
    &= \partial_r(v'(r))\left(\frac{dr}{\partial_{x_i}}\right) \frac{x_i}{r} + v'(r)\left( \frac{1}{r} - \frac{x_i}{r^2}(\partial_{x_i} r) \right) 
    = v'(r) \frac{x_i^2}{r^2} + v'r(r)\left(\frac{1}{r} - \frac{x_i^2}{r^3}\right)
  \end{align*}

  So we have \(\Delta u = \left( \sum_{i=1}^d d_{x_i}^2 \right) u = v''(r) + v'(r) (\frac{d}{r} - \frac{1}{r})\)

  Thus \(\Delta u = v'(r) + v(r) \frac{d-1}{r}\). We consider \(d \ge 2\). Laplace operator \(\Delta u = 0\) now becomes \(v''(r) + v'(r) \frac{d-1}{r} = 0\) \\
  \(\Rightarrow\) \(\log(v(r))' = \frac{v'(r)}{v(r)} = - \frac{d-1}{r} = -(d-1)(\log r)'\) (recall \(log(f)' = \frac{f'}{f}\)) \\
  \(\Rightarrow v'(r) = \frac{1}{v^{d-2} + \text{ const.}}\) \\
  \(\begin{cases}
    \frac{const}{r^{d-2}} + const xx + const &,d \ge 3 \\
    const \log(r) + const xx r + const &,d = 2
  \end{cases}\)
\end{rem}

\begin{defn} 
  (Fundamential Solution of Laplace Equation)
  \begin{align*}
    \Phi(x) = \begin{cases}
      - \frac{1}{2 \pi} \log(|x|) &, d = 2 \\
      \frac{1}{4 \pi |x|} &, d = 3 \\
      \frac{1}{(d-2) d |B_1|} \frac{1}{|x|^{d-2}} &, d \ge 3
    \end{cases}
  \end{align*}
  
  Where \( |B_1| \) is the Volume of the ball \(B_1(0) = B(0, 1) \subseteq \mathbb{R}^d\).

\end{defn}

\begin{rem}
  \(\Delta \Phi(x) = 0\) for all \(x \in \mathbb{R}^d\) and \(x \ne 0\). 
\end{rem}


\end{document}