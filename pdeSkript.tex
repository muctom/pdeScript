\documentclass[ngerman, BCOR=5mm]{scrreprt}

\KOMAoption{open}{left}

\usepackage{a4}
\setkomafont{sectioning}{\mdseries} 
\usepackage{tgtermes}

\usepackage[english]{babel}
%\usepackage{ngerman}
\usepackage[utf8]{inputenc}
\usepackage{amsmath, amssymb, mathtools}
\usepackage[arrow, matrix, curve]{xy}
\usepackage{graphicx}
\usepackage{color}
\usepackage{datetime}
\usepackage[hidelinks]{hyperref}
\usepackage{enumitem}
\usepackage{stmaryrd}
\usepackage{adjustbox}
\usepackage{fancybox}
\usepackage{makeidx}
\usepackage{blindtext}
\usepackage{wallpaper}
\usepackage{tikz}
\usepackage[new]{old-arrows}
\usepackage{wrapfig}
\usepackage{cutwin}
\usepackage{tocstyle}

\newtocstyle[KOMAlike][leaders]{alldotted}{}
\usetocstyle{alldotted}

\setlength{\parindent}{0em} 

\usepackage{amsthm}
\newtheoremstyle{tommy}% ⟨name ⟩
{15pt}% ⟨Space above ⟩1
{15pt}% ⟨Space below ⟩1
{\normalfont}% ⟨Body font ⟩
{}% ⟨Indent amount ⟩2
{\bfseries}% ⟨Theorem head font⟩
{}% ⟨Punctuation after theorem head ⟩
{0.7em}% ⟨Space after theorem head ⟩3
{}% ⟨Theorem head spec (can be left empty, meaning ‘normal’)⟩

\theoremstyle{tommy}
\newtheorem{defn}{Definition}
\newtheorem{satz}[defn]{Satz}
\newtheorem{lem}[defn]{Lemma}
\newtheorem{nota}[defn]{Notation}
\newtheorem{kor}[defn]{Korollar}
\newtheorem{folg}[defn]{Folgerung}
\newtheorem{prop}[defn]{Proposition}
\newtheorem{eg}[defn]{Example}
\newtheorem{bem}[defn]{Bemerkung}


\usepackage{chngcntr}
\counterwithin{defn}{chapter}


\renewcommand{\proofname}{Beweis}
\renewcommand\qedsymbol{$\blacksquare$}


%Neue Befehle allgemein
\newcommand{\IV}{\text{(IV)}}
\newcommand{\Aut}{\operatorname{Aut}}
\newcommand{\grad}{\operatorname{grad}}
\newcommand{\im}{\operatorname{im}}
\newcommand{\id}{\operatorname{id}}
\newcommand{\isoto}{\xrightarrow{\,\smash{\raisebox{-0.40ex}{\ensuremath{\scriptstyle\sim}}}\,}}
\newcommand{\Gal}{\operatorname{Gal}}
\newcommand{\Hom}{\operatorname{Hom}}
\newcommand{\Auto}{\operatorname{Aut}}
\newcommand{\End}{\operatorname{End}}
\newcommand{\chara}{\operatorname{char}}
\newcommand{\ggT}{\operatorname{ggT}}
\newcommand{\Mipo}{\operatorname{Mipo}}
\newcommand{\Quot}{\operatorname{Quot}}
\newcommand{\kgV}{\operatorname{kgV}}
\newcommand{\rg}{\operatorname{rg}}
\newcommand{\ord}{\operatorname{ord}}
\newcommand{\rang}{\operatorname{rg}}
\newcommand{\mo}{\operatorname{mod}}
\newcommand{\Tr}{\operatorname{Tr}}
\newcommand{\GL}{\operatorname{GL}}
\newcommand{\il}{\operatorname{il}}
\newcommand{\quot}{\operatorname{Quot}}
\newcommand*\longtwoheadrightarrow{\ensuremath{\relbar\joinrel\twoheadrightarrow}}
\newcommand{\Vol}{\operatorname{Vol}}
\newcommand{\Ok}{\mathcal{O}_K}
\newcommand{\dis}{\operatorname{d}}
\newcommand{\also}{\ $\Rightarrow$\ }
\DeclareMathOperator*{\prolim}{\underleftarrow\lim}

%\renewcommand*{\bibtitle}{Literaturverzeichnis}

\title{Partial Differerential Equations \\ Thanh Nam Phan \\ Winter Semester 2020/2021}
\author{Lecture notes \TeX ed by Thomas Eingartner}
\date{\today, \currenttime}

\begin{document}

\maketitle
\tableofcontents
\newpage



\chapter{Introduction}

A differential equation is an equation of a function and its derivatives. 

\begin{eg}
  \begin{align*}
    f: \mathbb{R} \to \mathbb{R} \\
    \begin{cases}
      f(t) = a f(t) \text{ for all } t \ge 0, a \in \mathbb{R} \\
      f(0) = a_0
    \end{cases}
  \end{align*}

  Solution: \(f(t) = a_0 e^{at}\) for all \(t \ge 0\). This is a linear ODE (Ordinary differential equation).
\end{eg}

\begin{eg} (Non-Linear ODE)
  \begin{align*}
    f: \mathbb{R} \to \mathbb{R} \\
    \begin{cases}
      f'(t) = 1 + f^2(t) \\
      f(0) = 1
    \end{cases}
  \end{align*}
  % We consider \(f(t) = \tan(t) = \frac{\sin(t)}{\cos(t)}\) because \(f'(t) = \frac{1}{\cos(t)} = 1 + \tan^2(t) = 1 + f^2(t)\). But this solution only \emph{good} in \((- \pi, \pi)\). \exists propblem to extend this to \(\mathbb{R} \to \mathbb{R}\). We have to be careful which the degree of domain of variables ... of solutions ... ?
\end{eg}

A PDE (Partial Differential Equation) is an equation of a function of 2 or more variables and its derivatives. Recall \(Omega \subseteq \mathbb{R}^d\), \(f: \Omega \to \{\mathbb{R}, \mathbb{C}\}\) open, 
\begin{align*}
  \partial_{x_1} f(x) = \lim \frac{f(x + he_i) - f(x)}{h}, e_i = (0,0, 1,, 0, 0) \in \mathbb{R}^d \\
  D^\alpha f(x) = \partial_{x_1}^{\alpha_1} \ddots \partial_{x_d}^{\alpha_d} f(x), |\alpha| = \sum_{i=1}^d |\alpha_i| \\
  Df = \nabla f = \text{ gradient of } f = (\partial{x_1} \dots \partial_{x_d}) \\
  \Delta f = \partial_{x_1}^2 + \dots + \partial^2_{x_d} f \\
  D^kf = (D^\alpha f)_{|\alpha| = k} \\
  D^2 f = (\partial_{x_i} \partial_{x_j} f)_{1 \le i, j \le d}
\end{align*}

% \begin{def}
%   Given a function \(F\). Then the equation of the form 
%   \[F(D^k u(x), D^{k-1} u(x), \dots, Du(x), u(x), x) = 0\]
%   with the unknown function \(u: \Omega \subseteq \mathbb{R}^d \to \mathbb{R}\) is called a \emph{PDE of order \(k\)}.
%   \begin{itemize}
%     \item Linear PDE: \(\sum_d a_\alpha(x) D^\alpha u(x) = 0\), \(a_\alpha\), \(u\) is unknown. 
%     \item Non-Linear PDE: More difficult: Semi-linear PDE
%     \(\sum_{|alpha| = k} a_\alpha(x) D^\alpha u(x) + F(D^{k-1}u, D^{k-2}u, \dots, Du, u, x) = 0\)
% \end{def}

Goals: For \emph{solving a PDE} we want to
\begin{itemize}
  \item Find an explizit solution! In many cases, it is impossible
  \item Prove a \emph{well-posted theory} (existence of solutions, uniqueness of solutions, continuous dependence of solutions on the data)
\end{itemize}

We have to notations of solution:
% \begin{enumerate}
%   \item Classical solution: The solution is smooth (continuous differentialble) (e.g. \(Delta u = f\) \leadsto ??? \(u \in C^2\))
%   \item Weak Solutions: The solution is not smooth/continuous
% \end{enumerate}

\(Omega \subseteq R^d\)
% \begin{align*}
%   C(\Omega) &= \( \{f: \ \Omega \to \mathbb{R}, f \text{ continuous} \}\) \\
%   C^k(\Omega) &= \( \{f: \ D^\alpha f \Omega \to \mathbb{R} \text{ is coninutous for all } |\alpha| \le k \}
% \end{align*}
% Classical solution of a PDE of order \(k \leadsto C^k\) solutions! \\
% \[ L^p(\Omega) = \{f: \ \Omega \to \mathbb{R} \text{ measurable lebesgue ???, \int_\Omega |f|^p < \infty, 1 \le p < p}\]???
% Sobolev Space: ???
Navier-Stokes equation:
\[
  \begin{cases}
    d_t u + u \nabla u - \Delta u = \nabla f, \quad f \text{ is known} \\
    \div(u) = \sum_{i=1}^d \frac{\partial}{\partial x_1} u_i(x) = 0
  \end{cases}
\]
open in 3D, exists smooth solution

\end{document}