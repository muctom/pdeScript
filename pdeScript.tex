% chktex-file 11
% chktex-file 35
% chktex-file 2
% chktex-file 10
% chktex-file 40

\documentclass{report}

\usepackage{a4}

\usepackage[english]{babel}
\usepackage[utf8]{inputenc}
\usepackage{amsmath, amssymb, mathtools, esint, bbold}
\usepackage{graphicx}
\usepackage{color}
\usepackage{datetime}
\usepackage[hidelinks]{hyperref}
\usepackage{enumitem}
\usepackage{stmaryrd}

\setlength{\parindent}{0em} 

\usepackage{amsthm}
\newtheoremstyle{tommy}% ⟨name ⟩
{15pt}% ⟨Space above ⟩1
{15pt}% ⟨Space below ⟩1
{\normalfont}% ⟨Body font ⟩
{}% ⟨Indent amount ⟩2
{\bfseries}% ⟨Theorem head font⟩
{}% ⟨Punctuation after theorem head ⟩
{0.7em}% ⟨Space after theorem head ⟩3
{}% ⟨Theorem head spec (can be left empty, meaning ‘normal’)⟩

\theoremstyle{tommy}
\newtheorem{defn}{Definition}
\newtheorem{thm}[defn]{Theorem}
\newtheorem{lem}[defn]{Lemma}
\newtheorem{nota}[defn]{Notation}
\newtheorem{cor}[defn]{Corrolary}
\newtheorem{prop}[defn]{Proposition}
\newtheorem{eg}[defn]{Example}
\newtheorem{rem}[defn]{Remark}
\newtheorem{ex}[defn]{Exercise}

% Counter
\usepackage{chngcntr}
\counterwithin{defn}{chapter}

\renewcommand\div{\operatorname{div}}
\newcommand{\dist}{\operatorname{dist}}
\renewcommand\qedsymbol{\(\blacksquare\)}

\title{Partial Differerential Equations \\ Thành Nam Phan \\ Winter Semester 2021/2022}
\author{Lecture notes \TeX{}ed by Thomas Eingartner}
\date{\today, \currenttime}

\begin{document}

\maketitle
\tableofcontents
\newpage
Please note that I write this lecture notes for my personal use. I may write things differently than presented in the lecture. This script also contains solutions for exercises (which may be wrong). Of course, I don't push them to GitHub while the exercises still can be handed in.
\newpage

\chapter{Introduction}

A differential equation is an equation of a function and its derivatives. 

\begin{eg}[Linear ODE]
  Let \(f: \mathbb{R} \to \mathbb{R}\),
  \begin{align*}
    \begin{cases}
      f(t) = a f(t) \text{ for all } t \ge 0, a \in \mathbb{R} \\
      f(0) = a_0
    \end{cases}
  \end{align*}
  is a linear ODE (Ordinary differential equation). The solution is: \(f(t) = a_0 e^{at}\) for all \(t \ge 0\).
\end{eg}

\begin{eg} (Non-Linear ODE) \(f: \mathbb{R} \to \mathbb{R}\)
  \begin{align*}
    \begin{cases}
      f'(t) = 1 + f^2(t) \\
      f(0) = 1
    \end{cases}
  \end{align*}
  Lets consider \(f(t) = \tan(t) = \frac{\sin(t)}{\cos(t)}\). Then we have \[f'(t) = \frac{1}{\cos(t)} = 1 + \tan^2(t) = 1 + f^2(t),\] but this solution only is \emph{good} in \((- \pi, \pi)\). It's a problem to extend this to \(\mathbb{R} \to \mathbb{R}\).
\end{eg}

A PDE (Partial Differential Equation) is an equation of a function of 2 or more variables and its derivatives.

\begin{rem}
  Recall for \( \Omega \subseteq \mathbb{R}^d\) open and \(f: \Omega \to \{\mathbb{R}, \mathbb{C}\}\) the notation of partial derivatives:
  \begin{itemize}
    \item \(\partial_{x_i} f(x) = \lim_{h \to 0} \frac{f(x + he_i) - f(x)}{h}, \text{where } e_i = (0, 0, \dots, 1, \dots, 0, 0) \in \mathbb{R}^d\)
    \item \(D^\alpha f(x) = \partial_{x_1}^{\alpha_1} \cdots \partial_{x_d}^{\alpha_d} f(x), \text{where } \alpha \in \mathbb{N}^d\)
    \item \(Df = \nabla f = (\partial{x_1}, \ldots, \partial_{x_d})\)
    \item \(\Delta f = \partial_{x_1}^2 + \cdots + \partial^2_{x_d} f\)
    \item \(D^k f = (D^\alpha f)_{|\alpha| = k},  \text{where } |\alpha| = \sum_{i=1}^d |\alpha_i|\)
    \item \(D^2 f = (\partial_{x_i} \partial_{x_j} f)_{1 \le i, j \le d}\)
  \end{itemize}
\end{rem}

\begin{defn}
  Given a function \( F \). Then the equation of the form 
  \begin{align*}
    F(D^k u(x), D^{k-1} u(x), \dots, Du(x), u(x), x) = 0 
  \end{align*}
  with the unknown function \(u:\ \Omega \subseteq \mathbb{R}^d \longrightarrow \mathbb{R}\) is called a \emph{PDE of order \(k\)}.
  \begin{itemize}
    \item Equations \(\sum_d a_\alpha(x) D^\alpha u(x) = 0\), where \(a_\alpha\) and \(u\) are unknown functions are called \emph{Linear PDEs}. 
    \item Equations \(\sum_{|\alpha| = k} a_\alpha(x) D^\alpha u(x) + F(D^{k-1}u, D^{k-2}u, \dots, Du, u, x) = 0\) are called \emph{semi-linear PDEs}.
  \end{itemize}
\end{defn}

Goals: For \emph{solving a PDE} we want to
\begin{itemize}
  \item Find an explizit solution! This is in many cases impossible.
  \item Prove a \emph{well-posted theory} (existence of solutions, uniqueness of solutions, continuous dependence of solutions on the data)
\end{itemize}

We have two notations of solutions:
\begin{enumerate}
  \item Classical solution: The solution is continuous differentiable (e.g. \( \Delta u = f  \leadsto u \in C^2  \))
  \item Weak Solutions: The solution is not smooth/continuous
\end{enumerate}

\begin{defn} (Spaces of continous and differentiable functions)
  Let \(\Omega \subseteq R^d\) be open
  \begin{align*}
    C(\Omega) &= \left\{ f: \ \Omega \to \mathbb{R} \mid f \text{ continuous} \right\} \\
    C^k(\Omega) &= \left\{ f: \ \Omega \to \mathbb{R} \mid D^\alpha f \text{ is continuous for all } |\alpha| \le k \right\}
  \end{align*}
\end{defn}

Classical solution of a PDE of order \(k \leadsto C^k\) solutions!
\begin{align*}
  L^p(\Omega) = \left\{ f: \ \Omega \to \mathbb{R} \text{ lebesgue measurable} \mid \int_\Omega |f|^p d\lambda < \infty, 1 \le p < \infty \right\}
\end{align*}

Sobolev Space:
\begin{align*}
  W^{k,p}(\Omega)= \left \{ f \in L^p(\Omega) \mid \forall \alpha \in \mathbb{N}^n \text{ with } |\alpha| \leq k: D^{\alpha}f \in L^p(\Omega) \text{exists}  \right \}
\end{align*}

In this course we will investigate
\begin{itemize}
  \item Laplace / Poisson Equation: \(-\Delta u = f\)
  \item Heat Equation: \(\partial_t u - \Delta u = f\)
  \item Wave Equation: \(\partial_t^2 - \Delta u = f\)
  \item Schrödinger Equation: \(i \partial_t u - \Delta u = f\)
\end{itemize}


\chapter{Laplace / Poisson Equation}

\section{Laplace Equation}
\(- \Delta u = 0\) (Laplace) or \(-\Delta u = f(x)\) (Poisson).

\begin{defn} (Harmonic Function)
  Let \(\Omega\) be an open set in \(\mathbb{R}^d\). If \(u \in C^2(\Omega)\) and \(\Delta u = 0\) in \(\Omega\), then \(u\) is a harmonic function in \(\Omega\).
\end{defn}

\begin{thm} (Gauss-Green Theorem)\label{gauss-green}
  Let \(A \subseteq \mathbb{R}^d\) open, \(\vec{F} \in C^1(A, \mathbb{R}^d)\) and \(K \subseteq A\) compact with \(C^1\) boundary. Then
  \[ \int_{\partial K} \vec{F} \cdot \vec{\nu}\ dS(x) = \int_K \div(\vec{F})\ dx \]
  where \(\nu\) is the outward unit normal vector field on \(\partial K\).
Thus
\begin{align*}
  \int_{\partial V} \nabla u \cdot \vec{\nu}\ dS(x)
  = \int_V \div(\nabla u) \ dx
  = \int_V \Delta u(x) \ dx
\end{align*}
for any \(V \subseteq \Omega \) open.
\end{thm}


\begin{thm} (Green's Identities)\label{green-identities}
  Let \(A \subseteq \mathbb{R}^d\) open, \(K \subseteq A\) d-dim.\ compactum with \(C^1\) boundary and \(f, g \in C^2(A)\)
  \begin{enumerate}
    \item Green's first identity (Partial Integration): \begin{align*}
      \int_K \nabla f \cdot \nabla g \, dx = \int_{\partial K} f \frac{\partial g}{\partial \nu} \, dS - \int_K f \Delta g \, dx
    \end{align*}
    where \(\frac{\partial g}{\partial \nu} = \partial_\nu g = \nu \cdot \nabla g\)
    \item Green's second identity: \begin{align*}
      \int_K f \Delta g - (\Delta f) g \, dx = \int_{\partial K} \left(f \frac{\partial g}{\partial \nu} - g \frac{\partial f}{\partial \nu}\right) \, dS
    \end{align*}
  \end{enumerate}
\end{thm}

\begin{ex}
  Let \(\Omega \subseteq \mathbb{R}^d\) open, let \(f: \Omega \to \mathbb{R}\) be continuous. Prove that if \( \int_B f(x) \ dx = 0 \), then \( u \equiv 0 \) in \(\Omega\).
\end{ex}

\begin{thm} (Fundamential Lemma of Calculus of Variations)
  Let \(\Omega \subseteq \mathbb{R}^d\) open, let \(f \in L^1(\Omega)\). If 
  \(\int_B f(x) \ dx = 0\) for all \(x \in B_r(x) \subseteq \Omega\), then \(f(x) = 0\) a.e. (almost everywhere) \(x \in \Omega\).
\end{thm}

\begin{rem} (Solving Laplace Equation)
  \(-\Delta u = 0\) in \(\mathbb{R}^d\). Consider the case when \(u\) is radial, i.e. \(u(x) = v(|x|)\), \(v: \mathbb{R} \to \mathbb{R}\). Denote \(r = |x|\), then 
  \[
    \frac{\partial r}{\partial x} 
    = \frac{\partial}{\partial x_i}  \left(\sqrt{x_1^2 + \dots + x_d^2}\right) \\
    = \frac{2 x_i}{2{\sqrt{x_1^2 + \dots + x_d^2}}} \\
    = \frac{x_i}{r}
  \]
  Then
  \begin{align*}
    \partial_{x_i} u &= \partial_{x_i} v = (\partial_r v) \frac{\partial r}{\partial {x_i}} 
    = v'(r) \frac{x_i}{r} \\
    \partial^2_{x_i} u 
    &= \partial_{x_i} \left(v(r)' \frac{x_i}{r}\right) 
    = (\partial_{x_i}v(r)') \frac{x_i}{r} + v'(r) \partial_{x_i} \left(\frac{x_i}{r}\right) \\
    &= (\partial_r v'(r))\left(\frac{dr}{\partial_{x_i}}\right) \frac{x_i}{r} + v'(r)\left( \frac{1}{r} - \frac{x_i}{r^2}(\partial_{x_i} r) \right) 
    = v'(r) \frac{x_i^2}{r^2} + v'r(r)\left(\frac{1}{r} - \frac{x_i^2}{r^3}\right)
  \end{align*}

  So we have \(\Delta u = \left( \sum_{i=1}^d d_{x_i}^2 \right) u = v''(r) + v'(r) (\frac{d}{r} - \frac{1}{r})\)

  Thus \(\Delta u = v'(r) + v(r) \frac{d-1}{r}\). We consider \(d \ge 2\). Laplace operator \(\Delta u = 0\) now becomes \(v''(r) + v'(r) \frac{d-1}{r} = 0\) \\
  \(\Rightarrow\) \(\log(v(r))' = \frac{v'(r)}{v(r)} = - \frac{d-1}{r} = -(d-1)(\log r)'\) (recall \(log(f)' = \frac{f'}{f}\)) \\
  \(\Rightarrow v'(r) = \frac{1}{v^{d-2} + \text{ const.}}\) \\
  \(\begin{cases}
    \frac{const}{r^{d-2}} + const xx + const &,d \ge 3 \\
    const \log(r) + const xx r + const &,d = 2
  \end{cases}\)
\end{rem}

\begin{defn} 
  (Fundamential Solution of Laplace Equation)
  \begin{align*}
    \Phi(x) = \begin{cases}
      - \frac{1}{2 \pi} \log(|x|), & d = 2 \\
      \frac{1}{(d-2) d |B_1|} \frac{1}{|x|^{d-2}}, & d \ge 3
    \end{cases}
  \end{align*}
  
  Where \( |B_1| \) is the Volume of the ball \(B_1(0) = B(0, 1) \subseteq \mathbb{R}^d\).

\end{defn}

\begin{rem}
  \(\Delta \Phi(x) = 0\) for all \(x \in \mathbb{R}^d\) and \(x \ne 0\). 
\end{rem}


\section{Poisson-Equation}
The Poisson-Equation is \(-\Delta u(x) = f(x)\) in \(\mathbb{R}^d\). The explicit solution is given by
\begin{align*}
  u(x) &= (\Phi \star f)(x)
  = \int_{\mathbb{R}^d} \Phi(x-y)f(y) \ dy 
  = \int_{\mathbb{R}^d} \Phi(y)f(x-y) \ dy
\end{align*}
This can be heuristically justifyfied with \[-\Delta (\Phi \star f) = (-\Delta \Phi) \star f = \delta_0 \star f = f\]


\begin{thm} \label{solution-for-poisson}
  Assume \(f \in C_c^2(\mathbb{R}^d)\). Then \(u = \Phi \star f\) satisfies that \(u \in C^2(\mathbb{R}^d)\) and \(- \Delta u(x) = f(x)\) for all \(x \in \mathbb{R}^d\)
\end{thm}


\begin{proof}
  By definition we have
  \begin{align*}
    u(x) &= \int_{\mathbb{R}^d} \Phi(y) f(x-y) \, dy.
  \end{align*}
  First we check that \(u\) is continuous: Take \(x_k \to x_0\) in \(\mathbb{R}^d\). We prove that \(u(x_n) \xrightarrow{n} u_0\), i.e.
  \begin{align*}
    \lim_{n \to \infty} \int_{\mathbb{R}^d} \Phi(y) f(x_n - y) \ dy = \int_{\mathbb{R}^d} \Phi(y) f(x_0 - y) \ dy
  \end{align*}
  This follows from the Dominated Convergence Theorem. More precisely:
  \begin{align*}
    \lim_{n \to \infty} \Phi(y) f(x_n -y) = \Phi(y) f(x_0 - y) \quad \text{ for all } y \in \mathbb{R}^d \setminus \{0\}
  \end{align*}
  and 
  \begin{align*}
    |\Phi(y) f(x-y)| &\le \| f ||_{L^\infty} \cdot \mathbb{1}(|y| \le R) \cdot |\Phi(y)| \in L^1(\mathbb{R}^d, dy)
  \end{align*}
  where \(R > 0\) depends on \(\{x_n\}\) and \(\operatorname{supp}(f)\) but independent of \(y\). Now we compute the derivatives:
  \begin{align*}
    \partial_{x_i} u(x) 
    &= \partial_{x_i} \int_{\mathbb{R}^d} \Phi(y) f(x-y) \ dy
    = \lim_{h \to 0} \int_{\mathbb{R}^d} \Phi(y) \frac{f(x + h e_i - y) - f(x-y)}{h} \ dy \\
    \text{(dom.\ conv.)} \quad &= \int \Phi(y) \partial_{x_i} f(x-y) \ dy \\
    \Rightarrow \quad D^\alpha u(x) &= \int_{\mathbb{R}^d} \Phi(y) D_x^\alpha f(x-y) \ dy \quad \text{for all } |\alpha| \le 2
  \end{align*}
  \(D^\alpha u(x)\)  is continuous, thus \(u \in C^2(\mathbb{R}^d)\).
  Now we check if this solves the Poisson-Equation:
  \begin{align*}
    - \Delta u(x) 
    &= \int_{\mathbb{R}^d} \Phi(y) (-\Delta_x) f(x-y) \, dy
    = \int_{\mathbb{R}^d} \Phi(y) (-\Delta_y) f(x-y) \, dy \\
    &= \int_{\mathbb{R}^d \setminus B(0, \epsilon)} \Phi(y) (-\Delta_x) f(x-y) \, dy + \int_{B(0, \epsilon)} \Phi(y) (-\Delta_x) f(x-y) \, dy \quad (\epsilon > 0 \text{ small})
  \end{align*}
  Now we come to the main part. We apply integration by parts (\ref{green-identities}):
  \begin{align*}
    &\int_{\mathbb{R}^d \setminus B(0, \epsilon)} \Phi(y)(- \Delta_y) f(x-y) \, dy \\
    &\quad = \int_{\mathbb{R}^d \setminus B(0, \epsilon)} (\nabla_y \Phi(y)) \cdot \nabla_y f(x-y) \, dy - \int_{\partial B(0, \epsilon)} \Phi(y) \cdot  \frac{\partial f}{\partial \vec{n}}(x-y) \, dS(y) \\
    &\quad = \int_{\mathbb{R}^d \setminus B(0, \epsilon)} \underbrace{(-\Delta_y \Phi(y))}_{=0} f(x-y) \, dy \\ &\qquad + \int_{\partial B(0, \epsilon)} \frac{\partial \Phi}{\partial \vec{n}}(y) f(x-y) \, dS(y) - \int_{\partial B(0, \epsilon)} \Phi(y) \frac{\partial f}{\partial \vec{n}}(x-y) \, dS(y)
  \end{align*}

  We have that \(\nabla_y \Phi(y) = -\frac{1}{d |B_1|} \frac{y}{|y|^d}\) and \( \vec{n} = \frac{y}{|y|} \text{ in } \partial B(0, \epsilon)\). This leads to
    \begin{align*}
      \frac{\partial \Phi}{\partial \vec{n}} &= \frac{1}{d |B_1|} \frac{1}{|y|^{d-1}} = \frac{1}{d |B_1| \epsilon^{d-1}} \quad \text{ for } y \in \partial B(0, \epsilon)
    \end{align*}

    Hence:
    \begin{align*}
      \int_{\partial B(0, \epsilon)} \frac{\partial \Phi}{\partial \vec{n}}(y) f(x-y) \ dS(y) 
      &= \frac{1}{d |B_1| \epsilon^{d-1}} \int_{\partial B(0, \epsilon)} f(x-y) \ dS(y) \\
      = \fint_{\partial B(0, \epsilon)} f(x-y) \ dS(y)
      &= \fint_{\partial B(x, \epsilon)} f(y) \ dS(y) 
      \xrightarrow{\epsilon \to 0} f(x)
    \end{align*}

      \begin{samepage}
        We have to regard the following error terms:
        \begin{itemize}
          \item \( 
              \begin{aligned}[t]
                \left| \int_{B(0, \epsilon)} \Phi(y) (- \Delta_y) f(x-y) \ dy\right| 
                &\le \int_{B(0, \epsilon)}|\Phi(y)| \underbrace{|-\Delta_y f(x-y)|}_{\le \|\Delta f\|_{L^\infty} \mathbb{1}(|y| \le R)} \ dy \\
                &\le \| \Delta f\|_{L^\infty} \int_{\mathbb{R}^d} \underbrace{|\Phi(y)| \mathbb{1}(|y| \le R)}_{L^1(\mathbb{R}^d)} \mathbb{1}(|y| \le \epsilon) 
                \xrightarrow{\epsilon \to 0} 0
              \end{aligned}
            \)
          Where \(R > 0\) depends on \(x\) and the support of \(f\) but is independent of \(y\).
        \item \( 
            \begin{aligned}[t]
              \left|\int_{\partial B(0, \epsilon)} \Phi(y) \frac{\partial f}{\partial \vec{n}}(x-y) \ dS(y)\right|
              &\le \|\nabla f\|_{L^\infty} \int_{\partial B(0, \epsilon)} |\Phi(y)| \ dy \\
              &\le \begin{cases}
                const \cdot \epsilon | \log \epsilon| \to 0, & d = 2 \\
                const \cdot \epsilon \to 0, & d \ge 3
              \end{cases}
            \end{aligned}
          \)
        \end{itemize}
      \end{samepage}
      Conclusion: \(-\Delta u(x) = f(x)\) for all \(x \in \mathbb{R}^d\) proved that \(u = \Phi \star f\) and \(f \in C_c^2(\mathbb{R}^d)\).
  \end{proof}

  Thus, if \(f \in C_c^2(\mathbb{R})\), then \(u = \Phi \star f\) satisfies \(u \in C^2(\mathbb{R}^2)\) and \(-\Delta u(x) = f(x)\) for all \(x \in \mathbb{R}^d\).


\begin{rem}
  The result holds for a much bigger class of functions \(f\). For example if \(f \in C_c^1(\mathbb{R})\) we can easily extend the previous proof:
  \begin{align*}
    \partial_{x_i} u = \int_{\mathbb{R}^d} \Phi(y) \partial_{x_i} f(x-y) \, dy \in C(\mathbb{R}^d) \Rightarrow u \in C^1(\mathbb{R}^d)
  \end{align*}
  Consequently: 
  \begin{align*}
    \partial_{x_i} \partial_{x_j} u 
    &= \partial_{x_i} \int_{\mathbb{R}^d} \Phi(y) \partial_{x_j} f(x-y) \, dy
    = \int_{\mathbb{R}^d} \partial_{x_i} \Phi(y) \partial_{x_j} f(x-y) \, dy \in C(\mathbb{R}^d)
  \end{align*}
  So we have \(u \in C^2(\mathbb{R}^d)\). Now we can compute
  \begin{align*}
    \Delta u = \sum_{i=1}^d \int_{\mathbb{R}^d} \partial_{x_i} \Phi(y) \partial_{x_i} f(x-y) \, dy \overset{(IBP)}{=} f(x).
  \end{align*}
\end{rem}

\begin{ex}
  Extend this to more general functions!
\end{ex}

\section{Equations in general domains}
\begin{thm} (Mean Value Theorem for Harmonic Functions)\label{mean-value-theorem} 
  Let \(\Omega \subseteq \mathbb{R}\) be open, let \( u \in C^2(\Omega)\) and \(\Delta u = 0\) in \(\Omega\). Then
  \begin{align*}
    u(x) 
    &= \fint_{B(x, r)} u
    = \fint_{\partial B(x, r)} u \quad \text{for all } x \in \Omega, B(x,r) \subseteq \Omega
  \end{align*}
\end{thm}

\begin{proof}
  Consider all \(r > 0\) s.t. \(B(x,r) \subseteq \Omega\),
  \begin{align*}
    f(r) &= \fint_{\partial B(x,r)} u
  \end{align*}
  We need to prove that \(f(r)\) is independent of \(r\). When it is done, then we immediately obtain
  \begin{align*}
    f(r) = \lim_{t \to 0} f(t) = u(x)
  \end{align*}
  as \(u\) is continuous. To prove that, consider
  \begin{align*}
    f'(r) 
    &= \frac{d}{dr} \left(\fint_{\partial B(0, r)} u(x+y) \, dS(y) \right) \\
    &= \frac{d}{dr} \left(\fint_{\partial B(0, 1)} u(x + rz) \, dS(z) \right) \\
    \text{(dom.\ convergence)} \quad & = \fint_{\partial B(0, 1)} \frac{d}{dr} [u(x + rz)] \, dS(z) \\
    &= \fint_{\partial B(0, 1)} \nabla u(x + rz) z \, dS(z) \\
    &= \fint_{\partial B(x, r)} \nabla u(y) \frac{y-x}{r} \, dS(y) \\
    &= \frac{1}{|B(x, r)|_{\mathbb{R}^d}} \int_{\partial B(x, r)} \nabla \cdot u(y) \cdot \vec{n_y} \, dS(y) \\
    \text{(Gauss-Green \ref{gauss-green})} \quad &= \frac{1}{|B(x, r)|_{\mathbb{R}^d}} \int_{B(x, r)} \underbrace{(\Delta u)(y)}_{= 0} \, dy = 0 \qedhere
  \end{align*}
\end{proof}

\begin{ex}
  In 1D\@: \(\Delta u = 0 \Leftrightarrow u'' = 0 \Leftrightarrow u(x) = ax + b\) (Linear Equation)
\end{ex}

\begin{rem}
  Recall the polar decomposition. Let \(x \in \mathbb{R}^d, x = (r,w), r = |x| > 0, \omega \in S^{d-1}\), then
  \begin{align*}
    \int_{B(0, r)} g(y) \, dy = \int_0^r \left(\int_{B(0, r)} g(y) \, dS(y) \right) dr
  \end{align*}
\end{rem}


\begin{rem}
  We already proved that for \(u\) harmonic we have \( u(x) = \fint_{\partial B(x,r)} u \, dy \). Now we have 
  \begin{align*}
    \int_{B(x, r)} u(y) \, dy 
    &= \int_{B(0, r)} u(x+y) \, dy \\
    \text{(Pol.\ decomposition)} \quad &= \int_0^r \left(\int_{\partial B(0, s)} u(x+y) \, dS(y)\right) ds \\
    &= \int_0^r \left(\int_{\partial B(x, s)} u(y) \, dS(y) \right) ds \\
    \text{(Mean value property)} \quad &= \int_0^r \left(|\partial B(x, s)| \, u(x) \right) ds
    = |B(x,r)| \, u(x)
  \end{align*}
  This implies
  \begin{align*}
    \fint_{B(x,r)} u(y) \, dy = u(x)
    \quad \text{for any \(B(x,r) \subseteq \Omega\).}
  \end{align*}
\end{rem}

\begin{rem}
  The reverse direction is also correct, namely if \(u \in C^2(\Omega)\) and
  \begin{align*}
    u(x) 
    &= \fint_{B(x, r)} u(y) \, dy
    = \fint_{\partial B(x,r)} u(y) \, dy
    \quad \text{for all } B(x,r) \subseteq \Omega,
  \end{align*}
  then \(u\) is harmonic, i.e. \(\Delta u = 0\) in \(\Omega\). (The proof is exactly like before)
\end{rem}

\begin{thm}[Maximum Principle]\label{maximum-principle}
  Let \(\Omega \subseteq \mathbb{R}^d\) be open, let \(u \in C^2(\Omega) \cap C(\bar \Omega)\), \(\Delta u = 0\) in \(\Omega\). Then
  \begin{enumerate}[label=\alph*)]
    \item \(\sup_{x \in \bar \Omega} u(x) = \sup_{x \in \partial \Omega} u(x)\)
    \item Assume that \(\Omega\) is connected. Then if there is a \(x_0 \in \Omega\) s.t. \( u(x_0) = \sup_{x \in \bar \Omega} u(x)\), then \( u \equiv const.\) in \(\Omega\).
  \end{enumerate}
\end{thm}

\begin{proof}
  Given \(U \subseteq \mathbb{R}^d\) open, we can write \(U = \bigcup_i U_i\), where \(U_i\) is open and connected.
  \begin{enumerate}
    \item[b)] Assume that \(\Omega\) is connected and there is a \(x_0 \in \Omega\) s.t. \(u(x_0) = \sup_{y \in \Omega} u(x)\). Define \(U = \{ x \in \Omega \mid u(x) = u(x_0)\} = u^{-1}(u(x_0))\). \(U\) is closed since \(u\) is continuous.
    Moreover, \(U\) is open by the mean-value theorem. I.e.~for all \(x \in U\) there is a \(r > 0\) s.t. \(B(x,r) \subseteq U \subseteq \Omega\).
    Since \(U\) is connected we get \(U = \Omega\), so \(u\) is constant in \(\Omega\). On the other hand, if there is no \(x_0 \in \Omega\) s.t. \(u(x_0) = \sup_{x \in \Omega}\) we have \(\forall x_0 \in \Omega: \quad u(x) < \sup_{x \in \bar \Omega} u(x) = \sup_{x \in \partial \Omega} u(x)\)
    \item[a)] Given \(\Omega \subseteq \mathbb{R}^d\) open, we can write \(\Omega = \bigcup_i \Omega_i\), where \(\Omega_i\) is open and connected. By b) we have
      \[\sup_{x \in \bar \Omega_i} u(x) = \sup_{x \in \partial \Omega_i} u(x), \quad \forall i\]
      So we can conclude
      \[\quad \sup_{x \in \bar \Omega} u(x) = \sup_{x \in \partial \Omega} u(x). \qedhere\]
  \end{enumerate}
\end{proof}

\begin{defn}
  \begin{itemize}
    \item If \(\Omega \subseteq \mathbb{R}^d\) is open, \(u \in C^2(\Omega)\), then \(u\) is called \emph{sub-harmonic} if \(\Delta u \ge 0\) in \(\Omega\).
    \item If \(\Delta u \le 0\), then \(u\) is called \emph{super-harmonic}.
  \end{itemize}
\end{defn}

\begin{ex}[E 1.4]
  Let \(\Omega \subseteq \mathbb{R}^d\) be open and \(u \in C^2(\Omega)\) be subharmonic.
  \begin{enumerate}[label=\alph*)]
    \item Prove that \(u\) satisfies the Mean Value Inequality
     \begin{align*}
       \fint_{\partial B(x, r)} u(y) \, dS(y)
       \ge \fint_{B(x, r)} u(y) \, dy
       \ge u(x)
     \end{align*}
     for all \(B(x,r) \subseteq \mathbb{R}^d\).
    \item Assume further that \(\Omega\) is connected and \(u \in C(\bar \Omega)\). Prove that \(u\) satisfies the strong maximum principle, namely either
    \begin{itemize}
      \item \(u\) is constant in \(\Omega\), or 
      \item \(\sup_{y \in \partial \Omega} u(y) > u(x)\) for all \(x \in \Omega\).
    \end{itemize}
  \end{enumerate}
\end{ex}

\begin{proof}[My Solution]
  \begin{enumerate}[label=\alph*)]
    \item Let \(f(r) = \fint_{\partial B(x, r)} u(y) \, dS(y)\), then we have
      \begin{align*}
        \partial_r f(r)
        &= \partial_r \fint_{\partial B(x,r)} u(y) \, dS(y) \\
        \text{(Dom. Convergence)} \quad &= \fint_{\partial B(x, r)} \partial_r u(y) \, dS(y) \\
        &= \fint_{\partial B(0,1)} \partial_r u(x+yr) \, dS(y) \\
        &= \fint_{\partial B(0,1)} \nabla u(x+yr) \cdot y \, dS(y) \\
        &= \fint_{\partial B(x, r)} \nabla u(y) \cdot \frac{y-x}{r} \, dS(y) \\
        &= \fint_{\partial B(x,r)} \nabla u(y) \cdot \vec{n}_y \, dS(y) \\
        \text{(Gauss-Green)} \quad &= \fint_{B(x, r)} \div(\nabla u(y)) \, dS(y) \\
        &= \fint_{B(x,r)} \underbrace{\Delta u(y)}_{\ge 0} \, dS(y) \ge 0
      \end{align*}
      So we can conclude that
      \[\fint_{\partial B(x, r)} u(y) \, dS(y) = f(r) \ge \lim_{r \to 0} f(r) = u(x).\]

      Now regard
      \begin{align*}
        \int_{B(x,r)} u(y) \, dy 
        &= \int_0^r \left(\int_{\partial B(x, r)} u(y) \, dS(y)\right) \, ds \\
        &= \int_0^r \left(|\partial B(x, r)| \fint_{\partial B(x,r)} u(y) \, dS(y)\right) \, ds \\
        &\ge \int_0^r |\partial B(x, r)| \cdot  u(x) \, dS(y) \\
        &= u(x) \int_0^r |\partial B(x, r)| \, dS(y)
        = u(x) |B(x,r)|.
      \end{align*}

      Thus we have
      \begin{align*}
        u(x) \le \fint_{B(x,r)} u(y) dy.
      \end{align*}

      Finally, lets regard
      \begin{align*}
        \int_{B(x, r)} u(y) \, dy
        &= \int_0^r \left(|\partial B(x,s)| \fint_{\partial B(x, s)} u(y) \, dS(y)\right) \, ds \\
        (\partial_r f(r) \ge 0) \quad &\le \int_0^r \left(|\partial B(x,s)| \fint_{\partial B(x, r)} u(y) \, dS(y)\right) \, ds \\
        &= \fint_{\partial B(x, r)} u(y)  \, dS(y) \int_0^r |\partial B(x,s)| \, ds \\
        &= \fint_{\partial B(x, r)} u(y)  \, dS(y) \cdot |B(x,s)|
      \end{align*}
      and we conclude
      \[\fint_{B(x, r)} u(y) \, dy \le \fint_{\partial B(x,r)} u(y) \, dS(y).\]
    \item Let \(x_0 \in \Omega\) s.t. \(u(x_0) = \sup_{x \in \Omega} u(x)\). Now, \begin{align*}
      \sup_{x \in \Omega} u(x) 
      &= u(x_0) 
      \le \fint_{\partial B(x_0, r)} u(y) \, dy \\
      &\le \fint_{\partial B(x_0, r)} \sup_{x \in \Omega} u(x) \, dy 
      = \sup_{x \in \Omega} u(x)
    \end{align*}
    Since \(u\) is continous we get \(u(y) = u(x_0)\) for all \(y \in B(x_0, r)\), so \(u\) is constant. \qedhere
  \end{enumerate}
\end{proof}

\begin{defn}
  The \emph{Poisson Equation} for given \(f, g\) on a bounded set is:
  \begin{align*}
    \begin{cases}
      - \Delta u = f, &\text{in } \Omega \\
      u = g, &\text{on } \partial \Omega
    \end{cases}
  \end{align*} 
\end{defn}

\begin{thm}(Uniqueness)
  Let \(\Omega \subseteq \mathbb{R}^d\) be bounded, open  and connected. Let \(f \in C(\Omega), g \in C(\partial \Omega)\). Then there exists \emph{at most} one solution \(u \in C^2(\Omega) \cap C(\bar \Omega)\), s.t. \begin{align*}
    \begin{cases}
      - \Delta u = f, &\text{in } \Omega \\
      u = g, &\text{on } \partial \Omega
    \end{cases}
  \end{align*}
\end{thm}

\begin{proof}
  Assume that we have two solutions \(u_1\) and \(u_2\). Then \(u \coloneqq u_1 - u_2\) is a solution to 
  \begin{align*}
    \begin{cases}
      - \Delta u = 0, &\text{in } \Omega \\
      u = 0 &\text{on } \partial \Omega
    \end{cases}
  \end{align*}
  By the maximum principle, we know that \(u = 0\) in \(\Omega\). More precisely, by the maximum principle we have \(\forall x \in \Omega\)
  \begin{align*}
    \sup_{x \in \Omega} u(x) \le \sup_{x \in \partial \Omega} u(x) = 0
    \quad \Rightarrow \quad
    u(x) \le 0
  \end{align*}
  Since \(-u\) satisfies the same property we have \(\forall x \in \Omega\):
  \begin{align*}
    \sup_{x \in \Omega}(-u(x)) \le \sup_{x \in \partial \Omega} (-u(x)) = 0
    \quad \Rightarrow \quad
    - u(x) \le 0
    \quad \Rightarrow \quad
    u(x) \ge 0
  \end{align*}
  So we geht \(u(x)  = 0\) in \(\Omega\).
\end{proof}

\begin{ex}[Bonus 1]\label{bonus-1}
  Let \(\Omega\) be open, connected and bounded in \(\mathbb{R}^d\). Let \(u \in C^2(\Omega) \cap C(\bar \Omega)\) s.t. 
  \begin{align*}
    \begin{cases}
      \Delta u = 0, &\text{in } \Omega \\
      u = g, &\text{on } \partial \Omega
    \end{cases}
  \end{align*}
  Prove that \begin{enumerate}[label=\alph*)]
    \item If \(g \ge 0\) on \(\partial \Omega\), then \(u \ge 0\) in \(\Omega\). 
    \item If \(g \ge 0\) on \(\partial \Omega\) and \(g \ne 0\), then \(u > 0\) in \(\Omega\).
  \end{enumerate}
\end{ex}

\begin{proof}[My Solution]
  \begin{enumerate}[label=\alph*)]
    \item We have that \(\Delta (-u) = 0\), so \(-u\) is harmonic in \(\Omega\). Since \(\Omega\) is open and bounded we can apply the Maximum Principle (\ref{maximum-principle}) and get that
    \[\sup_{x \in \bar \Omega} -u(x) \le \sup_{x \in \partial \Omega} -g(x) \le 0.\]
    This implies \(\inf_{x \in \Omega} u(x) \ge 0\), so \(u \ge 0\) in \(\Omega\).
    \item We prove this by contraposition. Assume there is a \(x_0 \in \Omega\) s.t. \(u(x_0) = 0\). Since we have \(u \ge 0\) on \(\Omega\) by a), it follows that
    \[0 = -u(x_0) = \sup_{x \in \bar \Omega} -u(x) \le \sup_{x \in \partial \Omega} -g(x) \le 0,\]
    so \(-u\) attains its maximum on \(\Omega\). Hence \(-u = 0 = u\) is constant by the strong maximum principle because \(\Omega\) is connected, in fact \(0 = u|_{\partial \Omega} = g\). \qedhere
  \end{enumerate}
\end{proof}

\begin{lem}[Estimates for derivatives]\label{estimates-of-derivatives}
  If \(u\) is harmonic in \(\Omega \subseteq \mathbb{R}^d\), \(\alpha \in \mathbb{N}_0^d\), \(|\alpha| = k\) and \(B(x_0, r) \subseteq \Omega\), then 
  \[|D^\alpha u(x_0)| \le \frac{c_k}{r^{d+k}} \int_{B(x_0, r)} |u| \, dy,\] where \(c_0 = \frac{1}{|B(0,1)|}\) and \(c_k = \frac{(2^{n+1}nk)^k}{|B(0,1)|}\) for \(k=1,\dots\)
\end{lem}

\begin{thm}[Regularity]
  Let \(\Omega\) be open in \(\mathbb{R}^d\). Let \(u \in C(\Omega)\) satisfy \(u(x) = \fint_{\partial B} u \, dy\) for any \(x \in B(x, r) \subseteq \Omega\). Then \(u\) is a harmonic function in \(\Omega\). Moreover, \(u \in C^\infty(\Omega)\) and \(u\) is analytic in \(\Omega\).
\end{thm}

% \begin{proof}
%   We use the convolution. For simlicity consider the case \(\Omega = \mathbb{R}^d\) first. Take \(\eta \in C_c^\infty(\mathbb{R}^d)\) with \(0 \le \eta \le 1\), \(\eta(x) = 0\) if \(|x| \ge 1\), \(\eta\) radial and \(\int \eta = 1\). Define \(\eta_\epsilon (x) = \epsilon^{-d} \eta(\epsilon^{-1} x)\) for all \(\epsilon > 0\). Then
%   \begin{align*}
%     \int_{\mathbb{R}^d} \eta_\epsilon = \int_{\mathbb{R}^d} \eta = 1
%   \end{align*}
%   We prove \(u_\epsilon \coloneqq \eta_\epsilon \star u = u\) for all \(\epsilon > 0\). By definition:
%   \begin{align*}
%     u_\epsilon(x) 
%     &= \int_{\mathbb{R}^d} \eta_\epsilon(x-y)u(y) \, dy \\
%     &= \int_0^\infty \left[\int_{\partial B(x, r)} \eta_\epsilon(x-y) u(y) \, dS(y)\right] dr \\
%     (\eta \text{ radial}) \quad &= \int_0^\infty \left[\eta_\epsilon(r) \int_{\partial B(x, r)} u(y) \, dS(y)\right] dr \\
%     \text{(Assumption)} \quad &= \int_0^\infty \eta_\epsilon(r)\, |\partial B(x, r)|\, u(x) \, dr \\
%     &= u(x) \int_0^\infty \eta_\epsilon(r) |\partial B(0, r)| \, dr \\
%     &= u(x) \int_{\mathbb{R}^d} \eta_\epsilon(y) \, dy = u(x)
%   \end{align*}

%   On the other hand, \(u_\epsilon = \eta_\epsilon \star u\) is \(C^\infty(\mathbb{R}^d)\). In fact \(D^\alpha(\eta_\epsilon \star u) = (D^\alpha \eta_\epsilon) \star u\) is continuous for any \(\alpha\) (Exercise). Then \(u \in C^\infty(\mathbb{R}^d)\), so \(u\) is harmonic in \(\mathbb{R}^d\), i.e. \(\Delta u = 0\) in \(\mathbb{R}^d\). \\

%   Consider now the general case where \(\Omega \subseteq \mathbb{R}^d\) is open. Take \(\epsilon > 0\) small and define \(\Omega_\epsilon = \{x \in \Omega \mid \dist(x, \partial \Omega) > \epsilon\}\). 
%   Define \[u_\epsilon(x) = \int_{\mathbb{R}^d} \eta_\epsilon(x-y) u(y) \, dy \quad \text{ for all } x \in \Omega_\epsilon\]
%   Recall that \(\eta_\epsilon(y) = 0\) if \(|y| \ge \epsilon\), then:
%   \[u_\epsilon(x) = \int_{B(x, \epsilon)} \eta_\epsilon(x-y)u(y) \, dy\]
%   is well-defined since \(B(x,\epsilon) \subseteq \Omega\) for all \(x \in \Omega_\epsilon\).
%   Then by the same computation using the polar-decomposition, we find that \(u_\epsilon(x) = u(x)\) for all \(x \in \Omega\). Note that \(u_\epsilon \in C^\infty(\Omega_\epsilon)\). Taking \(\epsilon \to 0\), we get \(u \in C^\infty(\Omega)\). Then we conclude that \(u\) is harmonic (We need to reverse the proof of the mean-value theorem).\\
%   To proof that \(u\) is analytic, we need to show that for all \(x_0 \in \Omega\), there is a \(r > 0\) s.t. \(B(x_0, r) \subseteq \Omega\) and \[u(x) = u(x_0) + \sum_{\alpha \ne 0} c_\alpha(x-x_0)^\alpha \quad \text{for  all } x \in B(x_0, r)\]
%   Here \(\alpha = (\alpha_1, \dots, \alpha_d), \alpha_i \in \{0, 1, 2, \dots\}\) and \(y^\alpha = y_1^{\alpha_1}y_2^{\alpha_2} \dots y_d^{\alpha_d}\). 
%   We want to prove that the series converges uniformly in \(B(x_0, r)\). Recall the Taylor expansion:
%   \[u(x) = u(x_0) + \sum_{0 < |\alpha| < N} D^\alpha u(x_0) \frac{(x-x_0)^\alpha}{\alpha!} + R_N(x)\]
%   where \(|\alpha| = \alpha_1 + \alpha_2 + \dots + \alpha_d\), \(\alpha! = \alpha_1! \cdots \alpha_d!\) and \[R_N(x) = \sum_{|\alpha| = N} \int_0^1 D^\alpha u(x_0 + t(x-x_0)) \frac{(x-x_0)^\alpha}{\alpha!} \, dt\]
%   New: Let \(x_0 \in \Omega\), take \(r > 0\), \(r < \frac{1}{L+1} \dist(x_0, \Omega^c)\) s.t. if \(x \in B(x_0, r)\), then \[B(x, Lr) \subseteq B(x_0, (L+1)r) \subseteq \Omega\]. With Lemma~\ref{estimates-of-derivatives} we get:
%   \begin{align*}
%     |D^\alpha u(x_0 + t(x-x_0))
%     &\le \frac{(c_d N)^N}{(Lr)^{d+N}} \int_{B(x, Lr)} |u|
%   \end{align*}
%   With \((x_0, r) \leadsto (x, Lr)\)
%   \begin{align*}
%     |R_N(x)| \le \sum_{|\alpha| = N} \frac{(c_d N)^N}{(Lr)^{d+N}} \frac{1}{\alpha!} \frac{1}{\alpha!} \tau^N \int_{B(x_0, (L+1)r) |u|}
%   \end{align*}

%   Thus 
%   \begin{align*}
%     \left(\frac{\tilde c_d N}{L}\right) \frac{1}{N!}
%     &\le \left(\frac{\tilde c_d N}{L}\right) \left(\frac{e}{N}\right)^N \quad \text{ if \(N\) large} \\
%     &= \left(\frac{\tilde c_d e}{L}\right)^N \xrightarrow{N \to \infty} 0 \quad \text{if } L > \tilde c_d e (L = L_d)
%   \end{align*}
%   We conclude that \[u(x) = u(x_0) + \sum_{\alpha \ne 0} \frac{D^\alpha u(x_0)}{\alpha!}(x-x_0)^\alpha\]
%   The series converges uniformly \(x \in B(x_0, r)\).
%   Now we proof the bound on derivatives. For \(\alpha = 0\)
%   \begin{align*}
%     |u(x_0)| = \left| \fint_{B(x_0, r)} u \right| \le \frac{1}{|B_1|r^d} \int_{B(x_0, r)} |u|
%   \end{align*}
%   For \(\alpha = 1:\) \(\Delta u = 0\) in \(\Omega\) \(\Rightarrow\) \(0 = \partial_{x_i} (\Delta u) = \Delta (\partial_{x_i} u)\), so \(\partial_{x_i} u\) is harmonic in \(\Omega\). Hence, by the mean-varlue theorem again:
%   \begin{align*}
%     \partial_{x_i} u(x_0) &= \fint_{B(x_0, \frac{r}{2})} \partial_{x_i} u = \frac{1}{|B_1| \frac{\frac{r}{2}}{2}^d} \int_{B(x_0, \frac{r}{2})} \partial_{x_i} u = \frac{1}{|B_1| \frac{r}{2}^d} \int_{\partial B(x_0, \frac{r}{2})}u n_i \, dS
%   \end{align*}
%   So we get:
%   \begin{align*}
%     |\partial_{x_i} u(x_0)| 
%     &= \frac{1}{|B_1| r^d} \int_{\partial B(x, r)} \, dS \|u\|_{L^\infty(\partial B(x_0, \frac{r}{2}))}  \\
%     &= \frac{|S_1|}{|B_1|\frac{r}{2}} \|u\|_{L^\infty(\partial B(x_0, \frac{r}{2}))}
%   \end{align*}
%   For any \(y \in \partial B(x_0, \frac{r}{2})\) by the mean value theorem, we get:
%   \begin{align*}
%     |u(y)| 
%     &= \left| \fint_{B(y, \frac{r}{2})} u\right|
%     &\le \frac{1}{|B_1| \left(\frac{r}{2}\right)^d} \int_{B(y, \frac{r}{2})} |u|
%     &\le \frac{1}{|B_1| \left(\frac{r}{2}\right)^d} \int_{B(x_0, r)} |u|
%   \end{align*}
%   Thus,
%   \begin{align*}
%     |\partial_{x_i} u(x_0)| 
%     &\le \frac{|S_1|}{|B_1| \left(\frac{r}{2}\right)} \frac{1}{|B_1|  \left(\frac{r}{2}\right)^d} \int_{B(x_0, r)} |u|
%     \le \frac{c_d}{r^{d+1}} \int_{B(x_0, r)} |u|
%   \end{align*}
%   Induction: Assume that we already proved the bound when \(|\alpha| = N-1\). Then:
%   \begin{align*}
%     \partial_{x_i} D^\alpha u 
%     &= D^\alpha(\underbrace{\partial_{x_i} u}_{\text{harmonic}})
%     = 0
%     \quad \Rightarrow \quad
%     D^\alpha u \text{ is harmonic}
%   \end{align*}
%   So we get
%   \begin{align*}
%     \partial_{x_i} (D^\alpha u) &= \fint_{B(x_0, \frac{r}{4})} \partial_{x_i} (D^\alpha u) \\
%     \Rightarrow \quad |\partial_{x_i} (D^\alpha u)| &\le  \frac{C_d}{r^{d+1}} \int_{B(x_0, \frac{r}{2})} |D^\alpha u|
%   \end{align*}
%   and by the induction hypothesis:
  
%   \begin{align*}
%     |D^\alpha u(x_0)| &\le \frac{c_d}{r} \|D^\alpha u\|_{L^\infty B(x_0, \frac{r}{2})} \\
%     &\le \frac{c_d}{r^{d+N-1}} \int_{B(x_0, r)} |u| \quad \forall x \in B\left(x_0, \frac{r}{2}\right)
%   \end{align*}
%   Then: \(|\partial_{x_i} D^\alpha u(x_0)| \le \frac{c_d}{r^{d+N}} \int_{B(x_0, r)} |u|\)
% \end{proof}

\begin{ex}[E 1.1: Proof the Gauss–Green formula]
  Let \(f \coloneqq (f_i)_1^d \in C^1(\mathbb{R}^d, \mathbb{R}^d)\). Prove that for every open ball \(B(y, r) \subseteq \mathbb{R}^d\) we have \[\int_{\partial B(y, r)} f(y) \cdot \nu_y \, dS(y) = \int_{B(y, r)}\div f \, dx.\]
  Here \(\nu_y\) is the outward unit normal vector and \(dS\) is the surface measure on the sphere.
\end{ex}

\begin{ex}[E 1.2]
  Let \(u \in C(\mathbb{R}^d)\) and \(\int_{B(x, r)} u = 0\) for every open ball \(B(x, r) \subseteq \mathbb{R}^d\). Show that \(u(x) = 0\) for all \(x \in \mathbb{R}^d\).
\end{ex}

\begin{proof}[My Solution]
  Assume there is a \(x_0 \in \mathbb{R}^d\) s.t. w.l.o.g. \(u(x_0) > 0\). Since \(u\) is continous there is a ball \(B(x_0, r)\) s.t. \(u(y) > \frac{u(x_0)}{2}\) for all \(y \in B(x_0, r)\). But then we get
  \begin{align*}
    \int_{B(x_0, r)} u(y) \, dy
    &\ge \int_{B(x_0, r)} \frac{u(x_0)}{2} \, dy
    = \frac{u(x_0)}{2} \, |B(x_0, r)| > 0. \qedhere
  \end{align*}
\end{proof}

\begin{ex}[E 1.3]
  Let \(f \in C_c^1(\mathbb{R}^d)\) with \(d \ge 2\) and \(u(x) \coloneqq (\Phi \star f)(x)\). Prove that \(u \in C^2(\mathbb{R}^2)\) and \(- \Delta u(x) = f(x)\) for all \(x \in \mathbb{R}^d\) (\ref{solution-for-poisson} was the same for \(f \in C_1(\mathbb{R})\))
\end{ex}

\begin{thm}[Liouville's Theorem] 
  If \(u \in C^2(\mathbb{R}^d)\) is harmonic and bounded, then \(u = const.\)
\end{thm}

\begin{proof}
  By the bound of the derivative \ref{estimates-of-derivatives} we have
  \begin{align*}
    |\partial_{x_i} u(x_0)| 
    &\le \frac{c_d}{r^{d+1}} \int_{B(x_0, r)} |u| \, dy \quad \forall x_0 \in \mathbb{R}^d\ \forall r > 0 \\
    &\le \|u\|_{L^\infty} \frac{c_d}{r^{d+1}} |B(x_0, r)| \\
    &\le \|u\|_{L^\infty} \frac{c_d}{r} \xrightarrow{r \to \infty} 0
  \end{align*}
  Thus \(\partial_{x_i} u = 0\) for all \(i = 1, 2, \dots d\) and \(u = const.\) in \(\mathbb{R}^d\)
\end{proof}

\begin{thm}[Uniqueness of solutions to Poisson Equation in \(\mathbb{R}^d\)]
  If \(u \in C^2(\mathbb{R}^d)\) is a bounded function and satisfies \(- \Delta u = f\) in \(\mathbb{R}^d\) where \(f \in C_c^2(\mathbb{R}^d)\), then we have
  \begin{align*}
    u(x) = \Phi \star f(x) + C = \int_{\mathbb{R}^d} \Phi(x-y)f(y) \, dy + C \quad \forall x \in \mathbb{R}^d
  \end{align*}
  where \(C\) is a constant and \(\Phi\) is the fundamental solution of the Laplace equation in \(\mathbb{R}^d\).
\end{thm}

\begin{proof}
  If we can prove that \(v\) is bounded, then \(v = const.\). We first need to show that \(\Phi \star f\) is bounded.
  \begin{align*}
    \Phi = \Phi_1 + \Phi_2 = \Phi \mathbb{1}(|x| \le 1) + \Phi(|x| \ge 1) \\
    \Phi \star f = \Phi_1 \star f + \Phi_2 \star f
  \end{align*}
  We have \(\Phi_1 \star f \in L^1(\mathbb{R}^d)\) and \(\Phi_2 \star f\) is bounded since \(\Phi \to 0\)  as \(|x| \to \infty\)  in \(d \ge 3\).
\end{proof}

\begin{ex} (Hanack's inequality)
  Let \(u \in C^2(\mathbb{R}^d)\) be harmonic and non-negative. Prove that for all open, bounded and connected \(\Omega \subseteq \mathbb{R}^d\), we have
  \begin{align*}
    \sup_{x \in \Omega} u(x) \le C_\Omega \inf_{x \in \Omega} u(x),
  \end{align*}
  where \(C_\infty\) is a finite constant depending only on \(\Omega\).
\end{ex}

\begin{proof} (Exercise)
  Hint: \(\Omega = B(x,r)\). General case cover \(\Omega\) by finitely many balls, one ball is inside \(\Omega\).
\end{proof}


\begin{defn}[Convolution]
  Let \(f, g: \mathbb{{R}^d \to \mathbb{R}}\) or \(\mathbb{C}\).
  \begin{align*}
    (f \star g)(x)
    &= \int_{\mathbb{R}^d} f(x-y) g(y) \, dy
    = \int_{\mathbb{R}^d} f(y) g(x-y) \, dy 
    = (g \star f)(x)
  \end{align*}
\end{defn}

\begin{rem}[Properties of the Convolution]
  \begin{itemize}
    \item \((f \star g)(x) = f \star (g \star h)\)
    \item \(\hat{f \star g} = \hat f \star \hat g\)
  \end{itemize}
\end{rem}


\begin{thm}[Young Inequality]
  If \(f \in L^1(\mathbb{R}^d)\) and \(g \in L^p(\mathbb{R}^d)\), where \(1 \le p \le \infty\), then \(f \star g \in L^p(\mathbb{R}^d)\) and \(\|f \star g \|_{L^p} \le \|f\|_{L^1} \|g\|_{L^p}\). More generally, if \(f \in L^p(\mathbb{R}^d), g \in L^q(\mathbb{R}^d)\), then \(f \star g \in L^1(\mathbb{R}^d)\), \(\| f \star g\|_{L^1} \le \|f\|_{L^p} \|g\|_{L^q}\), where \(1 \le p, q, r, \le \infty\), \(\frac{1}{p} + \frac{1}{q} = 1 + \frac{1}{r}\)
\end{thm}

\begin{proof}
  Let \(f \in L^1, g \in L^p\). With the Hölder Inequality \ref{hölder-inequality}, we have:
  % \begin{align*}
  %   \left|(f \star g)(x)| 
  %   &= \left| \int_{\mathbb{R}^d} f(x-y)g(y) \, dy \right| \\
  %   &\le \left(\int_{\mathbb{R}^d} |f(x-y)| \, dy\right)^{\frac{1}{q}} \left(\int_{\mathbb{R}^d} |f(x-x)| |g(y)|^p \, dy \right)^{\frac{1}{p}} \\
  %   &= \| f \|_{L^1}^{\frac{1}{q}} \left(\int_{\mathbb{R}^d} \dots \right)^{\frac{1}{p}}
  % \end{align*}
  \begin{align*}
    \|f \star g\|_{L^p}^p
    &= \int_{\mathbb{R}^d} |f \star g(x)|^p \, dx
    &\le \|f\|_{L^1}^{\frac{p}{q}} \int_{\mathbb{R}^d} \int_{\mathbb{R}^d} |f(x-y)||g(y)|^p \, dy \, dx \\
    &= \| f \|_{L^1}^{\frac{p}{q} + 1} \|g\|_{L^p}^p
  \end{align*}
  So we have \(\| f \star g \|_{L^p} \le \|f\|_{L^1} \|g\|_{L^p}\)
\end{proof}

\end{document}
