% chktex-file 11
% chktex-file 35
% chktex-file 2
% chktex-file 10
% chktex-file 40

\documentclass{report}

\usepackage{a4}

\usepackage[english]{babel}
\usepackage[utf8]{inputenc}
\usepackage{amsmath, amssymb, mathtools, esint, bbold, mathabx}
\usepackage{graphicx}
\usepackage{color}
\usepackage{datetime}
\usepackage[hidelinks]{hyperref}
\usepackage{enumitem}
\usepackage{stmaryrd}

\setlength{\parindent}{0em} 

\usepackage{amsthm}
\newtheoremstyle{tommy}% ⟨name ⟩
{15pt}% ⟨Space above ⟩1
{15pt}% ⟨Space below ⟩1
{\normalfont}% ⟨Body font ⟩
{}% ⟨Indent amount ⟩2
{\bfseries}% ⟨Theorem head font⟩
{}% ⟨Punctuation after theorem head ⟩
{0.7em}% ⟨Space after theorem head ⟩3
{}% ⟨Theorem head spec (can be left empty, meaning ‘normal’)⟩

\theoremstyle{tommy}
\newtheorem{defn}{Definition}
\newtheorem{thm}[defn]{Theorem}
\newtheorem{lem}[defn]{Lemma}
\newtheorem{nota}[defn]{Notation}
\newtheorem{cor}[defn]{Corrolary}
\newtheorem{prop}[defn]{Proposition}
\newtheorem{eg}[defn]{Example}
\newtheorem{rem}[defn]{Remark}
\newtheorem{ex}[defn]{Exercise}

% Counter
\usepackage{chngcntr}
\counterwithin{defn}{chapter}

\renewcommand\div{\operatorname{div}}
\newcommand{\dist}{\operatorname{dist}}
\newcommand{\four}{\mathcal{F}}
\newcommand{\supp}{\operatorname{supp}}
\newcommand{\sgn}{\operatorname{sgn}}
\newcommand{\Rd}{\mathbb{R}^d}
\newcommand{\Tr}{\operatorname{Tr}}
\renewcommand\qedsymbol{\(\blacksquare\)}

\title{Partial Differerential Equations \\ Thành Nam Phan \\ Winter Semester 2021/2022}
\author{Lecture notes \TeX{}ed by Thomas Eingartner}
\date{\today, \currenttime}

\begin{document}

  \maketitle
  \tableofcontents
  \newpage
  Please note that I write this lecture notes for my personal use. I may write things differently than presented in the lecture. This script also contains some of my personal solutions for exercises (which may be wrong).
  \newpage

  \chapter{Introduction}

  A differential equation is an equation of a function and its derivatives. 

  \begin{eg}[Linear ODE]
    Let \(f: \mathbb{R} \to \mathbb{R}\),
    \begin{align*}
      \begin{cases}
        f(t) = a f(t) \text{ for all } t \ge 0, a \in \mathbb{R} \\
        f(0) = a_0
      \end{cases}
    \end{align*}
    is a linear ODE (Ordinary differential equation). The solution is: \(f(t) = a_0 e^{at}\) for all \(t \ge 0\).
  \end{eg}

  \begin{eg} (Non-Linear ODE) \(f: \mathbb{R} \to \mathbb{R}\)
    \begin{align*}
      \begin{cases}
        f'(t) = 1 + f^2(t) \\
        f(0) = 1
      \end{cases}
    \end{align*}
    Lets consider \(f(t) = \tan(t) = \frac{\sin(t)}{\cos(t)}\). Then we have \[f'(t) = \frac{1}{\cos(t)} = 1 + \tan^2(t) = 1 + f^2(t),\] but this solution only is \emph{good} in \((- \pi, \pi)\). It's a problem to extend this to \(\mathbb{R} \to \mathbb{R}\).
  \end{eg}

  A PDE (Partial Differential Equation) is an equation of a function of 2 or more variables and its derivatives.

  \begin{rem}
    Recall for \( \Omega \subseteq \mathbb{R}^d\) open and \(f: \Omega \to \{\mathbb{R}, \mathbb{C}\}\) the notation of partial derivatives:
    \begin{itemize}
      \item \(\partial_{x_i} f(x) = \lim_{h \to 0} \frac{f(x + he_i) - f(x)}{h}, \text{where } e_i = (0, 0, \dots, 1, \dots, 0, 0) \in \mathbb{R}^d\)
      \item \(D^\alpha f(x) = \partial_{x_1}^{\alpha_1} \cdots \partial_{x_d}^{\alpha_d} f(x), \text{where } \alpha \in \mathbb{N}^d\)
      \item \(Df = \nabla f = (\partial{x_1}, \ldots, \partial_{x_d})\)
      \item \(\Delta f = \partial_{x_1}^2 + \cdots + \partial^2_{x_d} f\)
      \item \(D^k f = (D^\alpha f)_{|\alpha| = k},  \text{where } |\alpha| = \sum_{i=1}^d |\alpha_i|\)
      \item \(D^2 f = (\partial_{x_i} \partial_{x_j} f)_{1 \le i, j \le d}\)
    \end{itemize}
  \end{rem}

  \begin{defn}
    Given a function \( F \). Then the equation of the form 
    \begin{align*}
      F(D^k u(x), D^{k-1} u(x), \dots, Du(x), u(x), x) = 0 
    \end{align*}
    with the unknown function \(u:\ \Omega \subseteq \mathbb{R}^d \longrightarrow \mathbb{R}\) is called a \emph{PDE of order \(k\)}.
    \begin{itemize}
      \item Equations \(\sum_d a_\alpha(x) D^\alpha u(x) = 0\), where \(a_\alpha\) and \(u\) are unknown functions are called \emph{Linear PDEs}. 
      \item Equations \(\sum_{|\alpha| = k} a_\alpha(x) D^\alpha u(x) + F(D^{k-1}u, D^{k-2}u, \dots, Du, u, x) = 0\) are called \emph{semi-linear PDEs}.
    \end{itemize}
  \end{defn}

  Goals: For \emph{solving a PDE} we want to
  \begin{itemize}
    \item Find an explizit solution! This is in many cases impossible.
    \item Prove a \emph{well-posted theory} (existence of solutions, uniqueness of solutions, continuous dependence of solutions on the data)
  \end{itemize}

  We have two notations of solutions:
  \begin{enumerate}
    \item Classical solution: The solution is continuous differentiable (e.g. \( \Delta u = f  \leadsto u \in C^2  \))
    \item Weak Solutions: The solution is not smooth/continuous
  \end{enumerate}

  \begin{defn} (Spaces of continous and differentiable functions)
    Let \(\Omega \subseteq R^d\) be open
    \begin{align*}
      C(\Omega) &= \left\{ f: \ \Omega \to \mathbb{R} \mid f \text{ continuous} \right\} \\
      C^k(\Omega) &= \left\{ f: \ \Omega \to \mathbb{R} \mid D^\alpha f \text{ is continuous for all } |\alpha| \le k \right\}
    \end{align*}
  \end{defn}

  Classical solution of a PDE of order \(k \leadsto C^k\) solutions!
  \begin{align*}
    L^p(\Omega) = \left\{ f: \ \Omega \to \mathbb{R} \text{ lebesgue measurable } \middle| \int_\Omega |f|^p d\lambda < \infty,\ 1 \le p < \infty \right\}
  \end{align*}

  Sobolev Space:
  \begin{align*}
    W^{k,p}(\Omega)= \left \{ f \in L^p(\Omega) \mid \forall \alpha \in \mathbb{N}^n \text{ with } |\alpha| \leq k: D^{\alpha}f \in L^p(\Omega) \text{ exists}  \right \}
  \end{align*}

  In this course we will investigate
  \begin{itemize}
    \item Laplace / Poisson Equation: \(-\Delta u = f\)
    \item Heat Equation: \(\partial_t u - \Delta u = f\)
    \item Wave Equation: \(\partial_t^2 - \Delta u = f\)
    \item Schrödinger Equation: \(i \partial_t u - \Delta u = f\)
  \end{itemize}


  \chapter{Laplace / Poisson Equation}

  \section{Laplace Equation}
  \(- \Delta u = 0\) (Laplace) or \(-\Delta u = f(x)\) (Poisson).

  \begin{defn} (Harmonic Function)
    Let \(\Omega\) be an open set in \(\mathbb{R}^d\). If \(u \in C^2(\Omega)\) and \(\Delta u = 0\) in \(\Omega\), then \(u\) is a harmonic function in \(\Omega\).
  \end{defn}

  \begin{thm} (Gauss-Green Theorem)\label{gauss-green}
    Let \(A \subseteq \mathbb{R}^d\) open, \(\vec{F} \in C^1(A, \mathbb{R}^d)\) and \(K \subseteq A\) compact with \(C^1\) boundary. Then
    \[ \int_{\partial K} \vec{F} \cdot \vec{\nu}\ dS(x) = \int_K \div(\vec{F})\ dx \]
    where \(\nu\) is the outward unit normal vector field on \(\partial K\).
  Thus
  \begin{align*}
    \int_{\partial V} \nabla u \cdot \vec{\nu}\ dS(x)
    = \int_V \div(\nabla u) \ dx
    = \int_V \Delta u(x) \ dx
  \end{align*}
  for any \(V \subseteq \Omega \) open.
  \end{thm}


  \begin{thm} (Green's Identities)\label{green-identities}
    Let \(A \subseteq \mathbb{R}^d\) open, \(K \subseteq A\) d-dim.\ compactum with \(C^1\) boundary and \(f, g \in C^2(A)\)
    \begin{enumerate}
      \item Green's first identity (Integration by parts): \begin{align*}
        \int_K \nabla f \cdot \nabla g \, dx = \int_{\partial K} f \frac{\partial g}{\partial \nu} \, dS - \int_K f \Delta g \, dx
      \end{align*}
      where \(\frac{\partial g}{\partial \nu} = \partial_\nu g = \nu \cdot \nabla g\)
      \item Green's second identity: \begin{align*}
        \int_K f \Delta g - (\Delta f) g \, dx = \int_{\partial K} \left(f \frac{\partial g}{\partial \nu} - g \frac{\partial f}{\partial \nu}\right) \, dS
      \end{align*}
    \end{enumerate}
  \end{thm}

  \begin{ex}
    Let \(\Omega \subseteq \mathbb{R}^d\) open, let \(f: \Omega \to \mathbb{R}\) be continuous. Prove that if \( \int_B f(x) \ dx = 0 \), then \( u \equiv 0 \) in \(\Omega\).
  \end{ex}

  \begin{thm} (Fundamential Lemma of Calculus of Variations)
    Let \(\Omega \subseteq \mathbb{R}^d\) open, let \(f \in L^1(\Omega)\). If 
    \(\int_B f(x) \ dx = 0\) for all \(x \in B_r(x) \subseteq \Omega\), then \(f(x) = 0\) a.e. (almost everywhere) \(x \in \Omega\).
  \end{thm}

  \begin{rem} (Solving Laplace Equation)
    \(-\Delta u = 0\) in \(\mathbb{R}^d\). Consider the case when \(u\) is radial, i.e. \(u(x) = v(|x|)\), \(v: \mathbb{R} \to \mathbb{R}\). Denote \(r = |x|\), then 
    \[
      \frac{\partial r}{\partial x} 
      = \frac{\partial}{\partial x_i}  \left(\sqrt{x_1^2 + \dots + x_d^2}\right) \\
      = \frac{2 x_i}{2{\sqrt{x_1^2 + \dots + x_d^2}}} \\
      = \frac{x_i}{r}
    \]
    Then
    \begin{align*}
      \partial_{x_i} u &= \partial_{x_i} v = (\partial_r v) \frac{\partial r}{\partial {x_i}} 
      = v'(r) \frac{x_i}{r} \\
      \partial^2_{x_i} u 
      &= \partial_{x_i} \left(v(r)' \frac{x_i}{r}\right) 
      = (\partial_{x_i}v(r)') \frac{x_i}{r} + v'(r) \partial_{x_i} \left(\frac{x_i}{r}\right) \\
      &= (\partial_r v'(r))\left(\frac{dr}{\partial_{x_i}}\right) \frac{x_i}{r} + v'(r)\left( \frac{1}{r} - \frac{x_i}{r^2}(\partial_{x_i} r) \right) 
      = v'(r) \frac{x_i^2}{r^2} + v'r(r)\left(\frac{1}{r} - \frac{x_i^2}{r^3}\right)
    \end{align*}

    So we have \(\Delta u = \left( \sum_{i=1}^d d_{x_i}^2 \right) u = v''(r) + v'(r) (\frac{d}{r} - \frac{1}{r})\)

    Thus \(\Delta u = v'(r) + v(r) \frac{d-1}{r}\). We consider \(d \ge 2\). Laplace operator \(\Delta u = 0\) now becomes \(v''(r) + v'(r) \frac{d-1}{r} = 0\) \\
    \(\Rightarrow\) \(\log(v(r))' = \frac{v'(r)}{v(r)} = - \frac{d-1}{r} = -(d-1)(\log r)'\) (recall \(log(f)' = \frac{f'}{f}\)) \\
    \(\Rightarrow v'(r) = \frac{1}{v^{d-2} + \text{ const.}}\) \\
    \(\begin{cases}
      \frac{const}{r^{d-2}} + const xx + const &,d \ge 3 \\
      const \log(r) + const xx r + const &,d = 2
    \end{cases}\)
  \end{rem}

  \begin{defn} 
    (Fundamential Solution of Laplace Equation)
    \begin{align*}
      \Phi(x) = \begin{cases}
        - \frac{1}{2 \pi} \log(|x|), & d = 2 \\
        \frac{1}{(d-2) d |B_1|} \frac{1}{|x|^{d-2}}, & d \ge 3
      \end{cases}
    \end{align*}
    
    Where \( |B_1| \) is the Volume of the ball \(B_1(0) = B(0, 1) \subseteq \mathbb{R}^d\).

  \end{defn}

  \begin{rem}
    \(\Delta \Phi(x) = 0\) for all \(x \in \mathbb{R}^d\) and \(x \ne 0\). 
  \end{rem}


  \section{Poisson-Equation}
  The Poisson-Equation is \(-\Delta u(x) = f(x)\) in \(\mathbb{R}^d\). The explicit solution is given by
  \begin{align*}
    u(x) &= (\Phi \star f)(x)
    = \int_{\mathbb{R}^d} \Phi(x-y)f(y) \ dy 
    = \int_{\mathbb{R}^d} \Phi(y)f(x-y) \ dy
  \end{align*}
  This can be heuristically justifyfied with \[-\Delta (\Phi \star f) = (-\Delta \Phi) \star f = \delta_0 \star f = f\]


  \begin{thm} \label{solution-for-poisson}
    Assume \(f \in C_c^2(\mathbb{R}^d)\). Then \(u = \Phi \star f\) satisfies that \(u \in C^2(\mathbb{R}^d)\) and \(- \Delta u(x) = f(x)\) for all \(x \in \mathbb{R}^d\)
  \end{thm}


  \begin{proof}
    By definition we have
    \begin{align*}
      u(x) &= \int_{\mathbb{R}^d} \Phi(y) f(x-y) \, dy.
    \end{align*}
    First we check that \(u\) is continuous: Take \(x_k \to x_0\) in \(\mathbb{R}^d\). We prove that \(u(x_n) \xrightarrow{n} u_0\), i.e.
    \begin{align*}
      \lim_{n \to \infty} \int_{\mathbb{R}^d} \Phi(y) f(x_n - y) \ dy = \int_{\mathbb{R}^d} \Phi(y) f(x_0 - y) \ dy
    \end{align*}
    This follows from the Dominated Convergence Theorem. More precisely:
    \begin{align*}
      \lim_{n \to \infty} \Phi(y) f(x_n -y) = \Phi(y) f(x_0 - y) \quad \text{ for all } y \in \mathbb{R}^d \setminus \{0\}
    \end{align*}
    and 
    \begin{align*}
      |\Phi(y) f(x-y)| &\le \| f ||_{L^\infty} \cdot \mathbb{1}(|y| \le R) \cdot |\Phi(y)| \in L^1(\mathbb{R}^d, dy)
    \end{align*}
    where \(R > 0\) depends on \(\{x_n\}\) and \(\operatorname{supp}(f)\) but independent of \(y\). Now we compute the derivatives:
    \begin{align*}
      \partial_{x_i} u(x) 
      &= \partial_{x_i} \int_{\mathbb{R}^d} \Phi(y) f(x-y) \ dy
      = \lim_{h \to 0} \int_{\mathbb{R}^d} \Phi(y) \frac{f(x + h e_i - y) - f(x-y)}{h} \ dy \\
      \text{(dom.\ conv.)} \quad &= \int \Phi(y) \partial_{x_i} f(x-y) \ dy \\
      \Rightarrow \quad D^\alpha u(x) &= \int_{\mathbb{R}^d} \Phi(y) D_x^\alpha f(x-y) \ dy \quad \text{for all } |\alpha| \le 2
    \end{align*}
    \(D^\alpha u(x)\)  is continuous, thus \(u \in C^2(\mathbb{R}^d)\).
    Now we check if this solves the Poisson-Equation:
    \begin{align*}
      - \Delta u(x) 
      &= \int_{\mathbb{R}^d} \Phi(y) (-\Delta_x) f(x-y) \, dy
      = \int_{\mathbb{R}^d} \Phi(y) (-\Delta_y) f(x-y) \, dy \\
      &= \int_{\mathbb{R}^d \setminus B(0, \epsilon)} \Phi(y) (-\Delta_x) f(x-y) \, dy + \int_{B(0, \epsilon)} \Phi(y) (-\Delta_x) f(x-y) \, dy \quad (\epsilon > 0 \text{ small})
    \end{align*}
    Now we come to the main part. We apply integration by parts (\ref{green-identities}):
    \begin{align*}
      &\int_{\mathbb{R}^d \setminus B(0, \epsilon)} \Phi(y)(- \Delta_y) f(x-y) \, dy \\
      &\quad = \int_{\mathbb{R}^d \setminus B(0, \epsilon)} (\nabla_y \Phi(y)) \cdot \nabla_y f(x-y) \, dy - \int_{\partial B(0, \epsilon)} \Phi(y) \cdot  \frac{\partial f}{\partial \vec{n}}(x-y) \, dS(y) \\
      &\quad = \int_{\mathbb{R}^d \setminus B(0, \epsilon)} \underbrace{(-\Delta_y \Phi(y))}_{=0} f(x-y) \, dy \\ &\qquad + \int_{\partial B(0, \epsilon)} \frac{\partial \Phi}{\partial \vec{n}}(y) f(x-y) \, dS(y) - \int_{\partial B(0, \epsilon)} \Phi(y) \frac{\partial f}{\partial \vec{n}}(x-y) \, dS(y)
    \end{align*}

    We have that \(\nabla_y \Phi(y) = -\frac{1}{d |B_1|} \frac{y}{|y|^d}\) and \( \vec{n} = \frac{y}{|y|} \text{ in } \partial B(0, \epsilon)\). This leads to
      \begin{align*}
        \frac{\partial \Phi}{\partial \vec{n}} &= \frac{1}{d |B_1|} \frac{1}{|y|^{d-1}} = \frac{1}{d |B_1| \epsilon^{d-1}} \quad \text{ for } y \in \partial B(0, \epsilon)
      \end{align*}

      Hence:
      \begin{align*}
        \int_{\partial B(0, \epsilon)} \frac{\partial \Phi}{\partial \vec{n}}(y) f(x-y) \ dS(y) 
        &= \frac{1}{d |B_1| \epsilon^{d-1}} \int_{\partial B(0, \epsilon)} f(x-y) \ dS(y) \\
        = \fint_{\partial B(0, \epsilon)} f(x-y) \ dS(y)
        &= \fint_{\partial B(x, \epsilon)} f(y) \ dS(y) 
        \xrightarrow{\epsilon \to 0} f(x)
      \end{align*}

        \begin{samepage}
          We have to regard the following error terms:
          \begin{itemize}
            \item \( 
                \begin{aligned}[t]
                  \left| \int_{B(0, \epsilon)} \Phi(y) (- \Delta_y) f(x-y) \ dy\right| 
                  &\le \int_{B(0, \epsilon)}|\Phi(y)| \underbrace{|-\Delta_y f(x-y)|}_{\le \|\Delta f\|_{L^\infty} \mathbb{1}(|y| \le R)} \ dy \\
                  &\le \| \Delta f\|_{L^\infty} \int_{\mathbb{R}^d} \underbrace{|\Phi(y)| \mathbb{1}(|y| \le R)}_{L^1(\mathbb{R}^d)} \mathbb{1}(|y| \le \epsilon) 
                  \xrightarrow{\epsilon \to 0} 0
                \end{aligned}
              \)
            Where \(R > 0\) depends on \(x\) and the support of \(f\) but is independent of \(y\).
          \item \( 
              \begin{aligned}[t]
                \left|\int_{\partial B(0, \epsilon)} \Phi(y) \frac{\partial f}{\partial \vec{n}}(x-y) \ dS(y)\right|
                &\le \|\nabla f\|_{L^\infty} \int_{\partial B(0, \epsilon)} |\Phi(y)| \ dy \\
                &\le \begin{cases}
                  const \cdot \epsilon | \log \epsilon| \to 0, & d = 2 \\
                  const \cdot \epsilon \to 0, & d \ge 3
                \end{cases}
              \end{aligned}
            \)
          \end{itemize}
        \end{samepage}
        Conclusion: \(-\Delta u(x) = f(x)\) for all \(x \in \mathbb{R}^d\) proved that \(u = \Phi \star f\) and \(f \in C_c^2(\mathbb{R}^d)\).
    \end{proof}

    Thus, if \(f \in C_c^2(\mathbb{R})\), then \(u = \Phi \star f\) satisfies \(u \in C^2(\mathbb{R}^2)\) and \(-\Delta u(x) = f(x)\) for all \(x \in \mathbb{R}^d\).


  \begin{rem}
    The result holds for a much bigger class of functions \(f\). For example if \(f \in C_c^1(\mathbb{R})\) we can easily extend the previous proof:
    \begin{align*}
      \partial_{x_i} u = \int_{\mathbb{R}^d} \Phi(y) \partial_{x_i} f(x-y) \, dy \in C(\mathbb{R}^d) \Rightarrow u \in C^1(\mathbb{R}^d)
    \end{align*}
    Consequently: 
    \begin{align*}
      \partial_{x_i} \partial_{x_j} u 
      &= \partial_{x_i} \int_{\mathbb{R}^d} \Phi(y) \partial_{x_j} f(x-y) \, dy
      = \int_{\mathbb{R}^d} \partial_{x_i} \Phi(y) \partial_{x_j} f(x-y) \, dy \in C(\mathbb{R}^d)
    \end{align*}
    So we have \(u \in C^2(\mathbb{R}^d)\). Now we can compute
    \begin{align*}
      \Delta u = \sum_{i=1}^d \int_{\mathbb{R}^d} \partial_{x_i} \Phi(y) \partial_{x_i} f(x-y) \, dy \overset{(IBP)}{=} f(x).
    \end{align*}
  \end{rem}

  \begin{ex}
    Extend this to more general functions!
  \end{ex}

  \section{Equations in general domains}
  \begin{thm} (Mean Value Theorem for Harmonic Functions)\label{mean-value-theorem} 
    Let \(\Omega \subseteq \mathbb{R}\) be open, let \( u \in C^2(\Omega)\) and \(\Delta u = 0\) in \(\Omega\). Then
    \begin{align*}
      u(x) 
      &= \fint_{B(x, r)} u
      = \fint_{\partial B(x, r)} u \quad \text{for all } x \in \Omega, B(x,r) \subseteq \Omega
    \end{align*}
  \end{thm}

  \begin{proof}
    Consider all \(r > 0\) s.t. \(B(x,r) \subseteq \Omega\),
    \begin{align*}
      f(r) &= \fint_{\partial B(x,r)} u
    \end{align*}
    We need to prove that \(f(r)\) is independent of \(r\). When it is done, then we immediately obtain
    \begin{align*}
      f(r) = \lim_{t \to 0} f(t) = u(x)
    \end{align*}
    as \(u\) is continuous. To prove that, consider
    \begin{align*}
      f'(r) 
      &= \frac{d}{dr} \left(\fint_{\partial B(0, r)} u(x+y) \, dS(y) \right) \\
      &= \frac{d}{dr} \left(\fint_{\partial B(0, 1)} u(x + rz) \, dS(z) \right) \\
      \text{(dom.\ convergence)} \quad & = \fint_{\partial B(0, 1)} \frac{d}{dr} [u(x + rz)] \, dS(z) \\
      &= \fint_{\partial B(0, 1)} \nabla u(x + rz) z \, dS(z) \\
      &= \fint_{\partial B(x, r)} \nabla u(y) \frac{y-x}{r} \, dS(y) \\
      &= \frac{1}{|B(x, r)|_{\mathbb{R}^d}} \int_{\partial B(x, r)} \nabla \cdot u(y) \cdot \vec{n_y} \, dS(y) \\
      \text{(Gauss-Green \ref{gauss-green})} \quad &= \frac{1}{|B(x, r)|_{\mathbb{R}^d}} \int_{B(x, r)} \underbrace{(\Delta u)(y)}_{= 0} \, dy = 0 \qedhere
    \end{align*}
  \end{proof}

  \begin{ex}
    In 1D\@: \(\Delta u = 0 \Leftrightarrow u'' = 0 \Leftrightarrow u(x) = ax + b\) (Linear Equation)
  \end{ex}

  \begin{rem}
    Recall the polar decomposition. Let \(x \in \mathbb{R}^d, x = (r,w), r = |x| > 0, \omega \in S^{d-1}\), then
    \begin{align*}
      \int_{B(0, r)} g(y) \, dy = \int_0^r \left(\int_{B(0, r)} g(y) \, dS(y) \right) dr
    \end{align*}
  \end{rem}


  \begin{rem}
    We already proved that for \(u\) harmonic we have \( u(x) = \fint_{\partial B(x,r)} u \, dy \). Now we have 
    \begin{align*}
      \int_{B(x, r)} u(y) \, dy 
      &= \int_{B(0, r)} u(x+y) \, dy \\
      \text{(Pol.\ decomposition)} \quad &= \int_0^r \left(\int_{\partial B(0, s)} u(x+y) \, dS(y)\right) ds \\
      &= \int_0^r \left(\int_{\partial B(x, s)} u(y) \, dS(y) \right) ds \\
      \text{(Mean value property)} \quad &= \int_0^r \left(|\partial B(x, s)| \, u(x) \right) ds
      = |B(x,r)| \, u(x)
    \end{align*}
    This implies
    \begin{align*}
      \fint_{B(x,r)} u(y) \, dy = u(x)
      \quad \text{for any \(B(x,r) \subseteq \Omega\).}
    \end{align*}
  \end{rem}

  \begin{rem}
    The reverse direction is also correct, namely if \(u \in C^2(\Omega)\) and
    \begin{align*}
      u(x) 
      &= \fint_{B(x, r)} u(y) \, dy
      = \fint_{\partial B(x,r)} u(y) \, dy
      \quad \text{for all } B(x,r) \subseteq \Omega,
    \end{align*}
    then \(u\) is harmonic, i.e. \(\Delta u = 0\) in \(\Omega\). (The proof is exactly like before)
  \end{rem}

  \begin{thm}[Maximum Principle]\label{maximum-principle}
    Let \(\Omega \subseteq \mathbb{R}^d\) be open, let \(u \in C^2(\Omega) \cap C(\bar \Omega)\), \(\Delta u = 0\) in \(\Omega\). Then
    \begin{enumerate}[label=\alph*)]
      \item \(\max_{x \in \bar \Omega} u(x) = \max_{x \in \partial \Omega} u(x)\)
      \item Assume that \(\Omega\) is connected. Then if there is a \(x_0 \in \Omega\) s.t. \( u(x_0) = \max_{x \in \bar \Omega} u(x)\), then \( u \equiv const.\) in \(\Omega\).
    \end{enumerate}
  \end{thm}

  \begin{proof}
    Given \(U \subseteq \mathbb{R}^d\) open, we can write \(U = \bigcup_i U_i\), where \(U_i\) is open and connected.
    \begin{enumerate}
      \item[b)] Assume that \(\Omega\) is connected and there is a \(x_0 \in \Omega\) s.t. \(u(x_0) = \sup_{y \in \Omega} u(x)\). Define \(U = \{ x \in \Omega \mid u(x) = u(x_0)\} = u^{-1}(u(x_0))\). \(U\) is closed since \(u\) is continuous.
      Moreover, \(U\) is open by the mean-value theorem. I.e.~for all \(x \in U\) there is a \(r > 0\) s.t. \(B(x,r) \subseteq U \subseteq \Omega\).
      Since \(U\) is connected we get \(U = \Omega\), so \(u\) is constant in \(\Omega\). On the other hand, if there is no \(x_0 \in \Omega\) s.t. \(u(x_0) = \sup_{x \in \Omega}\) we have \(\forall x_0 \in \Omega: \quad u(x) < \sup_{x \in \bar \Omega} u(x) = \sup_{x \in \partial \Omega} u(x)\)
      \item[a)] Given \(\Omega \subseteq \mathbb{R}^d\) open, we can write \(\Omega = \bigcup_i \Omega_i\), where \(\Omega_i\) is open and connected. By b) we have
        \[\sup_{x \in \bar \Omega_i} u(x) = \sup_{x \in \partial \Omega_i} u(x), \quad \forall i\]
        So we can conclude
        \[\quad \sup_{x \in \bar \Omega} u(x) = \sup_{x \in \partial \Omega} u(x). \qedhere\]
    \end{enumerate}
  \end{proof}

  \begin{defn}
    \begin{itemize}
      \item If \(\Omega \subseteq \mathbb{R}^d\) is open, \(u \in C^2(\Omega)\), then \(u\) is called \emph{sub-harmonic} if \(\Delta u \ge 0\) in \(\Omega\).
      \item If \(\Delta u \le 0\), then \(u\) is called \emph{super-harmonic}.
    \end{itemize}
  \end{defn}

  \begin{ex}[E 1.4]
    Let \(\Omega \subseteq \mathbb{R}^d\) be open and \(u \in C^2(\Omega)\) be subharmonic.
    \begin{enumerate}[label=\alph*)]
      \item Prove that \(u\) satisfies the Mean Value Inequality
      \begin{align*}
        \fint_{\partial B(x, r)} u(y) \, dS(y)
        \ge \fint_{B(x, r)} u(y) \, dy
        \ge u(x)
      \end{align*}
      for all \(B(x,r) \subseteq \mathbb{R}^d\).
      \item Assume further that \(\Omega\) is connected and \(u \in C(\bar \Omega)\). Prove that \(u\) satisfies the strong maximum principle, namely either
      \begin{itemize}
        \item \(u\) is constant in \(\Omega\), or 
        \item \(\sup_{y \in \partial \Omega} u(y) > u(x)\) for all \(x \in \Omega\).
      \end{itemize}
    \end{enumerate}
  \end{ex}

  \begin{proof}[My Solution]
    \begin{enumerate}[label=\alph*)]
      \item Let \(f(r) = \fint_{\partial B(x, r)} u(y) \, dS(y)\), then we have
        \begin{align*}
          \partial_r f(r)
          &= \partial_r \fint_{\partial B(x,r)} u(y) \, dS(y) \\
          \text{(Dom. Convergence)} \quad &= \fint_{\partial B(x, r)} \partial_r u(y) \, dS(y) \\
          &= \fint_{\partial B(0,1)} \partial_r u(x+yr) \, dS(y) \\
          &= \fint_{\partial B(0,1)} \nabla u(x+yr) \cdot y \, dS(y) \\
          &= \fint_{\partial B(x, r)} \nabla u(y) \cdot \frac{y-x}{r} \, dS(y) \\
          &= \fint_{\partial B(x,r)} \nabla u(y) \cdot \vec{n}_y \, dS(y) \\
          \text{(Gauss-Green)} \quad &= \fint_{B(x, r)} \div(\nabla u(y)) \, dS(y) \\
          &= \fint_{B(x,r)} \underbrace{\Delta u(y)}_{\ge 0} \, dS(y) \ge 0
        \end{align*}
        So we can conclude that
        \[\fint_{\partial B(x, r)} u(y) \, dS(y) = f(r) \ge \lim_{r \to 0} f(r) = u(x).\]

        Now regard
        \begin{align*}
          \int_{B(x,r)} u(y) \, dy 
          &= \int_0^r \left(\int_{\partial B(x, r)} u(y) \, dS(y)\right) \, ds \\
          &= \int_0^r \left(|\partial B(x, r)| \fint_{\partial B(x,r)} u(y) \, dS(y)\right) \, ds \\
          &\ge \int_0^r |\partial B(x, r)| \cdot  u(x) \, dS(y) \\
          &= u(x) \int_0^r |\partial B(x, r)| \, dS(y)
          = u(x) |B(x,r)|.
        \end{align*}

        Thus we have
        \begin{align*}
          u(x) \le \fint_{B(x,r)} u(y) dy.
        \end{align*}

        Finally, lets regard
        \begin{align*}
          \int_{B(x, r)} u(y) \, dy
          &= \int_0^r \left(|\partial B(x,s)| \fint_{\partial B(x, s)} u(y) \, dS(y)\right) \, ds \\
          (\partial_r f(r) \ge 0) \quad &\le \int_0^r \left(|\partial B(x,s)| \fint_{\partial B(x, r)} u(y) \, dS(y)\right) \, ds \\
          &= \fint_{\partial B(x, r)} u(y)  \, dS(y) \int_0^r |\partial B(x,s)| \, ds \\
          &= \fint_{\partial B(x, r)} u(y)  \, dS(y) \cdot |B(x,s)|
        \end{align*}
        and we conclude
        \[\fint_{B(x, r)} u(y) \, dy \le \fint_{\partial B(x,r)} u(y) \, dS(y).\]
      \item Let \(x_0 \in \Omega\) s.t. \(u(x_0) = \sup_{x \in \Omega} u(x)\). Now, \begin{align*}
        \sup_{x \in \Omega} u(x) 
        &= u(x_0) 
        \le \fint_{\partial B(x_0, r)} u(y) \, dy \\
        &\le \fint_{\partial B(x_0, r)} \sup_{x \in \Omega} u(x) \, dy 
        = \sup_{x \in \Omega} u(x)
      \end{align*}
      Since \(u\) is continous we get \(u(y) = u(x_0)\) for all \(y \in B(x_0, r)\), so \(u\) is constant. \qedhere
    \end{enumerate}
  \end{proof}

  \begin{defn}
    The \emph{Poisson Equation} for given \(f, g\) on a bounded set is:
    \begin{align*}
      \begin{cases}
        - \Delta u = f, &\text{in } \Omega \\
        u = g, &\text{on } \partial \Omega
      \end{cases}
    \end{align*} 
  \end{defn}

  \begin{thm}(Uniqueness)
    Let \(\Omega \subseteq \mathbb{R}^d\) be bounded, open  and connected. Let \(f \in C(\Omega), g \in C(\partial \Omega)\). Then there exists \emph{at most} one solution \(u \in C^2(\Omega) \cap C(\bar \Omega)\), s.t. \begin{align*}
      \begin{cases}
        - \Delta u = f, &\text{in } \Omega \\
        u = g, &\text{on } \partial \Omega
      \end{cases}
    \end{align*}
  \end{thm}

  \begin{proof}
    Assume that we have two solutions \(u_1\) and \(u_2\). Then \(u \coloneqq u_1 - u_2\) is a solution to 
    \begin{align*}
      \begin{cases}
        - \Delta u = 0, &\text{in } \Omega \\
        u = 0 &\text{on } \partial \Omega
      \end{cases}
    \end{align*}
    By the maximum principle, we know that \(u = 0\) in \(\Omega\). More precisely, by the maximum principle we have \(\forall x \in \Omega\)
    \begin{align*}
      \sup_{x \in \Omega} u(x) \le \sup_{x \in \partial \Omega} u(x) = 0
      \quad \Rightarrow \quad
      u(x) \le 0
    \end{align*}
    Since \(-u\) satisfies the same property we have \(\forall x \in \Omega\):
    \begin{align*}
      \sup_{x \in \Omega}(-u(x)) \le \sup_{x \in \partial \Omega} (-u(x)) = 0
      \quad \Rightarrow \quad
      - u(x) \le 0
      \quad \Rightarrow \quad
      u(x) \ge 0
    \end{align*}
    So we geht \(u(x)  = 0\) in \(\Omega\).
  \end{proof}

  \begin{ex}[Bonus 1]\label{bonus-1}
    Let \(\Omega\) be open, connected and bounded in \(\mathbb{R}^d\). Let \(u \in C^2(\Omega) \cap C(\bar \Omega)\) s.t. 
    \begin{align*}
      \begin{cases}
        \Delta u = 0, &\text{in } \Omega \\
        u = g, &\text{on } \partial \Omega
      \end{cases}
    \end{align*}
    Prove that \begin{enumerate}[label=\alph*)]
      \item If \(g \ge 0\) on \(\partial \Omega\), then \(u \ge 0\) in \(\Omega\). 
      \item If \(g \ge 0\) on \(\partial \Omega\) and \(g \ne 0\), then \(u > 0\) in \(\Omega\).
    \end{enumerate}
  \end{ex}

  \begin{lem}[Estimates for derivatives]\label{estimates-of-derivatives}
    If \(u\) is harmonic in \(\Omega \subseteq \mathbb{R}^d\), \(\alpha \in \mathbb{N}_0^d\), \(|\alpha| = N\) and \(B(x_0, r) \subseteq \Omega\), then 
    \[|D^\alpha u(x)| \le \frac{(c_dN)^N}{r^{d+N}} \int_{B(x, r)} |u| \, dy\]
  \end{lem}

  \begin{proof}
    Induction: Assume \(|\alpha| = N-1\), Take \(|\alpha| = N\)
    \begin{align*}
      |D^\alpha u(x_0)| 
      &\le \frac{|S_1|}{|B_1|\frac{r}{N}} \| D^\beta u \|_{L^\infty (B(x_0, \frac{r}{n}))}, \quad D^\alpha u = \partial_{x_i}(D^\beta u)_{|\beta| = N-1}
    \end{align*}
    Note: \(x \in B(x_0, \frac{r}{N})\), so \(B(x, \frac{r(N-1)}{N}) \subseteq B(x_0, r)\). By the induction hypothesis:
    \begin{align*}
      \|D^\beta u\|_{L^\infty(B(x_0, \frac{r}{N}))} 
      &\le \frac{[c_d (N-1)]^{N-1}}{[r \frac{(N-1)}{N}]^{d+N-1}} \int_{B(x_0, r)} |u| \, dy
    \end{align*}
    The conclusion is:
    \begin{align*}
      |D^\alpha u (x_0)|
      &\le \frac{|S_1|}{|B_1| \frac{r}{N}} \frac{[c_d(N-1)]^{N-1}}{\left(r \frac{N-1}{N^d}\right)^{d + N - 1}} \int_{B(x_0, r)} |u| \, dy \\
      &= \frac{|S_1|}{|\beta_1|} \frac{c_d^{N-1}}{\left(\frac{r}{N}\right)^{d+N} (N-1)^d} \int_{B(x_0, r)} |u| \, dy \\
      &= \frac{|S_1|}{|\beta_1|} \frac{c_d^{N-1}}{\left(\frac{r}{N}\right)^{d+N} N^d} \left(\frac{N}{N-1}\right)^d \int_{B(x_0, r)} |u| \, dy \\
      &\le \frac{2^d |S_1|}{|B_1|} \frac{c_d^{N-1} N^N}{r^{d+N}} \int_{B(x_0, r)} |u| \, dy \qquad \text{if } c_d \ge \frac{2^d |S_1|}{|B_1|}
    \end{align*}
  \end{proof}

  \begin{thm}[Regularity]
    Let \(\Omega\) be open in \(\mathbb{R}^d\). Let \(u \in C(\Omega)\) satisfy \(u(x) = \fint_{\partial B} u \, dy\) for any \(x \in B(x, r) \subseteq \Omega\). Then \(u\) is a harmonic function in \(\Omega\). Moreover, \(u \in C^\infty(\Omega)\) and \(u\) is analytic in \(\Omega\).
  \end{thm}

  % \begin{proof}
  %   We use the convolution. For simlicity consider the case \(\Omega = \mathbb{R}^d\) first. Take \(\eta \in C_c^\infty(\mathbb{R}^d)\) with \(0 \le \eta \le 1\), \(\eta(x) = 0\) if \(|x| \ge 1\), \(\eta\) radial and \(\int \eta = 1\). Define \(\eta_\epsilon (x) = \epsilon^{-d} \eta(\epsilon^{-1} x)\) for all \(\epsilon > 0\). Then
  %   \begin{align*}
  %     \int_{\mathbb{R}^d} \eta_\epsilon = \int_{\mathbb{R}^d} \eta = 1
  %   \end{align*}
  %   We prove \(u_\epsilon \coloneqq \eta_\epsilon \star u = u\) for all \(\epsilon > 0\). By definition:
  %   \begin{align*}
  %     u_\epsilon(x) 
  %     &= \int_{\mathbb{R}^d} \eta_\epsilon(x-y)u(y) \, dy \\
  %     &= \int_0^\infty \left[\int_{\partial B(x, r)} \eta_\epsilon(x-y) u(y) \, dS(y)\right] dr \\
  %     (\eta \text{ radial}) \quad &= \int_0^\infty \left[\eta_\epsilon(r) \int_{\partial B(x, r)} u(y) \, dS(y)\right] dr \\
  %     \text{(Assumption)} \quad &= \int_0^\infty \eta_\epsilon(r)\, |\partial B(x, r)|\, u(x) \, dr \\
  %     &= u(x) \int_0^\infty \eta_\epsilon(r) |\partial B(0, r)| \, dr \\
  %     &= u(x) \int_{\mathbb{R}^d} \eta_\epsilon(y) \, dy = u(x)
  %   \end{align*}

  %   On the other hand, \(u_\epsilon = \eta_\epsilon \star u\) is \(C^\infty(\mathbb{R}^d)\). In fact \(D^\alpha(\eta_\epsilon \star u) = (D^\alpha \eta_\epsilon) \star u\) is continuous for any \(\alpha\) (Exercise). Then \(u \in C^\infty(\mathbb{R}^d)\), so \(u\) is harmonic in \(\mathbb{R}^d\), i.e. \(\Delta u = 0\) in \(\mathbb{R}^d\). \\

  %   Consider now the general case where \(\Omega \subseteq \mathbb{R}^d\) is open. Take \(\epsilon > 0\) small and define \(\Omega_\epsilon = \{x \in \Omega \mid \dist(x, \partial \Omega) > \epsilon\}\). 
  %   Define \[u_\epsilon(x) = \int_{\mathbb{R}^d} \eta_\epsilon(x-y) u(y) \, dy \quad \text{ for all } x \in \Omega_\epsilon\]
  %   Recall that \(\eta_\epsilon(y) = 0\) if \(|y| \ge \epsilon\), then:
  %   \[u_\epsilon(x) = \int_{B(x, \epsilon)} \eta_\epsilon(x-y)u(y) \, dy\]
  %   is well-defined since \(B(x,\epsilon) \subseteq \Omega\) for all \(x \in \Omega_\epsilon\).
  %   Then by the same computation using the polar-decomposition, we find that \(u_\epsilon(x) = u(x)\) for all \(x \in \Omega\). Note that \(u_\epsilon \in C^\infty(\Omega_\epsilon)\). Taking \(\epsilon \to 0\), we get \(u \in C^\infty(\Omega)\). Then we conclude that \(u\) is harmonic (We need to reverse the proof of the mean-value theorem).\\
  %   To proof that \(u\) is analytic, we need to show that for all \(x_0 \in \Omega\), there is a \(r > 0\) s.t. \(B(x_0, r) \subseteq \Omega\) and \[u(x) = u(x_0) + \sum_{\alpha \ne 0} c_\alpha(x-x_0)^\alpha \quad \text{for  all } x \in B(x_0, r)\]
  %   Here \(\alpha = (\alpha_1, \dots, \alpha_d), \alpha_i \in \{0, 1, 2, \dots\}\) and \(y^\alpha = y_1^{\alpha_1}y_2^{\alpha_2} \dots y_d^{\alpha_d}\). 
  %   We want to prove that the series converges uniformly in \(B(x_0, r)\). Recall the Taylor expansion:
  %   \[u(x) = u(x_0) + \sum_{0 < |\alpha| < N} D^\alpha u(x_0) \frac{(x-x_0)^\alpha}{\alpha!} + R_N(x)\]
  %   where \(|\alpha| = \alpha_1 + \alpha_2 + \dots + \alpha_d\), \(\alpha! = \alpha_1! \cdots \alpha_d!\) and \[R_N(x) = \sum_{|\alpha| = N} \int_0^1 D^\alpha u(x_0 + t(x-x_0)) \frac{(x-x_0)^\alpha}{\alpha!} \, dt\]
  %   New: Let \(x_0 \in \Omega\), take \(r > 0\), \(r < \frac{1}{L+1} \dist(x_0, \Omega^c)\) s.t. if \(x \in B(x_0, r)\), then \[B(x, Lr) \subseteq B(x_0, (L+1)r) \subseteq \Omega\]. With Lemma~\ref{estimates-of-derivatives} we get:
  %   \begin{align*}
  %     |D^\alpha u(x_0 + t(x-x_0))
  %     &\le \frac{(c_d N)^N}{(Lr)^{d+N}} \int_{B(x, Lr)} |u|
  %   \end{align*}
  %   With \((x_0, r) \leadsto (x, Lr)\)
  %   \begin{align*}
  %     |R_N(x)| \le \sum_{|\alpha| = N} \frac{(c_d N)^N}{(Lr)^{d+N}} \frac{1}{\alpha!} \frac{1}{\alpha!} \tau^N \int_{B(x_0, (L+1)r) |u|}
  %   \end{align*}

  %   Thus 
  %   \begin{align*}
  %     \left(\frac{\tilde c_d N}{L}\right) \frac{1}{N!}
  %     &\le \left(\frac{\tilde c_d N}{L}\right) \left(\frac{e}{N}\right)^N \quad \text{ if \(N\) large} \\
  %     &= \left(\frac{\tilde c_d e}{L}\right)^N \xrightarrow{N \to \infty} 0 \quad \text{if } L > \tilde c_d e (L = L_d)
  %   \end{align*}
  %   We conclude that \[u(x) = u(x_0) + \sum_{\alpha \ne 0} \frac{D^\alpha u(x_0)}{\alpha!}(x-x_0)^\alpha\]
  %   The series converges uniformly \(x \in B(x_0, r)\).
  %   Now we proof the bound on derivatives. For \(\alpha = 0\)
  %   \begin{align*}
  %     |u(x_0)| = \left| \fint_{B(x_0, r)} u \right| \le \frac{1}{|B_1|r^d} \int_{B(x_0, r)} |u|
  %   \end{align*}
  %   For \(\alpha = 1:\) \(\Delta u = 0\) in \(\Omega\) \(\Rightarrow\) \(0 = \partial_{x_i} (\Delta u) = \Delta (\partial_{x_i} u)\), so \(\partial_{x_i} u\) is harmonic in \(\Omega\). Hence, by the mean-varlue theorem again:
  %   \begin{align*}
  %     \partial_{x_i} u(x_0) &= \fint_{B(x_0, \frac{r}{2})} \partial_{x_i} u = \frac{1}{|B_1| \frac{\frac{r}{2}}{2}^d} \int_{B(x_0, \frac{r}{2})} \partial_{x_i} u = \frac{1}{|B_1| \frac{r}{2}^d} \int_{\partial B(x_0, \frac{r}{2})}u n_i \, dS
  %   \end{align*}
  %   So we get:
  %   \begin{align*}
  %     |\partial_{x_i} u(x_0)| 
  %     &= \frac{1}{|B_1| r^d} \int_{\partial B(x, r)} \, dS \|u\|_{L^\infty(\partial B(x_0, \frac{r}{2}))}  \\
  %     &= \frac{|S_1|}{|B_1|\frac{r}{2}} \|u\|_{L^\infty(\partial B(x_0, \frac{r}{2}))}
  %   \end{align*}
  %   For any \(y \in \partial B(x_0, \frac{r}{2})\) by the mean value theorem, we get:
  %   \begin{align*}
  %     |u(y)| 
  %     &= \left| \fint_{B(y, \frac{r}{2})} u\right|
  %     &\le \frac{1}{|B_1| \left(\frac{r}{2}\right)^d} \int_{B(y, \frac{r}{2})} |u|
  %     &\le \frac{1}{|B_1| \left(\frac{r}{2}\right)^d} \int_{B(x_0, r)} |u|
  %   \end{align*}
  %   Thus,
  %   \begin{align*}
  %     |\partial_{x_i} u(x_0)| 
  %     &\le \frac{|S_1|}{|B_1| \left(\frac{r}{2}\right)} \frac{1}{|B_1|  \left(\frac{r}{2}\right)^d} \int_{B(x_0, r)} |u|
  %     \le \frac{c_d}{r^{d+1}} \int_{B(x_0, r)} |u|
  %   \end{align*}
  %   Induction: Assume that we already proved the bound when \(|\alpha| = N-1\). Then:
  %   \begin{align*}
  %     \partial_{x_i} D^\alpha u 
  %     &= D^\alpha(\underbrace{\partial_{x_i} u}_{\text{harmonic}})
  %     = 0
  %     \quad \Rightarrow \quad
  %     D^\alpha u \text{ is harmonic}
  %   \end{align*}
  %   So we get
  %   \begin{align*}
  %     \partial_{x_i} (D^\alpha u) &= \fint_{B(x_0, \frac{r}{4})} \partial_{x_i} (D^\alpha u) \\
  %     \Rightarrow \quad |\partial_{x_i} (D^\alpha u)| &\le  \frac{C_d}{r^{d+1}} \int_{B(x_0, \frac{r}{2})} |D^\alpha u|
  %   \end{align*}
  %   and by the induction hypothesis:
    
  %   \begin{align*}
  %     |D^\alpha u(x_0)| &\le \frac{c_d}{r} \|D^\alpha u\|_{L^\infty B(x_0, \frac{r}{2})} \\
  %     &\le \frac{c_d}{r^{d+N-1}} \int_{B(x_0, r)} |u| \quad \forall x \in B\left(x_0, \frac{r}{2}\right)
  %   \end{align*}
  %   Then: \(|\partial_{x_i} D^\alpha u(x_0)| \le \frac{c_d}{r^{d+N}} \int_{B(x_0, r)} |u|\)
  % \end{proof}

  \begin{ex}[E 1.1: Proof the Gauss–Green formula]
    Let \(f \coloneqq (f_i)_1^d \in C^1(\mathbb{R}^d, \mathbb{R}^d)\). Prove that for every open ball \(B(y, r) \subseteq \mathbb{R}^d\) we have \[\int_{\partial B(y, r)} f(y) \cdot \nu_y \, dS(y) = \int_{B(y, r)}\div f \, dx.\]
    Here \(\nu_y\) is the outward unit normal vector and \(dS\) is the surface measure on the sphere.
  \end{ex}

  \begin{proof}[Solution]
    We proof this in d=3. Let \(f \in C^1(\mathbb{R}^3)\)
    \begin{align*}
      \int_{B(0, 1)} \partial_{x_3} f \, dx = \int_{\partial B(0, 1)} f x_3 \, dS(x), \quad x = (x_1, x_2, x_3) \in \mathbb{R}^3, \vec{n} = \frac{x}{|x|} \text{ on } \partial B(0,1)
    \end{align*}
    \begin{align*}
      B(0,1) &
      = \{x_1^2 + x_2^2 + x_3^2 \le 1 \} \\
      &= \{ x_1^2+ x_2^2 \le 1 - \sqrt{1 - x_1^2 - x_2^2} \le x_3 \le \sqrt{1 - x_1^2 - x_2^2}\}
    \end{align*}
    Then: 
    \begin{align*}
      \int_{B(0,1)} \partial_{x_3} f \, dx 
      &= \int_{x_1^2 + x_2^2 \le 1} \left(\int_{-\sqrt{1 - x_1^2 - x_2^2} \le x_3 \le \sqrt{1 - x_1^2 -x_2^2}} \partial_{x_3} f \, dx_3\right) \, dx_1 \, dx_2 \\
      &= \int_{x_1^2 + x_2^2 \le 1} \left[f(x_1, x_2, \sqrt{1 - x_1^2 - x_2^2})\right. \\
      &\qquad \left. - f(x_1, x_2, - \sqrt{1 - x_1^2 - x_2^2})\right] \, dx_1 \, dx_2
    \end{align*}
    Lets take polar coordinates in 2D:\begin{align*}
      x_1 &= r \cos \phi & r > 0, \phi \in [0, 2 \pi)\\
      x_2 &= r \sin \phi & \det \frac{\partial(x_1, x_2)}{\partial(r, \phi)} = r
    \end{align*}
    \begin{align*}
      (\star) &= \int_0^1 \int_0^{2 \pi} [f(r \cos \phi, r \sin \phi, r) - f(r \cos \theta, r \sin \phi, -r)] r \, dr \, d\phi
    \end{align*}
    On the other hand:
    \begin{align*}
      \int_{\partial B(0, 1)} f x_3 \, dS
    \end{align*}
    The polar coordinates in 3D are:
    \begin{align*}
      x_1 &= r \cos \phi sin \theta& r > 0, \phi \in (0, 2 \pi), \theta \in (0, \pi) \\
      x_2 &= r \sin \phi \sin \theta & \det \frac{\partial{x_1, x_2, x_3}}{\partial(r, \phi, t)} = r^2 \sin \theta\\
      x_3 &= \cos \theta
    \end{align*}
    Then: 
    \begin{align*}
      (\star \star) 
      &= \int_0^{2 \pi} \int_0^\pi f(\cos \phi \sin \theta, \sin \phi \sin \theta, \cos \theta) \sin \theta \cos \theta \, d \theta \, d \phi \\
      &= \int_0^{2 \pi} \left(\int_0^{\frac{\pi}{2}} + \int_{\frac{\pi}{2}}^\pi \, d \theta\right) \, d \phi \\
      (r = \sin \theta) \quad &= \int_0^{2 \pi} \int_0^1 f(r \cos \phi, r \sin \phi, \sqrt{1 - r^2}) r \, dr \, d \phi \\
      &\qquad - f(r \cos \phi, r \sin \phi, -\sqrt{1 - r^2}) r \, dr \, d\phi \qedhere
    \end{align*}
  \end{proof}

  \begin{ex}[E 1.2]
    Let \(u \in C(\mathbb{R}^d)\) and \(\int_{B(x, r)} u \, dy = 0\) for every open ball \(B(x, r) \subseteq \mathbb{R}^d\). Show that \(u(x) = 0\) for all \(x \in \mathbb{R}^d\).
  \end{ex}

  \begin{proof}[My Solution]
    Assume there is a \(x_0 \in \mathbb{R}^d\) s.t. w.l.o.g. \(u(x_0) > 0\). Since \(u\) is continous there is a ball \(B(x_0, r)\) s.t. \(u(y) > \frac{u(x_0)}{2}\) for all \(y \in B(x_0, r)\). But then we get
    \begin{align*}
      \int_{B(x_0, r)} u(y) \, dy
      &\ge \int_{B(x_0, r)} \frac{u(x_0)}{2} \, dy
      = \frac{u(x_0)}{2} \, |B(x_0, r)| > 0. \qedhere
    \end{align*}
  \end{proof}

  \begin{ex}[E 1.3]
    Let \(f \in C_c^1(\mathbb{R}^d)\) with \(d \ge 2\) and \(u(x) \coloneqq (\Phi \star f)(x)\). Prove that \(u \in C^2(\mathbb{R}^2)\) and \(- \Delta u(x) = f(x)\) for all \(x \in \mathbb{R}^d\) (\ref{solution-for-poisson} was the same for \(f \in C_1(\mathbb{R})\))
  \end{ex}

  % \begin{proof}[Solution]
  %   We regard d=3. 
    
  %   \begin{lem}
  %     If \(f \in C_c(\mathbb{R}^3)\), then \(\frac{1}{|x|} \star f \in C^1(\mathbb{R}^3)\) and \(\partial_{x_1} \left(\frac{1}{|x|} \star f\right) = \partial_{x_1} \left(\frac{1}{|x|}\right) \star f  = \left( \frac{- x_1}{|x|^3}\right) \star f\)
  %   \end{lem}

  %   \begin{proof}
  %     \begin{align*}
  %       \partial_{x_i} \left(\frac{1}{|x|} \star f\right) 
  %       &= \lim_{h \to 0} \frac{1}{h} \left[\int_{\mathbb{R}^3} \frac{1}{|x-y-he_i|} f(y) \, dy - \int_{\mathbb{R}^3} \frac{1}{|x-y|}f(y) \, dy\right] \\
  %       &= \lim_{h \to 0} \int_{\mathbb{R}^3} \frac{1}{h} \left[ \frac{1}{|x-y-he_i|} - \frac{1}{|x-y|}\right] f(y) \, dy \\
  %       &= \lim_{h \to 0} \int_{|x-y-he_i| \ge 2h} \frac{1}{h} \left[ \frac{1}{|x-y-he_i|} - \frac{1}{|x-y|}\right] f(y) \, dy \\
  %       &\qquad + \lim_{h \to 0} \int_{|x-y-he_i| < 2h} \frac{1}{h} \left[ \frac{1}{|x-y-he_i|} - \frac{1}{|x-y|}\right] f(y) \, dy
  %     \end{align*}
  %     \begin{align*}
  %       (I) \quad \lim_{h \to 0} \int_{\mathbb{R}^3} \frac{\mathbb{1}(|x-y-he_i| \ge 2H)}{h} \left[\frac{1}{|x-y-he_i|} - \frac{1}{|x-y|}\right] f(y) \, dy
  %     \end{align*}
  %     Dominated Convergence:
  %     \begin{align*}
  %       \lim_{h \to 0} (...) = \frac{\partial}{\partial x_i} \left(\frac{1}{|x-y|}\right) f(y) = \frac{-(x_i - y_i)}{|x-y|^3}f(y)
  %     \end{align*}
  %     and
  %     \begin{align*}
  %       \left| \frac{\mathbb{1}(|x-y-he_i| \ge 2h)}{h}\right| \left| \frac{1}{|x-y-he_i|} - \frac{1}{|x-y|}\right| |f(y)|
  %     \end{align*}
  %     Here:
  %     \begin{align*}
  %       \left| \frac{1}{|a|} - \frac{1}{|b|}\right| = \frac{||a|-|b||}{|a||b|} \le \frac{|a-b|}{|a||b|} \quad \forall a,b \in \mathbb{R}^3
  %     \end{align*}
  %     So we get
  %     \begin{align*}
  %       &\mathbb{1}(|x_3 - he_i) \ge 4h) \left|\frac{1}{|x-y-he_i|} - \frac{1}{|x-y|} \right| \\
  %       &\qquad \le \frac{|he_i|}{|x-y-he_i||x-y|} \mathbb{1}(|x-y-he_i| \ge 4h) \\
  %       &\qquad \le \frac{h}{\left(\frac{2}{3}\right)|x-y|}
  %     \end{align*}
  %     \begin{align*}
  %       (II) &= \left[ \int_{|x-y-he_i| < 4h} \frac{1}{h}\left[\frac{1}{|x-y-he_i|}-\frac{1}{|x-y|}\right]f(y) \, dy\right| \\
  %       &\le \int_{|x-y-he_i| < 4h} \frac{1}{h} \left| \frac{1}{|x-y-he_i|}-\frac{1}{|x-y|}\right| |f(y)| \, dy \\
  %       &\le \int_{|x-y-he_i| < 4h} \frac{1}{|x-y-he_i||x-y|}|f(y)| \, dy \\
  %       &\le \|f\|_{L^\infty} \int_{|x-y-he_i| < 4h} \left[\frac{1}{|x-y-he_i|^2}+\frac{1}{|x-y|^2}\right]\, dy
  %     \end{align*}
  %     Here:
  %     \begin{align*}
  %       \int_{|x-y-he_i| < 4h} \frac{1}{|x-y-he_i|^2} \, dy 
  %       &= \int_{|y| < 4h} \frac{1}{|y|^2} \, dy = const.\ h \xrightarrow{h \to 0} 0
  %     \end{align*}
  %     \begin{align*}
  %       \int_{|x-y-he_i| < 4h} \frac{1}{|x-y|^2} \, dy \\
  %       &\le \int_{|x-y| \le 5h} \frac{1}{|x-y|^2} \, dy \\
  %       &= \int_{|y| \le 5h} \frac{1}{|y|^2} \, dy ...
  %     \end{align*}
  %   \end{proof}
  %   To conclude: \(f \in C_c^1(\mathbb{R}^3)\). From the Lemma:
  %   \begin{align*}
  %     \frac{1}{|x|} \star f \in C^1
  %   \end{align*}
  %   and
  %   \begin{align*}
  %     \partial_{x_1} \left(\frac{1}{|x|} \star f\right)
  %     &= \frac{1}{|x|}\star\left(\partial_{x_i} f\right)
  %   \end{align*}
  %   \begin{align*}
  %     \frac{1}{|x|} \star \underbrace{(\partial_{x_i}f)} \in C^1
  %   \end{align*}
  %   and
  %   \begin{align*}
  %     \partial_{x_j} \frac{1}{|x|} \star \partial_{x_i}f =  ...
  %   \end{align*}
  % \end{proof}

  \begin{thm}[Liouville's Theorem] 
    If \(u \in C^2(\mathbb{R}^d)\) is harmonic and bounded, then \(u = const.\)
  \end{thm}

  \begin{proof}
    By the bound of the derivative \ref{estimates-of-derivatives} we have
    \begin{align*}
      |\partial_{x_i} u(x_0)| 
      &\le \frac{c_d}{r^{d+1}} \int_{B(x_0, r)} |u| \, dy \quad \forall x_0 \in \mathbb{R}^d\ \forall r > 0 \\
      &\le \|u\|_{L^\infty} \frac{c_d}{r^{d+1}} |B(x_0, r)| \\
      &\le \|u\|_{L^\infty} \frac{c_d}{r} \xrightarrow{r \to \infty} 0
    \end{align*}
    Thus \(\partial_{x_i} u = 0\) for all \(i = 1, 2, \dots d\) and \(u = const.\) in \(\mathbb{R}^d\)
  \end{proof}

  \begin{thm}[Uniqueness of solutions to Poisson Equation in \(\mathbb{R}^d\)]
    If \(u \in C^2(\mathbb{R}^d)\) is a bounded function and satisfies \(- \Delta u = f\) in \(\mathbb{R}^d\) where \(f \in C_c^2(\mathbb{R}^d)\), then we have
    \begin{align*}
      u(x) = \Phi \star f(x) + C = \int_{\mathbb{R}^d} \Phi(x-y)f(y) \, dy + C \quad \forall x \in \mathbb{R}^d
    \end{align*}
    where \(C\) is a constant and \(\Phi\) is the fundamental solution of the Laplace equation in \(\mathbb{R}^d\).
  \end{thm}

  \begin{proof}
    If we can prove that \(v\) is bounded, then \(v = const.\). We first need to show that \(\Phi \star f\) is bounded.
    \begin{align*}
      \Phi = \Phi_1 + \Phi_2 = \Phi \mathbb{1}(|x| \le 1) + \Phi(|x| \ge 1) \\
      \Phi \star f = \Phi_1 \star f + \Phi_2 \star f
    \end{align*}
    We have \(\Phi_1 \star f \in L^1(\mathbb{R}^d)\) and \(\Phi_2 \star f\) is bounded since \(\Phi \to 0\)  as \(|x| \to \infty\)  in \(d \ge 3\).
  \end{proof}

  \begin{ex} (Hanack's inequality)
    Let \(u \in C^2(\mathbb{R}^d)\) be harmonic and non-negative. Prove that for all open, bounded and connected \(\Omega \subseteq \mathbb{R}^d\), we have
    \begin{align*}
      \sup_{x \in \Omega} u(x) \le C_\Omega \inf_{x \in \Omega} u(x),
    \end{align*}
    where \(C_\infty\) is a finite constant depending only on \(\Omega\).
  \end{ex}

  \begin{proof} (Exercise)
    Hint: \(\Omega = B(x,r)\). General case cover \(\Omega\) by finitely many balls, one ball is inside \(\Omega\).
  \end{proof}


  \chapter{Convolution, Fourier Transform and Distributions}

  \section{Convolutions}
  \begin{defn}[Convolution]
    Let \(f, g: \mathbb{{R}^d \to \mathbb{R}}\) or \(\mathbb{C}\).
    \begin{align*}
      (f \star g)(x)
      &= \int_{\mathbb{R}^d} f(x-y) g(y) \, dy
      = \int_{\mathbb{R}^d} f(y) g(x-y) \, dy 
      = (g \star f)(x)
    \end{align*}
  \end{defn}

  \begin{rem}[Properties of the Convolution]
    \begin{itemize}
      \item \((f \star g)(x) = f \star (g \star h)\)
      \item \(\hat{f \star g} = \hat f \star \hat g\)
    \end{itemize}
  \end{rem}


  \begin{thm}[Young Inequality]
    If \(f \in L^1(\mathbb{R}^d)\) and \(g \in L^p(\mathbb{R}^d)\), where \(1 \le p \le \infty\), then \(f \star g \in L^p(\mathbb{R}^d)\) and \(\|f \star g \|_{L^p} \le \|f\|_{L^1} \|g\|_{L^p}\). More generally, if \(f \in L^p(\mathbb{R}^d), g \in L^q(\mathbb{R}^d)\), then \(f \star g \in L^1(\mathbb{R}^d)\), \(\| f \star g\|_{L^1} \le \|f\|_{L^p} \|g\|_{L^q}\), where \(1 \le p, q, r, \le \infty\), \(\frac{1}{p} + \frac{1}{q} = 1 + \frac{1}{r}\)
  \end{thm}

  \begin{proof}
    Let \(f \in L^1, g \in L^p\). With the Hölder Inequality \ref{hölder-inequality}, we have:
    % \begin{align*}
    %   \left|(f \star g)(x)| 
    %   &= \left| \int_{\mathbb{R}^d} f(x-y)g(y) \, dy \right| \\
    %   &\le \left(\int_{\mathbb{R}^d} |f(x-y)| \, dy\right)^{\frac{1}{q}} \left(\int_{\mathbb{R}^d} |f(x-x)| |g(y)|^p \, dy \right)^{\frac{1}{p}} \\
    %   &= \| f \|_{L^1}^{\frac{1}{q}} \left(\int_{\mathbb{R}^d} \dots \right)^{\frac{1}{p}}
    % \end{align*}
    \begin{align*}
      \|f \star g\|_{L^p}^p
      &= \int_{\mathbb{R}^d} |f \star g(x)|^p \, dx \\
      &\le \|f\|_{L^1}^{\frac{p}{q}} \int_{\mathbb{R}^d} \int_{\mathbb{R}^d} |f(x-y)||g(y)|^p \, dy \, dx \\
      &= \| f \|_{L^1}^{\frac{p}{q} + 1} \|g\|_{L^p}^p
    \end{align*}
    So we have \(\| f \star g \|_{L^p} \le \|f\|_{L^1} \|g\|_{L^p}\)
  \end{proof}

  \begin{thm}[Smoothness of the Convolution]
    If \(f \in C_c^\infty(\mathbb{R}^d),\ g \in L^p(\mathbb{R}^d),\ 1 \le p \le \infty\). Then \(f \star g \in C^\infty(\mathbb{R})\) and 
    \[D^\alpha (f\star g) = (D^\alpha f) \star g\]
    for all \(\alpha = (\alpha_1, \dots, \alpha_d), \alpha_i \in \{0, 1, 2, \dots\}\)
  \end{thm}

  \begin{proof}
    First we note that \(x \mapsto (f \star g)\) is continous as \(x_n \to x\) in \(\mathbb{R}^d\) since
    \begin{align*}
      (f \star g)(x_n) 
      &= \int_{\mathbb{R}^d} f(x_n - y) g(y) \, dy \xrightarrow{\text{dom. conv.}} \int_{\mathbb{R}^d} f(x-y) g(y) \, dy = (f \star g)(x)
    \end{align*}
    We can apply Dominated convergence because 
      \[f(x_n - y)g(y) \to f(x-y)g(y)\quad  \forall y \text{ as \(f\) is continuous and } x_n \to x\]
      and
      \[|f(x_n -y)\ g(y)| \le \|f\|_{L^\infty} |g(y)| \ \mathbb{1}(|y| \le R) \in L^1(\mathbb{R}^d).\]
      Where \(R > 0\) satisfies \(B(0, R) \supseteq \supp f + \sup_n |x_n|\).
      Now we can compute the derivatives:
      \begin{align*}
        \partial_{x_i} (f \star g)(x)
        &= \lim_{h \to 0} \frac{(f \star g)(x + he_i) - (f \star g)(x)}{h} \\
        &= \lim_{h \to 0} \int_{\mathbb{R}^d} \frac{f(x + he_i - y) - f(x-y)}{h} g(y) \, dy \\
        \text{(Dominated Convergence)} \quad &= \int_{\mathbb{R}^d}\lim_{h \to 0} \frac{f(x + he_i - y) - f(x-y)}{h} g(y) \, dy \\
        &= \int_{\mathbb{R}^d} (\partial_{x_i} f)(x-y) g(y) \, dy
      \end{align*}
      We could apply Dominated Convergence since
      \begin{align*}
        \frac{f(x + he_i -y) - f(x-y)}{h} g(y) \xrightarrow{h \to 0} (\partial_{x_i} f) (x-y) g(y) \quad \text{as \(f \in C^1\)} \\
        \left| \frac{f(x + h e_i -y) - f(x-y)}{h} g(y) \right| \le \| \partial_{x_i} f \|_{L^\infty} |g(y)| \ \mathbb{1}(|y| \le R) \in L^1(\mathbb{R}^d)
      \end{align*}
      where \(B(0, R) \supseteq \supp(f) + B(0, |x| + 1)\)
      and \(\partial_{x_i} (f \star g) = (\partial_{x_i} f) \star g \in C(\mathbb{R}^d)\) since \(\partial_{x_i} f \in C_c^\infty(\mathbb{R}^d)\).
      By induction we get \(D^\alpha (f \star g) = (D^\alpha f \star g) \in C(\mathbb{R}^d)\).
  \end{proof}


  \begin{rem}
      Question: Is there a \(f\) s.t. \(f \star g = g\) for all \(g\). In fact there is no regular function \(f\) that solves this formally:
      \[f \star g = g \Rightarrow \widehat{f \star g} = \hat g \Rightarrow \hat f \hat g = \hat g\ \Rightarrow \hat f = 1 \Rightarrow f \text{ is not a regular function!}\]

      However, if \(f\) is the Dirac-Delta Distribution, \(f = \delta_0\) then \(\delta_0 \star g = g\) for all \(g\). Formally:
      \begin{align*}
        \delta_0(x) &= \begin{cases}
          0 &x \ne 0 \\
          \infty &x = 0 \\
          \int \delta_0 = 1
        \end{cases}
      \end{align*}
      In fact, if \(f \in L^1(\mathbb{R}^d)\), \(\int f = 1\), \(f_\epsilon(x) = \epsilon^{-d} f(\epsilon^{-1} x)\), then \(f_\epsilon \to \delta_0\) in an appropriate sense and \(f_\epsilon \star g \to g\) for all \(g\) nice enough.
  \end{rem}
  
  \begin{thm}[Approximation by convolution]
    Let \(f \in L^1(\mathbb{R}^d)\), \(\int f = 1\), \(f_\epsilon(x) = \epsilon^{-d} f(\frac{x}{\epsilon})\). Then for all \(g \in L^p(\mathbb{R}^d)\), where \(1 \le p < \infty\), then
    \[f_\epsilon \star g \to g \quad \text{in } L^p(\mathbb{{R}^d})\]
  \end{thm}

  \begin{proof}\
    \begin{enumerate}[label=Step \arabic*:]
      \item Let \(f, g \in C_c(\mathbb{R}^d)\). Then
      \begin{align*}
        (f_\epsilon \star g)(x) - g(x) 
        &= \int_{\mathbb{R}^d} f_\epsilon(y) g(x-y) \, dy - \int_{\mathbb{R}^d} f_\epsilon(y) g(x) \, dy \\
        &= \int_{\mathbb{R}^d} f_\epsilon(y) (g(x-y) - g(x)) \, dy \\
        |(f_\epsilon \star g)(x) - g(x)| 
        &= \left| \int_{\mathbb{R}^d} f_\epsilon(y) (g(x-y) - g(x)) \, dy \right| \\
        &\le \int_{\mathbb{R}^d} |f_\epsilon(y)| |g(x-y) - g(x)| \, dy \\
        &\le \int_{|y| \le R_\epsilon} |f_\epsilon(y)||g(x-y) - g(x)| \, dy \\
        &\le \underbrace{\int_{|y| \le R_\epsilon} |f_\epsilon(y)| \, dy}_{\le \|f_\epsilon\|_{L^1} = \|f\|_{L^1}} \left[\sup_{|z| \le R} |g(x-z) - g(x)| \right] 
        \xrightarrow{\epsilon \to 0} 0
      \end{align*}
      We have Dominated Convergence since:
      \[(f_\epsilon \star g)(x) - g(x) \to 0 \quad \text{as } \epsilon \to 0\]
      and
      \[|f_\epsilon \star g(x) - g(x)| \le \|f\|_{L^1} \sup_{|z| \le R_\epsilon} |g(x-z) - g(x)| \le 2 \|f\|_1 \|g\|_{L^\infty} \mathbb{1}(|x| \le R_1).\]
      Where \(B(0, R_1) \supseteq \supp(g) + B(0, R_\epsilon)\), thus
      \(f_\epsilon \star g \to g\) in \(L^p(\mathbb{R}^d)\). To remove the technical assumptions \(f, g \in C_c(\mathbb{R}^d)\), then we use a density argument. We use the fact that \(C_c(\mathbb{R}^d)\) is dense in \(L^p(\mathbb{R}^d)\), \(1 \le p < \infty\).
      \item Let \(g \in C_c(\mathbb{R}^d), g \in L^p(\mathbb{R}^d)\). Then there is \(\{g_m\} \subseteq L^p(\mathbb{R}^d), g_m \to g\) in \(L^p(\mathbb{R}^d)\). Then 
      \begin{align*}
        \|f_\epsilon \star g - g\|_{L^p} 
        &\le \|f_\epsilon \star (g - g_m)\|_{L^p}  + \|f_\epsilon \star g_m - g_m \|_{L^p}+ \|g_m - g\|_{L^p} \\
        \text{(Young)} \quad &\le \|f_\epsilon\|_{L^1}\|g-g_m\|_{L^p}  + \|f_\epsilon \star g_m - g_m \|_{L^p}+ \|g_m - g\|_{L^p} \\
        &\le \|f\|_{L^1}\|g-g_m\|_{L^p}  + \|f_\epsilon \star g_m - g_m \|_{L^p}+ \|g_m - g\|_{L^p} \\
        &\le (\|f\|_{L^1} + 1) \|g - g_m\|_{L^p} + \|f \star g_m - g_m\|_{L^p}
      \end{align*}
      
      So we get:
      \begin{align*}
        \limsup_{\epsilon \to 0} \|f_\epsilon \star g - g\|_{L^p} 
        &\le (\|f\|_{L^p} + 1) \|g - g_m\|_{L^p} + \underbrace{\limsup_{\epsilon \to 0} \|f_\epsilon \star g_m - g_m\|_{L^p}}_{0 \text{by step 1.}} \\
        \xrightarrow{m \to \infty} 0
      \end{align*}
      \item Let \(f \in L^1(\mathbb{R}^d)\) and \(g \in L^p(\mathbb{R}^d)\). Take \(\{f_m\} \subseteq C_c(\mathbb{R}^d)\), s.t. 
      \begin{align*}
        \begin{cases}
          F_m \to g \in L^1(\mathbb{R}) \text{ as } m \to \infty \\
          \int_{\mathbb{R}^d} F_m = 1 (\text{it is possible since } \int_{\mathbb{R}^d}) f = 1)
        \end{cases}
      \end{align*} 
      Define \(F_{m, \epsilon}(x) = \epsilon^{-d} F_m(\epsilon^{-1} x)\) (recall \(f_\epsilon(x) = \epsilon^{-d} f(\epsilon^{-1} x)\)). Then:
      \begin{align*}
        f_\epsilon \star g - g &= (f_\epsilon - F_{m, \epsilon}) \star g + F_{m, \epsilon} \star g -g \\
        \Rightarrow \|f_\epsilon - g \|_{L^p} &\le \underbrace{\|f_\epsilon - F_{m, \epsilon} \star g\|_{L^p}}_{\mathclap{\overset{\text{Young}}{\le} \|f_\epsilon - F_{m, \epsilon}\|_{L^1} \|g\|_{L^p} = \|f - F_m\|_{L^1} \|g\|_{L^p}}} + \|F_{m, \epsilon} \star g - g\|_{L^p} \\
        \Rightarrow \limsup_{\epsilon \to 0} \|f_\epsilon \star g - g\|_{L^p} &\le \|f - F_m\|_{L^1} \|g\|_{L^p} = \|f - F_m\|_{L^1} \|g \|_{L^p} \qedhere
      \end{align*}
    \end{enumerate}
  \end{proof}

  
  \begin{lem}
    \(C_c(\mathbb{{R}^d})\) is dense in \(L^p(\mathbb{R}^d)\), \(1 \le p < \infty\)
  \end{lem}

  \begin{proof}
    For all \(g \in L^p(\mathbb{R}^d)\) there are \(g_m\) step functions and \(g_m \to m\) in \(L^p(\mathbb{R}^d)\),
    % \[g_m(x) = \sum_{\substack{\Omega \\ \text{{finite sum} \\ \Omega \subseteq \mathbb{R}^d \text{measurable}}}} \chi_\Omega(x) a_\Omega\]
    We can assume that \(\Omega\) is open and bounded and we want to approximate \(\chi_\Omega\) by \(C_c(\mathbb{R}^d)\). 
  \end{proof}

  \begin{lem}[Urnson]
    Define 
    \[\Omega_\epsilon = \{x \in \Omega \mid \dist(x, \partial \Omega) > \epsilon\}\]
    Then there is a \(\eta_\epsilon \in C_c(\mathbb{R}^d)\) s.t.
    \begin{align*}
      \begin{cases}
        0 \le \eta(x) \le 1 & \forall x \in \mathbb{R}^d \\
        \eta_\epsilon(x) = 1 & \text{if } x \in \Omega_\epsilon \\
        \eta_\epsilon(x) = 0 & \text{if } x \notin \Omega
      \end{cases}
    \end{align*}
  \end{lem}

  \begin{lem}[Gernal Version of Urnson]
    If \(A, B \subseteq \mathbb{R}^d\), \(A\) closed, \(B\) closed, \(A \cap B = \emptyset\). Then
    \[\eta(x) = \frac{\dist(x, A)}{\dist(x, A) + \dist(x, B)}\]
    Then \(\eta \in C(\mathbb{R}^d)\), \(0 \le \eta \le 1\) and \(\eta = 0\)if \(x \in B\), \(\eta = 1\) if \(x \in A\). App to \(A = \overline{\Omega_\epsilon} \subset \subset \Omega\) and \(B = \mathbb{R}^d \setminus \Omega\).
  \end{lem}

  
  \begin{thm}[Appendix C4 in Evans]
    Let \(\Omega\) be open in \(\mathbb{R}^d\) and for \(\epsilon > 0\) define
    \[\Omega_\epsilon = \{x \in \Omega \mid \dist(x, \mathbb{R}^d \setminus \Omega) > \epsilon\}\]
    Let \(f \in C_c^\infty(\mathbb{R}^d), \int_{\mathbb{R}^d} f = 1, \supp f \subseteq B(0, 1), f_\epsilon(x) = \epsilon^{-d} f(\epsilon^{-1} x) \supp\) is \(B(0, \epsilon)\). Then for all \(g \in L_{loc}^p(\Omega)\) (i.e. \(\mathbb{1}_K g \in L^p(\Omega) \forall K\) compakt set in \(\Omega\)), then:
    \begin{enumerate}[label=\alph*)]
      \item \(g_\epsilon(x) = (f_\epsilon \star g)(x) = \int_{\mathbb{R}^d} f_\epsilon(x-y) g(y) \, dy - \int_\Omega f_\epsilon(x-y) g(y) \, dy\) is well-defined in \(\Omega_\epsilon\) and \(g_\epsilon \in C^\infty(\Omega_\epsilon)\).
      \item \(g_\epsilon \to g\) in \(L_{loc}^p(\Omega)\) if \(1 \le p < \infty\) and \(g_\epsilon(x) \to g(x)\) almost everywhere \(x \in \Omega\).
      \item If \(g \in C(\Omega)\), then \(g_\epsilon(x) \to g(x)\) uniformly in any compact subset of \(\Omega\).
    \end{enumerate}
  \end{thm}

  \begin{proof}
    \begin{enumerate}[label=\alph*)]
      \item \(D^\alpha(g_\epsilon) = (D^\alpha f_\epsilon) \star g \in C(\Omega_\epsilon)\)
      % \item Replace \(g \mapsto \mathbb{1}_K g\) where \(K\) is a compact set \(\subbseteq \Omega\). Then \(\mathbb{1}_K g \in L^p\). Then our theorem \(f_\epsilon \star (\mathbb{1}_K g) \to \mathbb{1}_K g\) in \(L^p(\Omega)\). On the other hand \(\mathbb{1}_K[(f_\epsilon \star \mathbb{1}_K g) - (f_\epsilon \star g)] \to 0\) as \(\epsilon \to 0\) (exercise)
      \item Already proved in \(\mathbb{R}^d\) space.
    \end{enumerate}
  \end{proof}

  
  \begin{cor}[Lebesgue differentation theorem]
    If \(f \in L_{loc}^P(\mathbb{R}^d)\), then
    \begin{align*}
      \fint_{B(x, \epsilon)} |f(y) - f(x)|^p \, dy \to 0 \quad \text{ as } \epsilon \to 0
    \end{align*}
  \end{cor}


  \begin{ex}[E 2.1]
    Let \(u \in C^2(\mathbb{R}^2)\) be convex. I.e.
    \begin{align*}
      t u(x) + u(y) (1-t) \ge u(tx + (1-t)y) \forall x,y \in \mathbb{R}^d \forall t \in [0,1]
    \end{align*}
    \begin{enumerate}[label=\alph*)]
      \item Prove for all \(x \in \mathbb{R}^d\) that \(H(x) =  \) ...
    \end{enumerate}
  \end{ex}
  
  \begin{proof}[Solution]\
    \begin{enumerate}[label=\alph*]
      \item In 1D: If \(u\) is convex \(\Leftrightarrow\) \(u''(x) \ge 0\) for all \(x \in \mathbb{R}\). In general: Taylor expansion for all \(x,z \in \mathbb{R}^d\):
      \begin{align*}
        u(x) &= u(z) + \nabla u(z)(x-y) + \int_0^1 \sum_{|\alpha| = 2} D^\alpha u(z + s(x-z)) \frac{(x-z)^\alpha}{\alpha!} \, ds
      \end{align*}
      \(x = z + s(x-z), s= 1\)
      Use \(z = tx + (t-1)y\) \(\Rightarrow\) \(x-z = (1-t)(x-y)\)
      \begin{align*}
        t u(x) &= tu(z) + t\nabla u(z) (1-t)(x-y) + t\int_0^1 \sum_{|\alpha| = 2} D^\alpha u(z + s(x-z)) \frac{\left[(1-t)(x-y)\right]^\alpha}{\alpha!} \, ds \\
        (1-t) u(y) &= (1-t)u(z) + (1-t)\nabla u(z) t (y-x) + (1-t)\int_0^r \sum_{|\alpha| = 2} D^\alpha u(z + s(y-z)) \frac{[t(y-x)]^\alpha}{\alpha!} \, ds
      \end{align*}
      \begin{align*}
        \Rightarrow t u(x) + (1-t) u(y) = u(z) + t \int_0^1 ... + (1-t) \int_0^1 ... \\
        \Rightarrow t \int_0^1 ... + (1-t) \int_0^1 ... \ge 0 \forall x,y,t,z = tx + (1-t)y
      \end{align*}
      
      \begin{align*}
        t(1-t)^2 \int_0^1 \sum_{|\alpha| = 2} D^\alpha u(z + s(x-z)) \frac{(x-y)}{\alpha!} \, ds + (1-t) t^2 \int_0^1 \sum_{|\alpha| = 2} D^\alpha u(z + s(y-z)) \frac{(y-z)^\alpha}{\alpha!} \,ds \ge 0
      \end{align*}
      for all \(x,y \in \mathbb{R}^d, t \in [0,1], z = tx+ (1-t)y\). Divides for \(t(1-t)\)
      \begin{align*}
        (1-t) \int_0^1 \dots + + \int_0^1 \dots \ge 0
      \end{align*}
      Take \(t \to 0\)
      \begin{align*}
        \int_0^1 \sum_{|\alpha| = 2} D^\alpha u(y + s(x-y)) \frac{(x-y)^\alpha}{\alpha!} \, ds \ge 0 \forall x,y \in \mathbb{R}^d
      \end{align*}
      Take \(y = x +a, a \in \mathbb{R}^d\)
      \begin{align*}
        \int_0^1 \sum_{|\alpha| = 2} D^\alpha u(x +a+sa) \frac{a^\alpha}{\alpha!} \, ds \ge 0 \forall \epsilon > 0, \forall x,a \in \mathbb{R}^d
      \end{align*}
      Take \(\epsilon \to 0\)
      \begin{align*}
        \int_0^1 \sum_{|\alpha| = 2} D^\alpha u(x) \frac{a^\alpha}{\alpha!} \ge 0 \Rightarrow \sum_{i,j = 1, i \ne j}\partial_{x_i} \partial_{x_j} u(x) a_i a_j + \sum_{i=j=1}^d \partial_{x_i}^2 u(x) \frac{a_i^2}{2}
      \end{align*}
      We get
      \begin{align*}
        \frac{1}{2} a^T Ha \ge 0 \forall a(a_i)_{i=1}^d \in \mathbb{R}^d
      \end{align*}
      \item \(H(x) \ge 0 \Rightarrow \left(\partial_i \partial_j u\right) \ge 0 \Rightarrow Tr H(x) \ge 0 \Rightarrow \sum_{i=1}^d \partial_{x_i}^2 u(x) \ge 0 \Rightarrow \Delta u(x) \ge 0 \forall x \in \mathbb{R}^d\)
    \end{enumerate}
  \end{proof}

  
  \begin{ex}[E 2.2]
    
  \end{ex}

  \begin{proof}[Solution]
    Regard \(d = 3\). The function \(\frac{1}{|x|}\) is harmonic in \(\mathbb{R}^3 \setminus \{0\}\). We prove 
    \begin{align*}
      \fint_{\partial B(x, r)} \frac{dS(y)}{|y|} = \frac{1}{\max(|x|, r)}
    \end{align*}
    If \(|x| > r\), then \(0 \notin B(x,r + \epsilon)\). Then
    \[y \mapsto \frac{1}{|y|}\] is harmonic in \(B(x, r  + \epsilon)\). Then by the Mean Value Property:
    \[\fint_{\partial B(x,r)} \frac{dS(y)}{|y|} = \frac{1}{|x|}\]
    If \(|x| < r\): Then \(\frac{1}{|y|}\) is not harmonic in \(B(x,r)\) since \(0 \in B(x,r)\). Note
    \begin{align*}
      \fint_{\partial B(x,r)} \frac{d S(y)}{|y|} 
      &= \fint_{\partial B(0, r)} \frac{dS(y)}{|x-y|}
    \end{align*}
    This function depends on \(x\) only via \(|x|\). 
    \begin{align*}
      ... &= \fint_{\partial B(0,r)} \frac{dS(y)}{|Rx-Ry|}
    \end{align*}
    for all \(R\) rotation \(SO(3)\), \(dS(R_y) = dS(y)\)
    \begin{align*}
      &= \fint_{\partial B(0, r)} \frac{dS(y)}{|Rx - y|} \\
      &= \fint_{\partial B(0, r)} \frac{dS(y)}{|z - y|} \\
      \text{(Radial in z)} \quad &= \fint_{\partial B(0, |x|)} \left(\fint_{\partial B(0, r)} \frac{dS(y)}{|z-y|}\right) \, dS(z) \\
      (\text{Fubini}) \quad &= \fint_{\partial B(0,r)} \left(\fint_{\partial B(0, |x|)} \frac{dS(z)}{|z-y|}\right) \, dS(y) \\
      (\text{case 1 since \(|y| = r > |x|\))} \quad &= \fint_{\partial B(0,r)} \frac{1}{|y|} \, dS(y)
      = \frac{1}{r}
    \end{align*}
    If \(|x| = r\): Continuity: \(x \mapsto \fint_{\partial B(0, r)} \frac{dS(y)}{|x-y|}\)
  \end{proof}

  \begin{rem}
    For \(f \in C^{|\alpha|}, g \in C^{|\beta|}\):
    \begin{align*}
      D^{\alpha + \beta}(f \star g) = (D^\alpha f)\star(D^\beta g)
    \end{align*}
  \end{rem}

  
  \begin{lem}
    If \(d \ge 3\) and \(f: \mathbb{R}^d \to \mathbb{R}\) radial. Then:
    \begin{align*}
      \left(\frac{1}{|x|^{d-2}} \star f \right)(x) 
      = \int_{\mathbb{R}^d} \frac{f(y)}{|x-y|^{d-2}} \, dy \\
      = \int_{\mathbb{R}^d} \frac{f(y)}{\max(|x|^{d-2}, |y|^{d-2})}\, dy
    \end{align*}
  \end{lem}
  \begin{proof}
    (d=3) Polar coordinates:
    \begin{align*}
      \int_{\mathbb{R}^3} \frac{f(y)}{|x-y|}\, dy 
      &= \int_0^{\infty} \left[\int_{\partial B(0,1)} \frac{1}{|x-rw|} \, d\omega \right] \, f(r) \, dr \\
      (a) \quad &= \int_0^\infty \left[\int_{\partial B(0,1)} \frac{d \omega}{\max(|x|, r)}\right] f(r) \, dr \\
      &= \int_{\mathbb{R}^3} \frac{f(y)}{\max(|x|, |y|)} \, dy
    \end{align*}
    (b) (d=3) If \(f\) radial and non-negative
    \begin{align*}
      \int_{\mathbb{R}^3} \frac{f(y))}{|x-y|} &= \int_{\mathbb{R}^3} \frac{f(y)}{|x|} \, dy = \frac{(Sf?)}{|x|}
    \end{align*}
    Then
    \begin{align*}
      \int_{\mathbb{R}^3}\int_{\mathbb{R}^3} \frac{f_1(x-z_1)f_2(y-z_2)}{|x-y|} \, dx \, dy &= \int_{\mathbb{R}^3}\int_{\mathbb{R}^3} \frac{f_1(x) f_2(y)}{|x+ z_1 - y - z_2} \, dx \, dy \\
      &= \int_{\mathbb{R}^3} \left(\int_{\mathbb{R}^3} f_1(x) \, dx \right) f_2(y) \, dy \le \int_{\mathbb{R}^3} \frac{\left(\int_{\mathbb{R}^3} f_1\right)}{|y + z_2 - z_1|} f_2(y) \, dy \\
      &\le \frac{(\int_{\mathbb{R}^3} f_1)(\int_{\mathbb{R}^3} f_2)}{|z_1 - z_2|}
    \end{align*}
  \end{proof}

  
  \begin{ex}[Bonus 2]
    \begin{enumerate}[label=\alph*)]
      \item Prove that \(u(x) = \frac{1}{|x|}\) is sub-harmonic in \(\mathbb{R}^2 \setminus \{0\}\). 
      \item Prove that if \(f: \mathbb{R}^2 \to \mathbb{R}\) radial, non-negative, measurable:
      \begin{align*}
        \int_{\mathbb{R}^2} \frac{f(y)}{|x-y|} \, dy \ge \int_{\mathbb{R}^2} \frac{f(y)}{\max(|x|, |y|)} \, dy
      \end{align*}
    \end{enumerate}
  \end{ex}

  \section{Fourier Transformation}
  \begin{defn}[Fourier Transform]
    For \(f \in L^1(\mathbb{R}^d)\) define
    \[\four f(k) = \hat f(k) = \int_{\mathbb{R}^d} f(x) e^{- 2 \pi i k \cdot x} \, dx, \quad k\cdot x = \sum_{i=1}^d k_i x_i\]
  \end{defn}

  
  \begin{thm}[Basic Properties]
    \begin{enumerate}
      \item If \(f \in L^1(\mathbb{R}^d)\), then \(\hat f \in L^\infty(\mathbb{R}^d)\) and \(\| \hat f \|_{L^\infty} \le \|f\|_{L^1}\)
      \item For all \(f \in L^1(\mathbb{R}^d) \cap L^2(\mathbb{R}^d)\), \(\|\hat f\|_{L^2} = \|f\|_{L^2}\). Moreover, \(\four\) can be extended to be a unitary transforamtion \(L^2(\mathbb{R}^d) \to L^2(\mathbb{R}^d)\) s.t.
      \[\|\four g\|_{L^2} = \|f\|_{L^2} \quad \forall f \in L^2(\mathbb{R}^d)\]
      \item The inverse of \(F\) can be defined as
      \[(F^{-1}f)(x) = \check f (x) = \int_{\mathbb{R}^d} f(x) e^{2 \pi ik x} \, dk\] for all \(f \in L^1(\mathbb{R}^d) \cap L^2(\mathbb{R}^d)\)
      \item \(\widehat{D^\alpha f}(k) = (2 \pi i k)^\alpha \hat f(k)\) as \((2 \pi ik)^\alpha f(k) \in L^2(\mathbb{R}^d)\) (\(k^\alpha = k_1^{\alpha_1} \cdots k_\alpha^{\alpha_k}\))
      \item \(\widehat{f \star g}(k) = \hat f(k) \hat g (k)\) if \(f,g\) are nice enough.
    \end{enumerate}
  \end{thm}

  \begin{thm}[Hausdorff-Young-Inequality]\label{hausdorff-young}
    If \(1 \le p \le 2\), \(\frac{1}{p} + \frac{1}{p'} = 1\) and \(f \in L^p(\mathbb{R}^d) \cap L^1(\mathbb{R}^n)\) then
    \[\| \hat f \|_{L^{p'}} \le \|f\|_{L^p}\]
    and 
    \[\|\hat f\|_{L^p} \le \|f\|_{L^p} \quad \forall f \in L^p(\mathbb{R}^d)\]
  \end{thm}
  
  % \begin{rem}
  %   If \(1 \le p \le 2\) and \(f \in L^p(\mathbb{R}^d)\) we can write \(f = f_1 + f_2\) when \(f_1 \in L^1\), \(f_2 \in L^2\), e.g.
  %   \begin{align*}
  %     f = \underbrace{f \mathbb{1}(|f| \ge 1)}_{f_1} + \underbrace{f \mathbb{1}(|f| < 1)}_{f_2}
  %   \end{align*}
  %   \begin{align*}
  %     \int_{\mathbb{R}^d} |f_2|^2 \, dy &= \int_{\mathbb{R}^d}|f|^2 \mathbb{1}(|f| < 1) \le \int_{\mathbb{R}^d} |f|^p \, dy < \infty \\
  %     \int_{\mathbb{R}^d} |f_1| \, dy &= \int_{\mathbb{R}^d} |f| \mathbb{1}(|f| \ge 1) \le \int_{\mathbb{R}^d} |f|^p < \infty
  %   \end{align*}
  %   thus we can define \(\hat f = \hat f_1 + \hat f_2\) well defined in \(L^\infty(\mathbb{R}^d) + L^2(\mathbb{R}^d)\) what it in unclear why \(\hat f \in L^q\)?
  %   Proof of HY inequality. Weee nedd Riez-Theorem interpolation theorem. If \(1 \le p_0, p_1, q_0, q_1 \le \infty\), and \(\Omega \subseteq \mathbb{R}^d\) open and 
    % \[T: L^{p_0}(\Omega} + L^{p_1})(\Omega) \longrightarrow L^{q_0}(\Omega} + L^{q_1})(\Omega)\]
    % is a linear operator and
    % \begin{align*}
    %   T: \ L^{p_0} \to L^{q_0} \text{ and } \|T\|_{L^{p_1} \to L^{q_1}} \le 1, \quad \text{ for } i = 0, 1.
    %   \text{ Then } T: \ L^{p_0} \to L^{q_0} \text{ and } \|T\|_{L^{p_1} \to L^{q_1}} \le 1, \quad \text{ for any } 0 < \theta < 1, where \begin{cases}
    %     \frac{1}{p_0} = \frac{\theta}{p_0} + \frac{1 - \theta}{p_2} \\
    %     \frac{1}{q_0} = \frac{\theta}{q_0} + \frac{1 - \theta}{q_1}
    %   \end{cases}
    % \end{align*}
    % Consider the Foureir Transform:
    % \begin{align*}
    %   F: L^1 + L^1 \to L^2 + L^\infty 
    % \end{align*}
    % and
    % \begin{align*}
    %   \|F\|_{L^1 \to L^\infty} \le 1 \text{ as } \|\hat f\|_{L^\infty} \le \|f\|_{L^1} \forall f p \in L^1 \\
    %   \|F\|_{L^2 \to L^2} = 1 \text{ as } \| \hat f \|_{L^2} = \|f\|_{L^2} \forall f \in L^2 \\
    %   \Rightarrow \|F\|_{L^{p_0} \to L^{p_t}} \le 1 \forall \theta \in (0,1)
    % \end{align*}
    % \(p_0 = 1, p_1 = 2, q_0 = \infty, q_1 = 2\)
    % \begin{align*}
    %   \frac{1}{p_0} = \frac{\theta}{p_0} + \frac{1 - \theta}{p_2} = \theta + \frac{1 - \theta}{2} = \frac{1 + \theta}{2} \\
    %   \frac{1}{q_\theta} = \frac{\theta}{q_0} + \frac{1 - \theta}{q_1} = \frac{1 - \theta}{2} \\
    %   \Rightarrow \frac{1}{p_0} + \frac{1}{q_\theta}} = \frac{1 + \theta}{2} + \frac{1 - \theta}{2} = 1
    % \end{align*}
    % Thus: Let \(1 \le p,q,r \le 2\) s.t. \(\frac{1}{p} + \frac{1}{a} = 1 + \frac{1}{r}}\)
    % Let \(f \in L^p(\mathbb{R}^d)\), \(g \in L^q(\mathbb{R}^d)\). Recall \(f \star g \in L^r(\mathbb{R}^d)\) by Young. We have
    % \[\widehat{f \star g}(k) = \hat f(k) \hat g(k) \text{ a.e. k \in \mathbb{R}^d}\]
  
    
    % \begin{ex}
    %   Hint: first \(f,g \in C_c(\mathbb{R}^d)\) then use a density arugment)
    % \end{ex}

    % \begin{align*}
    %   \widehat{f \star g}(k) &= \int_{\mathbb{R}^d} (f \star g)(x) e^{-2 \pi i k x} \, dx = \int_{\mathbb{R}^d}\int_{\mathbb{R}^d} f(x-y) g(y) e^{-2 \pi i kx} \, dx \, dy \\
    %   = \int_{\mathbb{R}^d}\int_{\mathbb{R}^d} \left(f(x-y) e^{- 2 \pi i k(x-y)}\right) (g(y)) e^{- 2 \pi i k y}) \, dx \, dy \\
    %   &= \int_{\mathbb{R}^d} \left(\int_{\mathbb{R}^d} (f(x-y) e^{- 2 \pi i k (x-y)} \, dy)\right) (g(y) e^{ - 2 \pi i k y}) \, dy = \hat f(k) \hat g(k)
    % \end{align*}
    \begin{rem}
      We want to apply the Fourier transform to find the solution of a PDE, e.g. the Poisson-Equation:
      \begin{align*}
        - \Delta u = f \text{ in } \mathbb{R}^d
        \Rightarrow | 2 \pi k|^2 \hat u (k) = \hat f(k)
        \Rightarrow \hat u(k) = \frac{1}{|2 \pi k|^2} \hat f(k) 
      \end{align*}
      If we can find \(G\) s.t. \(\hat G(k)  = \frac{1}{|2 \pi k|^2}\), then
      \begin{align*}
        \hat u(k) &= \hat G(k) \hat f(k) = \widehat{G \star f} \\
        \Rightarrow u(x) &= (G \star f) (x) = \int_{\Rd} G(x-y)f(y)\, dy
      \end{align*}
      In fact \(G\) is the fundamential solution of laplace quation.
  \end{rem}

  
  \begin{thm}[Fourier Transform of \(\frac{1}{|x|^\alpha}\) for \(0 < \alpha < d\)] We have formally
    \begin{align*}
      \widehat{\frac{c_\alpha}{|x|^\alpha}} = \frac{c_{d-\alpha}}{|k|^{d - \alpha}} \quad \forall\ 0 < \alpha < d
    \end{align*}
    Here
    \begin{align*}
      c_\alpha = \pi^{- \frac{d}{2}} \Gamma \left(\frac{\alpha}{2} \right) = \pi^{- \frac{\alpha}{2}} \int_0^\infty e^{-\lambda} \lambda^{\frac{\alpha}{2}-1} \, d\lambda
    \end{align*}
    More precisely, for all \(f \in C_c^\infty(\mathbb{R}^d)\), 
    \begin{align*}
      \frac{c_\alpha}{|x|^\alpha}\star f = \left(\frac{c_{d-\alpha}}{|k|^{d - \alpha}} \hat f (k)\right)^\lor
    \end{align*}
    Moreover if \(\alpha > \frac{d}{2}\), then we also have
    \begin{align*}
      \left(\frac{c_\alpha}{|x|^\alpha} \star f\right)^\land = \frac{c_{d - \alpha}}{|f|^{d-\alpha}} \hat f (k)
    \end{align*}
  \end{thm}

  % \begin{rem}
  %   If \(f \in L^p\), \(1 \le p \le 2\) \(\Rightarrow\) \(f = \underbrace{f_1}_{\in L^1} + \underbrace{f_2}_{\in L^2} \Rightarrow \hat f = \hat f_1 + \hat f_2\) \\
  %   Question: If \(f = \tilde f_1 + \tilde f_2\), where \(\tilde f_1 \in L^1\) and \(\tilde f_2 \in L^2\), do we have that \(\hat f_1 + \hat f_2 = \widehat{\tilde{f_1}} + \widehat{\tilde{f_2}} \in L^1 \cap \L^2\)?
  %   In fact:
  %   \begin{align*}
  %     f_1 + f_2 &= \tilde f_1 + \tilde f_2  \\
  %     \Rightarrow \underbrace{f_1 - \tilde f_1}_{\in L^1} = \Rightarrow \underbrace{\tilde f_2 - f_2}_{\in L^2} \in L^1 \cap L^2 \\
  %     \Rightarrow \hat f_1 - \hat{\tilde f_1} = \widehat{\tilde f_2}- \hat f_2 \Rightarrow \hat f_1 + \hat f_2 = \widehat{\tilde f_1} + \widehat{\tilde f_2}.
  %   \end{align*}
  % \end{rem}
  
  \begin{lem}[Fourier Transform of Gaussians] In \(\Rd\),
    \[\widehat{e^{- \pi |x|^2}} = e^{- \pi |k|^2}\]
    More generally for all \(\lambda > 0\):
    \[\widehat{e^{- \pi \lambda^2 |x|^2}} = \lambda^{-d} e^{-\pi \frac{|k|^2}{\lambda^2}}\]
    (exercise)
  \end{lem}

  \begin{proof}[Proof of Theorem]
    Formally:
    \begin{align*}
      \frac{c_\alpha}{|x|^\alpha} &= \frac{1}{|x|^\alpha} \pi^{- \frac{\alpha}{2}} \int_0^\infty e^{-\lambda} \lambda^{\frac{\alpha}{2} -1} \, d \lambda = \int_0^\infty e^{- \pi \lambda |x|^2} \lambda^{\frac{\alpha}{2}-1} \, d \lambda \\
      \Rightarrow \frac{\hat c_\alpha}{|x|^\alpha}(k) &= \int_0^\infty \widehat{e^{- \pi \lambda |x|^2}}(k) \lambda^{\frac{\alpha}{2}-1} \, d \lambda = \int_0^\infty \lambda^{-\frac{d}{2}} e^{- \pi \frac{|k|^2}{\lambda}} \lambda^{\frac{\alpha}{2}-1} \, d \lambda \\
      (\lambda \to \frac{1}{\lambda}) \quad &= \int_0^\infty \lambda^{\frac{d}{2} e^{- \pi |k|^2 \lambda}} \lambda^{- \frac{\alpha}{2} + 1} \lambda^{-2} \, d\lambda \\
      &= \frac{c_{d-\alpha}}{|k|^{d-\alpha}}
    \end{align*}

    Let \(f \in C_c(\Rd)\). Then \(\left(\frac{1}{|x|^\alpha} \star f\right)(x) = \int_{\Rd} \frac{1}{|x-y|^\alpha} f(y) \, dy\) is well defined as \(\frac{1}{|x-y|} \in L_{loc}^1(\mathbb{R}^d, dy)\). It is bounded
    \begin{align*}
      \frac{1}{|x|^\alpha} \star f = \frac{1}{|x|^\alpha} \underbrace{\mathbb{1}(|x| \le 1)}_{\in L^\infty(\Rd)} \star \underbrace{f}_{L^\infty} + \underbrace{\frac{1}{|x|} \mathbb{1}(|x| > 1)}_{\in L^\infty} \star \underbrace{f}_{\in L^1} \in L^\infty(\Rd)
    \end{align*}
    When \(|x| \to \infty\):
    \begin{align*}
      \left(\frac{1}{|x|^\alpha} \star f\right)(x) &= \int_{\Rd} \frac{f(y)}{|x-y|^\alpha} \, dy = \int_{|y| \le R} \frac{f(y)}{|x-y|^\alpha} \, dy \sim \frac{\int_{\Rd} f(y) \, dy}{|x|^\alpha}
    \end{align*}

    Note that \(\frac{c_{d-\alpha}}{|k|^{d-\alpha}} \underbrace{\hat f(k)}_{\text{bounded}} \in L^1(\Rd)\).
    \begin{align*}
      (...) \mathbb{1}(|k| \le 1) + (...) \mathbb{1}(|k|> 1)
      \frac{1}{|k|^{d - \alpha}}|\hat f(k)| \mathbb{1}(|k| \le 1) \le \|f\|_{L^1} \frac{\mathbb{1}(|k| \le 1)}{|k|^{d-\alpha}} \in L^1(\Rd, dk) \\
      \frac{1}{|k|^{d-\alpha}} |\hat f(k)| \mathbb{1}(k >1) \le |\hat f(k)| \in L^2(\Rd, dK) \text{ as } f \in L^2(\Rd)
    \end{align*}

    \begin{lem}
      If \(f \in C_c^\infty(\Rd)\), then \(\hat f \in L^1(\mathbb{R}^d)\)
    \end{lem}

    \begin{proof}(Exercise)
      Hint: \(|\widehat{D^\alpha f}| = |2 \pi k|^{|\alpha|} |\hat f(k)| \leadsto |\hat f(k)| \le \frac{1}{|k|^{|k|}}\) as \(|k| \to \infty\).
    \end{proof}

    Compute:
    \begin{align*}
      \left(\frac{c_{d-\alpha}}{|k|^{d-\alpha}} \hat f(k)\right)^\lor (x) 
      &= \int_{\Rd} \frac{c_{d-\alpha}}{|k|^{d-\alpha}} \hat f(k) e^{2 \pi i k x} \, dk \\
      &= \int_{\Rd} \left(\int_0^\infty e^{-\pi |k|^2 \lambda} \lambda^{\frac{d-\alpha}{2}-1} \, d \lambda \right) \hat f (k) e^{2 \pi i k x} \, dk \\
      &= \int_0^\infty \left(\int_{\Rd} e^{-\pi |k|^2 \lambda} \hat f (k) e^{2 \pi i k x} \, dk \right) \lambda^{\frac{d-\alpha}{2}-1} \, d \lambda \\
      &= \int_0^\infty \left(e^{- \pi k^2 \lambda} \hat f(x)\right)^\lor \lambda^{\frac{d-\alpha}{2} - 1} \, d \lambda \\
      &= \int_0^\infty \left(\widehat{\lambda^{-\frac{d}{2}} e^{- \pi \frac{x^2}{\lambda}}}(k) \hat f(k)\right)^\lor \lambda^{\frac{d-\alpha}{2} - 1} \, d \lambda \\
      &= \int_0^\infty \left(\lambda^{- \frac{d}{2}} e^{- \pi \frac{x^2}{\lambda}} \star f \right) \lambda^{\frac{d-\alpha}{2} - 1} \, d \lambda \\
      &= \left(\int_0^\infty \lambda^{- \frac{d}{2}} e^{- \pi \frac{x^2}{\lambda}}\lambda^{\frac{d-\alpha}{2} - 1} \, d \lambda \right) \star f
    \end{align*}
    Assume \(d > \alpha > \frac{d}{2}\). Then \(\frac{c_\alpha}{|x|^\alpha} \star f \in L^\infty\) and behaves \(\frac{c_\alpha(\int f)}{|x|^\alpha}\) as \(|x| \to \infty\). This implies:
    \begin{align*}
      \int_{\mathbb{R}^d} \left| \frac{c_\alpha}{|x|^\alpha} \star f \right|^2 \le c + \int_{|x| \ge R} \frac{c}{|x|^{2d}} \, dx < \infty 
    \end{align*}
    Thus the Fourier Transform \(\widehat{\frac{c_\alpha}{|x|^\alpha}} \star f\) exists. Combining with 
    \begin{align*}
      \frac{c_\alpha}{|x|^\alpha} \star f = \left(\frac{c_{d-\alpha}}{|f|^{d-\alpha}} \hat f(k) \right)^\lor \\
      \Rightarrow \widehat{\frac{c_\alpha}{|x|^\alpha} \star f} = \frac{c_{d-\alpha}}{|k|^{d-\alpha}} \hat f (k)
    \end{align*}
  \end{proof}
  
  \begin{rem}
    If \(d \ge 3\)
    \begin{align*}
      \hat G(k) 
      &= \frac{1}{|2 \pi k|^2}  \\
      \Rightarrow G(x) 
      &= \left(\frac{1}{|2 \pi k|^2}\right)^\lor
      = \frac{1}{d(d-2(k) |x|^{d-2})} 
      = \Phi(x)
    \end{align*}
  \end{rem}

  \section{Theory of Distribution}
  Let \(\Omega \subseteq \mathbb{R}^d\) be open.
  \begin{itemize}
    \item \(D(\Omega) = C_c^\infty(\Omega)\) the space of test functions.
    \item \(\phi_n \to \phi\) in \(D(\Omega)\) if \(\exists K \subseteq \Omega\), \(\supp(\phi_n), \supp(\phi) \subseteq K\) and \(\|D^\alpha(\phi_n - \phi)\|_{L^\infty} \to 0\) for all \(\alpha = (\alpha_1, \dots, \alpha_d)\), \(d_i \in \{0, 1, 2, \dots\}\).
    \[D'(\Omega) = \{T: D(\Omega) \to \mathbb{R} \text{ or } \mathbb{C} \text{ linear and continuous}\}\] the space of distributions.
  \end{itemize}

  Motivation: \(L^2(\Omega)' = L^2(\Omega)\), \((L^p(\Omega))' = (L^q(\Omega))\), \(\frac{1}{p} + \frac{1}{q} = 1\).


  \begin{eg}["normal functions" are distributions]
    If \(f \in L_{loc}^1(\Omega)\), then \(T = T_f\) defined by:
    \[T(\phi) = \int_{\Omega} f(x) \phi(x) \, dx\]
    is a distribution for all \(\phi \in D(\Omega)\), i.e. \(T \in D'(\Omega)\). Indeed, it is clear that \(T(\phi)\) is well-defined for all \(\phi \in D(\Omega)\) and \(\phi \mapsto T(\phi)\) is linear. Let us check that \(\phi \mapsto T(\phi)\) is continuous. Take \(\phi_n \to \phi\) in \(D(\Omega)\) and prove that \(T(\phi_n) \to T(\phi)\). Since \(\phi_n \to \phi\) in \(D(\Omega)\), there is a compact \(K\) s.t. \(\supp(\phi_n), \supp(\phi) \subseteq K \subseteq \Omega\).
  \end{eg}

  Question: Why is \(f \mapsto T_f\) injective?

  \begin{lem}[Fundamental lemma of calculus of variants]
    Let \(\Omega \subseteq \mathbb{R}^d\) be open. If \(f, g \in L_{loc}^1(\Omega)\) and 
    \(\int_\Omega f \phi \, dy = \int_{\Omega} g \phi \, dy\) for all \(\phi \in D(\Omega) \), then \(f = g\) in \(L_{loc}^1(\Omega)\)
  \end{lem}

  % \begin{proof}
  %   By the linearity, it suffices to show that if \(f \in L_{loc}^1(\Omega)\) and \(\int_\Omega f \phi = 0\) for all \(\phi \in D(\Omega)\), then \(f = 0\).
  %   \begin{enumerate}[label=Step \arabic*)]
  %     \item (\(\Omega = \mathbb{R}^d\) and \(f \in L^1(\mathbb{R}^d)\))Take \(h \in C_c^\infty(\mathbb{R}^d)\), \(\int_{\mathbb{R}^d} h = 1\), \(h_\epsilon(x) = \epsilon^{-d}h(\epsilon^{-1}x)\) for all \(\epsilon > 0\). Then \(h_\epsilon \star f \to f\) in \(L^1(\mathbb{R}^d)\) as \(\epsilon \to 0\). On the other hand: \[h_\epsilon \star f(x) = \int_{\mathbb{R}^d} f(y) \underbrace{h_\epsilon(x-y)}_{\phi_{\epsilon, x}(y)} \, dy = 0\] as \(\phi_{\epsilon, x} \in C_c^\infty\).
  %     \item (\(\Omega \subseteq \mathbb{R}^d, f \in L_{loc}^1(\Omega)\)) For any \(\epsilon > 0\), 
  %     \[\Omega_\epsilon = \{x \in \Omega \mid \dist(x, \Omega^\complement) > \epsilon\}\]
  %     Take \(h \in C_c^\infty(\mathbb{R}^d)\), \(\supp f \subseteq \{|x| \le 1\}\) and \(\int_{\mathbb{R}^d} h = 1\), \(h_\epsilon(x) = \epsilon^{-d}h(\epsilon^{-1}x)\). Then for all \(g \in C_c^\infty(\Omega_\epsilon)\) for all \(\epsilon > 0\) small we have 
  %     \begin{align*}
  %       h_\epsilon \star (gf)(x) 
  %       &= \int_{\mathbb{R}^d} \underbrace{h_\epsilon(x-y) g(y)}_{\phi_{\epsilon, x}(y)}f(y) \, dy = 0 \quad \text{as } \phi_{\epsilon, x} \in C_c^\infty(\Omega)
  %     \end{align*}
  %     but ...
  %   \end{enumerate}
  % \end{proof}
  
  \begin{eg}[Dirac delta function]
    Let \(\Omega \subseteq \mathbb{R}^d\) open. Define \(T: D(\Omega) \to \mathbb{R}\) or \(\mathbb{C}\) by \(T(\phi) = \phi(x_0)\). Let \(x_0 \in \Omega\). Then \(T \in D'(\Omega)\) and we denote it by \(\delta_{x_0}\). It is clear that \(\phi \mapsto T(\phi) = \phi(x_0)\) is well-defined and linear for all \(\phi \in D(\Omega)\). Take \(\phi_n \to \phi\) in \(D(\Omega)\) and prove \(T(\phi_n) \to T(\phi)\), i.e. \(\phi_n(x_0) \to \phi(x_0)\) (obvious.)
  \end{eg}

  \begin{eg}[Principle Value]
    The function \(f(x) = \frac{1}{x}\) is not in \(L_{loc}^1(\mathbb{R})\), but we can still define
    \[\int_{\mathbb{R}} f(x) \phi(x) \, dx = \int_{\mathbb{R}} \frac{\phi(x)}{x} \, dx\] for all \(\phi \in D(\mathbb{R})\) s.t. \(\phi(0) = 0\). In fact, \[\phi(x) = |\phi(x)-\phi(0)| \le (\sup |\phi'|)(x),\] so \(\frac{|\phi(x)|}{|x|} \in L^\infty(\mathbb{R})\) and compactly supported. So \(\frac{\phi(x)}{x} \in L^1(\mathbb{R})\).
    Define \(T: D(\mathbb{R}) \to \mathbb{R}\) or \(\mathbb{C}\) by \begin{align*}
      T(\phi) 
      &= \lim_{\epsilon \to 0} \int_{|x| \ge \epsilon} \frac{\phi(x)}{x} \, dx \quad \forall \phi \in D(\mathbb{R}) \text{ s.t. } \phi(0) = 0
    \end{align*}
    We denote \(T = p.v. \left(\frac{1}{x}\right)\). We check that \(T \in D'(\mathbb{R})\): For all \(\epsilon > 0\) we have  \[\left|\frac{\phi(x)}{x}\right| \le \frac{\|\phi\|_{L^\infty}}{\epsilon}\] for all \(|x| \ge \epsilon\) and \(\phi\) is compactly supported. So we get for all \(\epsilon > 0\):
    \begin{align*}
      \mathbb{1}(|x| \ge \epsilon)\frac{\phi(x)}{x} \in L^1(\mathbb{R}) \leadsto \int_{|x| \ge \epsilon} \frac{\phi(x)}{x} \, dx < \infty
    \end{align*}
    We can write:
    \begin{align*}
      \int_{|x| \ge \epsilon} \frac{\phi(x)}{x} \, dx 
      = \int_{|x|\ge 1} \frac{\phi(x)}{x} \, dx + \int_{\epsilon \le |x|\le 1} \frac{\phi(x)}{x}\, dx
    \end{align*}
    The second part can be written as:
    \begin{align*}
      \int_{\epsilon \le |x| \le 1} \frac{\phi(x)}{x} \, dx
      &= \int_{\epsilon}^1 \frac{\phi(x)}{x} \, dx + \int_{-1}^{-\epsilon} \frac{\phi(x)}{x} \, dx 
      = \int_\epsilon^1 \frac{\phi(x) - \phi(-x)}{x}\, dx
    \end{align*}
    Since \(\phi \in C_c^\infty(\mathbb{R})\) it holds that \(|\phi(x) - \phi(-x)|  \le 2 \|\phi'\|_{L^\infty}(x)\).
    \begin{align*}
      \Rightarrow \frac{\phi(x) - \phi(-x)}{x} \in L^\infty(\mathbb{R})
      \Rightarrow \frac{\phi(x) - \phi(-x)}{x} \in L^1(0, 1) \\
      \Rightarrow \int_0^1 \frac{\phi(x) - \phi(-x)}{x} \, dx = \lim_{\epsilon \to 0} \int_\epsilon^1 \frac{\phi(x) - \phi(-x)}{x} \, dx
    \end{align*}
  \end{eg}

  \begin{rem}
    The function \(\frac{1}{|x|^d}\) is not in \(L_{loc}^1(\mathbb{R}^d)\) but \(\exists T \in D'(\mathbb{R}^d)\) s.t. \(T(\phi) = \int_{\mathbb{R}^d} \frac{\phi(x)}{|x|^d} \, dx\) for all \(\phi \in C_c^\infty(\mathbb{R}^d)\) s.t. \(\phi(0) = 0\)
  \end{rem}

  Let in the following \(\mathbb{K} \in \{\mathbb{R}, \mathbb{C}\}\).

  \begin{defn}[Derivatives of distributions] 
    Let \(\Omega \subseteq \mathbb{R}^d\) and \(T \in D'(\Omega)\). Define for \(\alpha \in \mathbb{N}^d\): 
    \begin{align*}
      D^\alpha T: \ D(\Omega) &\longrightarrow \mathbb{K} \\
      \phi &\longmapsto (-1)^{|\alpha|}T(D^\alpha \phi)
    \end{align*}
    
  \end{defn}
  Motivation: \(f \in C_c^\infty(\Omega)\)
  \begin{align*}
    \int_\Omega(D^\alpha f) \phi = (-1)^{|\alpha|} \int_\Omega f(D^\alpha \phi)
  \end{align*}
  \glqq{}If the classical derivative exists, then it is the same as the distributional derivative.\grqq{} We write
  \[(D^\alpha T)(\phi) = T_{D^\alpha f} (\phi) = (-1)^{|\alpha|}T_f(D^\alpha \phi).\]
  
  \begin{rem}
    For all \(T \in D'(\Omega)\) it holds \(D^\alpha T \in D'(\Omega)\) for all \(\alpha \in \mathbb{N}^d\). Clearly
    \[\phi \longmapsto (D^\alpha T)(\phi) = (-1)^{|\alpha|}T(D^\alpha \phi)\] is linear. Moreover, if \(\phi_n \to \phi\) in \(D(\Omega)\), then \(D^\alpha \phi_n \to D^\alpha \phi \text{ in } D(\Omega)\), so
    \begin{align*}
      (D^\alpha T)(\phi_n) = (-1)^{|\alpha|} T(D^\alpha \phi_n) \xrightarrow{n \to \infty} (-1)^{|\alpha|} T(D^\alpha \phi) = (D^\alpha T)(\phi)
    \end{align*}
  \end{rem}

  \begin{eg}
    Consider \(f: x \mapsto |x|\), then \(f \in C(\mathbb{R})\) but \(f \notin C^1(\mathbb{R})\). However, 
    \[f'(x) = g(x) = \begin{cases}
      1 & x \ge 0 \\ -1 &x < 0
    \end{cases} \in L_{loc}^1\]
    Lets check \(f' = g\), i.e. \(-f(\phi') = f'(\phi) = g(\phi)\) for all \(\phi \in D(\mathbb{R})\). Thus we need to prove:
    \begin{align*}
      - \int_{\mathbb{R}} f(x) \phi'(x) \, dx = \int_{\mathbb{R}} g(x) \phi(x) \, dx \quad \forall \phi \in D(\mathbb{R})
    \end{align*}
    namely:
    \begin{align*}
      \underbrace{- \int_{\mathbb{R}} |x| \phi'(x) \, dx}_{\coloneqq (\star)} &= \int_0^\infty \phi(x) \, dx - \int_{-\infty}^0 \phi(x) \, dx
    \end{align*}
    Now we have
    \begin{align*}
      (\star) = - \int_0^\infty x \phi'(x) \, dx + \int_{-\infty}^0 x \phi'(x) \, dx.
    \end{align*}
    By integration by parts:
    \begin{align*}
      \int_0^\infty x \phi'(x) \, dx = \underbrace{[x \phi(x)]_0^\infty}_{= 0} - \int_0^\infty \phi(x) \, dx = - \int_0^\infty \phi(x) \, dx
    \end{align*}
    and similary:
    \begin{align*}
      \int_{-\infty}^0 x \phi'(x) \, dx = - \int_{-\infty}^0 \phi(x) \, dx
    \end{align*}
    Thus \(f' = g\) in \(D'(\Omega)\). We claim that \(g' = 2 \delta_0\) in \(D'(\mathbb{R})\). In fact, for all \(\phi \in D(\mathbb{R})\), then:
    \begin{align*}
      g'(\phi) 
      &= - g(\phi') 
      = - \int_{\mathbb{R}} g\phi' \, dx
      = - \int_{-\infty}^0 (-1) \phi'  \, dx - \int_0^\infty (1) \phi'  \, dx\\
      &= - \int_0^\infty \phi'  \, dx + \int_{-\infty}^0 \phi' \, dx
      = [\phi(0) - \underbrace{\phi(\infty)}_{= 0}] + [\phi(0) - \underbrace{\phi(-\infty)}_{= 0}]
      = 2 \phi(0) = 2 \delta_0 (\phi)
    \end{align*}
    So \(g' = 2 \delta_0\) in \(D'(\mathbb{R})\).
  \end{eg}

  \begin{ex}
    Prove that \((D^\alpha \delta_x)(\phi) = (-1)^{|\alpha|}(D^\alpha \phi)(x)\) for all \(\phi \in D(\mathbb{R})\) for all \(x \in \mathbb{R}\).
  \end{ex}

  \begin{defn}[Convergence of distributions]
    Let \(\Omega \subseteq \mathbb{R}^d\) be open, then
    \[T_n \xrightarrow{n \to \infty} T\] in \(D'(\Omega)\) if \(T_n(\phi) \xrightarrow{n \to \infty} T(\phi)\) for all \(\phi \in D(\Omega)\).
  \end{defn}

  \begin{ex}
    Let \(f \in L^1(\mathbb{R}^d)\), \(\int f = 1\) For \(\epsilon > 0\), define \(f_\epsilon(x) = \epsilon^{-d}f(\epsilon^{-1} x)\). Then: \(f_\epsilon \to \delta_0\) in \(D'(\Omega)\).
  \end{ex}
  
  \begin{ex}
    Let \(\Omega \subseteq \mathbb{R}^d\) be open and \(T_n \to T\) in \(D'(\Omega)\). Then:
    \(D^\alpha T_n \to D^\alpha T\) in \(D'(\Omega)\) for all \(\alpha = (\alpha_1, \dots, \alpha_d)\)
  \end{ex}
  
  \begin{defn}[Convolution of distributions]
    Let \(T \in D'(\mathbb{R})\) and \(f \in L_c^\infty(\mathbb{R}^d)\). Define \[(T \star f)(y) = T(f_y)\]
    We write \(f_y(x) = f(x-y)\) and \(\tilde f(x) = f(-x)\).
  \end{defn}
  
  \begin{thm}
    Let \(T \in D'(\mathbb{R})\). Then for all \(f \in D(\mathbb{R})\):
    \begin{enumerate}
      \item \(y \mapsto T(f_y)\) is \(C^\infty(\mathbb{R}^d)\) and 
      \[D_y^\alpha(T(f_y)) = (D^\alpha T)(f_y) = (-1)^{|\alpha|}T(D^\alpha f_y)\]
      \item For all \(g \in L^1(\mathbb{R}^d)\) and compactly supported, then
      \[\int_{\mathbb{R}^d} g(y) T(f_y) \, dy = T(\underbrace{f \star g}_{\in C_c^\infty(\mathbb{R})})\]
    \end{enumerate}
  \end{thm}

  \begin{proof}
    \begin{enumerate}
      \item We prove that \(y \mapsto T(f_y)\) is continuous. Take \(y_n \to y\) in \(\mathbb{R}^d\), then:\[T(f_{y_n}) \to T(f_y)\] since \(f_{y_n} \to f_y\) in \(D(\mathbb{R}^d)\). We check this: Since \(f \subseteq C_c^\infty(\mathbb{R}^d)\), it holds that \(\supp f \subseteq B(0, R) \subseteq \mathbb{R}^d\). Since \(y_n \to y\) in \(\mathbb{R}^d\). We have \(\sup_n |y_n| < \infty\). Thus \(f_{y_n}, f_y\) are supported in \(\overline{B(0, R + \sup_n |y_n|)} = K\) compact. Moreover
      \begin{align*}
        |f_{y_n}(x) - f_y(x)|
        &= |f(x-y_n) - f(x-y))|
        \le \| \nabla f \|_{L^\infty} \|y_n - y\| \to 0
      \end{align*}
      So we get \(\| f_{y_n} - f_y\|_{L^\infty} \to 0\)
      Similary:
      \[\|D^\alpha f_{y_n} - D^\alpha f_n \|_{L^\infty} \to 0\] \qedhere
    \end{enumerate}
  \end{proof}


  \begin{ex}[E 3.1 Lebesgue Differentiation Theorem]
    Let \(f \in L_{loc}^1(\mathbb{R}^d)\). Prove that that for almost every \(x \in \mathbb{R}^d\):
    \[\fint_{B(x,r)} |f(x) - f(y)| \, dy \xrightarrow{r \to 0} 0\]
  \end{ex}

  \begin{proof}
    Clearly the same result holds with \(\mathbb{R}^d \leadsto \Omega \subseteq \mathbb{R}^d\) open. Also it suffices to consider \(f \in L^1(\mathbb{R}^d)\). From the last time discussion, by a density argument there exists \(r_n \to 0\) s.t.
    \[\fint_{B(x, r_n)} |f(y) - f(x)| \, dy = 0\]
    for a.e. \(x \in \mathbb{R}^d\). We prove that for all \(\epsilon > 0\), te set \(A_\epsilon = \{x \in \mathbb{R}^d \mid \limsup_{r \to 0} \fint_{B(x,r)} |f(y) - f(x)| \, dy > \epsilon\}\) has measure \(0\). This will imply that 
    \[\bigcup_{n = 1}^\infty A_{\frac{1}{n}} = \left\{x \in \mathbb{R}^d \mid \limsup_{r \to 0} \int_{B(x,r)} |f(y) - f(x)| \, dy > 0\right\}\] has measure \(0\), which is what wie want to show. First, we show that \(|A_\epsilon| = 0\): Take \(\{f_n\} \subseteq C_c^\infty\), \(f_n \to f\) in \(L^1(\mathbb{R}^d)\). By the triangle inequality:
    \begin{align*}
      |f(y) - f(x)| \le |f(y) - f_n(y)| + |f_n(y) - f_n(x)| + |f_n(x) - f(x)|
    \end{align*}
    So we get 
    \begin{align*}
      &\fint_{B(x,r)} |f(y) - f(x)| \, dy \\
      \quad &\le \fint_{B(x,r)}|f(y) - f_n(y)| \, dy + \fint_{B(x, r)} |f_n(y) - f_n(x)| + |f_n(x) - f(x)| \\
      \Rightarrow \quad \limsup_{r \to 0} ... &\le \limsup_{r \to 0} (\dots) +0 + |f_n(x) - f(x)|
    \end{align*}
    Thus, for all \(x \in A_\epsilon\), then:
    \begin{align*}
      \limsup_{r \to 0} \fint_{B(x, r)} |f_n(y) - f(y)| \, dy + |f_n(x) - f(x)| > 2 \epsilon
    \end{align*}
    Observation: If \(a, b \ge 0\), \(a + b > 2 \epsilon\) then either \(a > \epsilon\) or \(b > \epsilon\). Therefore \(A_\epsilon \subseteq \left(S_{n, \epsilon} \bigcup \tilde S_{n, \epsilon}\right)\), where
    \begin{align*}
      S_{n, \epsilon} &= \{x \mid |f_n(x) - f(x)| > \epsilon\} \\
      \tilde S_{n, \epsilon} &= \{x \mid \limsup_{r \to 0} \fint_{B(x,r)} |f_n(y) - f(y)| \, dy > \epsilon\}
    \end{align*}
    Consequently: \(|A_\epsilon| \le |S_{n, \epsilon}| + |\tilde S_{n, \epsilon}|\) for all \(n \ge 1\). By the Markov / Chebyshev inequality:
    \begin{align*}
      |S_{n, \epsilon}| 
      &\le \int_{S_{n, \epsilon}} \frac{|f_n(x) - f(x)|}{\epsilon} \, dx
      = \int_{\mathbb{R}^d} \frac{|f_n(x) - f(x)|}{\epsilon} \, dx
      = \frac{\|f_n - f\|_{L^1}}{\epsilon}
    \end{align*}
    We want to prove a simpler bound for \(\tilde S_{n \epsilon}\). For all \(x \in \tilde S_{n \epsilon}\): 
    \begin{align*}
      \limsup_{r \to 0} \fint_{B(x, r)} |f_n(x) - f(y)| \, dy > \epsilon
    \end{align*}
    So there is a \(r_x \in (0,1)\) s.t.
    \[\fint_{B(x,r_x) = B_x} |f_n(y) - f(y)| \, dy > \epsilon\]
    Thus \(\tilde S_{n \epsilon} \subseteq \left(\bigcup_{x \in \tilde S_{n, \epsilon}} B_x\right)\). 
    
    \begin{lem}[Vitali Covering]
      If \(F\) is a collection of balls in \(\mathbb{R}^d\) with bounded radius, then there exists a sub-collection \(G \subseteq F\) s.t.
      \begin{itemize}
        \item \(G\) has disjoint balls
        \item \(\bigcup_{B \in F} B \subseteq \bigcup_{B \in G} 5B, 5B(x,r) = B(x, 5r)\)
      \end{itemize}
    \end{lem}

    \begin{rem}
      The condition of the boundedness of the radius is necessary. Otherwise, consider \(\{B(0,n)\}_{n=1}^\infty\)
    \end{rem}

    Here consider \(F = \{B_x\}_{x \in \tilde S_{n \epsilon}}\). With the vitali covering leamm there is a \(G \subseteq F\) s.t. \(G \) contains disjoint balls and:
    \[\tilde S_{n, \epsilon} \subseteq \bigcup_{B \in F} B \subseteq \bigcup_{B \in G} 5 B\]
    So we get
    \[|\tilde S_{n, \epsilon}| \le |\bigcup_{B \in G} 5B| \le \sum_{B \in G} |5B| = \sum_{B \in G} 5^d|B|\]
    On the other hand, for all \(B \in G \subseteq F\): 
    \begin{align*}
      \fint_B |f_n(y) - f(y)| \, dy > \epsilon \Rightarrow \int_B |f_n - f| > \epsilon|B|
    \end{align*}
    This implies:
    \begin{align*}
      \sup_{B \in G} \int_B |f_n - f| > \epsilon \sum_{B \in G} |B|
    \end{align*}
    Since balls in \(G\) are disjoint:
    \begin{align*}
      \int_{\mathbb{R}^d} \ge \int_{\bigcup_{B \in G}} |f_n - f| \, dy 
      > \epsilon \sum_{B \in G} |B|
      \ge \frac{\epsilon}{5^d} |\tilde S_{n, \epsilon}|
    \end{align*}
    So 
    \[|\tilde S_{n \epsilon}| \le \frac{5^d}{\epsilon} \|f_n - f\|_{L^1}\]
    In summary: 
    \[|A_\epsilon| \le |S_{n, \epsilon}| + |\tilde S_{n, \epsilon}| \le \frac{5^d + 1}{\epsilon} \|f_n - f\|_{L^1} \to 0\]
    as \(n \to \infty\). So \(|A_\epsilon| = 0\) for all \(\epsilon > 0\)
  \end{proof}
  
  \begin{rem}
    \begin{enumerate}
      \item The proof can be done by using the Besicovitch covering lemma: For all \(E \subseteq \mathbb{R}^d\) s.t. \(E\) is bounded. Let \(F = \) collection of balls s.t. for all \(x \in E\) there is a \(B_x \in F\) s.t. \(x\) is the center of \(B_x\). There is a sub-collection \(G \subseteq F\) s.t.
      \begin{itemize}
        \item \(E \subseteq \bigcup_{B \in G} B\)
        \item Any point in \(E\) belongs to at most \(C_d\) balls in \(C_T\) (\(C_d\) depends only on \(\mathbb{R}^d\)), i.e. 
        \begin{align*}
          \mathbb{1}_E(x) \le \sum_{B \in G} \mathbb{1}_B(x) \le C_d \mathbb{1}_E(x) \forall x
        \end{align*}
      \end{itemize}
      \item By a simpler argument we can prove the weak \(L^1\)-estimate:
      \[\{x \mid f^\star(x) > \epsilon\} \le \frac{c_d}{\epsilon} \|f\|_{L^1(\mathbb{R}^d)}\] (Hardy-Littlewood maximal function)
    \end{enumerate}
  \end{rem}

  \begin{ex}[E 3.2]
    Let \(1 \le p, q, r \le 2\), \(\frac{1}{p} + \frac{1}{q} = 1 + \frac{1}{r}\). Recall that if \(f \in L^p(\mathbb{R}^d)\), \(g \in L^q(\mathbb{R}^d)\), then \(f \star g \in L^r(\mathbb{R}^d)\) by Young's Inequality, and its Fourier transform is well-defined by the Hausdorff-Young inequality. Prove that 
    \[\widehat{f \star g}(k) = \hat f(k) \hat g(k) \quad \forall k \in \mathbb{R}^d\]

    Hint: In the lecture we already discussed the case \(f, g \in C_c(\mathbb{R}^d)\).
  \end{ex}

  \begin{proof}[Solution] \
    \begin{enumerate}[label=Step \arabic*)]
      \item \(f, g \in C_c^\infty(\mathbb{R}^d)\) (Fubini)
      \item \(f \in L^p, g \in L^q\), find \(f_n, g_n \in C_c^\infty\) s.t. \(f_n \to f\) in \(L^p\), \(g_n \to g\) in \(L^q\). \(\widehat{f_n \star g_n} = \hat f_n \hat g_n\) pointwise a.e. we have
      \begin{align*}
        \text{(Hausdorff-Young)} \quad &\|\widehat{f \star g} - \widehat{f_n \star g_n}\|_{L^{r'}} \\
        &\le \|\widehat{f \star g} - \widehat{f_n \star g_n}\|_{L^r} \\
        &= \|(f - f_n) \star g_n + f_n \star (g_n - g) \|_{L^r} \\
        &\le \| (f-f_n) \star g_n \|_{L^r} + \| f_n \star (g_n - g)\|_{L^r} \\
        \text{(Young)} &\le \|f - f_n\|_{L^p} \|g_n\| + \|f_n\|_{L^p} \|g_n - g\|_{L^p} \xrightarrow{n \to \infty} 0
      \end{align*}
      Moreover:
      \begin{align*}
        \|\hat f_n \hat g_n - \hat f \hat g \|_{L^{r'}} 
        &= \|(\hat f_n \hat f)\hat g_n + \hat f (\hat g_n - \hat g) \|_{L^{r'}} \\
        \text{(Hölder)} \quad &\le \|\hat f_n - \hat f \|_{L^{p'}} \| \hat g_n \|_{L^{q'}} + \| \hat f \|_{L^{q'}} \\
        \text{(Hausdorff-Young (\ref{hausdorff-young}))} \quad &\le \| f_n - f \|_{L^p}\|g_n\|_{L^q} + \|f\|_{L^p} \|g_n - g\|_{L^p} \xrightarrow{n \to \infty} 0
      \end{align*}
      So \(\hat f_n \hat g_n \to \hat f \hat g \text{ in } L^{r'}\)
      \(\widehat{f \star g} = \hat f \hat g\) in \(L^{r'}\)
      \(\frac{1}{r'} = \frac{1}{p'} + \frac{1}{q'}\) \qedhere
    \end{enumerate}
  \end{proof}

  
  \begin{ex}[E 3.3]
    \(f \in C_c^\infty(\mathbb{R}^d)\). Prove \(|\hat f(k)| \le \frac{C_N}{(1 + |k|)^N}\)
  \end{ex}

  \begin{proof}[Solution]
    Since \(f \in C_c^\infty\) we have that \(D^\alpha f \in C_c^\infty\).
    Recall
    \begin{align*}
      \widehat{D^\alpha f}(k) &= (-2 \pi ik)^\alpha \hat f (k)
    \end{align*}
    For example
    \begin{align*}
      \widehat{- \Delta f}(k) &= |2 \pi i k|^2 \hat f (k) \\
      (\text{Induction}) \leadsto \widehat{(-\Delta)^N f}(k) &= |2 \pi k|^{2N} \hat f (k)
    \end{align*}
    So we can conclude
    \[ \hat f(k) = \frac{\widehat{(-\Delta)^N}f(k)}{|2 \pi k|^{2N}} \forall k \in \mathbb{R}^d\]
    \begin{enumerate}
      \item \(f \in C_c^\infty \subseteq L^1(\mathbb{R}^d) \Rightarrow \hat f \in L^\infty\)
      \item \((-\Delta)^N f \in C_c^\infty \subseteq L^1(\mathbb{R}^d) \Rightarrow \widehat{(-\Delta)^N f} \in L^\infty\)
    \end{enumerate}
    Conclusion: \(\hat f(k) \le \begin{cases}
      C &\forall k \\ \frac{C_N}{|k|^{2N}} &\forall k
    \end{cases}\)
    So \(\hat f(k) \le \frac{C_N}{(1+|k|)^N}\)
  \end{proof}

  \begin{ex}[E 3.4]
    
  \end{ex}

  \begin{proof}
    Siehe Goodnotes
  \end{proof}
  
  \begin{ex}[Bonus 3]
    Let \(f \in L^1(\mathbb{R}^d)\) such that
    \[|\hat f(k)| \le \frac{C_N}{(1 + |k|)^N}\]
    for all \(k \in \mathbb{R}^d\), for all \(N \ge 1\). (\(C_N\) is independent of \(k\)). Prove that \(f \in C^\infty(\mathbb{R}^d)\)
  \end{ex}
  (\(f \in C^\infty\)) i.e. \(\exists \tilde f \in C^\infty\) s.t. \(f = \tilde f\) a.e.

  \begin{thm}
    Take \(T \in D'(\mathbb{R}), f \in C_c^\infty(\mathbb{R}^d) = D(\mathbb{R}^d)\), \(f_y(x) = f(x-y)\)
    \begin{enumerate}[label=\alph*)]
      \item \(y \mapsto T(f_y) \in C^\infty(\mathbb{R}^d)\) and \(D_y^\alpha(T(f_y)) = (D^\alpha T) (f_y) = (-1)^{|\alpha|} T(D_x^\alpha f_y)\)
      \item \(\forall g \in L^1(\mathbb{R}^d)\) and compactly supported 
      \[\int_{\mathbb{R}^d} g(y) T(f_y) \, dy = T(\underbrace{f \star g}_{\in C_c^\infty})\]
    \end{enumerate}
  \end{thm}

  \begin{proof}
    \begin{enumerate}[label=\alph*)]
      \item \(y \mapsto T(f_y)\) is continuous since \(y_n \to y\) in \(\mathbb{R}^d\), then 
      \(f_{y_n} \to f_y\) implies \(T(f_{y_n}) \to T(f_y)\). Let's check that \(y \mapsto T(f_y) \in C^1\):
      \begin{align*}
        \lim_{h \to 0} \frac{T(f_{y-he_i})-T(f_y)}{h}
        &= \lim_{h \to 0} T \left(\frac{f_{y-he_i} - f_y}{h}\right)
      \end{align*}
        We have \(\frac{f_{y - h e_i} - f_y}{h} \xrightarrow{h \to 0} (\partial_i f)_y\) in \(D(\mathbb{R}^d)\)
      \begin{itemize}
        \item \(\exists K\) compact set such that \(\supp(f_{y - e_i} - f_y)\), \(\supp \partial_i f \subseteq K\) as \(|h|\) small.
        \item \(\begin{aligned}[t]
          &\frac{f_{y - h e_i}(x)- f_y(x)}{h} - (\partial_i f)_y(x) \\
          &\quad = \frac{f(x-y+he_i) - f(x-y)}{h} - (\partial_i f)(x-y) \\
          &\left| \int_0^1 \partial_i f(x-y+the_i) \, dt - \partial_i f(x-y) \right| \xrightarrow{h \to 0} 0 \text{ uniformly in } x
        \end{aligned}\) \\
        Similary:
        \begin{align*}
          &\left| D_x^\alpha \left(\frac{f(x-y+he_i) - f(x-y)}{h} - (\partial_i f)(x-y)\right) \right| \\
          &= \left| \frac{D^\alpha f(x-y+he_i) - D^\alpha f(x-y)}{h} - \partial_i(D^\alpha f)(x-y) \right| \xrightarrow{h \to 0} 0 
        \end{align*}
        uniformly in \(x\). Conclude: 
        \begin{align*}
          \lim_{h \to 0} \frac{T(f_{y - he_i}) - T(f_y)}{h} \xrightarrow{h \to 0} T((\partial_i f)_y) \in C(\mathbb{R}^d)
        \end{align*}
        So we geht that \(y \mapsto T(f_y) \in C^1\) and \(- \partial_{y_i} T(f_y) = T((\partial_i f)_y)\)
      \end{itemize}
      By induction: 
      \begin{align*}
        D_y^\alpha T(f_y) = (-1)^{|\alpha|} T((D^\alpha f)_y) = (D^\alpha T)(f_y) \quad \forall \alpha \in \mathbb{N}^d
      \end{align*}
      \item Heuristic: \(T = T(x)\)
      \begin{align*}
        \int_{\mathbb{R}^d} g(y) T(f_y) \, dy 
        &= \int_{\mathbb{R}^d} g(y) \left(\int_{\mathbb{R}^d}T(x) f(x-y) \, dx \right)\, dy \\
        &= \int_{\mathbb{R}^d} T(x) \left(\int_{\mathbb{R}^d} g(y) f(x-y) \, dy \right) \, dx \\
        &= \int_{\mathbb{R}^d} T(x) (f \star g)(x) \, dx = T(f \star g)
      \end{align*}
      \begin{enumerate}[label=Step \arabic*:]
        \item \(g \in C_c^\infty(\mathbb{R}^d)\) \begin{align*}
          (\text{Rieman Sum}) \quad \int_{\mathbb{R}^d} g(y) T(f_y) \, dy 
          &= \lim_{\Delta_N \to 0} \Delta_N \sum_{j=1}^N g(y_j) T(f_{y_j}) \\
          &= \lim_{\Delta_N \to 0} T\left(\Delta_N \sum_{j=1}^N g(y_j) f_{y_j}\right) \\
          &= T(f\star g)
        \end{align*}
        because
        \begin{align*}
          \lim_{\Delta_N \to 0} \Delta_N \sum_{j=1}^N g(y_j) f_{y_j} (x) \to (f\star g)(x) \text{ in } D(\mathbb{R}^d) \\
          \lim_{\Delta_N \to 0} \Delta_N \sum_{j=1}^N g(y_j) f(x-y_j) \xrightarrow{\text{Riemann}} \int_{\mathbb{R}^d} g(y) f(x-y) \, dy = (f \star g)(x)
        \end{align*}
        Proof of: 
        \begin{align*}
          \lim_{\Delta_N \to 0} \Delta_N \sum_{j=1}^N g(y_j) f(x-y_j) \to (f\star g)(x) \text{ in } D(\mathbb{R}^d)
        \end{align*}
        \begin{enumerate}[label=\arabic*)]
          \item Since \(f, g \in C_c^\infty\) we have \(f \star g \in C_c^\infty\). And we have \[x \mapsto \Delta_N \sum_{j=1}^N g(y_j) f(x-y_j) \in C^\infty\] since \(f \in C^\infty\) supported in \((\supp g + \supp f)\). So all functions are \(C_c^\infty\) and supported in \((\supp g + \supp f)\).
          \item 
          \begin{align*}
            \left| \lim_{\Delta_N \to 0} \Delta_N \sum_{j=1}^N g(y_j) f(x-y_j) - \int_{\mathbb{R}^d} g(y) f(x-y) \, dy \right| \xrightarrow{\Delta_N \to 0} 0
          \end{align*}
          uniformly in \(x\). (Result from the Riemann-Sum)
          \item \begin{align*}
            &\left| D_x^\alpha(\Delta_N \sum_{j=1}^N g(y_j) f(x-y) - (f \star g)(x)) \right| \\
            &= \left| \Delta_N \sum_{j=1}^N g(y_j) D^\alpha f(x-y) - (D^\alpha f)\star g(x) \right| \xrightarrow{\Delta_N \to 0} 0
          \end{align*}
          uniformly in \(x\) for all \(\alpha\).
        \end{enumerate}
        \item Take \(g \in L^1(\mathbb{R}^d)\) and compactly supported. Then \(\exists \{g_n\} \subseteq C_c^\infty(\mathbb{R}^d)\), \(\supp g_n \subseteq \supp g + B(0,1)\) such that \(g_n \to g\) in \(L^1(\mathbb{R}^d)\). By Step 1: 
        \begin{align*}
          \int_{\mathbb{R}^d} g_n(y) T(f_y) \, dy = T(g_n \star f)
        \end{align*}
        Take \(n \to \infty\):
        \begin{align*}
          \int_{\mathbb{R}^d} g_n(y) T(f_y) \, dy \to \int_{\mathbb{R}^d} g(y) T(f_y) \, dy
        \end{align*}
        since \(g_n \to g\) in \(L^1\) compactly supported and \(y \mapsto T(f_y) \in C^\infty \subseteq L^\infty(K)\). Moreover (exercise):
        \[\underbrace{g_n \star f}_{\in C_c^\infty} \to g \star f \quad \text{ in } D(\mathbb{R}^d)\]
        So \(T(g_n \star f) \xrightarrow{n \to \infty} T(g \star f)\).
        Finally we optain:
        \[\int g(y) T(f_n) \, dy = T(g \star f)\qedhere\] 
      \end{enumerate}
  \end{enumerate}
  \end{proof}
  
  \begin{thm}
    Let \(\Omega \subseteq \mathbb{R}^d\) be open. Let \(T \in D'(\Omega)\) and \(f \in C_c^\infty(\Omega)\). Denote
    \[\Omega_f = \{y \in \mathbb{R}^d \mid \supp f_y = y + \supp f \subseteq \Omega\}\]
    \begin{enumerate}[label=\alph*)]
      \item \(y \mapsto T(f_y) \in C^\infty(\Omega_f)\) and \(D_y^\alpha(T(f_y)) = (D^\alpha T)(f_y) = (-1)^{|\alpha|}T((D^\alpha f)_y)\)
      \item For all \(g \in L^1(\Omega_g)\) compactly supported in \(\Omega_f\) and it holds:
      \[\int_\Omega g(y) T(f_y) \, dy = T(f \star g).\]
    \end{enumerate}
  \end{thm}
  
  \begin{thm}
    Let \(T \in D'(\Omega)\) s.t. \(\nabla T = 0\) in \(D'(\Omega)\). Then: \(T = const.\) in \(\Omega\).
  \end{thm}

  \begin{proof}
    \((\Omega = \mathbb{R}^d)\) for all \(f \in C_c^\infty\), \(y \mapsto T(f_y) \in C^\infty(\mathbb{R}^d)\) and \(\partial_{y_i} T(f_y) = (\partial_j T)(f_y) = 0\) for all \(i = 1, \dots, d\). Then by the result of the theorem for \(C^\infty\) functions,
    \(y \mapsto T(f_y) = const\) independent of \(y\). Consequently:
    \begin{align*}
      T(f_y) = T(f_0) = T(f) \quad \forall y \in \mathbb{R}^d \ \forall f \in C_c^\infty(\mathbb{R}^d)
    \end{align*}
    For any \(g \in C^\infty(\mathbb{R}^d)\):
    \begin{align*}
      \left(\int_{\mathbb{R}^d} g \, dy\right) T(f)
      &= \int_{\mathbb{R}^d} g(y) T(f_y) \, dy
      = T(f \star g)
      = T(g \star f)
      = \left(\int_{\mathbb{R}^d} f \, dy\right) T(g)
    \end{align*}
    So \(\frac{T(f)}{\int_{\mathbb{R}^d}f}\) is independent of \(f\) (as soon as \(\int f \ne 0\)). So we get that \(T(f) = const \int_{\mathbb{R}^d} f\), where const is independent of \(f\).
  \end{proof}
  
  \begin{rem}
    If \(u \in C^1(\mathbb{R}^d)\), then:
      \begin{align*}
        u(x+y) - u(x) 
        &= \int_0^1 \sum_{j=1}^d y_j (\partial_j u)(x + ty_j) \, dt
        = \int_0^1 y \nabla u(x+ty) \, dt
      \end{align*}
      So we get that if \(\nabla u = 0\), then \(u(x+y) - u(x) = 0\) for all \(x,y\), so \(u = const.\)
  \end{rem}

  \begin{thm}[Taylor expansion for distributions]
    Let \(
      T \in D'(\mathbb{R}^d)\) and \(f \in C_c^\infty(\mathbb{R}^d)\). Then \(y \mapsto T(f_y) \in C^\infty\)
       and 
    \[T(f_y) - T(f) = \int_0^1 \sum_{j=1}^d y_j(\partial_j T) (f_{ty}) \, dt.\]
    In particular, if \(g \in L_{loc}^1\) and \(\nabla g \in L_{loc}^1\), then \(\forall y \in \mathbb{R}^d\):
    \[g (x+y) - g(x) = \int_0^1 g(x + ty) y \, dt\]
    for a.e. \(x \in \mathbb{R}^d\).
  \end{thm}

  \begin{proof}
    \(y \mapsto T(f_y)\) is \(C^\infty\) and \(\frac{d}{dt} [T(f_{ty})] = (\nabla T)(f_{ty}) y\) So we get
    \begin{align*}
      T(f_y) - T(f) 
      &= \int_0^1 \frac{d}{dt}(T(f_{ty})) \, dt \\
      &= \int_0^1 (\nabla T)(f_{ty}) y \, dt \\
      &= \int_0^1 \sum_{j=1}^d (\partial_j T)(f_{ty}) y_j \, dt \qedhere
    \end{align*}
  \end{proof}
  
  \begin{cor}
    Let \(g \in L_{loc}^1(\mathbb{R}^d)\) s.t. \(\partial_j g\in L_{loc}^1(\mathbb{R}^d)\) for all \(j = 1, 2, \dots, d\) (i.e. \(g \in W_{loc}^{1,1}(\mathbb{R}^d)\)). Then for all \(y \in \mathbb{R}^d\):
    \begin{align*}
      g(x+y) - g(x) &= \int_0^1 y \cdot \nabla g(x + ty) \, dt \\
      &= \int_0^1 \sum_{j=1}^d y_j \partial g(x+ty) \, dt
    \end{align*}
    for a.e. \(x\).
  \end{cor}

  \begin{proof}
    For all \(f \in C_c^\infty\) we have 
    \begin{align*}
      \int_{\mathbb{R}^d} f(x) [g(x+y) - g(x)] \, dx
      &= \int_{\mathbb{R}^d} g(x) [f(x-y) - f(x)] \, dx \\
      &= g(f_y) - g(f) \\
      &= \int_0^1 \sum_{j=1}^d y_j (\partial_j g)(f_{ty}) \, dt \\
      &= \int_0^1 \sum_{j=1}^d y_j \int_{\mathbb{R}^d} \sum_{j=1}^d y_j \left[\int_{\mathbb{R}^d} (\partial_j g)(x) f_{ty}(x) \, dx\right] \\
      &= \int_0^1 \sum_{j=1}^d y_i \left[\int_{\mathbb{R}^d}(\partial_j g)(x+ty) f(x) \, dx\right] \, dt \\
      &= \int_{\mathbb{R}^d} f(x) \left[\int_0^1 \sum_{j=1}^d y_j \partial_j g(x+ty) \, dt \right] \, dx
    \end{align*}
    For all \(\phi \in C_c^\infty\): \( = g(x+y) - g(x)\) a.e. \(x \in \mathbb{R}^d\).
  \end{proof}

  \begin{rem}
    If \(T \in D'(\Omega)\), \(\Omega \subseteq \mathbb{R}^d\) open, if \(y \nabla T = 0\), then \(T = const\).
  \end{rem}

  \begin{thm}[Equivalence of the classical and distributional derivatives]
      Let \(\Omega \subseteq \mathbb{R}^d\). Then the following are equivalent:
      \begin{enumerate}
        \item \(T \in D'(\Omega)\) s.t. \(\partial_{x_i} T = g_i \in C(\Omega)\) for all \(i = 1, \dots, d\).
        \item \(T = f \in C^1(\Omega)\) and \(g_i = \partial_{x_i} f\)
      \end{enumerate}
  \end{thm}

  \begin{proof} \
    \begin{itemize}
      \item [(2) \(\Rightarrow\) (1):] If \(T = f \in C^1(\Omega)\), then: \(\partial_{x_i} f \in C(\Omega)\).
      \begin{align*}
        \partial_{x_i} T(\phi) &= - T(\partial_{x_i} \phi) 
        = - \int_\Omega f(\partial_{x_i} \phi) = \int_\Omega (\partial_{x_i} f) \phi
      \end{align*}
      for all \(\phi \in D(\Omega)\), so \(\partial_{x_i} T = \partial_{x_i} f\).
      \item [(1) \(\Rightarrow\) (2):] Why is \(T = f \in C^1(\Omega)\)? As \(\partial_{x_i} f = g_i\): 
      \begin{align*}
        f(x+y) - f(x) 
        &= \int_0^1 \nabla f(x+ty) y \, dt
        = \int_0^1 \sum_{i=1}^d g_i(x + ty) y_i \, dt
      \end{align*}
      So we get
      \begin{align*}
        f(y) = f(0) + \int_0^1 \sum_{i=1}^d g_i(ty) g_i \, dt.
      \end{align*}
      We expect that \(f \in C^1\) and \(\partial_{x_i} f = g_i\). But this is not trivial to prove.
    \end{itemize}
    \begin{align*}
      \frac{f(y + he_i) - f(y)}{h} 
      &= \int_0^1 \sum_{i=1}^d [g_i(ty + the_i)(y_i + h \delta_{ij})] \, dt \\
      &= \int_0^1 g_i(ty + the_i) \, dt + \int_0^1 \sum_{j \ne i} \frac{[g_i(ty + the_i) - g_i(ty)]}{h} y_i \, dt \\
      &\xrightarrow{h \to 0} \int_0^1 g_i(ty) \, dt  + \text{is difficult ...}
    \end{align*}
    Lets take \(\phi \in C_c^\infty\), then:
    \begin{align*}
      T(\phi_{y}) - T(\phi)
      &= \int_0^1 \underbrace{\nabla T}_{(g_i)_{i=1}^d} (\phi_{ty}) y \, dt \\
      &= \int_0^1 \sum_{i=1}^d \left(\int_\Omega g_i(x) \smash{\underbrace{\phi_{ty} }_{\mathclap{= \phi(x - ty)}}} \, dx \right) \, dt \\
      &= \int_{\mathbb{R}^d} \left(\sum_{i=1}^d \int_0^1 g_i(x) \phi(x-ty) y_i \ dt\right) \, dx \\
      &= \int_{\mathbb{R}^d} \left(\sum_{i=1}^d \int_0^1 g_i(x + ty) \phi(x) y_i \, dt\right) \, dx \\
      &= \int_{\mathbb{R}^d} \left(\sum_i \int_0^1 g_i(x + ty) y_i \, dt \right) \phi(x) \, dx
    \end{align*}
    Integrating against \(\psi(y)\) with \(\psi \in C_c^\infty\):
    \begin{align*}
      &\int_{\mathbb{R}^d} T(\phi_y) \psi(y) \, dy - T(\phi) \int_{\mathbb{R}^d} \psi(y) \, dy  \\
      &\quad = \int_{\mathbb{R}^d} \left(\int_{\mathbb{R}^d} \sum_{i} \int_0^1 g_i (x + ty) y_i \psi(y) \, dt \, dy \right) \psi(x) \, dx
    \end{align*}
    \begin{align*}
      &\Rightarrow T(\phi \star \psi) - T(\phi)\int \psi = \dots \\
      &\Rightarrow \int_{\mathbb{R}^d} T(\psi_y) \phi(y) \, dy - T(\phi) \int \psi = \dots
    \end{align*}
    Take \(\psi \in C_c^\infty(\mathbb{R}^d)\) such that \(\int \psi = 1\). Then:
    \begin{align*}
      T(\phi)
      &= \int_{\mathbb{R}^d} \underbrace{T(\psi_x) - \left(\int_{\mathbb{R}^d} \sum_{i=1}^d \int_0^1 g_i(x + ty) y_i \psi(y) \, dt \, dy\right)}_{\smash{f(x)}} \phi(x) \, dx
    \end{align*}
    for all \(\phi \in C_c^\infty\), so \(T = f \in C(\Omega)\). Thus \(T = f \in C(\Omega)\) and \(\partial_{x_i} T = g_i \in C(\Omega)\). Then we need to prove that \(f \in C^1(\Omega)\) and \(\partial_{x_i} f = g_i\) (classical derivative). Since \(f \in W_{loc}^{1,1}\):
    \begin{align*}
      f(x + y) - f(x)
      &= \int_0^1 \sum_{i=1}^d g_i(x + ty)y_i \, dt \quad \forall x,y
    \end{align*}
    In particular:
    \begin{align*}
      \frac{f(x + h e_i) - f(x)}{h}
      &= \int_0^1 \frac{1}{h} \sum_{i=1}^d g_i (x + the_i) h \delta_{ij} \, dt \\
      &= \int_0^1 g_i(x + the_i) \, dt \xrightarrow{h \to 0} g_i(x)
    \end{align*}
    So we get \(\partial_{x_i} f(x) = g_i(x) \in C(\Omega)\) in the classical sense. So \(f \in C^1(\Omega)\).
  \end{proof}
  
  \begin{defn}[Sobolev Spaces]
    Let \(\Omega \subseteq \mathbb{R}^d\) be open. We define for \(1 \le p\le \infty\):
    \begin{align*}
      W^{1, p}(\Omega)
      &= \{f \in L^p(\Omega) \mid \partial_{x_i} f \in L^p(\Omega)\ \forall i=1, \dots, d \} \\
      W^{k, p}(\Omega)
      &= \{f \in L^p(\Omega) \mid D^\alpha f \in L^p(\Omega) \ \forall |\alpha|\le k \} \\
      W^{k, p}_{loc}(\Omega)
      &= \{f \in L_{loc}^p(\Omega) \mid D^\alpha f \in L_{loc}^p(\Omega) \ \forall |\alpha|\le k \}
    \end{align*}
  \end{defn}
  
  \begin{thm}[Approximation of \(W_{loc}^{1,p}(\Omega)\) by \(C^\infty(\Omega)\)]
    Let \(\Omega \subseteq \mathbb{R}^d\) be open, let \(f \in W_{loc}^{1,p}(\Omega)\). Then there exists \(\{f_n\} \subseteq C^\infty(\Omega)\) such that \(f_n \to f \) in \(W_{loc}^{1,p}(\Omega)\), i.e. for all \(K \subseteq \Omega\) compact: \(\|f_n - f\|_{L^p(K)} + \sum_{i=1}^d \| \partial_{x_i}(f_n - f)\|_{L^p(K)} \to 0\) .
  \end{thm}

  \begin{proof}
    Case \(\Omega = \mathbb{R}^d\): Take \(g \in C_c^\infty\), \(\int g = 1\), \(g_\epsilon(x) = \epsilon^{-d} g(\epsilon^{-1}x)\). Then \(g_\epsilon \star f \in C_c^\infty\). Since \(f \in L_{loc}^p(\Omega)\) we have \(g_\epsilon \star f \to f\) in \(L_{loc}^p\) as \(\epsilon \to 0\). Moreover \(\partial_{x_i} (g_\epsilon \star f) = (g_\epsilon \star \partial_{x_i} f)\xrightarrow{\epsilon \to 0} \partial_{x_i} f\) in \(L_{loc}^p\). Then we can take \(f_n = g_{\frac{1}{n} \star f}\).
  \end{proof}

  \begin{rem}
    In general, if we want to compute the distributional derivative \(D^\alpha f\), then we can find \(f_n \to f\) in \(D'(\Omega)\) and compute \(D^\alpha f_n\). Then \(D^\alpha f_n \to D^\alpha f\) in \(D^\alpha (\Omega)\). As an example we can compute \(\nabla |f|\) with \(f \in W_{loc}^{1,p}(\Omega)\).
    \begin{align*}
      (\nabla |f|)(x)
      &= \begin{cases}
        \nabla f(x) & f(x) > 0 \\
        - \nabla f(x) & f(x) < 0 \\
        0 & f(x) = 0
      \end{cases}
    \end{align*}
  \end{rem}
  
  \begin{thm}[Chain Rule]
    Let \(G \in C^1(\mathbb{R}^d)\) with \(|\nabla G|\) is bounded. Let \(f = (f_i)_{i=1}^d \subseteq W_{loc}^{1,p}(\Omega)\). Then \(x \mapsto G(f(x)) \in W_{loc}^{1,p}(\Omega)\) and
    \begin{align*}
      \partial_{x_i} G(f)
      &= \sum_{k=1}^d (\partial_k G)(f) \cdot \partial_{x_i} f_k \quad \text{ in } D'(\Omega)
    \end{align*}
    Moreover, if \(G(0) \in L^p(\Omega)\) (i.e. either \(|\Omega| < \infty\) or \(G(0) = 0\)), then if \(f = (f_i)_{i=1}^d \subseteq W^{1,p}(\Omega)\), then \(G(f) \in W^{1,p}(\Omega)\).
  \end{thm}

  \begin{proof}
    Since \(G \in C^1\) we have that \(G\) is bounded in any compact set. Moreover \(\|\nabla G\|_{L^\infty} < \infty\) implies:
    \[|G(f) - G(0)| \le \|\nabla G\|_{L^\infty} |f| \in L^p_{loc}\]
  
    So \(G(f) \in L_{loc}^p\). Let us compute \(\partial_{x_i} G(f)\). Let \(\{f^{(n)}\}_{n=1}^\infty \subseteq C^\infty\) such that \(f^{(n)} \to f\) in \(W_{loc}^{1,p}\), then: 
    \begin{align*}
      |G(f^{(n)}) - G(f)|
      \le \|\nabla G\|_{L^\infty} |f^{(n)} - f| \to 0 \text{ in  \(L_{loc}^p\)}
    \end{align*}
    So \(G(f^{(n)}) \to G(f)\) in \(L_{loc}^p\), thus \(\partial_{x_i} G(f^{(n)}) \to \partial_{x_i} G(f)\) in \(D'(\Omega)\). On the other hand, by the standard Chain-Rule for \(C^1\)-functions:
    \begin{align*}
      \partial_{x_i} G(f^{(k)})
      &= \sum_{k=1}^d \underbrace{\partial_k G(f^{(k)})}_{(\text{b.d.} \to \partial_k G(f))} \underbrace{\partial_i f_k^{(n)}}_{(\to \partial_i f_k \text{ in } L^p(\Omega))} \to \sum_{k=1}^d \partial_k G(f) \partial_i f_k \text{ in }L_{loc}^p(\Omega)
    \end{align*}
    Thus
    \begin{align*}
      \partial_{x_i} G(f) = \sum_{k=1}^d \underbrace{\partial_k G(f)}_{\in L^\infty} \underbrace{\partial_i f_k}_{\in L_{loc}^p} \in L_{loc}^p \text{ in } D'(\Omega)
    \end{align*}
    So \(G(f) \in W_{loc}^{1,p}(\Omega)\). Aussume that \(G(0) \in L^p(\Omega)\) (i.e. \(|\Omega| < \infty\) or \(G(0) = 0\)). If \(f \in W^{1,p}(\Omega)\), then \(G(f) \in W^{1,p}(\Omega)\) since
    \begin{align*}
      |G(f) - G(0) | \le \|\nabla G\|_{L^\infty} |f| \in L^p \Rightarrow G(f) \in L^p
    \end{align*}
    and 
    \begin{align*}
      \partial_{x_i} G(f) = \sum_k \underbrace{\partial_k G}_{\in L^\infty} \underbrace{\partial_i f_k}_{\in L^p} \in L^p \Rightarrow G(f) \in W^{1,p}(\Omega)
    \end{align*}
  \end{proof}
  
  \begin{thm}[Derivative of absolute value] Let \(\Omega \subseteq \mathbb{R}^d\) be open. Let \(f \in W^{1,p}(\Omega)\). 
    Then \(|f| \in W^{1,p}(\Omega)\)
    and  if \(f\) is real-valued:
    \begin{align*}
      (\nabla |f|) (x) =
      \begin{cases}
        \nabla f(x) & f(x) > 0 \\
        -\nabla f(x) & f(x) < 0 \\
        0 & f(x) = 0
      \end{cases}
    \end{align*}
  \end{thm}

  \begin{proof}
    Exercise. Hint: Use the Chain-Rule for \(G_\epsilon(x) = \sqrt{\epsilon^2 + x^2} - \epsilon \to |x|\) as  \(\epsilon \to 0\)
  \end{proof}

  \section{Distribution vs. measures}
  Let \(\mu\) be a Borel measure in \(\mathbb{R}^d\) s.t. \(\mu(K) < \infty\) for all compact \(K \subseteq \mathbb{R}^d\). Then define 
  \begin{align*}
    T: \ D(\mathbb{R}^d) &\longrightarrow \mathbb{C} \\
    \phi &\longmapsto \int_{\mathbb{R}^d} \phi(x) \, d\mu(x) \quad \forall \phi \in C_c^\infty
  \end{align*}
  \(\leadsto\) T is a distribution since if \(\phi_n \to \phi\) in \(D(\Omega)\), then
  \begin{align*}
    |T(\phi_n) - T(\phi)| 
    \le \int_{\mathbb{R}^d} |\phi_n - \phi| \, d\mu(x) \le \|\phi_n - \phi\|_{L^\infty} \left(\int_{K} \, d \mu \right) \xrightarrow{n \to \infty} 0
  \end{align*}
  
  \begin{eg}
    \(\partial_0\) in \(D'(\mathbb{R}^d)\) is a Borel probability measure.
  \end{eg}  
  
  \begin{thm}[Positive distributions are measures] 
    Let \(\Omega \subseteq \mathbb{R}^d\) be open, let \(T \in D'(\Omega)\). Assume \(T \ge 0\), i.e. \(T(\phi) \ge 0\) for all \(\phi \in D(\Omega)\) satisfying \(\phi(x) \ge 0\) for all \(x\). Then there is a Borel positive measure \(\mu\) on \(\Omega\) such that \(\mu(K) < \infty\) for all \(K \subseteq \Omega\) compact and:
    \[T(\phi) = \int_\Omega \phi(x) \, d\mu (x) \quad \forall \phi \in D^(\Omega)\]
  \end{thm}

  \begin{proof}See Lieb-Loss Analysis. Sketch: If \(O \subseteq \mathbb{R}^d\) is open, then 
    \[\mu(O) = \sup \{T(\phi) \mid \phi \in D(\Omega), 0 \le \phi \le 1, \supp \phi \subseteq O\}\]
    For all \(A \subseteq \Omega\) (not necessarily open),
    \[\mu(A) = \inf \{\mu(O) \mid O \text{ open}, A \subseteq O\}\]
    The mapping \(\mu: 2^\Omega \to [0, \infty]\) is an outer measure, i.e.
    \begin{enumerate}
      \item \(\mu(\emptyset) = 0\)
      \item \(\mu(A) \le \mu(B)\) if \(A \subseteq B\)
      \item \(\mu \left(\bigcup_{i=1}^\infty A_i\right) \le \sum_{i=1}^\infty \mu(A_i)\)
    \end{enumerate}
    From the outer measure we can find a \(\sigma\)-algebra \(\Sigma\) and \(\mu\) is a measure on \(\Omega\) s.t. \(E\) is measurable iff \[\mu(E) = \mu(E \cap A) + \mu(E \cap A^\complement).\]
    So all open sets are measurable, thus outer regularity (by def \(\mu(A) = \inf\{\mu(O) \mid O \text{ open } \supseteq A\)), so inner regularity \(\mu(A) = \sup\{\mu(K) \mid K \text{ compact} \subseteq A\}\). 
  \end{proof}

  \begin{ex}[E 4.1]
    Prove that if \(T_n \to T\) in \(D'(\mathbb{R}^d)\), then \(D^\alpha T_n  \to D^\alpha T\) in \(D^\alpha(\mathbb{R}^d)\) for all \(\alpha \in \mathbb{N}^d\).
  \end{ex}

  \begin{ex}[E 4.2]
  \end{ex}

  \begin{ex}[E 4.3]
    \(f \in L^1(\mathbb{R}^d)\), \(\int f = 1\) \(f_\epsilon(x) = \epsilon^{-d} f(\epsilon^{-1}x)\). Then \(f_\epsilon \to \delta_0\) in \(D'(\mathbb{R}^d)\). 
  \end{ex}

  \begin{ex}[E 4.4]
    Let \(\{f_n\} \subseteq L^1\), \(\supp f \subseteq B(0,1), f_n \to f\) in \(L^1\). Prove for all \(g \in C_c^\infty\) that \(f_n \star g \to f \star g\) in \(D(\mathbb{R}^d)\).
  \end{ex}

  \begin{proof}[Solution]
    Since \(f_n \in L^1\), \(\supp f \subseteq B(0,1)\) and \(g \in C_c^\infty\) we have \(f_n \star g \in C_c^\infty\) and \[\supp(f_n \star g) \subseteq (\supp g) + \overline{B(0,1)} = K.\] Since \(f_n \to f\) in \(L^1\) there is a subsequence \(f_{n_k} \to f\) almost everywhere, so \(f \supp\) in \(\overline{B(0,1)}\), so \(f \star g \in C_c^\infty\), \(\supp(f \star g) \subseteq K\). We have:
    \begin{align*}
      |f_n \star g(x) - f \star g(x)| 
      &= \left|\int_{\mathbb{R}^d} (f_n(y)-f(y))g(x-y) \, dy \right| \\
      &\le \int_{\mathbb{R}^d} |f_n(y) - f(y)| |g(x-y)| \, dy \\
      &\le \|g\|_{L^\infty} ||f_n - f\|_{L^1} 
      \xrightarrow{n \to \infty} 0
    \end{align*}
    thus \(\|f_n \star g - f \star g\|_{L^\infty} \to 0\). Similary:
    \begin{align*}
      \|D^\alpha(f_n \star g) - D^\alpha (f\star g)\|_{L^\infty} 
      &= \|f_n \star \underbrace{(D^\alpha g)}_{\in C_c^\infty} - f \star (D^\alpha g)\|_{L^\infty} \xrightarrow{n \to \infty} 0
    \end{align*}
    for all \(\alpha \in \mathbb{N}^d\), so \(f_n \star g \to f \star g\) in \(D(\mathbb{R}^d)\).
  \end{proof}

  \begin{ex}[E 4.5]
    Compute distributional derivatives \(f', f''\) of \(f(x) = x|x-1|\).
  \end{ex}

  \begin{proof}[Solution]
    We prove \(f'(x) = g(x) \coloneqq \begin{cases}
      2x - 1 & x > 1 \\  1- 2x & x < 1
    \end{cases}\). Take \(\phi \in C_c^\infty(\mathbb{R}^d)\).
    \begin{align*}
      -f'(\phi) 
      &= - \int_{\mathbb{R}^d} f \phi' \, dy \\
      &= - \int_{-\infty}^1 f \phi' \, dy - \int_1^\infty f \phi' \, dy  \\
      &= [f \phi]_{- \infty}^1 - \int_{- \infty}^1 f' \phi \, dy + [f \phi]_1^\infty - \int_1^\infty f' \phi \, dy \\
      &= [f \phi]_{- \infty}^1 - \int_{- \infty}^1 g \phi \, dy + [f \phi]_1^\infty - \int_1^\infty g \phi \, dy \\
      &= f(1-) \phi(1) - f(1+)\phi(1) - \int_{\mathbb{R}^d} g \phi \, dy \\
      &= 0  - \int_{\mathbb{R}^d} g \phi \, dy \\
    \end{align*}
    Now we compute \(f'' = g'\). Take \(\phi \in C_c^\infty(\mathbb{R}^d)\):
    \begin{align*}
      -(g')(\phi) 
      &= \int_{\mathbb{R}^d} g\phi' \, dy \\
      &= \int_{- \infty}^1  g\phi' \, dy + \int_1^\infty  g\phi' \, dy \\
      &= [g(1-) - g(1+)] \phi(1) - \int_{-\infty}^1g' \phi \, dy - \int_1^\infty g' \phi \, dy \\
      &= [g(1-) - g(1+)] \phi(1) - \int_{-\infty}^1(-2) \phi \, dy - \int_1^\infty 2 \phi \, dy \\
      &= - 2 \phi(1) + \int_{-\infty}^\infty [2 \mathbb{1}_{(-\infty, 1)}(x) - 2 \mathbb{1}_{(1,  \infty)} (x)]\phi(x) \, dx \\
      &= -2 \delta_1(\phi) + \int_{-\infty}^\infty [2 \mathbb{1}_{(-\infty, 1)}(x) - 2 \mathbb{1}_{(1,  \infty)} (x)]\phi(x) \, dx \\
      \Rightarrow g' &= \underbrace{2 \delta_1}_{\notin L_{loc}^1} - \underbrace{2 \mathbb{1}_{(-\infty, 1)} + 2 \mathbb{1}_{(1, \infty)}}_{\int L_{loc}^1} \qedhere
    \end{align*}
  \end{proof}

  \chapter{Weak Solutions and Regularity}
  \begin{defn}
    Consider the linear PDE:
    \[\sum_{\alpha} c_\alpha D^\alpha u(x) = F(x), \quad c_\alpha \text{ constant}, F \text{ given}\]
    A function \(u\) is called a weak solution (a distributional solution) if 
    \[\sum_\alpha c_\alpha D^\alpha u = F \quad \text{in } D'(\Omega).\]
    Namely,
    \[\sum_\alpha (-1)^{|\alpha|} c_\alpha \int_\Omega u D^\alpha \phi = \int_\Omega F \phi, \quad \forall \phi \in D(\Omega)\]
  \end{defn}

  Regularity: Given some condition on the data \(F\), what can we say about the smoothness of \(u\)? Can we say that the equation holds in the classical sense? We derived \(G\) (the solution of the Laplace Equation) before in two ways:
  \begin{enumerate}
    \item \(\Delta G(x) = 0\) for all \(x \ne 0\), assuming \(G(x) = G(|x|)\) and \(d \ge 2\)
    \item \(\hat G(k) = \frac{1}{|2\pi k|^2}\) for \(d \ge 3\)
  \end{enumerate}
  
  \begin{thm}
    For all \(d \ge 1\) we have \(G \in L_{loc}^1(\mathbb{R^d})\) and \(- \Delta G = \delta_0\) in \(D'(\mathbb{R}^d)\).
  \end{thm}

  \begin{proof}
    Take \(\phi \in D(\mathbb{R}^d)\). Then:
    \begin{align*}
      (- \Delta G_y)(\phi) 
      &= G_y(-\Delta \phi)
      = \int_{\mathbb{R}^d} G_y(x) (-\Delta \phi)(x) \, dx \\
      &= \int_{\mathbb{R}^d} G(y-x) (-\Delta \phi)(x) \, dx \\
      &= [G \star (-\Delta \phi)](y) = (-\Delta)(G \star \phi)(y)
    \end{align*}
    Recall for all \(f \in C^2\), \(-\Delta (G \star f) = f\) pointwise. So we can conclude \(-\Delta G_y = \delta_y\) in \(D'(\mathbb{R}^d)\).
  \end{proof}
  
  \begin{rem}
    In \(d = 1\), \(G(x) = - \frac{1}{2}|x|\), so \(-G'(x) = \sgn(x)/2\), so \(-G''(x) = \delta_0\).
  \end{rem}

  \begin{rem}
    Formally: 
    \begin{align*}
      -\Delta(G_y \star \phi) = (-\Delta G_y) \star \phi(x) = (\delta_0 \star \phi)(x) = \int \delta_0(y) \phi(x - y) \, dy = \delta_0 (\phi(x-\bullet))
    \end{align*}
  \end{rem}

  \begin{thm}[Poisson's equation with \(L_{loc}^1\) data]\label{PoissonsequationwithL1locdata}
    Let \(f \in L_{loc}^1(\mathbb{R}^d)\) s.t. \(\omega_d f \in L^1(\mathbb{R}^d)\) where 
    \[\omega_d(x) = \begin{cases}
      1 + |x| & d = 1 \\ \log(1 + |x|) & d=2 \\ \frac{1}{1 + |x|^{d-2}} & d \ge 3,
    \end{cases}\]
    then \(u(x) = (G\star f)(x) \in L_{loc}^1(\mathbb{R}^d)\). Moreover \(-\Delta u = f\) in \(D'(\mathbb{R}^d)\). In fact, \(u \in W_{loc}^{1,1}(\mathbb{R}^d)\) and:
    \[\partial_{x_i} u(x) = (\partial_{x_i} G) \star f(x) = \int_{\mathbb{R}^d} (\partial_{x_i} G)(x-y)f(y) \, dy\]
  \end{thm}

  \begin{rem}
    We can also replace \(\mathbb{R}^d\) by \(\Omega\) and get \(- \Delta u = f\) in \(D'(\Omega)\). 
  \end{rem}

  \begin{proof}[Proof of Theorem \ref{PoissonsequationwithL1locdata}]
    First we check that \(u \in L_{loc}^1\). Take any Ball \(B(0, R) \subseteq \mathbb{R}^d\), prove \(\int_B |u| \, dy < \infty\). We have 
    \begin{align*}
      \int_B |u| \, dy 
      &= \int_B \left| \int_{\mathbb{R}^d} G(x-y) f(y) \, dy \right| \, dx \\
      &\le \int_B \int_{\mathbb{R}^d} |G(x-y)||f(y)| \, dy \, dx \\
      &= \int_{\mathbb{R}^d} \left(\int_B |G(x-y) \, dx\right)|f(y)| \, dy
    \end{align*}
    If \(y \notin B = B(0,R)\), then by Newtons's theorem (Mean-value theorem):
    \begin{align*}
      \int_{B(0, R)} |G(x-y)| \, dx
      &= |B(0, R)| |G(y)|
      \le C|B| \omega_d(y)
    \end{align*}
    If \(y \in B\), then \(|y| \le R\), so \(|x-y| \le 2 R\) if \(x \in B\).
    \begin{align*}
      \int_{B(0,R)}|G(x-y)| \, dx
      &\le \int_{|x-y|\le 2 R} |G(x-y)| \, dx
      = \int_{|z| \le 2 R} |G(z)| \, dz 
      \le c_R
    \end{align*}
    as \(G \in L_{loc}^1\). Thus
    \begin{align*}
      \int_B |u| \, dy 
      \le c_B \int_{|y|\ge R} \omega_d (y) |f(y)|\, dy + c_B \int_{|y|\le R} |f(y)| \, dy < \infty
    \end{align*}
    Let us prove \(- \Delta u = f\) in \(D'(\mathbb{R}^d)\). Take \(\phi \in D(\mathbb{R}^d)\). Then:
    \begin{align*}
      (-\Delta u)(\phi)
      &= u(-\Delta \phi)\\
      &= \int_{\mathbb{R}^d} u(x) (-\Delta \phi)(x) \, dx \\
      &= \int_{\mathbb{R}^d}\int_{\mathbb{R}^d} G(x-y) f(y) (-\Delta \phi)(x) \, dx \, dy\\
      &= \int_{\mathbb{R}^d}\int_{\mathbb{R}^d} G(y-x) f(y) (-\Delta \phi)(x) \, dx \, dy \\
      &= \int_{\mathbb{R}^d} [G \star (-\Delta \phi)](y) f(y) \, dy \\
      &= \int_{\mathbb{R}^d} -\Delta(G\star \phi)(y) f(y) \, dy \\
      &= \int_{\mathbb{R}^d} \phi(y) f(y) \, dy
    \end{align*}
    So \(-\Delta u = f\) in \(D'(\mathbb{R}^d)\). We check that \(\partial_i G \star f \in L_{loc}^1(\mathbb{R}^d)\). Note that 
    \[|\partial_i G(x)| \le c \frac{1}{|x|^{d-1}} \in L_{loc}^1(\mathbb{R}^d)\]
    and
    \[\int_{B(0,R)} |\partial_i G(x-y)| \, dx \le \begin{cases}
      C_r \omega_d(y) & |y| \ge R \\ C_r & |y| \le R
    \end{cases}\]
    So \(\int_{B(0,R)} |(\partial_i G\star f)|(y) < \infty\) for all \(R > 0\). For all \(\phi \in D(\mathbb{R}^d)\): 
    \begin{align*}
      -(\partial_i u)(\phi) 
      &= u(\partial_i \phi)
      = \int_{\mathbb{R}^d} u(x) \partial_i \phi(x) \, dx \\
      &= \int_{\mathbb{R}^d} \int_{\mathbb{R}^d} G(x-y) f(y) \partial_i \phi(x) \, dx \, dy \\
      &= \int_{\mathbb{R}^d} \int_{\mathbb{R}^d} G(y-x) f(y) \partial_i \phi(x) \, dx \, dy \\
      &= \int_{\mathbb{R}^d} (G \star \partial_i^y \phi)(y) f(y) \, dy \\
      &= \int_{\mathbb{R}^d}(\partial_i^y G \star \phi)(y) f(y) \, dy \\
      &= \int_{\mathbb{R}^d} \int_{\mathbb{R}^d} \partial_i^y G(y-x) f(y) \phi(x) \, dx \, dy \\
      &= \int_{\mathbb{R}^d} \int_{\mathbb{R}^d} -(\partial_i G)(x-y) f(y) \phi(x) \, dx \, dy \\
      &= - \int_{\mathbb{R}^d} (\partial_i G \star f)(x) \phi(x) \, dx
    \end{align*}
    So \(\partial_i u = \partial_i G \star f \in L_{loc}^1(\mathbb{R}^d)\). Thus \(u \in L_{loc}^1\), \(\partial_i u \in L_{loc}^1\) for all \(i\). So \(u \in W_{loc}^{1,1}(\mathbb{R}^d)\).
  \end{proof}

  Regularity: We consider the Laplace Equation \(\Delta u = 0\) in \(\mathbb{R}^d\).

  
  \begin{lem}[Weyl]\label{weyl}
    If \(\Omega \subseteq \mathbb{R}^d\) open and \(T \in D'(\Omega)\) s.t. \(\Delta T = 0\) in \(D'(\Omega)\), then: \(T = f \in C^\infty(\Omega)\) and \(f\) is a harmonic function.
  \end{lem}

  \begin{proof}
    (\(\Omega = \mathbb{R}^d\)). Take \(\phi \in C_c^\infty\), then \(y \mapsto T(\phi_y) = T(\phi(-y))\) is \(C^\infty\) and \(\Delta_y T(\phi_y) = T((\Delta \phi)_y) = (\Delta T)(\phi_y)=0\). Take \(g \in C_c^\infty\), \(g\) is radial. Then:
    \begin{align*}
      \int_{\mathbb{R}^d} T(\phi_y) g(y) \, dy 
      \overset{\text{(exercise)}}{=} \int_{\mathbb{R}^d} T(\phi)g(y) \, dy 
      = T(\phi) \left(\int_{\mathbb{R}^d} g \, dy \right)
    \end{align*}
  
  
    \begin{ex}
      Let \(f \in C^\infty(\mathbb{R}^d)\) be a harmonic function and \(g \in C_c^\infty\), \(g\) is radial. Then:
      \begin{align*}
        \int_{\mathbb{R}^d} f(x) g(x) \, dx = f(0) \left(\int_{\mathbb{R}^d} g(x) \, dx\right)
      \end{align*}
    \end{ex}

    On the other hand:
    \begin{align*}
      \int_{\mathbb{R}^d} T(\phi_y) g(y) \, dy = T(\phi \star g) = T(g \star \phi) = \int_{\mathbb{R}^d} T(g_y) \phi(y)\, dy
    \end{align*}

    Take \(\int_{\mathbb{R}^d} g \, dy = 1\), then:
    \begin{align*}
      T(\phi) 
      &= \int_{\mathbb{R}^d} T(g_y) \phi(y) \, dy 
    \end{align*}
    For all \(\phi \in C_c^\infty\). Then \(T = T(g_y) \in C^\infty\)
  \end{proof}

  Now lets regard the Poisson Equation \(- \Delta u = f\) in \(D'(\mathbb{R}^d)\).

  \begin{rem}
    Any solution has the form \(u = G \star g + h\) where \(\Delta h = 0\) in \(D'(\mathbb{R}^d)\). By Weyls Lemma (\ref{weyl}), \(h \in C^\infty\), then we only need to consider the regularity of \(G \star f\).
  \end{rem}

  \begin{rem}
    The regularity is a \emph{local question}, namely if we write \[f = f_1 + f_2 = f \phi + f(1-\phi),\] where \(\phi = 1\) in a ball \(B\) and \(\phi \in C_c^\infty\).
  \end{rem}
  Then \(G \star f = G \star f_1 + G \star f_2\). Here \(f_2 = f(1 - \phi) = 0\) in \(B\). With Weyls Lemma (\ref{weyl}), \(G \star f_2 \in C^\infty\).

  \begin{thm}[Low Regularity of Poisson Equation]\label{Low Regularity for Poisson Equation}
    Lef \(f \in L^p(\mathbb{R}^d)\) and compactly supported. Then
    \begin{enumerate}[label=\alph*)]
      \item If \(p \ge 1\), then \begin{itemize}
        \item \(G \star f \in C^1(\mathbb{R}^d)\) if \(d = 1\).
        \item \(G \star f \in L_{loc}^q(\mathbb{R}^d)\) for any \(q < \infty\) if \(d = 2\).
        \item \(G \star f \in L_{loc}^q(\mathbb{R}^d)\) for \(q < \frac{d}{d-2}\) if \(d \ge 3\).
      \end{itemize}
      \item If \(\frac{d}{2} < p \le d\), then \(G \star f \in C_{loc}^{0, \alpha}(\mathbb{R}^d)\) for all \(0 < \alpha < 2 - \frac{d}{p}\), i.e.
      \begin{align*}
        |(G \star f)(x) - (G \star f)(y) | \le C_k|x-y|^\alpha \quad \forall x,y \in K
      \end{align*}
      with \(K\) compact in \(\mathbb{R}^d\).
      \item If \(p > d\), then \(G \star f \in C_{loc}^{1, \alpha} (\mathbb{R}^d)\) for all \(0 < \alpha < 1 - \frac{d}{p}\).
    \end{enumerate}
    where \(G\) is den fundamental solution of the laplace equation.
  \end{thm}
  
  \begin{eg}
    Let \(r = |x|\)
    \begin{align*}
      u(x) 
      &= \omega(r) = \log(|\log(r)|)
    \end{align*}
    if \(0 < r < \frac{1}{2}\), so \(u\) is well-defined in \(B = B(0, \frac{1}{2})\). We conclude:
    \begin{align*}
      - \Delta_{\mathbb{R}^3} u(x) = - \omega''(r) - \frac{2 \omega'(r)}{r} = f(x) \in L^{\frac{3}{2}(B)}
    \end{align*}
    But the Theorem (b) tells us that if \(f \in L^{\frac{3}{2}}\) then \(u\) is continuous but \(u \notin C(B)\).
  \end{eg}

  \begin{proof}[Proof of theorem \ref{Low Regularity for Poisson Equation}]
    \begin{enumerate}[label=\alph*)]
      \item (\(p=1\)) Why is \(G \star f \in L_{loc}^q\)? Recall from the proof of Youngs inequality: \begin{align*}
        |(G\star f)(x)| 
        &= \left| \int_{\mathbb{R}^d} G(x-y) f(y) \, dy \right| \\
        (\text{Hölder}) \quad &= \left(\int_{\mathbb{R}^d} |G(x-y)|^q |f(y)| \, yd\right)^{\frac{1}{q}}\left(\int_{\mathbb{R}^d} |f(y)| \, dy\right)^{\frac{1}{q'}}
      \end{align*}
      Where \(\frac{1}{q} + \frac{1}{q'} = 1\). Then:
      \begin{align*}
        |(G \star f)(x)|^q 
        &\le C \int_{\mathbb{R}^d} |G(x-y)|^q |f(y)| \, dy
      \end{align*}
      For any Ball \(B = B(0, R) \subseteq \mathbb{R}^d\):
      \begin{align*}
        \int_B |G \star f(x)|^q \, dx
        &\le C \int_B \left(\int_{\mathbb{R}^d} |G(x-y)|^q |f(y)| \, dy\right) \, dx \\
        &= C \int_{\mathbb{R}^d} \left(\int_B |G(x-y)|^q \, dx \right) |f(y)| \, dy
      \end{align*}
      \(G(x) \sim \frac{1}{|x|^{d-2}} \leadsto |G|^{q} = \frac{1}{|x|^{(d-2)q}} \in L_{loc}^1(\mathbb{R}^d)\) if \((d-2)q < 2 \Leftrightarrow q < \frac{d}{d-2}\). Here, \(y \in \supp f \), so \(|y| \le R_1\), then \(|x-y| \le R + R\) if \(|x| \le R\). With \(y \in \supp f\), this implies:
      \begin{align*}
        \int_{B(0,R)}|G(x-y)|^q \, dx \le \int_{|z| \le R + R_1} |G(z)|^q \, dz < \infty
      \end{align*}

      \item \begin{align*}
        (G \star f)(x) - (G \star f)(y) 
        &= \int_{\mathbb{R}^d} (G(x-z)-G(y-z)) f(z) \, dz
      \end{align*}
      So \begin{align*}
        |G \star f(x) - (G\star f)(y)| 
        &\le C \int_{\mathbb{R}^d} \left| \frac{1}{|x-z|^{d-2}} - \frac{1}{|y-z|^{d-2}}\right| |f(z)| \, dz
        \end{align*}
        for all \(x, y \in \mathbb{R}^d\):
        \begin{align*}
          \left| \frac{1}{|x|^{d-2}} - \frac{1}{|y|^{d-2}} \right|
          &= \left| \left( \frac{1}{|x|} - \frac{1}{|y|}\right) \left(\frac{1}{|x|^{d-3}} + \cdots + \frac{1}{|y|^{d-3}}\right) \right| \\
          &\le C \frac{||x|-|y||}{|x||y|} \max \left(\frac{1}{|x|^{d-3}}, \frac{1}{|y|^{d-3}}\right) \\
          &= C \frac{|x-y|}{|x||y|} \max \left(\frac{1}{|x|^{d-3}}, \frac{1}{|y|^{d-3}}\right) \\
          &\le C \max(|x|, |y|)^{1-\alpha} \frac{|x-y|^\alpha}{|x||y|} \max \left(\frac{1}{|x|^{d-3}}, \frac{1}{|y|^{d-3}}\right)
        \end{align*}
        as
        \begin{align*}
          ||x|-|y|| 
          &\le \min \left(|x-y|, \max(|x|, |y|)\right)
          \le |x-y|^\alpha \max(|x|, |y|)^{1-\alpha}
        \end{align*}
        Thus, for all \(x,y \in \mathbb{R}^d\):
        \begin{align*}
          \left| \frac{1}{|x|^{d-2}}- \frac{1}{|y|^{d-2}}\right| 
          &\le C |x-y|^\alpha \frac{\max(|x|, |y|)^{1-\alpha}}{|x||y|} \max \left(\frac{1}{|x|^{d-3}}, \frac{1}{|y|^{d-3}}\right) \\
          &\le C |x-y|^\alpha \max \left(\frac{1}{|x|^{d-2+\alpha}}, \frac{1}{|y|^{d-2+\alpha}}\right)
        \end{align*}
        So we get \begin{align*}
          \left| \frac{1}{|x-y|^{d-2}}- \frac{1}{|y-z|^{d-2}}\right|
          &\le C |x-y|^\alpha \max \left(\frac{1}{|x-z|^{d-2+\alpha}}, \frac{1}{|y-z|^{d-2+\alpha}}\right)
        \end{align*}
        Therefore:
        \begin{align*}
          &|G \star f(x) - G \star f(y)| \\
          &\quad\le C \int_{\mathbb{R}^d}|x-y|^\alpha \max \left(\frac{1}{|x-z|^{d-2+\alpha}}, \frac{1}{|y-z|^{d-2+\alpha}}\right) |f(z)| \, dz \\
          &\quad\le C |x-y|^\alpha \left(\sup_{\xi \in \mathbb{R^d}} \int_{\mathbb{R}^d} \frac{1}{|\xi - z|^{d-2+\alpha}} |f(z)| \, dz\right)
        \end{align*}
        Claim: If \(f \in L^p(\mathbb{R}^d)\) is compactly supported, \(d \ge p > \frac{d}{2}\), then: 
        \begin{align*}
          \sup_{\xi \in \mathbb{R}^d} \int_{\mathbb{R}^d} \frac{1}{|\xi - z|^{d - 2 + d}}|f(z)| \, dz < \infty
        \end{align*}
        for all \(0 < \alpha < 2 - \frac{d}{p}\). Assume \(\supp f \subseteq \overline{B(0, R_1)}\). Consider 2 cases:
        \begin{itemize}
          \item If \(|\xi| > 2 R_1\), then: \(|\xi - z| \ge R_1\) for all \(z \in B(0, R_1)\). Hence: \begin{align*}
            \int_{\mathbb{R}^d} \frac{1}{|\xi -z|^{d-2+\alpha}}|f(z)| \, dz
            &\le \frac{1}{R_1^{d-2+\alpha}} \|f\|_{L^1} < \infty
          \end{align*}
          \item If \(|\xi| \le 2 R_1\), then: \(|\xi - z| \le 3 R_1\) for all \(z \in B(0, R_1)\):
          \begin{align*}
            \int_{\mathbb{R}^d} \frac{1}{|\xi - z|^{d-2+\alpha}} |f(z)| \, dz
            &\le \int_{|\xi - z| \le 3 R_1} \frac{1}{|\xi - z|^{d - 2 + \alpha}} |f(z)| \, dz \\
            (\text{Hölder}), \left(\frac{1}{p} + \frac{1}{q} = 1\right) \quad &\le \left(\int_{\mathbb{R}^d}|f(z)|^p \, dz\right)^{\frac{1}{p}}\\
            &\quad \cdot \left(\int_{|\xi - z| \le 3 R_1} \frac{1}{|\xi - z|^{(d - 2 + \alpha)q}}\right)^{\frac{1}{q}} \\
            &= \|f\|_{L^p} \left(\int_{|z| \le 3 R_1} \frac{1}{|z|^{(d-2+\alpha)q}} \, dz\right)^{\frac{1}{q}} < \infty
          \end{align*}
        \end{itemize}
        \item (\(d \ge 3\)) We already know: \begin{align*}
          \partial_i(G \star f) =(\partial_i G \star f) \in L_{loc}^1(\mathbb{R}^d)
        \end{align*}
        as \(\omega_d f \in L^1(\mathbb{R}^d)\). We claim that \(\partial_i G \star f \in C^{0, \alpha}(\mathbb{R}^d)\). So \(G \star f \in C^{1, \alpha}(\mathbb{R}^d)\) by the equivalence between the classical and the distributional derivatives. Exercise. Hint:
        \begin{align*}
          |\partial_i G\star f(x) - \partial_i G \star f(y)|
          &\le \int_{\mathbb{R}^d} |\partial_i G(x-z) - \partial_i G(y-z)||f(z)| \, dz,
        \end{align*}
        \(\partial_i G(x) = \frac{-x_i}{d|B_1||x|^d}\). \(\leadsto\) Need to estimate \(|\partial_i G(x) - \partial_i G(y)| \le C|x-y|^\alpha\). \qedhere
    \end{enumerate}
  \end{proof}

  \begin{thm}[High regularity for Poisson's equation]\label{High regularity for Poisson's equation}
    Let \(f \in C^{0, \alpha}(\mathbb{R}^d)\), \(0 < \alpha < 1\) be compactly supported. Then \(G \star f \in C^{2, \alpha}(\mathbb{R}^d)\).
  \end{thm}

  \begin{rem}
    \((- \Delta u = f)\) and \(f \in C(\mathbb{R}^d)\) does not imply that \(u \in C^2(\mathbb{R}^d)\). (exercise)
  \end{rem}

  \begin{rem}
    If \(f \in C^{k, \alpha}(\mathbb{R}^d)\), \(k \in \{0, 1, \dots\}\), \(0 < \alpha < 1\) is compactly supported, then \(G \star f \in C^{k + 2, \alpha}(\mathbb{R}^d)\). This more general statement is a consequence of the theorem since
    \[D^\beta(G \star f) = G \star \underbrace{(D^\beta f)}_{\in C^{0, \alpha}}\]
    for all \(\beta = (\beta_1, \dots, \beta_d)\), \(|\beta| \le k\).
  \end{rem}

  \begin{proof}[Proof of theorem \ref{High regularity for Poisson's equation}]
    Since \(f \in L^p\) for all \(p \le \infty\) by the low regularity (\ref{Low Regularity for Poisson Equation}) we have \(G \star f \in C^{1, \alpha}\) and \(\partial_i (G \star f) = \partial_i G \star f\)
    in the classical sense. We will compute the distributional derivatives \(\partial_i \partial_j (G \star f)\) and prove that they are Hölder continuous. Compute \(\partial_j \partial_i (G \star f)\): For all \(\phi \in C_c^\infty(\mathbb{R}^d)\) we have
    \begin{align*}
      -(\partial_j \partial_i G \star f)(\phi)
      &= (\underbrace{\partial_i(G \star f)}_{\in C})(\partial_j \phi) \\
      &= \int_{\mathbb{R}^d}((\partial_i G)\star f)(x) \partial_j \phi(x) \, dx \\
      &= \int_{\mathbb{R}^d}\int_{\mathbb{R}^d} \partial_i G(x-y) f(y) \partial_j \phi(x) \, dx \, dy \\
      &= \int_{\mathbb{R}^d} f(y) \left[\int_{\mathbb{R}^d} \partial_i G(x-y)\partial \phi(x) \, x\right] \, dy \\
      &\overset{?}{=} \int_{\mathbb{R}^d} \square \phi(y) \, dy
    \end{align*}
    Recall: \(\partial_i G(x) = \frac{-x_i}{d|B_1||x|^d}, \partial_i \partial_j G(x) = \frac{1}{|B_1|}\left[\frac{x_i x_j}{|x|^2}- \frac{\delta_{ij}}{d}\right] \frac{1}{|x|^d}\). We have:
    \begin{align*}
      \int_{\mathbb{R}^d} \partial_i G(x-y) \partial_j \phi(x) \, dx
      &= \lim_{\epsilon \to 0^+}\int_{|x-y| \ge \epsilon} \partial_i G(x-y) \partial_j \phi(x) \, dx
    \end{align*}
    By dominated convergence we have \(|\partial_i G(x-y) \partial_j \phi(x)| \in L^1(dx)\).
    By the Gauss-Green-Theorem (\ref{gauss-green}) for all \(\epsilon > 0\):
    \begin{align*}
      &\int_{|x-y| \ge \epsilon} \partial_i G(x-y) \partial_j \phi(x) \, dx \\
      &\quad= \int_{\partial B(y, \epsilon)}\partial_i G(x-y) \phi(x) \omega_j \, dS(x) - \int_{|x-y| \ge \epsilon} \partial_j \partial_i G(x-y) \phi(x) \, dx
    \end{align*}
    Where \(\omega = \frac{x-y}{|x-y|}\). For the boundary term:
    \begin{align*}
      - \int_{\partial B(y, \epsilon)} \partial_i G(x-y) \phi(x) \omega_j \, dS(x)
      &= - \epsilon^{d-1} \int_{\partial B(0, 1)} \partial_i G(\epsilon \omega) \phi(y + \epsilon \omega) \omega_j \, d\omega \\
      (\star) \quad &= \int_{\partial B(0, 1)} \frac{1}{d|B_1|} \omega_i \omega_j \phi(y + \epsilon \omega) \, d \omega \\
      &\xrightarrow{\epsilon \to 0} \int_{\partial B(0, 1)} \frac{1}{d|B_1|} \omega_i \omega_j \phi(y) \, d\omega \\
      &= \frac{1}{d} \delta_{i,j} \phi(y)
    \end{align*}
    \((\star)\) \ \(\partial_i G(x) = \frac{- x_i}{d|B_1||x|^d}\)
    , so \(\partial_i G(\epsilon \omega) = - \frac{- \omega_i}{d |B_1|} \frac{1}{\epsilon^{d-1}}.\) for all \(|\omega| = 1\). \\
    
    Now we split:
    \begin{align*}
      &- \int_{|x-y| \ge \epsilon} \partial_i \partial_j G(x-y) \phi(x) \, dx \\
      &\quad= - \int_{|x-y| \ge 1}  \partial_i \partial_j G(x-y) \phi(x) \, dx - \int_{1 \ge |x-y| \ge \epsilon}  \partial_i \partial_j G(x-y) \phi(x) \, dx
    \end{align*}
    The key observation is: \(\int_{\partial B(0, r)} \partial_i \partial_j G(x) \, dx = 0\) since \[\partial_i \partial_j G(x) = \frac{1}{|B_1|}\left(\omega_i \omega_j - \frac{\partial_{ij}}{d}\right) \frac{1}{|x|^d},\] \(\omega = \frac{x}{|x|}\). For example if \(i = 1, j = 2, r = 1\):
    \begin{align*}
      \int_{\partial B(0, 1)} \partial_1 \partial_2 G(x) \, dS(x) 
      &= \frac{1}{|B_1|} \int_{\partial B(0, 1)} \omega_1 \omega_2 \, d \omega,
    \end{align*}
    \(\partial B(0,1) = \{\omega \mid |\omega| = 1\}\). Consider: \(\omega \mapsto R \omega, (\omega_1, \dots, \omega_d) \mapsto (- \omega_1, \omega_2, \dots, \omega_d)\). Then
    \begin{align*}
      -\int_{1 \ge |x-y| \ge \epsilon} \partial_i \partial_j G(x-y) \phi(y) \, dx = 0.
    \end{align*}
    So 
    \begin{align*}
      - \int_{1 \ge |x-y| \ge \epsilon} \partial_i \partial_j G(x-y) \phi(x) \, dx &= - \int_{1 \ge |x-y|\ge \epsilon} \partial_i \partial_j G(x-y) (\phi(x) - \phi(y)) \, dx
    \end{align*}
    In summary:
    \begin{align*}
      \partial_i \partial_j (G \star f)(\phi)
      &= \int_{\mathbb{R}^d} f(y) \left(\int_{\mathbb{R}^d} \partial_i G(x-y) \partial_j \phi(x) \, dx\right) \, dy \\
      &= \int_{\mathbb{R}^d} f(y) \frac{1}{d} \partial_{ij} \phi(y) \, dy \\
      & \quad - \int_{\mathbb{R}^d} f(y) \left(\int_{|x-y| > 1} \partial_i \partial_j G(x-y) \phi(x) \, dx\right) \\
      & \quad - \int_{\mathbb{R}^d} \left[\lim_{\epsilon \to 0} \int_{1 \ge |x-y| \ge \epsilon} \underbrace{\partial_i \partial_j G(x-y) (\phi(x) - \phi(y)) \, dx}_{\smash{\le \frac{C}{|x-y|^d}|x-y| \|\nabla \phi\|_{L^\infty} \le \frac{C}{|x-y|^{d-1}} \in L_{loc}^1(dx) \forall y}}\right] \, dy \\
      &= \int_{\mathbb{R}^d} \frac{\delta_{ij}}{d} f(x) \phi(x) \, dx
      - \int_{\mathbb{R}^d} \phi(x) \left(\int_{|x-y| > 1} \partial_i \partial_j G(x-y) f(y) \, dy\right) \, dx \\
      & \quad - \int_{\mathbb{R}^d} \phi(x) \left[\int_{|x-y| \le 1} \partial_i \partial_j G(x-y) (f(y)-f(x))\, dy\right] \, dx
    \end{align*}
    Conclusion: 
    \begin{align*}
      \partial_i \partial_j (G \star f) (x) &= - \frac{\delta_{ij}}{d} f(x) + \int_{|x-y| > 1} \partial_i \partial_j G(x-y) f(y) \, dy \\
      &\quad + \int_{|x-y| \le 1} \partial_i \partial_j G(x-y) \left(f(y) - f(x)\right) \, dy
    \end{align*}
    The first term \(f \in C^{0, \alpha}\). The second term is also at least \(C^{0, \alpha}\) since \(\partial_i \partial_j G(x)\) is smooth as \(|x| > 1\). We need to prove that the thirt term
    \begin{align*}
      W_{ij}(x) &= \int_{|x-y| \le 1} \partial_i \partial_j G(x-y)(f(y) - f(x)) \, dy
    \end{align*}
    is Hölder-continuous, \(|W_{ij}(x) - W_{ij(y)}| \le C |x-y|^\alpha\). Recall:
    \begin{align*}
      |\partial_i \partial_j G(x-y) (f(y)-f(x))| \le C \frac{1}{|x-y|^d}|x-y|^\alpha = \frac{C}{|x-y|^{d-\alpha}} \in L_{loc}^1(dy)
    \end{align*}
    We write 
    \begin{align*}
      W_{ij}(x) 
      &= \int_{|x-y| \le 1} \partial_i \partial_j G(x-y) (f(y)-f(x)) \, dy \\
      &= \int_{|z| \le 1} \partial_i \partial_j G(z) (f(x+z)-f(x)) \, dz
    \end{align*}
    So we get: \begin{align*}
      W_{ij} - W_{ij}(y) &= \int_{|z| \le 1} \partial_i \partial_j G(z) (f(x+z)-f(y+z)-f(x)+f(y)) \, dz
    \end{align*}
    Easy thought: Use \(\partial_i \partial_j G(z)| \le \frac{C}{|z|^d}\) and 
    
    \begin{align*}
      &|f(x+z) - f(y+z) - f(x) + f(y)| \\
      &\quad\le 
      \begin{cases}
        |f(x+z) - f(x)| + |f(y+z)-f(y)| \le C|z|^\alpha \\
        |f(x+z) - f(y+z)| + |f(x) - f(y)| \le C|x-y|^\alpha
      \end{cases}
    \end{align*}
    Thus: 
    
    \begin{align*}
      |W_{ij}(x) - W_{ij}(y)| 
      &\le C \int_{|z| \le 1} \frac{1}{|z|^d} \min(|z|^\alpha, |x-y|^\alpha) \, dz \\
      &\le C \int_{|z| \le 1} \frac{1}{|z|^d} (|z|^\alpha)^\epsilon (|x-y|^\alpha)^{1-\epsilon}, \quad 0 < \epsilon < 1 \\
      &\le C \left(\int_{|z| \le 1} \frac{1}{|z|^{d-\alpha \epsilon}}\right)|x-y|^{\alpha(1-\epsilon)} \\
      &\le C_\epsilon |x-y|^{\alpha (1 - \epsilon)}
    \end{align*}

    thus it is easy to prove \(|W_{ij}(x) - W_{ij}(y)| \le C_\alpha |x-y|^\alpha\) for all \(\alpha' \le \alpha\). However, to get \(\alpha' = \alpha\) we need a more precise estimate. We split:
    \begin{align*}
      W_{ij}(x) - W_{ij}(y) 
      &= \int_{|z| \le 1} \dots = \int_{|z| \le min(4|x-y|, 1)} + \int_{4|x-y| < |z| \le 1}
    \end{align*}
    For the first domain:
    \begin{align*}
      &\int_{|z| \le 4 |x-y|} |\partial_{ij} G(z)| |f(x+z)-f(y+z)-f(y)+f(x)| \, dz \\
      &\quad \le C \int_{|z| \le 4|x-y|} \frac{1}{|z|^d} |z|^\alpha \, dz
      = const \cdot |x-y|^\alpha
    \end{align*}
    For the second domain:
    \begin{align*}
      &\int_{4|x-y| < |z| \le 1} \partial_{ij} G(z) (f(x+z) - f(y+z) + f(y)  f(x)) \, dz \\
      &\quad = \int_{4|x-y| < z \le 1} \partial_{ij} G(z) (f(x+z) - f(y+z)) \, dz = (\ldots)
    \end{align*}
    since \(\int_{4|x-y|<|z|\le1} \partial_{ij}G(z) \, dz = 0\). Then
    \begin{align*}
      (\ldots) &= \int_{4|x-y| < |z-x| \le 1} \partial_{ij} G(z-x) f(z) \, dz - \int_{4|x-y| < |z-y| \le 1} \partial_{ij} G(z-y) f(z) \, dz.
    \end{align*}
    Denote \(A = \{z \mid 4 |x-y| < |z-x| \le 1\}\), \(B = \{z \mid 4|x-y| < |z-y| \le 1\}\). Consider 
    \begin{align*}
      &\int_A \partial_{ij} G(z-x) f(z) \, dz - \int_B \partial_{ij} G(z-y) f(z) \, dz \\
      &\quad = \int_{A\setminus B} + \int_{B \setminus A} + \int_{A \cap B} (\partial_{ij} G(z-x) - \partial_{ij} G(z-y))f(z) \, dz
    \end{align*}
    Lets regard the intersection. We have \begin{align*}
      \partial_{ij} G(x) &= \frac{1}{|B_1|}\frac{1}{|x|^d} (\omega_i \omega_j - \frac{1}{d} \delta_{ij}) \\
      |\partial_{ij}G(x) - \partial_{ij}G(y)| &\le C|x-y| \left(\frac{1}{|x|^{d+1}} + \frac{1}{|y|^{d+1}}\right)
    \end{align*}
    Now, 
    \begin{align*}
      |\partial_{ij} G(z-x) - \partial_{ij} G(z-y)|
      &\le C |x-y| \left(\frac{1}{|z-x|^{d+1}} + \frac{1}{|z-y|^{d+1}}\right)
    \end{align*}
    So we have 
    \begin{align*}
      &\left| \int_{A \cap B} (\partial_{ij} G(z-x) - \partial_{ij} G(z-y)) f(z) \, dz \right| \\
      &\quad \le C \int_{A \cap B} |x-y| \left(\frac{1}{|z-x|^{d+1}} + \frac{1}{|z-y|^{d+1}}\right) |f(z)| \, dz = (\ldots)
    \end{align*}
    Now we replace \(f(z)\) by \(f(z) - f(x)\), then:
    \begin{align*}
      &\left| \int_{A \cap B} (\partial_{ij} G(z-x) - \partial_{ij} G(z-y)) (f(z)-f(x)) \, dz \right| \\
      &\quad \le C \int_{A \cap B} |x-y| \left(\frac{1}{|z-x|^{d+1}} + \frac{1}{|z-y|^{d+1}}\right) |z-x|^\alpha \, dz \\
      &\quad = C \underbrace{\int_{A \cap B} |x-y| \frac{1}{|z-x|^{d + 1 - \alpha}} \, dz}_{(I)} + \underbrace{C \int_{A \cap B} |x-y| \frac{1}{|z-y|^{d+1}}|z-x|^\alpha \, dz}_{(II)}
    \end{align*}
    Now, \begin{align*}
      (I) &\le C |x-y| \int_{4 |x-y| < |z-x| \le 1} \frac{1}{|z - x|^{d+1-\alpha}} \, dz \\
      &= C |x-y| \int_{4 {x-y} < |z| \le 1} \frac{1}{|z|^{d+1-\alpha}} \, dz \\
      &\le C|x-y| \int_{4|x-y|}^1 \frac{1}{r^{d+1-\alpha}}r^{d-1} \, dr \\
      &= C|x-y| \int_{4|x-y|}^1 \frac{1}{r^{2-\alpha}} \, dr \\
      &\le C |x-y| \left[-1 + \frac{1}{(4|x-y|)^{1-\alpha}}\right] \\
      &\le C|x-y|^\alpha
    \end{align*}

    \begin{align*}
      (II) &\le C |x-y| \int_{A \cap B} \frac{1}{|z-y|^{d+1}} |z-x|^\alpha \, dz \\
      &\le C |x-y| \int_{A \cap B} \frac{1}{|z-y|^{d+1}} \left(|z-y|^\alpha + |x-y|^\alpha\right) \, dz \\
      &\le \underbrace{C |x-y| \int_B \frac{1}{|z-y|^{d+1-\alpha}} \, dz}_{\text{similar to (I)}} +\ C |x-y|^{1+\alpha} \int_B \frac{1}{|z-y|^{d+1}} \, dz \\
    \end{align*}
    and \begin{align*}
      C |x-y|^{1+\alpha} \int_B \frac{1}{|z-y|^{d+1}} \, dz
      &\le \int_{4|x-y|} \frac{1}{r^{d+1}} r^{d-1} \, dr 
      \le \frac{C}{|x-y|}
    \end{align*}
    Consider \(A \setminus B\): 
    \begin{align*}
      \left| \int_{A \setminus B} \right| \le C \|f\|_{L^\infty} \int_{A \setminus B} \frac{1}{|z-x|^d} \, dz
    \end{align*}
    where
    \begin{align*}
      A &= \{z \mid 4 |x-y| < |z-x| \le 1\} \\
      B &= \{z \mid 4 |x-y| < |z-y| \le 1\} \\
      A \setminus B &= \{z \in A \mid |z-y| \le 4 |x-y|\} \cup \{z \in A \mid |z-y| > 1\} = E_1 \cup E_2 \\
    \end{align*}
    for 
    \begin{align*}
      E_1 &= \{z \mid |z-y| \le 4 |x-y| < |z-x| \le 1\} \\
      &\subseteq \{ z \mid 4 |x-y| \le |x-z| \le 5 |x-y| \}.
    \end{align*}
    \(|x-z| \le |x-y| + |y-z| \le 5 |x-y|\) in \(E_1\).
    We have 
    \begin{align*}
      \int_{E_1} \frac{1}{|z-x|^d} \, dz 
      &\le \int_{4|x-y| \le |x-z| \le 5 |x-y|} \frac{1}{|z-x|^{d-\alpha}} \, dz \\
      &= \int_{4|x-y| \le |z| \le 5 |x-y|} \frac{1}{|z|^{d-\alpha}} \, dz \\
      &= \int_{4|x-y|} \frac{1}{r^d} r^{d-1} \, dr \\
      &= \int_{4|x-y|} \frac{1}{r^{1-\alpha}} \, dr \\
      &\le C |x-y|^\alpha
    \end{align*}
    Now in \(E_2\): \(|z-x| \ge |z-y| - |y-x| \ge 1 - |y-x|\).
    \begin{align*}
      \int_{E_2} \frac{1}{|z-x|^{d-\alpha}} \, dz
      &\le \int \frac{1}{|z-x|^{d-\alpha}} \, dz
      = \int_{1-|x-y|}^1 \frac{1}{r^{d-\alpha}} r^{d-1} \, dr \\
      &\le const. \left| 1 - \frac{1}{(1 - |x-y|)^\alpha} \right| \le C |x-y|^\alpha
    \end{align*}
  \end{proof}

  \begin{ex}[E 5.1]
    Prove that if \(f\) is a harmonic function in \(\mathbb{R}^d\) and \(g \in C_c(\mathbb{R}^d)\) is radial, then
    \[\int_{\mathbb{R}^d} f(x) g(x) \, dx = f(0) \int_{\mathbb{R}^d} g(x) \, dx\]
  \end{ex}

  \begin{proof}[Solution]
    \(x = r \omega, r > 0, |\omega| = 1\)
    \begin{align*}
      \int_{\mathbb{R}^d} f(x) g(x) \, dx &\overset{{\text{(Polar)}}}{=} \int_0^\infty \left(\int_{\partial B(0,1)} f(r\omega) g(r\omega) \, d \omega\right) \, dr \\
      &= \int_0^\infty \left(g_0(r) \int_{\partial B(0, 1)} f(r \omega) \, d \omega   \right) \, dr \\
      \text{(Mean value theorem (\ref{mean-value-theorem}))} \quad &= \int_0^\infty \left(g_0(r) f(0) \int_{\partial B(0, 1)}\, d \omega\right) \, dr \\
      &= f(0) \int_0^\infty \left( \int_{\partial B(0,1)} g(r \omega) \, d \omega \right) \, dr \\
      &= f(0) \int_{\mathbb{R}^d} g(x) \, dx \qedhere
    \end{align*}
  \end{proof}

  \begin{rem}
    Let \(g \in C_c(\mathbb{R}^d)\) be radial. Why is \(\int_{\mathbb{R}^3} \frac{g(x)}{|x|} \,dx \ne \infty\)? Because \(f(x) = \frac{1}{|x|}\) is harmonic in \(\mathbb{R}^d \setminus \{0\}\) and sub-harmonic in \(\mathbb{R}^d\), \(- \Delta f= c \delta_0\).
  \end{rem}

  \begin{ex}[E 5.2]
    Let \(1 \le p < \infty\). Let \(\Omega \subseteq \mathbb{R}^d\) be open. Consider the Sobolev Space
    \begin{align*}
      W^{1, p}(\Omega) &= \{f \in L^p(\Omega) \mid \partial_{x_i} f \in L^p(\Omega), \ \forall i = 1, 2, \dots, d\}
    \end{align*}
    with the norm
    \[\| f \|_{W^{1,p}} = \|f\| + \sum_{i=1}^d \| \partial_{x_i} f\|_{L^p(\Omega)}.\]
    Prove that \(W^{1,p}(\Omega)\) is a Banach space. Here \(x = (x_i)_{i=1}^d \in \mathbb{R}^d\). Hint: You can use the fact that \(L^p(\Omega)\) is a Banach Space.
  \end{ex}

  \begin{proof}[Solution]
    \(W^{1,p}(\Omega) \subseteq L^p(\Omega) \times L^p(\Omega) \cdots \times L^p(\Omega) = (L^p(\Omega))^{d+1}\). For an element \(f \in W^{1,p}(\Omega)\) we can think of it as \(f \mapsto (f, \partial_1 f, \partial_2 f, \dots, \partial_d f)\), so \(W^{1,p}(\Omega)\)  is a subspace of \((L^p(\Omega))^{d+1}\), which is a norm-space. Why is \(W^{1,p}(\Omega)\) closed in \((L^p(\Omega))^{d+1}\)? Take \(\{f_n\}_{n=1}^\infty \subseteq W^{1,p}(\Omega)\) such that \(f_n \to f\) in \(L^p\) an \(\partial_i f_n \to g_i\) in \(L^p\) for all \(i = 1, \dots, d\). We prove that \((f, g_1, \dots, g_d) \in W^{1,p}(\Omega)\), i.e. \(f \in W^{1,p}\) and \(g_i = \partial_i f\) for all \(i = 1, \dots, d\). We know that \(f_n \to f\) in \(L^p(\Omega)\), so \(f_n \to f\) in \(D'(\Omega)\) and \(\partial_i f_n \to \partial_i f\) in \(D'(\Omega)\). On the other hand we have \(partial_i f_n \to g_i\) in \(L^p(\Omega)\), so \(\partial_i f_n \to g_i\) in \(D'(\Omega)\). So we get \(\partial_i f = g_i \in L^p(\Omega)\) for all \(i=1, \dots, d\) in \(D'(\Omega)\). So we can conclude \(f \in W^{1,p}(\Omega)\) and \(\partial_i f = g_i\) for all \(i = 1, \dots, d\).
  \end{proof}

  \begin{ex}[E 5.3]
    Let \(f\) be a real-valued function in \(W^{1,p}(\mathbb{R}^d)\) for some \(1 \le p < \infty\). Prove that \(|f| \in W^{1,p}(\mathbb{R}^d)\) and 
    \begin{align*}
      (\nabla |f|)(x) &= \begin{cases}
        \nabla f(x) & f(x) > 0 \\ - \nabla f(x) & f(x) < 0 \\ 0 & f(x) = 0
      \end{cases}.
    \end{align*}
  \end{ex}

  \begin{proof}[Solution]
    Consider \(G_\epsilon(t)= \sqrt{\epsilon^2 + t^2} - \epsilon\) for \(\epsilon > 0\), \(t \in \mathbb{R}\). Clearly we have \(G_\epsilon(t) \to |t|\) as \(\epsilon \to 0\) and 
    \begin{align*}
      G_\epsilon'(t) &= \frac{2t}{2 \sqrt{\epsilon^2 + t^2}} = \frac{t}{\sqrt{\epsilon^2 + t^2}},
    \end{align*}
    so \(|G_\epsilon'(t)| \le 1\), \(G_\epsilon(0) = 0\). By the chain rule, \(G_\epsilon(f) \in W^{1,p}(\mathbb{R}^d)\) and 
    \begin{align*}
      (\partial_i G_\epsilon(f))(x) &= G_\epsilon'(f) \partial_if(x) = \frac{f(x)}{\sqrt{\epsilon^2 + f^2(x)}} \partial_i f(x) \in L^p(\mathbb{R}^d)
    \end{align*}
    for all \(i = 1, \dots, d\). Note then when \(\epsilon \to 0\) that \(G_\epsilon(f)(x) \to |f(x)|\) pointwise, so \(G_\epsilon(f) \to |f|\) in \(L^p(\mathbb{R}^d)\). \(|G_\epsilon(f)(x) - G_\epsilon(0)| \le |f(x)| \in L^p(\mathbb{R}^d)\) by dominated convergence. \begin{align*}
      \partial_i G_\epsilon(f)(x) 
      &= \frac{f(x)}{\sqrt{\epsilon^2 + f^2(x)}} \partial_i f(x) \xrightarrow{\epsilon \to 0} g_i(x) \coloneqq \begin{cases}
        \partial f_i(x) & f(x) > 0 \\ - \partial_i f(x) & f(x) < 0 \\ 0 & f(x) = 0
      \end{cases} \\
      |\partial_i G_\epsilon(f)(x)|
      &\le \left| \frac{f(x)}{\sqrt{\epsilon^2 + f^2(x)}}\right| |\partial_i f(x)| \le |\partial_i f(x)| \in L^p(\mathbb{R}^d)
    \end{align*}
    So we get \(\partial_i G_\epsilon(f) \xrightarrow{\epsilon \to 0} g_i\) in \(L^p(\mathbb{R}^d)\) by Dominated Convergence. So we conclude: \(\partial_i(|f|) = g_i \in L^p(\mathbb{R}^d)\) for all \(i = 1, \dots, d\), so \(|f| \in W^{1,p}(\mathbb{R}^d)\), \(|f| \in L^p\).
  \end{proof}

  \begin{ex}[E 5.4]
    Let \(\Omega \subseteq \mathbb{R}^d\) be open and bounded, \(f \in L^1(\Omega)\), 
    \begin{align*}
      u(x) &= \int_\Omega G(x-y)f(y) \, dy
    \end{align*}
    Let \(- \Delta u = f\) in \(D'(\Omega)\), \(u \in L_{loc}^1(\Omega)\), \(f \in L_{loc}^1(\mathbb{R}^d)\) and \(\omega_d f \in L^1(\mathbb{R}^d)\), where \[\omega_d (x) = \begin{cases}
      1 + |x| & d = 1 \\ \log(1 + |x|) & d = 1 \\ \frac{1}{(1 + |x|)^{d-2}} & d \ge 3
    \end{cases}.\]
    Prove that \begin{align*}
      G \star f &= \int_{\mathbb{R}^d} G(x-y) f(y) \, dy \in L_{loc}^1(\mathbb{R}^d) 
    \end{align*}
    and \(- \Delta (G \star f) = f\) in \(D'(\mathbb{R}^d)\).
  \end{ex}

  \begin{proof}[Solution]
    Define \(\tilde f = \mathbb{1}_\Omega(x) f(x) = \begin{cases}
      f(x) & x \in \Omega \\ 0 & x \notin \Omega
    \end{cases}\). Then 
    \begin{align*}
      u(x) &= \int_{\Omega} G(x-y)f(y) \, dy = \int_{\mathbb{R}^d} G(x-y) \tilde f(y) \, dy = (G \star \tilde f)(x)
    \end{align*}
    We have \(u \in L_{loc}^1(\mathbb{R}^d)\), so \(u \in L^1(\Omega)\).
    Then \(- \Delta u = \tilde f\) in \(D'(\mathbb{R}^d)\), so \(- \Delta u = f\) in \(D'(\Omega)\). Claim: \(- \Delta u = f\) in \(D'(\mathbb{R}^d)\), so \(- \Delta u = f\) in \(D'(\Omega)\) if \(\Omega \subseteq \mathbb{R}^d\), \(\tilde f|_\Omega = f\). Take \(\phi \in C_c^\infty(\Omega)\). We need: \((- \Delta u)(\phi) \overset{?}{=} \int_\Omega f \phi\). We have \(\phi \in C_c^\infty(\Omega)\), so \(\phi C_c^\infty(\mathbb{R}^d)\). This implies:
    \begin{align*}
      (- \Delta u)(\phi) 
      &= \int_{\mathbb{R}^d} \tilde f \phi 
      = \int_{\substack{\Omega, \\ \supp \phi \subseteq \Omega}} \tilde f \phi = \int_\Omega f \phi \qedhere
    \end{align*}
  \end{proof}

  \begin{ex}[E 5.5]
    Let \(B = B\left(0, \frac{1}{2}\right) \subseteq \mathbb{R}^3\). Consider \(u: B \to \mathbb{R}\), defined by 
    \begin{align*}
      u(x) &= \log|\log|x||.
    \end{align*}
    Prove that the distributional derivative \(f = - \Delta u\) is a function in \(L^{\frac{3}{2}}(B)\).
  \end{ex}

  \begin{proof}[Solution]
    \begin{align*}
      \omega(r) &= \log(-\log(r)), \quad \text{for } r \in \left(0, \frac{1}{2}\right) \\
      \omega'(r) &= \frac{1}{- \log(r)} \left(- \frac{1}{r}\right) = \frac{1}{r \log{r}} \\
      \omega''(r) &= - \frac{1}{(r \log(r))^2}(r \log(r))' = - \frac{\log(r)+1}{(r \log r)^2}
    \end{align*}
    So we have 
    \begin{align*}
      - \Delta u = w''(r) = \frac{1}{(r \log r)^2} - \frac{1}{r^2 \log(r)} = f(r)
    \end{align*}
    % Originally:
    % \[- \Delta u = - \omega''(r) - \frac{2 \omega'(r)}{r} = \frac{\log(r) + 1}{(r \log(r))^2} - \frac{2}{r^2 \log (r)} = \frac{1}{(r \log r)^2} - \frac{1}{r^2 \log(r)} = f(r)\]
    We show that \(f \in L^{\frac{3}{2}}: \) 
    \begin{align*}
      \int_B |f(x)|^{\frac{3}{2}} \, dx 
      &= const \int_0^{\frac{1}{2}} \left| \frac{1}{r^2 (\log r)^2} - \frac{1}{r^2 \log{r}} \right|^{\frac{3}{2}} r^2 \, dr \\
      &\tilde < \int_0^{\frac{1}{2}} \frac{1}{r} \left| \frac{1}{(\log(r))^2} - \frac{1}{(\log(r))} \right|^{\frac{3}{2}} \, dr \\
      \left(\begin{aligned}
        r = e^{-x}, \\
        x \in (\log(2), \infty), \\
        dr = -e^{-x} \, dx
      \end{aligned}\right) \quad &\tilde < \int_{\log(2)}^\infty e^x \left(\frac{1}{x^2} + \frac{1}{x}\right)^{\frac{3}{2}} e^{-x} \, dx \\
      &\tilde < \int_{\log(2)}^\infty \frac{1}{x^{\frac{3}{2}}} \, dx < \infty
    \end{align*}
    Where \(\tilde <\) means \emph{up to a constant}. Now, \(u(x) = \omega(r) = \log(-\log(r))\). 
    \begin{align*}
      - \Delta u(x) &= f(r) = \frac{1}{r^2(\log(r))^2} - \frac{1}{r^2\log(r))}
    \end{align*}
    for all \(x \ne 0, |x| = r < \frac{1}{2}\). Why is \(- \Delta u(x) = f\) in \(D'(B)\)? Take \(\phi \in C_c^\infty(B)\), check: \(\int_B u(-\Delta \phi) = \int_B f \phi\). 
    \begin{align*}
      \int_{|x| < \frac{1}{2}} u (- \Delta \phi) \, dx = \lim_{\epsilon \to 0^+} \int_{\epsilon < |x| < \frac{1}{2}} u(x) (- \Delta \phi)(x) \, dx
    \end{align*}
    by Dominated convergence. \(u \in L^1(B)\). For all \(\epsilon > 0\):
    \begin{align*}
      \int_{\epsilon < |x| < \frac{1}{2}} u(x) (-\Delta \phi)(x) \, dx 
      &= \int_{|x|>\epsilon} u(x) (-\Delta \phi)(x) \, dx \\
      &= \int_{\partial B(0, \epsilon)}u(x) \nabla \phi(x) \frac{x}{|x|} \, dS(x) + \int_{|x| > \epsilon} \nabla u(x) \nabla \phi(x) \, dx
    \end{align*}
    The boundary term vanishes as \(\epsilon \to 0\) since 
    \begin{align*}
      \left|u(x) \nabla \phi(x) \frac{x}{|x|}\right| \le \|\nabla \phi \|_{L^\infty} |u(x)| = C \log|\log(r)|
    \end{align*}
    \begin{align*}
      \left| \int_{\partial B(0, \epsilon)} u(x) \nabla \phi(x) \frac{x}{|x|} \, dS(x) \right| 
      &\le C \int_{\partial B(0, \epsilon)} \log |\log(\epsilon)| \, dS(x) \\
      &= C \log |\log \epsilon| \underbrace{|\partial B(0, \epsilon)|}_{\sim \epsilon^2} \xrightarrow{\epsilon \to 0} 0
    \end{align*}
    \begin{align*}
      &\int_{|x|> \epsilon} \nabla u(x) \nabla \phi(x) \, dx 
      = \sum_{i=1}^d \int_{|x|> \epsilon} \partial_i u(x) \partial_i \phi(x) \, dx \\
      &\quad= \sum_{i=1}^d \left(- \int_{\partial B(0, \epsilon)} \partial_i u(x) \phi(x) \frac{x_i}{|x|} \, dS(x) - \int_{|x|> \epsilon} \underbrace{\partial_i \partial_i u(x)}_{f(x)} \phi(x) \, dx\right)
    \end{align*}
    The boundary term vanishes as \(\epsilon \to 0\) as 
    \begin{align*}
      \left| \int_{\partial B(0, \epsilon)} \partial u(x) \phi(x) \frac{x_i}{|x|}\, dS(x) \right|
      &\le \|\phi\|_{L^\infty} \int_{\partial B(0, \epsilon)} |\partial_i u(x)| \, dS(x) \\
      (\star) \quad &\le C \frac{1}{|\epsilon \log(r)|} |\partial B(0, \epsilon)| \to 0
    \end{align*} as \(\epsilon \to 0\). 
    \((\star) u = u(r), u(x) = \omega(|x|), \partial_i u(x) = \omega(|x|)\frac{x_i}{|x|}\), \(|\partial_i u(x)| \le |\omega(|x|)| = \left|\frac{1}{r\log(r)}\right|\). Finally:
    \begin{equation*}
      \int_{|x|> \epsilon} f(x) \phi(x) \, dx \xrightarrow{\epsilon \to 0} \int_{\mathbb{R}^d} f(x) \phi(x) \, dx
    \end{equation*}
    Since \(f \phi \in L^1\) and Dominated Convergence.
  \end{proof}

  \begin{ex}[Bonus 5]
    Construct \(u \in L^1(\mathbb{R}^3)\) compactly supported s.t. \(- \Delta u \in L^{\frac{3}{2}}(\mathbb{R}^3)\) and \(u\) is not continuous at \(0\).
  \end{ex}
  Hint: Related to E 5.5. \(u_0(x) = \omega(r) = \log(|\log(r)|)\) if \(0 < r = |x| < \frac{1}{2}\). Consider \(\chi u_0\) where \(\chi \in C_c^\infty\), \(\chi = 0\) if \(|x| > \frac{1}{2}\), \(\chi = 1\) if \(|x| < \frac{1}{4}\). You can prove that \(\Delta (\chi u_0) = (\Delta \chi) u_0 + 2 \nabla \chi \nabla u_0 + \chi(\underbrace{\Delta u_0}_{\in L^{\frac{3}{2}}})\) in \(D'(\mathbb{R}^3)\). (almost everywhere, in distributional sense, integration by parts)

  \begin{thm}[Regularity on Domains] Let \(\Omega \subseteq \mathbb{R}^d\) be open. Assume \(u, f \in D'(\Omega)\) such that \(- \Delta u = f\) in \(D'(\Omega)\).
    \begin{enumerate}[label=\alph*)]
      \item If \(f \in L_{loc}^1(\Omega)\), then \begin{itemize}
        \item \(u \in C^1(\Omega)\) if \(d = 1\)
        \item \(u \in L_{loc}^q(\Omega)\) for all \(q < \infty\) if \(d = 2\)
        \item \(u \in L_{loc}^q(\Omega)\) for all \(q < \frac{d}{d-2}\) if \(d \ge 3\)
      \end{itemize}
      \item If \(f \in L_{loc}^q(\Omega)\), \( d \ge p < \frac{d}{2}\), then \(u \in C_{loc}^{0, \alpha}(\Omega)\), where \(0 < \alpha < 2 - \frac{d}{p}\)
      \item If \(f \in L_{loc}^p(\Omega)\), \(p > d\)f, then \(u \in C_{loc}^{1, \alpha}(\Omega)\), where \(0 \le \alpha < 1 - \frac{d}{p}\)
      \item If \(f \in C_{loc}^{0, \alpha}(\Omega)\) for some \(0 < \alpha < 1\), then \(u \in C_{loc}^{2, \alpha}(\Omega)\)
      \item If \(f \in C_{loc}^{m, \alpha}(\Omega)\), then \(u \in C_{loc}^{m+2, \alpha}(\Omega)\)
    \end{enumerate}
  \end{thm}

  \begin{proof}
    Let \(\mathbb{K} \in \{\mathbb{R}, \mathbb{C}\}\).
    Take a ball \(\overline B \subseteq \Omega\). Define \(f_B: \mathbb{R}^d \to \mathbb{K}\), \begin{align*}
      f_B(x) &= (\mathbb{1}_B f)(x) = \begin{cases}
        f(x) & x \in B \\ 0 & x \notin B
      \end{cases}
    \end{align*}
    Then if \(f \in L_{loc}^1(\Omega)\), \(f_B\) is compactly supported. From the previous theorems: \(G \star f_B \in L_{loc}^1(\mathbb{R}^d)\) and \(- \Delta (G \star f_B) = f_B\) in \(D'(\mathbb{R}^d)\). On the other hand, \(- \Delta u = f\) in \(D'(\Omega)\), so \(- \Delta(u - G \star f_B) = 0\) in \(D'(B)\). Indeed, for all \(\phi \in C_c^\infty(B)\), then:
    \begin{align*}
      (- \Delta u)(\phi) &= \int_\Omega f \phi = \int_B f_B \phi = - \int_{\mathbb{R}^d} f_B \phi = (-\Delta)(G \star f_B)(\phi)
    \end{align*}
    Then \(- \Delta u = - \Delta (G \star f_B)\) in \(D'(B)\). Then \(u - G \star f_B\) is harmonic in \(B\) and by Weyls lemma we have \(u - G \star f_B \in C^\infty(B)\). So the smoothness of \(u\) in \(B\) is the same to that of \(G \star f\).
  \end{proof}


  \begin{ex}[E 6.1]
    Show that If \(\chi \in C^\infty(\mathbb{R}^d)\), then \(f \in W^{1,p}(\mathbb{R}^d)\), \(1 \le p < \infty\), then \(\chi f \in W_{loc}^{1, p}(\mathbb{R}^d)\) and 
    \[\partial_i (\chi f) = (\partial_i \chi) f + \chi(\partial_i f) \quad \text{in } D'(\mathbb{R}^d)\]
  \end{ex}

  \begin{proof}[Solution]
    \(\chi f \in L_{loc}^p(\mathbb{R}^d)\) obvious. \(\partial (\chi f) \in L_{loc}^p(\mathbb{R}^d)\) is nontrivial but follows from \(\partial_i (\chi f) = \underbrace{(\partial_i \chi) f + \chi (\partial f)}_{\in L_{loc}^p}\) in \(D'(\mathbb{R}^d)\). To compute the distributional derivative \(\partial_i (\chi f)\), then: Take \(\phi \in C_c^\infty(\mathbb{R}^d)\): 
    \begin{align*}
      - \int_{\mathbb{R}^d} \chi f (\partial \phi)
      &= \int_{\mathbb{R}^d}(?) \phi 
    \end{align*}
    We have 
    \begin{align*}
      - \int_{\mathbb{R}^d}\chi f(\partial_i \phi) 
      &= - \int_{\mathbb{R}^d} f(\chi \partial_i \phi) \\
      (\chi \partial_i \phi = (\partial_i \chi)\phi + \chi(\partial_i \phi)) \quad &= - \int_{\mathbb{R}^d} f \left(\partial_i(\chi \phi)-(\partial_i \chi) \phi\right) \\
      &= - \int_{\mathbb{R}^d} f \partial_i(\underbrace{\chi \phi}_{\in C_c^\infty}) + \int_{\mathbb{R}^d} f (\partial_i \chi) \phi \\
      &= \int_{\mathbb{R}^d} (\partial_i f) \chi \phi + \int f(\partial_i \chi) \phi \\
      &= \int_{\mathbb{R}^d} ((\partial_i f) \chi + f (\partial_i \chi)) \phi
    \end{align*}
    So \(\partial_i (\chi f) = (\partial_i f) \chi + f (\partial_i \chi)\) in \(D'(\mathbb{R}^d)\).
  \end{proof}

  \begin{rem}
    Question: If \(\chi \in C^1(\mathbb{R}^d)\), \(f \in W^{1,p}(\mathbb{R}^d)\). Is this it still correct that \(\partial_i (\chi f) = (\partial_i \chi) f + \chi(\partial_i f) \quad \text{in } D'(\mathbb{R}^d)\)?
  \end{rem}
  
  \begin{proof}
    It suffices to show that we still can apply intergration by parts.
    \begin{align*}
      (\star) \quad - \int f \partial_i g
      &\overset{?}{=} \int(\partial_i f) g
    \end{align*}
    Approximation: \((\star)\) is correct if \(g \in C_c^\infty\) \begin{itemize}
      \item If \(g \in C_c^1\), there is \(\{g_n\} \subseteq C_c^\infty\) s.t. \(g_n \to g\) in \(W_{loc}^{1,p}\), \(\frac{1}{p} + \frac{1}{q} = 1\).
      \begin{align*}
        \int (\partial_i g) f \xleftarrow{n \to \infty} - \int \underbrace{f}_{L^p} \underbrace{\partial g_n}_{\to \partial_i g \text{ in } L^q} = \int \underbrace{(\partial_i f)}_{\in L^p} \underbrace{g_n}_{\to g \text{ in } L^q} \xrightarrow{n \to \infty} \int (\partial_i f) g
      \end{align*}
    \end{itemize}
  \end{proof}

  \begin{ex}[E 6.2]
    \(\mathbb{R}^2\), \(G(x) = - \frac{1}{2 \pi} \log |x|\). Let \(f \in L^p(\mathbb{R}^d)\), compactly supported. Define \(u(x) =(G \star f)(x) = \int_{\mathbb{R}^2} G(x-y) f(y) \, dy\)
    \begin{enumerate}
      \item If \(p = 1\), then \(u \in L_{loc}^q(\mathbb{R}^2)\) for all \(q < \infty\).
      \item If \(p > 2\), then \(u \in C^{1, \alpha}\) with \(0 < \alpha < 1 - \frac{2}{p}\).
    \end{enumerate}
  \end{ex}

  \begin{proof}[Solution]
    \begin{enumerate}
      \item Take any ball \(B = B(0,R)\) and:
      \begin{align*}
        \int_B |u(x)|^q \ dx 
        &= \int_B \left(\int_{\mathbb{R}^d} |G(x-y)| |f(y)| \, dy\right)^q \, dx \\
        &\le C \int_{B} \left(\int_{\mathbb{R}^2} |G(x-y)|^q |f(y)| \, dy\right) \, dx  \\
        &= C \int_{\mathbb{R}^2} \left(\int_B |G(x-y)|^q \, dx\right) |f(y)| \, dy
      \end{align*}
      Recall from the proof of Youngs inequality:
      \begin{align*}
        |u(x)| &= \left| \int_{\mathbb{R}^2} G(x-y) f(y) \, dy \right| \\
        &\le \int_{\mathbb{R}^2} |G(x-y)||f(y)| \, dy \\
        &\le \left(\int_{\mathbb{R}^2} |G(x-y)|^q |f(y)| \, dy\right)^{\frac{1}{q}} \left(\int_{\mathbb{R}^2} |f(y)| \, dy\right)^{\frac{1}{q}}, \quad \frac{1}{q} + \frac{1}{q'} = 1
      \end{align*}
      Assume \(\supp f \subseteq \overline{B(0, R)}\). Then if \(y \in \supp f\) and \(x \in B(0,R)\), then \(|x-y| \le |x| + |y| \le R + R_1\). For all \(y \in \supp f\):
      \begin{align*}
        \int_{B(0, R)} |G(x-y)|^q \, dx
        &\le \int_{|x-y| \le R + R_1} |G(x-y)|^q \, dx \\
        &= \int_{|z| \le R + R_1} |G(z)|^q \, dz < \infty
      \end{align*}
      as \(G \in L_{loc}^q\) (\(|G(z)| = \frac{1}{2 \pi} |\log(z)| \le \frac{C_{R + R_1, \epsilon}}{|z|^\epsilon}\) for all \(|z| \le R + R_1\)), so 
      \begin{align*}
        \int_{|z| \le R + R_1} |G(z)|^q \le C_{R + R_1, \epsilon} \int_{|z| \le R + R_1} \frac{1}{|z|^{\epsilon q}} \, dz < \infty
      \end{align*}
      if \(\epsilon q < 2\).
      \item Recall \(\partial_i u \in L_{loc}^1(\mathbb{R}^2)\) and:
      \begin{align*}
        \partial_i u(x) &= (\partial_i G \star f)(x) 
        = c \int_{\mathbb{R}^2} \frac{x_i - y_i}{|x-y|^2} f(y) \, dy
      \end{align*}
      First we show \(\partial_i u \in C^{0, \alpha}\): 
      \begin{align*}
        |\partial_i u(x) - \partial_i u(z)|
        &= \left| C \int_{\mathbb{R}^2} \left(\frac{x_i - y_i}{|x-y|^2} - \frac{z_i - y_i}{|z-y|^2}\right) \, f(y) \, dy \right| \\
        &\le C \int_{\mathbb{R}^2} \left| \frac{x_i y_i}{|x-y|^2} - \frac{z_i - y_i}{|z-y|^2} \right| |f(y)| \, dy \\
        &\overset{?}{\le} C |x-y|^\alpha
      \end{align*}
      Note that 
      \begin{align*}
        \left| \frac{x_i - y_i}{|x-y|^2} - \frac{z_i - y_i}{|z-y|^2} \right| 
        &= \left| (x_i  - y_i) \left(\frac{1}{|x-y|^2} - \frac{1}{|z-y|^2}\right) + \frac{x_i - z_i}{|z-y|^2} \right| \\
        &\le |x_i - y_i| \left| \frac{1}{|x-y|^2} - \frac{1}{|z-y|^2} \right| + \frac{|x_i - z_i|}{|z-y|^2} \\
        &\le C|z-x|^\alpha \left(\frac{1}{|x-y|^{1 + \alpha}} + \frac{1}{|z-y|^{1 + \alpha}} + \frac{|x-z|}{|z-y|^{2}}\right)
      \end{align*}
      Here \(|x_i - z_i| \le |x-z|\) and \(|x_i - y_i| \le |x-y|\) and:
      \begin{align*}
        \smash{\underbrace{\left| \frac{1}{|x-y|^2} - \frac{1}{|z-y|^2} \right|}_{\text{sym } x \leftrightarrow z}}
        &= \left| \frac{1}{|x-y|} - \frac{1}{|z-y|}\right| \left| \frac{1}{|x-y|} + \frac{1}{|z-y|}\right| \\
        &= \frac{||z-y| - |x-y|}{|x-y||z-y|} \left| \frac{1}{|x-y|} + \frac{1}{|z-y|} \right| \\
        &\le |z-x|^\alpha \frac{\max(|z-y|, |x-y|)^{1-\alpha}}{|x-y| |z-y|} \left(\frac{1}{|x-y|} + \frac{1}{|z-y|}\right) \\
        &\le C |z-x|^\alpha \left(\frac{1}{|x-y|^{2 + \alpha}} + \frac{1}{|z-y|^{2 + \alpha}}\right)
      \end{align*}
    \end{enumerate}
    By the symmetrie \(x \leftrightarrow z\):
    \begin{align*}
      LHS &\le C|z-x|^\alpha \left(\frac{1}{|x-y|^{1 + \alpha}} + \frac{1}{|z-y|^{1+\alpha}}\right) + \frac{|x-y|}{|x-y|^2} \\
      \Rightarrow LHS &\le C \dots + |x-z| \min\left(\frac{1}{|z-y|^2}, \frac{1}{|x-y|^2}\right) \\
      & \le (|x-y| + |z-y|)^{1-\alpha} \\
      & C |z-x|^\alpha \left(\frac{1}{|x-y|^{1+\alpha}} + \frac{1}{|z-y|^{1+\alpha}}\right)
    \end{align*}
    In summary:
    \begin{align*}
      |\partial_i u(x) - \partial_i u(z)| 
      &\le C \int_{\mathbb{R}^2} \left| \frac{x_i - y_i}{|x-y|^2} - \frac{z_i - y_i}{|z-y|^2}\right||f(y)| \, dy \\
      &= C |x-y|^\alpha \int_{\mathbb{R}^2} \left(\frac{1}{|x-y|^{1+\alpha}} + \frac{1}{|z-y|^{1+\alpha}}\right) |f(y)| \, dy
    \end{align*}
    Consider if \(|x| > 2 R_1\):
    \begin{align*}
      \int_{\mathbb{R}^2} \frac{1}{|x-y|^{1+\alpha}} |f(y)| \, dy 
      &\le \int_{\mathbb{R}^2} \frac{1}{R_1^{1+\alpha}} |f(y)| \, dy \le C
    \end{align*}
    \(\supp f \subseteq B(0, R_1)\). If \(|x| < 2 R_1\), then \(|x-y| \le 3R\) if \(y \in B(0, R_1)\). Hence:
    \begin{align*}
      &\int_{|x-y| \le 3 R_1} \frac{1}{|x-y|^{1+\alpha}} |f(y)| \, dy \\
      &\quad \le \left(\int_{|x-y| \le 3 R_1} \frac{1}{|x-y|^{(1+\alpha)p'}}\right)^{\frac{1}{p'}} \left(\int |f(y)|^p \, dy\right)^{\frac{1}{p}} \\
      &\quad= \int_{|z| \le 3 R_1} \frac{1}{|z|^{(1+\alpha)p'}} \,dz < \infty
    \end{align*}
    So \(\alpha < 1 - \frac{2}{p}\).
  \end{proof}

  \begin{ex}[E 6.3]  
    Let \(f \in C_{loc}^{0,\alpha}\) and \(- \Delta u = f\) in \(D'(\Omega)\). Prove \(u \in C_{loc}^{2, \alpha}(\Omega)\).
  \end{ex}

  \begin{proof}[Solution]
    Take an open ball \(B \subseteq \bar B \subseteq \Omega\). We prove \(u \in C^{2, \alpha}(B)\). There is an open \(\Omega_B\) s.t. \(\bar B \subseteq \bar \Omega_B \subseteq \Omega\). Then there is a \(\chi_B \in C_c^\infty(\mathbb{R}^d)\) s.t. \(\chi_B(x) = 1\) if \(x \in B\) and \(\chi_B(x) = 0\) if \(x \notin \Omega_B\). Define
    \begin{align*}
      f_B(x) = \chi_B(x)f(x): \ \mathbb{R}^d \to \mathbb{R}
    \end{align*}
    We prove that \(f_B \in C^{0, \alpha}(\mathbb{R}^d)\). Since \(f \in C_{loc}^{0,\alpha}(\Omega)\) we have \(f \in C^{0,\alpha}(\Omega)\), so \(|f(x)-f(y)| \le C|x-y|^\alpha\) for all \(x,y \in \Omega_B\). Then:
    \begin{align*}
      |f_B(x) - f_B(y)| 
      &= |\chi_B(x) f(x) - \chi_B(y) f(y)| \\
      &\le |(\chi_B(x) - \chi_B(y))f(x) + \chi_B(y)(f(x)-f(y))| \\
      &\le C |x-y|^\alpha \|f\|_{L^\infty} + C\|\chi\|_{L^\infty(\Omega_B)}|x-y|^\alpha
      \le C_{\Omega_B}|x-y|^\alpha
    \end{align*}
    What about other cases? If \(x, y\) are bot not in \(\Omega_B\), then \(|f_B(x) - f_B(y)| =0\), then if \(x \in \Omega_B\) and \(y \notin \Omega_B\): \(|f_B(x) - f_B(y)| = |f_B(x)| = |\chi_B(x)||f(x)| = |\chi_B(x) - \chi_B(y)||f(x)| \le C |x-y|^\alpha\). Conclusion: \(|f_B(x) - f_B(y)| \le C|x-y|^\alpha\) for all \(x,y \in \mathbb{R}^d\), i.e. \(f_B \in C^{0,\alpha}(\mathbb{R}^d)\). Also \(f_B\) is compactly supported. By a theorem in the lecture: \(G \star f_B \in C^{2, \alpha}(\mathbb{R}^d)\). Finally: \(-\Delta u = f\) in \(D'(\Omega)\), \(-\Delta (G \star f_B) = f_B\) in \(D'(\mathbb{R}^d)\). So we conclude \(-\Delta u = f = f_B = - \Delta (G\star f_B) \) in \(D'(B)\). \(-\Delta(u - G \star f_B)= 0\) in \(D'(B)\), so \(u - G \star f_B \in C^\infty(B)\), so \(u \in C^{2,\alpha}(B)\).
  \end{proof}

  \begin{ex}[E 6.4]
    \(u, f \in L^2(\mathbb{R}^d)\), \(-\Delta u = f \) in \(D'(\mathbb{R}^d)\). Prove \(u \in W^{2,2}(\mathbb{R}^d)\), \(\|u\|_{W^{2,2}(\mathbb{R}^d)} \le C \left(\|u\|_{L^2} + \|f\|_{L^2}\right)\).
    \begin{align*}
      W^{2,2}(\mathbb{R}^d)
      &= \{g \in L^2(\mathbb{R}^d) \mid D^\alpha g \in L^2 \text{ for all } |\alpha| \le 2\} \\
      &= \{g \in L^2(\mathbb{R}^d) \mid \widehat{D^\alpha g}(k) = (-2 \phi ik)^\alpha \hat g(k) \in L^2(\mathbb{R}^d) \text{ for all } |\alpha| \le 2\} \\
      &= \{g \in L^2(\mathbb{R}^d) \mid (1 + |k|^2) \hat g (k) \in L^2(\mathbb{R}^d)\}
    \end{align*}
    \(\|u\|_{W^{2,2}(\mathbb{R}^d)}\) is comparable \(\int_{\mathbb{R}^d} (1 + |k|^2)^2 |\hat g(k)|^2 \, dk\). If \(D^\alpha g \in L^2\), then \(\widehat{D^\alpha g}(k) = (-2 \pi i k)^\alpha \hat g(k)\). For any \(\phi \in C_c^\infty(\mathbb{R}^d)\): 
    \begin{align*}
      \int \widehat{D^\alpha g} (k) \hat \phi (k) , dk
      &= \int (D^\alpha g) \phi = (-1)^{|\alpha|} \int g (D^\alpha \phi) \\
      &= (-1)^{|\alpha|} \int \overline{\widehat{g}}(k) \widehat{D^\alpha} \phi(k) \\
      &= (-1)^{|\alpha|} \int \overline{\widehat{g}}(k)(-2 \pi i k)^\alpha \hat \phi(k) \, dk
    \end{align*}
  \end{ex}
  so \(\hat D^\alpha g(k) = (-1)^{|k|} \hat g (k) \overline{(-2 \pi i k x)^\alpha} = \hat g(k)(-2 \pi ik)^\alpha\). This implies:
  \begin{align*}
    \|u\|_{W^{2,2}(\mathbb{R}^d)} 
    &\le C \int_{\mathbb{R}^d} (1 + |k|^2)^2 |\hat u(k)|^2 \, dk \\
    &= C \left(\|u\|_{L^2}^2  + \int_{\mathbb{R}^d} |k|^4 |\hat u(k)|^2 \, dk\right) \\
    &\le C \left(\|u\|_{L^2}^2 + \|f\|_{L^2}^2\right) \\
    &\le C(\| u\|_{L^2} + \|f||_{L^2})^2
  \end{align*}

  \begin{rem}[Bonus 6]
    Let \(f,g \in W^{1,2}(\mathbb{R}^d)\). Prove that \(fg \in W^{1,1}(\mathbb{R}^d)\) and 
    \[\partial_i (fg) = (\partial_i f) g + f(\partial_i g) \quad \text{ in } D'(\mathbb{R}^d)\]
  \end{rem}


  \chapter{Existence for Poisson's Equation on Domains}

  Let \(\Omega \subseteq \mathbb{R}^d\) be open. Consider Poisson's equation.
  \[\begin{cases}
    - \Delta u = f & \text{in } \Omega \\ u = g & \text{on } \partial \Omega
  \end{cases}\]
  for given data \((f, g)\) and \(u\) the unknown function.
  \begin{itemize}
    \item Classical solutions: \(f \in C^2(\bar \Omega) \leadsto\) explicit representation formula.
    \item Weak solution: \(f \in L^p( \Omega)\), \(g \in L^p(\partial \Omega)\) \(\leadsto\) \(u \in W^{2,p}(\Omega)\). We are going to establish the existence by \emph{Energy Methods}. (Calculus of variations)
  \end{itemize}

  \begin{defn}[\(C^1\)-Domains]
    Let \(\Omega \subseteq \mathbb{R}^d\) be open. We say that \(\Omega\) is of class \(C^1\) (i.e. \(\partial \Omega \in C^1\)) if for all \(x_0 \in \partial \Omega\) there is a bijective function \(h: U \to Q\), where \begin{itemize}
      \item \(x_0 \in U\) open in \(\mathbb{R}^d\)
      \item \(Q = \{x = (x_1, \dots, x_d) = (x', x_d)\} \in \mathbb{R}^{d-1} \times \mathbb{R} \mid |x'| < 1, |x_d| < 1 \}\)
      \item \(h \in C^1(\bar U)\) and \(h^{-1} \in C^1(\bar Q)\) (\(C^1\)-diffeomorphism)
      \item \(h(U) = Q\)
      \begin{align*}
        h(U \cap \Omega) &= Q_+ = Q \cap \mathbb{R}_+^d = \{x = (x', x_d) \in Q \mid x_d > 0\} \\
        h(U \cap \partial \Omega) &= Q_0 = Q \cap \partial \mathbb{R}_+^d = \{x = (x', x_d) \in Q \mid x_d = 0\} \\
        h(U \setminus \bar \Omega) &= Q_- = Q \cap \mathbb{R}_-^d = \{x = (x', x_d) \in Q \mid x_d < 1\}
      \end{align*}
      (From Brezis' book)
    \end{itemize}
  \end{defn}
  
  \begin{rem}
    The set \(Q\) can be replaced by a ball, i.e. \(\Omega\) is of \(C^1\) if for all \(x_0 \in \partial \Omega\) there is a function \(U \to B(0,1) \subseteq \mathbb{R}^d\).
    \begin{itemize}
      \item \(x_0 \in U\) with \(U \subseteq \mathbb{R}^d\) open.
      \item \(h \in C^1(\bar U), h^{-1} \in C^1(\overline{B(0,1)})\)
      \item \(h(U \cap \Omega) = B(0,1) \cap \mathbb{R}_+^d\), \(h(U \cap \partial \Omega) = B(0,1) \cap \mathbb{R}^d\).
    \end{itemize}
  \end{rem}

  \begin{rem}[An equivalent definition form Evan's book App. C]
    Let \(\Omega \subseteq \mathbb{R}^d\) be open. Then \(\Omega\) is \(C^1\) if for all \(x_0 \in \partial \Omega\) there is a \(r > 0\) and a \(C^1\)-function \(\gamma: \mathbb{R}^{d-1} \to \mathbb{R}\) s.t. (upon relabeling and reorienting the axes if necessary) such that:
    \begin{align*}
      \Omega \cap B(x_0, r) &= \{x = (x', x_d) \in B(x_0, r) \mid x_d < \gamma(x_0)\}
    \end{align*}
  \end{rem}

  \begin{proof}[Proof of the equivalence of the two definitions]\
    \begin{itemize}
      \item [Def. 2 \(\Rightarrow\) Def. 1:] In fact, given \(x_0 \in \partial \Omega\) and \(\gamma\) we can define
      \begin{align*}
        h(x', x_d) &= (x', x_d - \gamma(x')) \in C^1(\mathbb{R}^d, \mathbb{R}^d) \\
        h^{-1}(x', x_d) &= (x', x_d + \gamma(x')) \in C^1(\mathbb{R}^d, \mathbb{R}^d)
      \end{align*}
      \item [Def. 1 \(\Rightarrow\) Def. 2:] We need the inverse function theorem and the implicit function theorem. Let \(x_0 \in \partial \Omega\), let \(h: U \to B(0,1)\) as in Def. 1. Denote \(h = (h_1, h_2, \dots, h_d)\). Since \(h\) is invertible near \(x_0\), by the inverse function theorem we have for the Jacobi matrix \(J h(x_0) = (\partial_j h_i(x_0))_{1 \le i,j \le d}\) is invertible. So we have \(\nabla h_d(x_0) = (\partial_j h_d(x_0))_{1 \le j \le d} \ne \vec{0}^{\mathbb{R}^d}\), so there is a \(j \in \{1, 2, \dots, d\}\) s.t. \(\partial_j h_d(x_0) \ne 0\). By relabeling and reorienting the axes, we can assume that \(\partial_d h_d(x_0) > 0\). By continuity there is a \(r > 0\) such that \(\partial_d h_d(x) > 0\) for all \(x \in B(x_0, r)\). Define \(\gamma: \mathbb{R}^{d-1} \to \mathbb{R}\) s.t. in \(B(x_0, r)\):
      \begin{equation*}
        x = (x', x_d) \in \partial \Omega
        \Longleftrightarrow h_d(x', x_d) = 0
        \Longleftrightarrow x_d = \gamma(x'),
      \end{equation*}
      \(h_d: \mathbb{R}^d \to \mathbb{R}\). This gives a solution \(\gamma\) if \(\partial_d h_d > 0\) in \(B(x_0, r)\). (For implicit function theorem, \(\partial_d h_d(x_0) \ne 0\))
      Question: Why in \(B(x_0,r)\)?
      \begin{align*}
        x = (x', x_d) \in \Omega
        \Longleftrightarrow x_d > \gamma(x')
      \end{align*}
      Since \(\partial_d h_d(x) > 0\) for all \(x \in B(x_0, r)\) we have that \(x_d \mapsto h_d(x', x_d)\) is strictly increasing, hence 
      \begin{align*}
        &x = (x', x_d) \in \Omega \\
        \Longleftrightarrow \quad&h(x', x_d) \in \mathbb{R}_+^d \\
        \Longleftrightarrow \quad&h_d(x', x_d) > 0 = h_d(x', \gamma(x')) \\
        \Longleftrightarrow \quad& x_d > \gamma(x') \qedhere
      \end{align*}
    \end{itemize}
  \end{proof}
  
  \begin{thm}[Gauss-Green formula / Integration by parts]
    Let \(\Omega \subseteq \mathbb{R}^d\) be open and bounded. Then
    \begin{enumerate}
      \item For all \(u, v \in C^1(\bar \Omega)\):
      \begin{align*}
        \int_\Omega (\partial_i u) v &= - \int_\Omega u(\partial_i v) + \int_{\partial \Omega} uv n_i \, dS,
      \end{align*}
      where \(\vec{n} = (n_i)_{i=1}^d\) is the outwarded unit normal vector.
    \item For all \(u,v \in C^2(\bar \Omega)\):
    \begin{align*}
      \int_\Omega u(-\Delta v) &= \int_\Omega \nabla u \nabla v - \int_{\partial \Omega} u \frac{\partial v}{\partial \vec{n}} \, dS
    \end{align*}
    where \(\frac{\partial v}{\partial \vec{n}} = \nabla v \vec{n} = \sum_{i=1}^d \partial_i v n_i\). 
  \end{enumerate}
  \end{thm}

  Classical solutions via Green's function:
  \begin{align*}
    \begin{cases}
      - \Delta u = f & \text{in } \Omega \\ u = g & \text{on } \partial \Omega
    \end{cases}
  \end{align*}
  Let \(\Omega \subseteq \mathbb{R}^d\) be open, bounded, \(\partial \Omega \in C^1\). Assume there exists a \(u \in C^2(\bar \Omega)\), \(f \in C(\bar \Omega)\), \(g \in C(\partial \Omega)\). Let \(G\) be the fundamential solution of the Laplace Equation in \(\mathbb{R}^d\). We use integration by parts in \(\Omega \setminus B(x, \epsilon)\):
  \begin{align*}
    &\int_{\Omega \setminus B(x, \epsilon)} u(y)(-\Delta G)(y-x) \, dy \\
    &\quad= \int_{\Omega \setminus B(x, \epsilon)} \nabla u(y) \nabla G(y-x) \, dy - \int_{\partial \Omega \cup \partial B(x, \epsilon)} u(y) \frac{\partial G}{\partial \vec{n}} (y-x) \, dS(y) \\
    &\int_{\Omega \setminus B(x, \epsilon)} G(y-x)(-\Delta u)(y) \, dy \\
    &\quad= \int_{\Omega \setminus B(x,\epsilon)} \nabla G(y-x)\nabla u(y) \, dy - \int_{\partial \Omega \cup \partial B(x,\epsilon)}G(y-x) \frac{\partial u}{\partial \vec{n}}(y) \, dS(y)
  \end{align*}
  This implies:
  \begin{align*}
    &\int_{\Omega \setminus B(x,\epsilon)} \left[u(y) (-\Delta G(y-x))-G(y-x)(-\Delta u)(y)\right] \, dy  \\
    &\quad= - \int_{\partial \Omega \cup \partial B(x, \epsilon)} \left[u(y) \frac{\partial G}{\partial \vec{n}}(y-x) - G(y,x) \frac{\partial u}{\partial \vec{n}}(y)\right] \, dS(y)
  \end{align*}
  for all \(x \in \Omega, x \in B(x, \epsilon) \subseteq \Omega\). When \(\epsilon \to 0\), then the left hand side converges to \(- \int_\Omega G(y-x) f(y) \, dy\) and the right hand side (for \(d \ge 2\)) we have \(\partial_j G(y) = \frac{-y_j}{d|B_1||y|^d}\), so
  \begin{align*}
    \frac{\partial G}{\partial \vec{n}}
    = \nabla G \vec{n} 
    = \nabla G(y) \left(\frac{-y}{|y|}\right)
    = \sum_{j=1}^d \frac{-y_i}{d|B_1||y|^d} \frac{-y_j}{|y|}
    = \frac{1}{d|B_1||y|^{d-1}} \text{ on} \partial B(0, \epsilon)
  \end{align*}
  so we have 
  \begin{align*}
    \frac{\partial G}{\partial \vec{n}}(y-x) = \frac{1}{d|B_1| \epsilon^{d-1}}
  \end{align*}
  on \(\partial B(x, \epsilon)\). Hence
  \begin{align*}
    \int_{\partial B(x, \epsilon)} u(y) \frac{\partial G}{\partial \vec{n}}(y-x) \, dS(y) 
    &= \frac{1}{d|B_1| \epsilon^{d-1}} \int_{\partial B(x, \epsilon)} u(y) \, dS(y) \\
    &= \fint_{\partial B(x,\epsilon)} u(y) \, dS(y)
    \xrightarrow{\epsilon \to 0} u(x) 
  \end{align*}
  On the other hand:
  \begin{align*}
    \left|\int_{\partial B(x, \epsilon)} G(y-x) \frac{\partial u(y)}{\partial \vec{n}} \, dS(y)\right|
    &\le C \epsilon^{d-1} \sup_{|z| = \epsilon} |G(z)| \xrightarrow{\epsilon \to 0} 0
  \end{align*}
  since \(|G(z)|\le \frac{C}{|z|^{d-2}}\) if \(d \ge 3\), \(|G(z)|\le C |\log(z)| \) if \(d = 2\) and \(|G(z)|\le C|z|\) if \(d = 1\). In summary: 
  \begin{align*}
    - \int_\Omega G(y-x) f(y) \, dy 
    = - \int_{\partial \Omega} \left[u(y) \frac{\partial G}{\partial \vec{n}} (y-x) - G(y-x) \frac{\partial u}{\partial \vec{n}}(y)\right] \, dS(y) - u(x) \\
    \Leftrightarrow u(x) = \int_\Omega G(y-x) f(y) \, dy + \int_{\partial \Omega} \left[G(y-x) \frac{\partial u}{\partial \vec{n}}(y) - g(y) \frac{\partial G}{\partial \vec{n}}(y-x)\right] \, dS(y)
  \end{align*}
  Problem: We don't know anything about \(\frac{\partial u}{\partial \vec{n}}\) on \(\partial \Omega\). Trick: We can resolve that by using the \emph{corrector} function: \(\Phi_x = \Phi_x(y)\) which solves:
  \[\begin{cases}
    - \Delta \Phi_x = 0 & \text{in } \Omega \\ \Phi_x(y) = G(y-x) &\text{on } \partial \Omega
  \end{cases}\]
  We assume that \(\Phi_x\) exists. 

  \begin{defn}[Green's function]
    \(\tilde G(x-y) = G(y-x) - \Phi_x(y)\) for all \(x,y \in \Omega\), \(x \ne y\).
  \end{defn}
  

  \begin{ex}[E 7.1]
    Let \(\Omega \subseteq \mathbb{R}^d\) be open and bounded with \(C^1\) boundary. For \(x \in \Omega\), assume there exist \(\Phi_x(y)\), \(y \in \bar \Omega\), s.t. 
    \begin{align*}
      \begin{cases}
        \Delta_y \Phi_x(y) = 0 \\ \Phi_x(y) = G(y-x)
      \end{cases},
    \end{align*}
    \(G(z) = \frac{1}{d(d-2)|B_1||z|^{d-2}}, d \ge 3\). Prove that \(\Phi_x(y) = \Phi_y(x)\) for all \(x,y \in \Omega\). Then \(\tilde G(x,y) = G(y-x) - \Phi_x(y)\) is symmetric, i.e. \(\tilde G(x,y) = \tilde G(y,x)\).
  \end{ex}

  \begin{proof}[Solution]
    Assume \(x \ne y\). Define \begin{align*}
      f(z) &= \tilde G(x,z) = G(z-x) - \Phi_x(z) \\
      g(z) &= \tilde G(y,z) = G(z-y) - \Phi_y(z)
    \end{align*}
    Integration by parts:
    \begin{align*}
      \int_{\Omega \setminus (B(x, \epsilon) \cup B(y, \epsilon))} (f \Delta g - g \Delta f) 
      &= \int_{\partial \Omega \cup \partial B(x,\epsilon) \cup \partial B(y, \epsilon)} \left(f \frac{\partial g}{\partial \vec{n_z}} - g \frac{\partial f}{\partial \vec{n_z}}\right) \, dS(z) \\
      &= \int_{\partial B(x, \epsilon) \cup \partial B(y, \epsilon)} \left(f \frac{\partial g}{\partial \vec{n_z}} - g \frac{\partial f}{\partial \vec{n_z}}\right)\, dS(z)
    \end{align*}
    Consider \(f \frac{\partial g}{\partial \vec{n_z}}\) on \(\partial B(x,\epsilon)\). Since \(g\) is only singular at \(y\), so \(\left|\frac{\partial g}{\partial \vec{n}}\right| \le C\) on \(\partial B(x,\epsilon)\). This implies:
    \begin{align*}
      \int_{\partial B(x,\epsilon)} \left|f \frac{\partial g}{\partial \vec{n_z}}\right| \, dS(z) 
      &\le C \int_{\partial B(x,\epsilon)}|f| \, dS(z) \\
      &\le C \int_{\partial B(x,\epsilon)} \left(\frac{1}{|x-z|^{d-2}} + \|\Phi_x\|_{L^\infty(\Omega)}\right) \, dS(z) \\
      &\le C \epsilon^{d-1} \left(\frac{1}{\epsilon^{d-2}} + 1\right)
      \le C\epsilon \xrightarrow{\epsilon \to 0} 0
    \end{align*}
    Consider \(f \frac{\partial g}{\partial \vec{n_z}}\) on \(\partial B(y, \epsilon)\). Decompose \(\frac{\partial g}{\partial \vec{n}} = \left[\nabla_z G(z-y) - \nabla_z \Phi_y(z)\right]\frac{(z-y)}{|z-y|}\). Since \(\Phi_y(z)\) is harmonic in \(\Omega\), we have that 
    \begin{align*}
      \int_{\partial B(y,\epsilon)} \left| f \nabla_z \Phi_y(z) \frac{-(z-y)}{|z-y|} \right|
      \le C \int_{\partial B(y,\epsilon)}|f|
      \le C \epsilon^{d-1} \xrightarrow{\epsilon \to 0}0
    \end{align*}
    Thus the main contribution from \(f \frac{\partial g}{\partial \vec{n}}\) is 
    \begin{align*}
      &\int_{\partial B(y, \epsilon)} f(z) \nabla_z G(z-y) \frac{-(z-y)}{|z-y|} \, dS(z) \\
      &\quad= \int_{\partial B(y, \epsilon)} f(z) \frac{-(z-y)}{d |B_1||z-y|^d} \frac{-(z-y)}{|z-y|} \, dS(z) \\
      &\quad = \frac{1}{d|B_1|\epsilon^{d-1}} \int_{\partial B(y,\epsilon)} f(z) \, dS(z) \\
      &\quad= \fint_{\partial B(y,\epsilon)} f(z) \, dS(z) = f(y)
    \end{align*}
    In summary:
    \begin{align*}
      \int_{\partial B(x, \epsilon) \cup \partial B(y,\epsilon)} f \frac{\partial g}{\partial \vec{n_z}} \, dS(z) \xrightarrow{\epsilon \to 0} f(y)
    \end{align*}
    Similary:
    \begin{align*}
      \int_{\partial B(x,\epsilon) \cup \partial B(y, \epsilon)} g \frac{\partial f}{\partial \vec{n_z}} \, dS(z) \xrightarrow{\epsilon \to 0} g(x)
    \end{align*}
    So we have that \(f(y) = g(x)\), so \begin{align*}
      f(y) &= G(y-x) - \Phi_x(y) \\
      g(x) &= G(x-y) - \Phi_y(x).
    \end{align*}
    So \(\Phi_x(y) = \Phi_y(x)\) for all \(x \ne y \in \Omega\). This implies \(\Phi_x(y) = \Phi_y(x)\) for all \(x,y \in \Omega\).
  \end{proof}
  
  
  \begin{thm}
    Let \(\Omega \subseteq \mathbb{R}^d\) be open, bounded and \(C^1\). If \(u \in C^2(\Omega)\) solves 
    \begin{align*}
      \begin{cases}
        - \Delta u = f & \text{in } \Omega \\
        u = g & \text{on } \partial \Omega
      \end{cases},
    \end{align*}
    then 
    \begin{align*}
      u(x)
      &= - \int_{\partial \Omega} g(y) \frac{\partial \tilde G}{\partial \vec{n_y}}(x,y) \, dS(y) + \int_{\Omega} \tilde G(x,y) \, dy    \end{align*}
  \end{thm}

  \begin{proof}
    We need to prove: 
    \begin{align*}
      \int_\Omega \Phi_x(y) f(y) \, dy + \int_{\partial \Omega} \left(-g(y) \frac{\partial \Phi_x(y)}{\partial \vec{n}_y} + G(y-x) \frac{\partial u}{\partial \vec{n}}(y)\right) = 0
    \end{align*}
    By integration by parts:
    \begin{align*}
      \int_\Omega \Phi_x(y) f(y) \, dy 
      &= \int_\Omega \Phi_x(y) (-\Delta u(y)) \, dy  \\
      &= \int_\Omega \left[\Phi_x(y) (-\Delta u(y)) + (\Delta \Phi_x(y)) u(y)\right] \, dy \\
      (\Delta \Phi_x(y) = 0) \quad &= \int_{\partial \Omega} \left(-\Phi_x(y) \frac{\partial u}{\partial \vec{n}} + \frac{\partial \Phi_x(y)}{\partial \vec{n}} \smash{\underbrace{u(y)}_{g(y)}}\right) \, dS(y) \qedhere
    \end{align*}
  \end{proof}
  How can we compute \(\Phi_x(y)\)? It is not easy for general domains. But let us prove on two cases:
  \begin{itemize}
    \item \(\Omega = \mathbb{R}_+^d\) (half-space)
    \item \(\Omega = B(0,r)\) (a ball)
  \end{itemize}
  \section{Green's function on the upper half plane}
  We use the following notation:
  \begin{align*}
    \mathbb{R}_+^d &= \{x = (x_1, x_2, \dots, x_d) = (x', x_d) \in \mathbb{R}^{d-1} \times \mathbb{R} \mid x_d > 0\} \\
    \partial \mathbb{R}_+^d &= \{x = (x', x_d) \mid x_d = 0\} = \mathbb{R}^{d-1} \times \{0\}
  \end{align*}
  For all \(x \in \mathbb{R}^d\) we want to find the correction function \(\Phi_x(y)\) with \(y \in \overline{\mathbb{R}_{+}^d}\) s.t.
  \begin{align*}
    \begin{cases}
      + \Delta_y \Phi_x(y) = 0 & \text{in } \mathbb{R}_+^d \\ \Phi_x(y) = G(y-x) & \text{in } \partial \mathbb{R}_+^d
    \end{cases}
  \end{align*}

  
  \begin{defn}[Reflection for \(\mathbb{R}_+^d\)] 
    For all \(x = (x', x_d) \in \mathbb{R}^d\), \(\tilde x = (x', -x_d) \in \mathbb{R}^d\), (if \(x \in \mathbb{R}_+^d \Rightarrow \tilde x \mathbb{R}_-^d\)) 
  \end{defn}
    
  Claim: \(\Delta_y \Phi_x(y) = G(y - \tilde x)\) is a corrector function.
  \begin{itemize}
    \item \(\Delta_y \Phi_x(y) = \Delta_y G(y-\tilde x) = 0\) for all \(y \in \mathbb{R}_+^d\) for all \(x \in \mathbb{R}_+^d\) (as \(\tilde x \in \mathbb{R}_-^d = \mathbb{R}^d \setminus \overline{\mathbb{R}_+^d}\))
    \item \(\Phi_x(y) = G(y-\tilde x) = G(y-x)\) on \(y \in \partial \mathbb{R}_+^d\). In fact, \(y \in \partial \mathbb{R}_+^d\), so \(y_d = 0\), so 
    \begin{align*}
      G(y - \tilde x) = G_0(|y-\tilde x|) = G_0\left(\sqrt{\sum_{i=1}^{d-1} |x_i - y_i|^2 +|x_d|^2}\right) = G_0(|y-x|)
    \end{align*}
  \end{itemize}
  Consider \(f = 0\) and 
  \begin{align*}
    \begin{cases}
      - \Delta = 0 &\text{in } \mathbb{R}_+^d \\ u = g & \text{on } \partial \mathbb{R}_+^d
    \end{cases}
  \end{align*}
  Then we expect \[u(x) = - \int_{\partial \Omega} g(y) \frac{\partial \tilde G}{\partial \vec{n_y}}(x,y) \, dS(y)\]
  We compute
  \begin{align*}
    \frac{\partial \tilde G}{\partial \vec{n_y}}(x-y) = \sum_{j=1}^d \frac{\partial \tilde G}{\partial y_j}(x,y) \vec{n_j} = - \frac{\partial \tilde G}{\partial y_d}(x,y) = \frac{\partial }{\partial y_d}(G(y-\tilde x) - G(y-x)) = \dots
  \end{align*}
  because \(\tilde G(x,y) = G(y-x) - \Phi_x(y) = G(y-x) - G(y-\tilde x)\). 
  \begin{align*}
    \dots &= \frac{1}{d |B_1|} \left[\frac{-(y_d - \tilde x_d)}{|y-\tilde x|^d} - \frac{-(y_d-x_d)}{|y-x|^d}\right] \\
    (y \in \partial \mathbb{R}_+^d) &\quad= \frac{1}{d|B_1|} \left[\frac{\tilde x_d}{|y-x|} - \frac{x_d}{|y-x|^d}\right]
    = \frac{- 2 x_d}{d |B_1| |y-x|^d}
  \end{align*}
  We expect
  \begin{align*}
    u(x) = - \int_{\partial \mathbb{R}_+^d} g(y) \frac{\partial \tilde G}{\partial \vec{n_y}} (x,y) \, dS(y) = \int_{\partial \mathbb{R}_+^d} g(y) \frac{2x_d}{d |B_1| |y-x|^d} \, dS(y)
  \end{align*}
  
  \begin{thm}
    Assume \(g \in C(\mathbb{R}^{d-1}) \cap L^\infty(\mathbb{R}^{d-1})\) Then 
    \begin{align*}
      u(x) &= \int_{\partial \mathbb{R}_+^d} g(y) K(x,y) \, dS(y)
    \end{align*}
    and 
    \begin{align*}
      K(x,y) &= \frac{2 x_d}{d|B_1||y-x|^d} \quad \text{for all \(x \in \mathbb{R}_+^d\).}
    \end{align*}
    satisfies that \(u \in C^\infty(\mathbb{R}_+^d) \cap L^\infty(\mathbb{R}_+^d)\) and 
    \begin{align*}
      \begin{cases}
        \Delta u = 0 & \text{in } \mathbb{R}_+^d \\ \lim_{\substack{x \to 0 \\ x \in \mathbb{R}_+^d}} u(x) = g(x_0) & \forall x_0 \in \partial \mathbb{R}_+^d
      \end{cases}
    \end{align*}
  \end{thm}
  
  \begin{proof}
    Claim: For all \(y \in \partial \mathbb{R}_+^d\), \(x \mapsto K(x,y)\) is harmonic in \(\mathbb{R}_+^d\) (i.e. \(\Delta_x K(x,y) = 0\) in \(\mathbb{R}_+^d\))
    \begin{itemize}
      \item Argument from Evans: \begin{align*}
        K(x,y) = - \frac{\partial}{\partial y_d}, \ \tilde G(y-x) = - \frac{\partial}{\partial y_d}(G(y-x)-G(y-\tilde x))
      \end{align*}
      We know that for all \(x \in \mathbb{R}_+^d\), \(y \mapsto \tilde G(y,x)\) is haromnic in \(\mathbb{R}_+^d \setminus \{x\}\). By symmetry we have \(\tilde G(y,x) = \tilde G(x,y)\) for all \(x,y \in \mathbb{R}^d_+\). So for all \(y \in \mathbb{R}_+^d\), \(x \mapsto \tilde G(y,x)\) is harmonic in \(\mathbb{R}_+^d \setminus \{y\}\). Then for all \(y \in \mathbb{R}_+^d:\) \(- \frac{\partial}{\partial y_d} \tilde G(y,x)= K(x,y)\) is harmonic \(x \in \mathbb{R}_+^d \setminus \{y\}\). By a limit argument, for all \(y \in \partial \mathbb{R}_+^d\), \(x \mapsto K(x,y)\) is harmonic for all \(x \in \mathbb{R}_+^d\).
      \item  A direct proof:
      \begin{align*}
        K(x,y) = \frac{2 x_d}{d |B_1|} \frac{1}{|x-y|^d}
      \end{align*}
      for all \(x \in \mathbb{R}_+^d\), \(y \in \partial \mathbb{R}_+^d\). For \(i \ne d\), \(x=(x_1, \dots, x_d)\), 
      \begin{align*}
        \partial_{x_i} K(x,y) &= \frac{2 x_d}{d |B_1|} \frac{(-d)}{|x-y|^{d+1}}\frac{x_i-y_i}{|x-y|} = \frac{-2x_d}{|B_1|}\frac{x_i - y_i}{|x-y|^{d+2}} \\
        \partial_{x_i}^2 K(x,y) &= - \frac{2 x_d}{|B_1|} \left[\frac{1}{|x-y|^{d+1}} - \frac{(d+2)}{|x-y|^{d+3}}(x_i-y_i) \frac{(x_i-y_i)}{|x-y|}\right] \\
        &= - \frac{2 x_d}{|B_1|} \left[\frac{1}{|x-y|^{d+1}} - \frac{(d+2)}{|x-y|^{d+4}}(x_i-y_i)^2\right]
      \end{align*}
      Moreover:
      \begin{align*}
        \partial_{x_d} K(x,y) 
        &= \frac{2}{d|B_1|} \frac{1}{|x-y|^d} + \frac{2 x_d}{d|B_1|}(-d) \frac{(x_d-y_d)}{|x-y|^{d+2}} \\
        (y_d = 0) \quad &= \frac{2}{d|B_1|} \frac{1}{|x-y|^d} + \frac{2x_d^2}{|B_1||x-y|^{d+2}} \\
        \partial_{x_d}^2K(x,y) &= \frac{-2}{|B_1|} \frac{(x_d-y_d)}{|x-y|^{d+2}} + \frac{4 x_d}{|B_1||x-y|^{d+2}} - \frac{2(d+2){|B_1|}} x_d^2 \frac{(x_d - y_d)}{|x-y|^{d+4}}
      \end{align*}
      Then: \begin{align*}
        \Delta_x K(x,y) 
        &= \sum_{i=1}^{d-1} \partial_{x_i}^2 K(x,y) + \partial_{x_i}^2 K(x,y) \\
        &= - \frac{2 x_d}{|B_1|} \left[\frac{d-1}{|x-y|^{d+2}} - (d+2) \sum_{i=1}^{d-1} \frac{(x_i - y_i)^2}{|x-y|^{d+4}} \right. \\
        &\quad + \left. \frac{1+2}{|x-y|^{d+2}} - \frac{(d+2)x_d(x_d-y_d)}{|x-y|^{d+4}}\right] \\
        &= - \frac{2 x_d}{|B_1|} \left[\frac{d+2}{|x-y|^{d+2}} - (d+2) \frac{1}{|x-y|^{d+4}} \left(\underbrace{\sum_{i=1}^d |x_i - y_i|^2}_{\smash{|x-y|^2}}\right)\right] = 0
      \end{align*}
      for all \(x \in \mathbb{R}_+^d\), \(y \in \partial \mathbb{R}_+^d\).
      Claim (exercise) for all \(x \in \mathbb{R}_+^d\), 
      \begin{align*}
        \int_{\partial \mathbb{R}_+^d} K(x,y) \, dy = 1 
      \end{align*}
      Consider 
      \begin{align*}
        u(x) = \int_{\partial \mathbb{R}_+^d} K(x,y) g(y) \, dy, \quad x \in \mathbb{R}_+^d
      \end{align*}
      Since \(g \in L^\infty(\mathbb{R}^{d-1}) = L^\infty(\partial \mathbb{R}_+^d)\) and \(K(x,y) \ge 0\), hence 
      \begin{align*}
        |u(x)| \le \left(\int_{\partial \mathbb{R}_+^d} K(x,y) \, dy\right) \|g\|_{L^\infty}
      \end{align*}
      Thus \(\|u\|_{L^\infty} \le \|g\|_{L^\infty}\). Moreover 
      \begin{align*}
        D_x^\alpha u(x) &= \int_{\partial \mathbb{R}_+^d} D_x^\alpha K(x,y) g(y) \, dy
      \end{align*}
      bounded, so \(u \in C^\infty(\mathbb{R}_+^d)\), \(x \mapsto K(x,y)\) is smooth as \(x \ne y\).
      \begin{align*}
        \Delta_x u(x) &= \int_{\partial \mathbb{R}_+^d} \underbrace{\Delta_x K(x,y)}_{= 0} g(y) \, dy = 0
      \end{align*}
      So \(u\) is harmonic in \(\mathbb{R}^d_+\). (\(\Rightarrow u \in C^\infty\) by Weyl's lemma). Take \(x_0 \in \partial \mathbb{R}_+^d\) and \(x \in \mathbb{R}_+^d\). Then:
      \begin{align*}
        |u(x)-g(x_0)| 
        &= \left| \int_{\partial \mathbb{R}_+^d} K(x,y)(g(y) - g(x_0)) \, dy \right| \\
        &\le \int_{\partial \mathbb{R}_+^d} K(x,y) |g(y)-g(x_0)| \, dy \\
        &= \underbrace{\int_{|y-x_0| \le L |x-x_0|}}_{(I)} + \underbrace{\int_{|y-x_0| > L|x-x_0|}}_{(II)}
      \end{align*}
      \begin{align*}
        (I) &= \int_{|y-x_0| \le L |x-x_0|} K(x,y) |g(y) - g(x_0)| \, dy \\
        &= \sup_{|y-x_0| \le L|x-x_0|} |g(y) - g(x_0)| \xrightarrow{x \to x_0} 0 \quad \forall L > 0
      \end{align*}
      \((II)\): If \(|y-x_0| > L|x-x_0|\), then \(|y-x| > \frac{1}{2} |y-x_0| > \frac{L}{2}|x-x_0|\) if \(L \ge 2\).
      \begin{align*}
        &\int_{|y-x_0| > |L|x-x_0|} K(x,y) |g(y) - g(x_0)| \, dy \le C \int_{y \in \partial \mathbb{R}_+^d} \frac{x_d}{|x_0-y|} \, dy \\
        &\quad C x_d \int_{\substack{z \in \mathbb{R}^{d-1}\\|z| > L|x-x_0|}} \frac{1}{|z|^d} \, dz = const. \frac{x_d}{L|x-x_0|} \le \frac{const.}{L} \xrightarrow{L \to \infty} 0
      \end{align*}
      \(x_d = |x_d - (x_0)_d| \le |x-x_0|\)
    \end{itemize}
  \end{proof}

  \section{Green's function for a ball}
  Let \(B = B(0,1)\). For all \(x \in B,\) for all \(y \in \bar B\) we want to find the corrector function \(\Phi_x(y)\) s.t.
  \[\begin{cases}
    \Delta_y \Phi_x(y) = 0 & \text{in } B \\
    \Phi_x(y) = G(y-x) & \text{on } \partial B
  \end{cases}\] 
  where for \(d \ge 3\): \(G(z) = \frac{1}{d(d-2)|B_1||z|^{d-2}}\).

  \begin{defn}[Reflection / Duality through the sphere \(\partial B\)]
    For all \(x \in \mathbb{R}^d \setminus \{0\}\) we define \(\tilde x = \frac{x}{|x|^2}\). Clearly we have for all \(x \in B\) that if \(|x| < 1\), then \(|\tilde x| = \left| \frac{x}{|x|^2}\right| = \frac{1}{|x|} > 1\), so \(\tilde x \notin \bar B\) 
    % and \((\tilde x)^\sim = \frac{\tild x}{|\tilde x|^2 = \frac{\frac{x}{|x|^2}}{\left| \frac{x}{|x|^2}}\right|} = \frac{\frac{x}{|x|^2}}{\left(\frac{1}{|x|}\right)^2} = x\)
  \end{defn}

  \begin{lem}
    For \(d \ge 3\) the function \(\Phi_x(y) = G(|x|(y-\tilde x))\) is a corrector function.
  \end{lem}

  \begin{proof}
    \begin{align*}
      \Phi_x(y) &= \frac{1}{d(d-2)|B_1| |x|^{d-2} |y-\tilde x|^{d-2}}
    \end{align*}
    for all \(x \in B, x \ne 0\), for all \(y \in \bar B\). Then clearly \(y \mapsto \Phi_x(y)\) is harmonic in \(B\) (Since \(\frac{1}{|z|^{d-2}}\) is harmonic in \(\mathbb{R} \setminus {0}\)). Let's check the boundary: Let \(y \in \partial B\), i.e. \(|y| = 1\). Then
    \begin{align*}
      ||x|(y-\tilde x)| 
      &= |x| \left|y- \frac{x}{|x|^2}\right|  \\
      &= |x| \sqrt{|y|^2 - 2 \frac{xy}{|x|^2} + \left| \frac{x}{|x|^2}\right|^2} \\
      &= \sqrt{|x|^2 |y|^2 - 2xy + 1} \\
      (|y| = 1) \quad &= \sqrt{|x|^2 - 2xy + |y|^2} 
      = |x-y|
    \end{align*}
    Thus \(\Phi_x(y) = G(|x| |y-\tilde x|) = G(y-x)\) for all \(0 \ne x \in B\), for all \(y \in \partial B\). Let's compute the Poisson kernel: If want to solve
    \[\begin{cases}
      - \Delta u = 0 & \text{in } B \\ u = g &\text{on } \partial B
    \end{cases}\]
    then \[u(x) = - \int_{\partial B} \frac{\partial \tilde G}{\partial \vec{n}_y}(x,y) g(y) dS(y).\]
    \(\tilde G(x,y) = G(y-x) - \Phi_x(y) = G(y-x)-G(|x|(y-\tilde x))\) for all \(x \in B \setminus \{0\}\), \(y \in \bar B\).
    \begin{align*}
      \frac{\partial \tilde G}{\partial \vec{n}_y} = \sum_{i=1}^d \partial_{y_i} \tilde G y_i
    \end{align*}
    Here \begin{align*}
      \partial_{y_i} \tilde G &= \partial_{y_i} G(y-x) - \partial_{y_i} [G(|x|(y-\tilde x))] \\
      &= \frac{- (y_i -x_i)}{d|B_1||y-x|^d} + \frac{y_i-\tilde x_i}{d|B_1||x|^{d-2}|y-\tilde x|^d} \\
      \Rightarrow \frac{\partial \tilde G}{\partial \vec{n}_y} &= \sum_{i=1}^d [\dots] y_i \\
      &= \frac{-y(y-x))}{d|B_1||y-x|^d} + \frac{y(y-\tilde x)}{d|B_1||x|^{d-2}|y-\tilde x|^d} \\
      &= \frac{1}{d|B_1||y-x|^d} (-y(y-x) + y(y-\tilde x) |x|^2) \\
      &= \frac{1}{d|B_1||y-x|^d}[-|y|^2 + xy + |y|^2 |x|^2 - xy] \\
      &= \frac{-1 + |x|^2}{d|B_1||y-x|^d}
    \end{align*}
    as \(y \in \partial B\).
  \end{proof}

  \begin{thm}[Poisson Formula for a Ball] Let \(B = B(0,1)\), \(g \in C(\partial B)\). Define for all \(x \in B\):
    \begin{align*}
      u(x) &= \int_{\partial B} K(x,y) g(y) \, dS(y),
    \end{align*}
    \(K(x,y) = - \frac{\partial \tilde G}{\partial \vec{n}_y}(x,y) = \frac{1-|x|^2}{d|B_1||y-x|^d}\) for all \(x \in B\), for all \(y \in \partial B\). Then \(u \in C^\infty(B)\), \(\Delta u = 0\) and for all \(x_0 \in \partial B\) we have \(\lim_{\substack{x \to x_0 \\ x \in B}} u(x) = g(x_0)\). This holds for all \(d \ge 2\).
  \end{thm}


  \begin{proof}
    We need to check: \begin{enumerate}
      \item For all \(y \in \partial B\), \(x \mapsto K(x,y)\) is harmonic in \(B\).
      \item \(\int_{\partial B} K(x,y) \, dS(y) = 1\) for all \(x \in B\) (exercise)
    \end{enumerate}
    Now for all \(x \in B\), for all \(y \in \partial B\):
    \begin{align*}
      K(x,y) &= \frac{1-|x|^2}{d|B_1||y-x|^d} \\
      \partial_{x_i} K(x,y) &= \frac{-2x_i}{d|B_1|} \frac{1}{|x-y|^d} - \frac{1-|x|^2}{|B_1|} \frac{x_i - y_i}{|x-y|^{d+2}} \\
      \partial_{x_i}^2 K(x,y) &= - \frac{2}{d|B_1|} \frac{1}{|x-y|^d} + \frac{2 x_i}{|B_1|} \frac{x_i - y_i}{|x-y|^{d+2}} + \frac{2 x_i}{|B_1|} \frac{x_i - y_i}{|x-y|^{d+2}} \\
      &\quad - \frac{1-|x|^2}{|B_1|} \frac{1}{|x-y|^{d+2}} + \frac{1+|x|^2}{|B_1|} (d+2) \frac{(x_i-y_i)^2}{|x-y|^{d+4}}\\
      \Delta_x K &= \sum_{i=1}^d \partial_{x_i}^2 K = - \frac{2}{|B_1|} \frac{1}{|x-y|^d} + \frac{4 x(x-y)}{|B_1||x-y|^{d+2}} \\
      &\quad - \frac{d(1-|x|^2)}{|B_1|} \frac{1}{|x-y|^{d+2}} + (d+2) \frac{1-|x|}{|B_1|} \frac{1}{|x-y|^{d+2}} \\
      &= \frac{2}{|B_1||x-y|^{d+2}} [-|x|^2 + 2xy - |y|^2 + 2|x|^2 - 2xy + 1 - |x|^2] \\
      &= \frac{2}{|B_1||x-y|^{d+2}} [-|x|^2 + 2xy - |y|^2 + 2|x|^2 - 2xy + 1 - |x|^2]
    \end{align*}
    \(1 - |y|^2 = 0\) as \(y \in \partial B\). Thus \(\Delta_x K(x,y) = 0\), for all \(x \in B\), for all \(y \in \partial B\).
    \begin{align*}
      |u(x)| &= \left| \int_{\partial B}K(x,y) g(y) \, dS(y) \right| \le \|g\|_{L^\infty(\partial B)}
    \end{align*}
    \(\int_{\partial B} K(x,y) , dS(y) = \|g\|_{L^\infty}\), 
    \begin{align*}
      \Delta_x u(x) &= \int_{\partial B} \underbrace{\Delta_x K(x,y)}_{= 0} g(y) \, dS(y) = 0
    \end{align*}
    Take \(x \in B\), \(x \to x_0 \in \partial B\).
    \begin{align*}
      |u(x) - g(x_0)| 
      &= \left| \int_{\partial B} K(x,y)(g(y) - g(x_0)) \, dS(y) \right| \\
      &\le \int_{A_1} + \int_{A_2} K(x,y) |g(y) - g(x_0)| \, dS(y),
    \end{align*}
    where
    \begin{align*}
      A_1 &= \{y \in \partial B \mid |y - x_0| \le |x-x_0|^\alpha\} \\
      A_2 &= \{y \in \partial B \mid |y-x_0| > |x-x_0|^2\}
    \end{align*}
    On \(A_1\) we have:
    \begin{align*}
      \int_{A_1} \dots \le \sup_{\substack{|z-x_0| \le |x-x_0|^\alpha \\ z \in \partial B}} \int_{\partial B} K(x,y) \, dS(y) \xrightarrow{x \to x_0} 0
    \end{align*}
    since \(G \in C(\partial B)\). On \(A_2\):
    \[
      |y-x_0| > |x-x_0|^\alpha
    \]
    \[
      \Rightarrow |y-x| \ge |y-x_0|-|x-x_0| \ge |x-x_0|^\alpha - |x-x_0| \ge \frac{1}{2} {x-x_0}^\alpha
    \]
    if \(\alpha < 1\) and \(|x-x_0|\) small. So we get 
    \begin{align*}
      K(x,y) &= \frac{1-|x|^2}{d|B_1||x-y|^d} \le C \frac{1-|x|^2}{|x-x_0|^{d\alpha}} \le C |x-x_0|^{1-d\alpha}
    \end{align*}
    Thus 
    \begin{align*}
      \int_{A_2} K(x,y) |g(y)-g(x_0)| \, dS(y) \le C \|g\|_{L^\infty} |x-x_0|^{1-d\alpha} \xrightarrow{x \to x_0} 0
    \end{align*}
    if \(1 - d\alpha > 0\) \(\Leftrightarrow\) \(\alpha < \frac{1}{d}\).
  \end{proof}

  \begin{ex}[E 7.2]
    Define \(\mathbb{R}_+^d = \{(x', x_d) \in \mathbb{R}^{d-1} \times \mathbb{R} \mid x_d > 0\}\). Let \(K(x,y) = \frac{2 x_d}{d |B_1| |x-y|^d}\) for all \(x \in \mathbb{R}_+^d, y \in \partial \mathbb{R}_+^d = \{(y', 0) \mid y' \in \mathbb{R}^{d-1}\} \simeq \mathbb{R}^{d-1}\). Prove
    \begin{align*}
      \int_{\partial \mathbb{R}_+^d} K(x,y) \, dS(y) = 1 \quad \forall x \in \mathbb{R}_+^d
    \end{align*}
  \end{ex}

  \begin{proof}[Solution]
    Denote \(x = (x', x_d)\), \(y = (y', 0)\), \(x', y' \in \mathbb{R}^{d-1}\), \(x_d > 0\).
    \begin{align*}
      \int_{\partial \mathbb{R}_+^d} K(x,y) \, dS(y) &= \int_{\mathbb{R}^{d-1}} \frac{2 x_d}{d |B_1|\left(|x'-y'|^2 + x_d^2\right)^{\frac{d}{2}}} \, dy' = \dots
    \end{align*}
    as \(|x-y| = |(x'-y', x_d)| = \sqrt{|x'-y'|^2 + x_d^2}\).
    \begin{align*}
      (y'-x' \mapsto y') \quad \dots &= \int_{\mathbb{R}^{d-1}} \frac{2 x_d}{d |B_1|\left(|y'|^2 + x_d^2\right)^{\frac{d}{2}}} \, dy' \\
      (y' = x_d z) \quad &= \int_{\mathbb{R}^{d-1}} \frac{2 x_d}{d |B_1| \left(x_d^2 (|z|^2 + 1)\right)^{\frac{d}{2}}} \left(x_d^{d-1}\right) \, dz \\
      &= \int_{\mathbb{R}^{d-1}} \frac{2}{d|B_1|\left(|z|^2 + 1\right)^{\frac{d}{2}}} \, dz \\
      &= \int_0^\infty \frac{2 \omega_{d-1}}{d|B_1|} \frac{1}{(r^2+1)^{\frac{d}{2}}} r^{d-2} \, dr \\
      &= \frac{2 \omega_{d-1}}{\omega_d} \int_0^\infty \frac{1}{\left(r^2+1\right)^{\frac{d}{2}}} r^{d-2} \, dr
    \end{align*}
    Set \(d = 2\): \(\omega_1 = 1, |\omega_2| = 2 \pi\) 
    \begin{align*}
      \frac{2}{\pi} \int_0^\infty \frac{1}{r^2+1} \, dr 
      &= \frac{2}{\pi} \int_0^{\frac{\pi}{2}} \frac{1}{(\tan t)^2 +1} [(\tan t)^2 +1] \, dt = 1
    \end{align*}
    we we set \(r = \tan t, t \in \left(0, \frac{\pi}{2}\right), \frac{dr}{dt} = (\tan t)' = 1 + (\tan t)^2\)
  \end{proof}
  For \(d = 3\):
  \begin{align*}
    \frac{2 \cdot 2 \pi}{4 \pi} \int_0^\infty \frac{1}{(r^2 + 1)^{\frac{3}{2}}} r \, dr
    &= \int_0^\infty \frac{d}{dr} \left[\frac{-1}{(r^2+1)^{\frac{1}{2}}}\right] \, dr = \left. \frac{-1}{(r^2 + 1)^{\frac{1}{2}}}\right]_0^\infty = 1
  \end{align*}

  \begin{ex}[7.3]
    Let \(g \in C(\partial \mathbb{R}_+^d) \cap L^\infty(\partial \mathbb{R}_+^d)\) (\(\partial \mathbb{R}_+^d \simeq \mathbb{R}^{d-1}\)).
    \begin{align*}
      u(x) &= \int_{\partial \mathbb{R}_+^d} K(x,y) g(y) \, dS(y)
      &K(x,y) &= \frac{2 x_d}{d |B_1| |x-y|^d}, x \in \mathbb{R}_+^d
    \end{align*}
    Prove that if \(g(y) = |y|\), if \(|y| \le 1\), then \(|\nabla u|\) is unbounded in \(B(0,r) \cap \mathbb{R}_+^d\) for all \(r > 0\).
  \end{ex}

  \begin{proof}[Solution]
    \begin{align*}
      \partial_{x_d} u(x) &= \int_{\partial \mathbb{R}_+^d} \partial x_d K(x,y) g(y) \, dy \quad \forall x \in \mathbb{R}_+^d \\
      &= \frac{2}{d |B_1|} \int_{\partial \mathbb{R}_+^d} \left[\frac{1}{|x-y|^d} - \frac{dx_d^2}{|x-y|^{d+2}}\right] g(y) \, dy \\
      &= \frac{2}{d |B_1|} \int_{\partial \mathbb{R}_+^d} \frac{1}{|x-y|^{d+2}} [|x-y|^2 - dx_d^2] g(y) \, dy \\
      &= \frac{2}{d|B_1|} \int_{\partial \mathbb{R}_+^d} \frac{1}{(|x'-y'|+x_d^2)^{\frac{d+2}{2}}} \left[|y'|^2 - (d-1) x_d^2\right] g(y) \, dy
    \end{align*}
    Assume that \(\partial_d u\) is bounded in \(B(0,r) \cap \mathbb{R}_+^d\) Then:
    \begin{align*}
      |u(0,x_d) - \underbrace{u(0,0)}_{g(0) = 0}| \le C |x_d|
    \end{align*}
    if \(x_d\) small. Consider:
    \begin{align*}
      \limsup_{x_d \to 0^+} \frac{u(0,x_d)}{x_d} 
      &= \limsup_{x_d \to 0^+} c \int_{\mathbb{R}^{d-1}} \frac{1}{(|y'|^2 + x_d^2)^{\frac{d}{2}}} g(y) \, dy' \\
      &\ge \int_{\mathbb{R}^{d-1}} \frac{1}{|y'|^d} g(y) \, dy 
      = \int_{|y'| \le 1} + \int_{|y'| > 1} \\
      &to \int_{\mathbb{R}^{d-1}} \frac{1}{|y'|^{d-1}} \, dy' = \infty \qedhere
    \end{align*}
  \end{proof}

  \begin{ex}[Bonus 7]
    Recall the Poisson kernel on a ball \(B(0,r) \subseteq \mathbb{R}^d\):
    \begin{align*}
      K(x,y) = \frac{r^2 - |x|^2}{d|B_1|r} \frac{1}{|x-y|^d}
    \end{align*}
    for all \(x \in B(0,r)\), \(y \in \partial B(0,r)\). Prove:
    \begin{align*}
      \int_{\partial B(0,r)} K(x,y) \, dS(y) = 1
    \end{align*}
    for all \(x \in B(0,r)\). (It suffices if you can prove \(d = 2\) and \(d = 3\))
  \end{ex}

  \section{Energy Method}

  Consider \(u \in C^2(\Omega)\) for \(\Omega \subseteq \mathbb{R}^d\) open, bounded and with \(C^1\) boundary and \begin{align*}
    \begin{cases}
      - \Delta u = f &\text{in } \Omega \\ u = g &\text{ on} \partial \Omega.
    \end{cases}
  \end{align*}
  Take \(\phi \in C_c^\infty(\Omega)\), then by integration by parts:
  \begin{align*}
    0 &= \int_\Omega(-\Delta u - f) \phi = \int_\Omega \nabla u \nabla \phi - \int_\Omega f \phi
  \end{align*}
  Key observation: This is the \emph{derivative} of the energy functional 
  \begin{align*}
    E(u) &= \frac{1}{2} \int_\Omega|\nabla u|^2 - \int_\Omega f u
  \end{align*}
  If \(u\) is a minimizer of \(E\), then it solves the equation \(-\Delta u = f\) in \(\Omega\). The boundary condition \(u = g\) does not appear on \(E\), but this is encoded in the set of \emph{admissible functions}. (The set of candidates of solutions). For the classical solutions, we have 
  
  \begin{thm}(Dirichlet's principle) Let \(\Omega \subseteq \mathbb{R}^d\) be open, bounded with \(C^1\)-boundary. Let \(f \in C(\bar \Omega)\) and \(g \in C(\partial B)\). Then the following statements are equivalent:
    \begin{enumerate}
        \item \(u \in C^2(\bar \Omega)\) solves \(\begin{cases}
          - \Delta u = f &\text{in } \Omega \\ u = g &\text{on } \partial \Omega
        \end{cases}\)
        \item \(u\) is a minimizer of the variational problem \(E = \inf_{v \in A} E(v)\), where 
        \[E(v) = \frac{1}{2} \int_\Omega |\nabla v|^2 - \int_\Omega f v,\]
        \(A = \{v \in C^2(\bar \Omega) \mid v = g \text{ on } \partial \Omega\}.\)
    \end{enumerate}
    Moreover there is at most a solution / minimizer (uniqueness). 
  \end{thm}

  \begin{proof}
    The result holds even for complex-valued functions. Let us write the proof for real-valued functions.
    \begin{itemize}
      \item[1. \(\Rightarrow\) 2.:] Let \(u \in C^2(\bar \Omega)\) be a solution of \(\begin{cases}
        - \Delta u = f &\text{in } \Omega \\ u = g &\text{on } \partial \Omega
      \end{cases}\). Then we prove \(E(u) \le E(v)\) for all \(v \in A\). If \(v \in A\), then \(u - v = 0\) on \(\partial \Omega\). Using this and \(-\Delta u = f\) in \(\Omega\), we have:
      \begin{align*}
        0 
        &= \int_\Omega (-\Delta u - f)\cdot(u-v) \, dy  \\
        \text{(Part. Int.)} \quad &= \int_\Omega \nabla u (\nabla u - \nabla v) \, dy - \int_\Omega f (u-v) \, dy \\
        &= \left[\frac{1}{2} \int_\Omega |\nabla u|^2 \, dy - \int_\Omega fu \, dy\right] - \left[\frac{1}{2} \int_\Omega |\nabla v|^2 \, dy - \int_\Omega f v \, dy\right] \\
        &\quad+ \frac{1}{2} \int_\Omega |\nabla u|^2 + \frac{1}{2} \int_\Omega  \\
        &= E(u) - E(v) + \frac{1}{2} \underbrace{\int_\Omega |\nabla u - \nabla v|^2}_{\ge 0}
      \end{align*}
    
      \(E(u) \le E(v)\), so \(u\) is a minimizer of \(\inf_{v \in A} E(v)\). Moreover \(u\) is the unique minimizer on \(A\). Since \(E(u) = E(v)\) we have \(\int_\Omega|\nabla (u-v)^2 = 0\), so \(u-v = const.\), so \(u-v = 0\) in \(\bar \Omega\).

      \item[2. \(\Rightarrow\) 1.:] Assume that \(u\) is a minimizer of \(\inf_{v \in A} E(v)\). Then \(E(u) \le E(v)\) for all \(v \in A\). Take \(\phi \in C_c^\infty(\Omega)\), then \(u + t \phi \in A\) for all \(t \in \mathbb{R}\).
      \begin{enumerate}[label=\(\Rightarrow\)]
        \item \(E(u) \le E(u + t \phi)\) for all \(t \in \mathbb{R}\)
        \item \(t \mapsto E(u + t \phi)\) has a minimizer at \(t = 0\)
        \item \(\begin{aligned}[t]
          0 &= \frac{d}{dt} E(u + t \phi)|_{t = 0}  \\
          &= \frac{d}{dt}\left. \left(\frac{1}{2} \int_\Omega|\nabla u + t \nabla \phi|^2 - \int_{\Omega} f(u + t \phi)\right)\right|_{t=0} \\
          &= \frac{d}{dt} \left. \left(\frac{1}{2} \int_\Omega |\nabla u|^2 + t^2 |\nabla \phi|^2 + 2 t \nabla u \nabla \phi - \int_{\Omega} f(u + t \phi)\right)\right|_{t=0} \\
          &\int_\Omega \nabla u \nabla \phi - \int_\Omega f \phi = \int_\Omega (- \Delta u - f) \phi
        \end{aligned}\) \\
        for all \(\phi \in C_c^\infty(\Omega)\). So \(- \Delta u - f = 0\) in \(\Omega\) and \(u = g\) since \(u \in A\).
      \end{enumerate}
    \end{itemize}
    
    Direct method of calculus of variations. Think \(f: \mathbb{R} \to \mathbb{R}\), \(f \in C(\mathbb{R})\), \(f(x) \to \infty\) as \(|x| \to \infty\). There is a \(x_0 \in \mathbb{R}\) s.t. \(f(x_0) = \inf_{x \in\mathbb{R}} f(x)\).
    \begin{enumerate}[label=Step \arabic*:]
      \item \(E = \inf_{x \in \mathbb{R}} f(x) > - \infty\)
      \item Take a minizing sequence \(\{x_n\} \subseteq \mathbb{R}\), \(f(x_n) \to E\). Up to a subsequence \(x_n \to x_0\) in \(\mathbb{R}\) (compactness)
      \item Lower semicontinuity \(E = \liminf_{n \to \infty} f(x_n) \ge f(x_0)\)
    \end{enumerate}
    If we apply the direct method to \(\inf_{v \in A}E(v)\), 
    \begin{align*}
      E(v) = \frac{1}{2} \int_{\Omega} |\nabla v|^2 - \int_\Omega f v,
    \end{align*}
    \(A = \{v \in C^2(\bar \Omega), v=g \text{ on } \partial \Omega\}\)

    \begin{enumerate}[label=Step \arabic*:]
      \item Easy \(E = \int_{v \in A} E(v) > - \infty\)
      \item There is a minimizing sequence \(\{v_n\} \subseteq A\) s.t. \(E(v_n) \to E\). We don't know if there is a subsequence of \(\{v_n\}\) that converges to \(u \in A\). The lack of compactness is a serious problem! We need to find the rigt set \(A\)! Consider again \begin{align*}
        \begin{cases}
          - \Delta u = f & \text{in } \Omega \\ u = g &\text{on } \partial \Omega
        \end{cases}
      \end{align*}
      Consider the simple case \(g = 0\). 
      \(\Delta u -f \) in \(\Omega\) \(\Leftrightarrow\) \(\nabla u \nabla \phi\)...
      The right set \(A\) should be \(A = \{v \mid \int_\Omega |\nabla v|^2 < \infty, v = 0 \text{ on } \partial \Omega\}.\) Rigorously we take \(W_0^{1,2}(\Omega) = \overline{C_c^\infty(\Omega)}W^{1,2}(\Omega)\) (Notation: \(H_0^1 = W_0^{1,2}, H^1 = W^{1,2}\)) Recall that \(W^{1,p}\) is a banach space with norm \(\|f\|_{W^{1,p}(\Omega)} = \|f\|_{L^p(\Omega)} + \|\nabla f\|_{L^p(\Omega)}\). We know that \(C_c^\infty(\Omega)\) is dense in \(W_{loc}^{1,p}(\Omega)\), i.e. for all \(u \in W^{1,p}_{loc}(\Omega)\) there is \(\|u_n\| \subseteq C_c^\infty\) s.t. \(u_n \to u\) in \(W^{1,p}(K)\) for all \(K \subseteq \Omega\) compact. However in general \(C_c^\infty(\Omega)\) is not dense in \(W^{1,p}(\Omega)\), i.e. \(W_0^{1,p}(\Omega) = \overline{C_c^\infty(\Omega)}W^{1,p}(\Omega) \subsetneq W^{1,p}(\Omega)\). Clearly \(W_0^{1,p}\) is a closed subspace of \(W^{1,p}(\Omega) \to W_0^{1,p}(\Omega)\)  is a Banach space with \(\|\cdot\|_{W^{1,p}(\Omega)}\). Why does \(W_0^{1,p}(\Omega)\) encode the \(0\)-boundary condition? Note that by definition for all \(u \in W_0^{1,p}(\Omega)\) there is a sequence \(\{u_n\} \subseteq C_c^\infty(\Omega)\), \(u_n \to u\) in \(W^{1,p}(\Omega)\) up to a subsequence \(u_n(x) \to u(x)\) for almost every \(x \in \Omega\). Note \(u_n|_{\partial \Omega} = 0 \nrightarrow u|_{\partial \Omega} = 0\) since \(\partial \Omega\) must be of 0-measure. \qedhere
    \end{enumerate}
  \end{proof}

  \begin{thm}[Characterization for \(W_0^{1,p}\)]\label{characterization-for-w01p}
    Let \(\Omega\) be open, bounded with \(C^1\)-boundary. Let \(u \in W^{1,p}(\Omega)\cap C(\bar \Omega)\). Then the following statements are equivalent:
    \begin{enumerate}[label=\alph*)]
      \item \(u = 0\) on \(\partial \Omega\)
      \item \(u \in W_0^{1,p}(\Omega)\)
    \end{enumerate}
    (Later we will remove the condition \(C(\bar \Omega)\) by introducing the \emph{Trace operator}.)
  \end{thm}

  
  \begin{rem}
    If \(d = 1\), it holds that \(W^{1,p} \subseteq C(\bar \Omega)\). Then the theorem gives a full characterization for \(W_0^{1,p}\), but if \(d \ge 2\), then in general \(W^{1,p} \nsubseteq C(\Omega)\). (later)
  \end{rem}


  \begin{proof}[Proof of theorem \ref{characterization-for-w01p}]\
    \begin{itemize}
      \item [a) \(\Rightarrow\) b):]
      \begin{lem}
        If \(u \in W^{1,p}(\Omega)\) and \(\supp u \subseteq \Omega\), then \(u \in W_0^{1,p}(\Omega)\).
      \end{lem}
      \begin{proof}
        Since \(K \coloneqq \supp u\) is a compact subset in \(\Omega\), we can find a function \(\chi \in C_c^\infty(\Omega)\), \(\chi = 1\) on \(K\). Moreover since \(u \in W^{1,p}(\Omega)\), there is a sequence \(\{u_n\} \subseteq C_c^\infty(\Omega)\) s.t. \(u_n \to u\) in \(W_{loc}^{1,p}(\Omega)\). We claim that \(\chi u_n \to \chi u\) in \(W_{loc}^{1,p}(\Omega)\). (exercise, \(\nabla (\chi u) = \nabla \chi u + \chi \nabla u\)). This implies \(\chi u_n \to u\) in \(W^{1,p}(\supp \chi)\), thus \(\chi u_n \to u\) in \(W^{1,p}(\Omega)\), so \(u \in W_0^{1,p}(\Omega)\).
      \end{proof}
      Assume \(u \in W^{1,p}(\Omega) \cap C(\bar \Omega)\) and \(u = 0\) on \(\partial \Omega\). Take \(G \in C^1(\mathbb{R})\) s.t. \(|G(t)| \le t\) for all \(t\), \(G(t) = t\) if \(t \ge 2\) and \(G(t) = 0\) if \(t \le 1\). Then let
      \begin{align*}
        u_n(x) &\coloneqq \frac{1}{n} G(n u(x)) \in W^{1,p}(\Omega) \\
        \overset{\text{(Chain-rule)}}{\Rightarrow} \quad \nabla u_n(x) &= \frac{1}{n} G'(nu(x)) n \nabla u(x) = G'(n u(x)) \nabla u(x)
      \end{align*}
      Moreover, \(u_n\) is compactly supported in \(\Omega\), so \(u_n \in W_0^{1,p}(\Omega)\) by the lemma and \(u_n \to u\) in \(W^{1,p}(\Omega)\), so \(u \in W_0^{1,p}(\Omega)\) since \(W_0^{1,p}\) is a closed space. Recall that \(u \in C(\bar \Omega)\) and \(u = 0\) on \(\partial \Omega\). Thus for all \(\epsilon > 0\) there is a compact \(K_\epsilon \subseteq \Omega\) s.t. \(\sup_{x \in \Omega \setminus K_\epsilon} |u(x)| \le \epsilon\). For any given \(n \in \mathbb{N}\), \(u_n(x) \ne 0\), so \(G(nu(x)) \ne 0\). This implies \(n|u(x)| > 1\), hence \(|u(x)| > \frac{1}{n}\). Thus \(u_n(x) = 0\) for all \(x\) such that \(|u(x)| \le \frac{1}{n}\), so \(\supp u_n \subseteq K_{\frac{1}{n}}\) compact in \(\Omega\). Next, let us check \(u_n \to u\) in \(W^{1,p}(\Omega)\).
      \begin{align*}
        \int_\Omega |u_n(x) - u(x)|^p \, dx \to 0
      \end{align*}
      since \(u_n(x) = \frac{1}{n} G(nu(x)) \xrightarrow{n \to \infty} u(x)\) for all \(x \in \Omega\) and \(|u_n(x)| \le \frac{1}{n} |G(nu(x))| \le \frac{1}{n} |nu(x)| \le |u(x)| \in L^p(\Omega)\).
      \begin{align*}
        \int_\Omega|\nabla u_n(x) - \nabla u(x)|^p \, dx 
        &= \int_\Omega |G'(nu(x)) - 1|^p |\nabla u(x)|^p \, dx \to 0
      \end{align*}
      as \(|G'(v(x))-1| \to 0\) for all \(x\) s.t. \(u(x) \ne 0\) and \(\nabla u(x) = 0\) on \(\{x \mid u(x) = 0\}\).  (exercise)
      \item [(b) \(\Rightarrow\) (a):] Let \(u \in W^{1,p}(\Omega) \cap C(\bar \Omega)\) and \(u \in W_0^{1,p}(\Omega)\). Then we prove \(u = 0\) on \(\partial \Omega\). Lets regard the case \(\Omega = Q_+ = \{(x',x_d) \mid \mathbb{R}^{d-1} \times \mathbb{R} \mid |x'|<1, 0 < x_d < 1\}\). 
      We prove that if \(u \in W_0^{1,p} \)
      \((Q_+) \cap C(\overline {Q_+)}\), then \(u = 0\) on \(Q_0 = \{(x', 0) \mid x' \in \mathbb{R}^{d-1}, |x'| < 1\}\). Since \(u \in W_0^{1,p}(Q_+)\) there is \(\{u_n\} \subseteq C_c^\infty(Q_+)\) s.t. \(u_n \to u\) in \(W^{1,p}(Q_+)\) for all \(x = (x', x_d) \in Q_+\), then:
      \begin{align*}
        u_n(x', x_d) = \underbrace{u_n(x', 0)}_{= 0} + \int_0^{x_d} \partial_d u_n(x', t) \, dt
      \end{align*}
      Hence 
      \begin{align*}
        |u_n(x', x_d)| &\le \int_0^{x_d} |\partial_d u_n(x', t) \, dt
      \end{align*}
      This implies:
      \begin{align*}
        &\int_{0 < x_d < \epsilon} \int_{|x'| \le 1} |u_n(x', x_d)| \, dx' \, dx_d \\
        &\quad \le \int_{0 < x_d < \epsilon} \int_{|x'| < 1} \left(\int_0^{x_d}|\partial_d u_n(x', t)| \, dt\right) \, dx' \, dx_d \\
        &\quad \le \epsilon \int_{|x'| < 1} \int_0^\epsilon |\partial_d u_n(x', t)| \, dx' \, dt
      \end{align*}
      \begin{align*}
        \Rightarrow \frac{1}{\epsilon} \int_0^\epsilon \int_{|x'| \le 1} |u_n(x', x_d)| \, dx' \, dx_d 
        &\le \int_0^\epsilon \int_{|x'| < 1} |\partial_d u_n(x', x_d)| \, dx' \, dx_d
      \end{align*}
      for all \(n \in \mathbb{N}\), \(\epsilon > 0\). Take now \(n \to \infty\), use \(u_n \to u\) in \(W^{1,p}(\Omega)\). Then:
      \begin{align*}
        \frac{1}{\epsilon} \int_0^\epsilon \int_{|x'| \le 1} |u(x', x_d)| \, dx' \, dx_d
        &\le \int_0^\epsilon \int_{|x'| < 1} |\partial_x u_n(x', x_d)| \, dx' \, dy
      \end{align*}
      for all \(\epsilon > 0\). Take \(\epsilon \to 0\):
      \begin{align*}
        \int_{|x'| \le 1} |u(x', 0)| \, dx' \le 0
      \end{align*}
      here we use \(u \in C(\bar \Omega)\) for the left side and Dominated Convergence for the right side. Thus \(u(x', 0) = 0\) for all \(|x'| \le 1\), i.e. \(u = 0\) on \(\partial \Omega\). Let's regard the general case: Let \(\Omega\) be open, bounded and with \(C^1\)-boundary. Lets define \emph{local charts} By definition for all \(x \in \partial \Omega\), there is a \(U_x\) open, such there is a bijecitve map \(h: U_x \to Q\), and \(h, h^{-1}\) are \(C^1\). Then clearly \(\partial \Omega \subseteq \bigcup_{x \in \partial \Omega} U_x\). Since \(\partial \Omega\) is compact, there is a finite subcover \(\{U_i\}_{i=1}^N\) s.t. \(\partial \Omega \subseteq \bigcup_{i=1}^N U_i\). We can find \(U_0\) open s.t. \(\bar U_0 \subseteq \Omega\) and \(\Omega \subseteq \bigcup_{i=0}^N U_i\).

      \begin{lem}
        There is a sequence \(\{x_i\}_{i=0}^N \subseteq C^\infty(\mathbb{R}^d)\) s.t.
        \begin{enumerate}
          \item \(\chi_i \ge 0\), \(\sum_{i=0}^N \chi_i = 1\) in \(\mathbb{R}^d\) (\(\{\chi_i\}\) is a partition of unity)
          \item For all \(i=1, \dots, N\), \(\supp \chi_i\) is in \(U_i\), i.e. \(\chi_i \in C_c^\infty(U_i)\).
          \item \(i=0\), \(\supp \chi_0 \subseteq \mathbb{R}^d \setminus \partial \Omega\) and \(\chi_0 \setminus \Omega \in C_c^\infty(\Omega)\). (exercise)
        \end{enumerate}
        Given \(u \in W_0^{1,p}(\Omega) \cap C(\bar \Omega)\). Then \(u = \sum_{i=0}^N \chi_i u\), where \(\chi_i \ge 0\), \(\chi_0 \in C_c^\infty(\Omega)\), \(\chi_i \in C_c^\infty(U_i)\). Since \(\chi_0 u\) is supported in a copact set inside \(\Omega\), \(\chi_0 u = 0\) on \(\partial \Omega\). It remains to show that for all \(i=1, \dots, N\), \(\chi_i u = 0\) on \(U_i \cap \partial \Omega\). Then \(\chi_i u(h^{-1}x) \in W_0^{1,p}(Q) \cap C(\bar \Omega)\). This implies \(\chi_i u (h^{-1}x) = 0\) on \(Q_0\), so \(\chi_i u(x) = 0\) on \(U_i \cap \partial \Omega\). 
        Why \(W_0^{1,p}(U_i \cap \Omega) \to W_0^{1,p}(Q_+)\). If \(v \in W_0^{1,p}(U_i \cap \Omega)\), then \(v_n \to v, v_n \in C_c^\infty\). \(v_n \circ h^{-1} \to v \circ h^{-1} \Rightarrow v \circ h^{-1} \in W_0^{1,p}(Q_+)\)
      \end{lem}
    \end{itemize}
  \end{proof}

  \begin{ex}[E 8.1]
    Let \(u \in W_{loc}^{1,1}(\mathbb{R}^d)\). Let \(B = u^{-1}(\{0\})\). Prove that \(\nabla u(x) = 0\) for a.e. \(x \in B\). 
  \end{ex}

  \begin{proof}[Solution]
    We have already seen that if \(f, g \in W_{loc}^{1,1}(\mathbb{R}^d)\), then \(\max(f,g) \in W_{loc}^{1,1}\). This implies that if \(u = u^+ - u^- \in W^{1,1}_{loc}\), then \(u^+, v^+ \in W_{loc}^{1,1}\) since \(u^+ = \max(u, 0)\) and \(u^- = \max(-u, 0)\). We have that \(\nabla u = \nabla u^+ - \nabla u^-\). Claim:
    \begin{align*}
      \nabla u^+ = \begin{cases}
        0 & u(x) \le 0 \\ \nabla u & u(x) > 0
      \end{cases} \quad 
      \nabla u^- = \begin{cases}
        0 & u(x) \ge 0 \\ \nabla u & u(x) < 0
      \end{cases}
    \end{align*}    
    \begin{align*}
      \int_{\mathbb{R}^d} (\partial_i u^+) \phi 
      &= - \int_{\mathbb{R}^d} u^+ \partial_i \phi 
      = -\int_{\{u(x) \le 0\}} 0 \partial_i \phi - \int_{\{u(x) > 0\}} u \partial_i \phi \\
      &= \int_{\{u(x) \le 0\}}0 \phi + \int_{\{u(x) > 0\}} \partial_i u \phi
    \end{align*}
    Alternative way: We showed for \(f \in W^{1,p}(\mathbb{R}^d)\), that 
    \begin{align*}
      \nabla |f|(x) &= \begin{cases}
        (\nabla f)(x) &f(x) > 0 \\ - (\nabla f)(x) & f(x) < 0 \\ 0 & f(x) = 0
      \end{cases}
    \end{align*}
    \(u_+ = \frac{1}{2}(u + |u|)\). Hence \(\nabla u_+ = \frac{1}{2}(\nabla u + \nabla |u|)\).
    Remark: If \(A \subseteq \mathbb{R}\) has measure zero, then \(\nabla u 1 _{\{u(x) \in A\}} = 0\) a.e. (Th. 6.19 Lieb-Loss Analysis)
  \end{proof}

  \begin{ex}[E 8.2]
    Let \(\Omega, U \subseteq \mathbb{R}^d\) be open, \(U \cap \Omega \ne \emptyset\), \(u \in W_0^{1,p}(\Omega)\), \(1 \le p < \infty\), \(\chi \in C_c^\infty(U)\). Prove: \(\chi u \in W_0^{1,p}(\Omega \cap U)\)
    Hint: Recall \(W_0^{1,p}(\Omega) = \overline{C_c^\infty(\Omega)}^{\|\cdot\|_{W^{1,p}}}\)
  \end{ex}

  \begin{proof}[Solution]
    By definition there is a sequence \((u_n)_{n \in \mathbb{N}} \subseteq C_c^\infty(\Omega)\) s.t. \(u_n \xrightarrow[n \to \infty]{\|\cdot\|_{W^{1,p}}} u\) , i.e.
    \begin{align*}
      \|u_n - u\|_p + \|\nabla u_n - \nabla u \|_{p} \xrightarrow{n \to \infty} 0.
    \end{align*}
    Define \(f_n: \mathbb{R}^d \to \mathbb{C}\), \(f_n(x) \coloneqq u_n(x) \chi(x)\). Note \(f_n \in C_c^\infty (\Omega \cap U)\) for all \(n \in \mathbb{N}\). Claim: \((f_n)_{n \in \mathbb{N}}\) is Cauchy with respect to \(\|\cdot \|_{W^{1,p}}\). Proof: 
    \begin{align*}
      \|f_n - f_m\|_p = \|\chi(u_n - u_m)\|_p \le \|\chi\|_\infty \underbrace{\|u_n - u_m\|_p}_{\xrightarrow{n,m \to \infty} 0} \xrightarrow{n,m \to \infty} 0 
    \end{align*}
    \(\nabla f_n = \nabla (\chi u_n) = (\nabla \chi) u_n + \chi \nabla u_n\)
    \begin{align*}
      \|\nabla f_n - \nabla f_m\|_p 
      &\le \|\nabla \chi(u_n - u_m)\|_p + \|\chi(\nabla u_n - \nabla u_m)\|_p \\
      &\le \|\nabla \chi\|_\infty \underbrace{\|u_n - u_m\|_p}_{\xrightarrow{n,m \to \infty} 0} + \underbrace{\|\chi\|}_{< \infty} \underbrace{\|\nabla u_n - \nabla u_m\|_p}_{\xrightarrow{n,m \to \infty}0}\xrightarrow{n,m \to \infty)} 0
    \end{align*}
    Thus, there is a \(f \in W_0^{1,p}(\Omega \cap U)\) s.t. \(\|f_n - f\|_{W^{1,p}} \xrightarrow{n \to \infty} 0\). We know:
    \begin{align*}
      \|f_n - \chi u\|_{L^p} &= \|\chi u_n - \chi u\|_p \\
      &\le \|\chi\|_\infty \underbrace{\|u_n - u\|_p}_{\to 0} \xrightarrow{n \to \infty} 0
    \end{align*}
    Since limits in \(L^p\) are unique, we get \(\chi u = f \in W_0^{1,p}(\Omega \cup U)\).
  \end{proof}

  \begin{ex}[E 8.3]
    Let \(\Omega, U \subseteq \mathbb{R}^d\) open and bounded, \(h: \bar U \to \bar \Omega\) \(C^1\)-diffeomorphisms, \(u \in W_0^{1,p}(\Omega)\), \(1 \le p < \infty\). Prove (\(x \mapsto u(h(x)) \in W_0^{1,p}(U)\).
  \end{ex}
  
  \begin{proof}[Solution]
    Since \(u \in W_0^{1,p}(\Omega)\) there is a sequence \((u_n)_{n \in \mathbb{N}} \subseteq C_c^\infty(\Omega)\) s.t. \begin{align*}
      \|u - u_n\|_p + \|\nabla u - \nabla u_n\|_p \xrightarrow{n \to \infty} 0
    \end{align*}
    Define for all \(n \in \mathbb{N}\) \(f_n: U \to \mathbb{C}\), \(f_n(x) = u_n(h(x))\). Note \(f_n \in C_c^1(U)\).
    Claim 1: \((f_n)_{n \in \mathbb{N}}\) is Cauchy wrt. \(\|\cdot\|_{W^{1,p}}\). \begin{align*}
      \|f_n - f_m\|_p^p &= \int_U |u_n(h(x)) - u_m(h(x))|^p \, dx \\
      &= \int_\Omega |u_n(y) - u_m(y)|^p \, dy \underbrace{|\det(Dh^{-1})(y)|}_{\le C < \infty} \xrightarrow{n,m \to \infty} 0 \\
      (\nabla f_n)(x) &= \nabla (u_n(h(x))) = (\nabla u_n) (h(x)) (Dh)(x)
    \end{align*}
    \begin{align*}
      \|\nabla f_n - \nabla f_m\|_p^p &= \int_U \left|\left[(\nabla u_n)(h(x)) - (\nabla u_m)(h(x))\right]\smash{\underbrace{(Dh)(x)}_{bdd.}}\right|^p \, dx \\
      &\le C \int_U \left| (\nabla u_n)(h(x)) - (\nabla u_m)(h(x)) \right|^p \, dx \\
      &= C \int_\Omega \left|(\nabla u_n)(y) - (\nabla u_m)(y)\right|^p \underbrace{|\det Dh^{-1}(y)|}_{\le \tilde C} \, dx
      \xrightarrow{n,m \to 0}0
    \end{align*}
    Claim 2: \(\|f_n - u \circ h\|_p \xrightarrow{n \to \infty} 0\).
    \begin{align*}
      \|f_n - u\circ h\|_p 
      &= \int_U |u_n(h(x)) - u(h(x))|^p \, dx \\
      &= \int_\Omega |u_n(y)- u(y)|^p \underbrace{|\det Dh^{-1}(y)|}_{\le C}\, dy \xrightarrow{n \to \infty} 0
    \end{align*}
    Conclusion: Since \((f_n)_{n \in \mathbb{N}} \subseteq C_c^1(U)\) is Cauchy with respect to \(\|\cdot\|_{W^{1,p}}\), there is a \(f \in W_0^{1,p}(U)\) s.t. \(f_n \smash{\xrightarrow[\|\cdot\|_{W^{1,p}}]{n \to \infty}f}\). Since limits in \(L^p\) are unique by claim 2 we get \(u \circ h = f \in W_0^{1,p}(U)\).
  \end{proof}

  \begin{ex}[E 8.4]
    Let \(\Gamma \subseteq \mathbb{R}^d\) be compact, \(\{U_i\}_{i=1}^N\) open s.t. \(\Gamma \subseteq \bigcup_{i=1}^N U_i\). Prove: There exists \(\{\chi_i\}_{i=0}^N \subseteq C^\infty(\mathbb{R}^d)\) s.t. \begin{enumerate}
      \item \(\chi_i \ge 0\) for all \(i\), \(\sum_{i=0}^N \chi_i = 1\) 
      \item \(\supp(\chi_i) \subseteq U_i\) for all \(i \in \{1, \dots, N\}\)
      \item \(\supp(\chi_0) \subseteq \mathbb{R}^d \setminus \Gamma\)
    \end{enumerate}
  \end{ex}

  \begin{proof}[Solution]
    WLOG assume that \(U_i \ne \emptyset\) for all \(i\). If \(\Gamma \ne 0\), then \(\chi_0 = 1\) does the job. Now suppose \(\Gamma \ne \emptyset\). Let \(\psi \in C_c^\infty(B_1(0))\), \(\psi \ge 0\), \(\int \psi = 1\), \(\psi|_{B_{\frac{1}{2}}(0)} > 0\) and for \(\epsilon > 0\) let \(\psi_\epsilon(x) = \frac{1}{\epsilon^d} \psi \left(\frac{x}{\epsilon}\right)\), so \(\int \psi_\epsilon = 1\). Define \begin{align*}
      \tilde d \coloneqq \sup \{\tilde{\tilde d} > 0 \mid \forall x \in \Gamma \exists i \in \{1, \dots, N\} s.t. \dist(x, U_i^c) \ge \tilde{\tilde d}\}
    \end{align*}
    Claim 1: \(\tilde d  > 0\) Suppose this was not true. Then there is a sequence \((x_n)_{n \in \mathbb{N}} \subseteq \Gamma\) s.t. for all \(i \in \{1, \dots, N\}\), 
    \begin{align*}
      \dist(x_n, U_i^c) < \frac{1}{n}
    \end{align*}
    Since \(\Gamma\) is compact, there is a subsequence, which we call \(x_n\) again, s.t. \(x_n \xrightarrow{n \to \infty} \bar x\) for asome \(\Gamma\). By \(\Gamma \subseteq \bigcup_{i=1}^N U_i\) there is a \(\epsilon_{\bar x} > 0\) and \(i \in \{1, \dots, N\}\) s.t. \(B_{\epsilon_{\bar x}}(\bar x) \subseteq U_i\ \lightning\).
    Define \(d \coloneqq \min\{\tilde d, 1\} > 0\). For all \(\epsilon > 0\), for all \(A \subseteq \mathbb{R}^d\): \((A)_\epsilon \coloneqq \{x \in A \mid \dist(x, A^c) \ge \epsilon\}\). for every \(i \in \{1, \dots, N\}\) define \(\phi_i: U_i \to [0,\infty)\) by
    \begin{align*}
      \phi_i(x) \coloneqq \mathbb{1}_{(U_i \cap B_R(0))_{\frac{d}{4}}} \star \phi_{\frac{d}{4}}
    \end{align*}
    Note \(\phi_i \in C_c^\infty(U_i)\) and \((U_i \cap B_R(0))_{\frac{d}{4}} \subseteq (\supp(\phi_i))^0\). Define \(\phi_0: \mathbb{R}^d \setminus \Gamma \to [0, \infty)\) by \(\phi_0(x) = \mathbb{1}_{(\mathbb{R}^1 \setminus \Gamma)_{\frac{d}{4}}} \star \psi_{\frac{d}{4}}\). Again, \(\phi_0 \in C^\infty(\mathbb{R}^d \setminus \Gamma)\), \(\supp(\phi_0))^0 \supseteq (\mathbb{R}^d \setminus \Gamma)_{\frac{d}{4}}\), \(\supp(\phi_0) \subseteq \mathbb{R}^d \setminus \Gamma\). 
    Claim 2: For all \(x \in \mathbb{R}^d\) there is a \(i \in \{0, 1, \dots, N\}: \phi_i (x) > 0\). Proof: By construction, we know for \(i \in \{1, \dots, N\}\) that \(\phi_i\) is \(>0\) on \((U_i \cap B_R(0))_{\frac{d}{4}}\). Moreover \(\phi_0 > 0\) on \((\mathbb{R}^d \setminus \Gamma)_{\frac{d}{4}}\). thus, we are done if we can show taht \(\bigcup_{i=1}^N (U_i \cap B_R(0))_{\frac{d}{4}} \cup (\mathbb{R}^d \setminus \Gamma)_{\frac{d}{4}} = \mathbb{R}^d\). Suppose there is a \(x \in \mathbb{R}^d \setminus A\). Then \(\dist(x, \Gamma) < \frac{d}{4}\). Since \(\Gamma \subseteq B_{\frac{R}{2}}(0)\) and \(R > 2\) and \(d \le 1\).
    \begin{align*}
      |x-0|
      &\le \dist(x, \Gamma) + \frac{R}{2}
      < \frac{d}{4} + \frac{R}{2} = R - \frac{d}{4} - \frac{R}{2} + \frac{d}{2} < R - \frac{d}{4} - \frac{2}{2} + \frac{1}{2} < R - \frac{d}{4}
    \end{align*}
    Thus \(x \in (B_R(c))_{\frac{d}{4}}\). Thus, we are done if we can show that \(x \in (U_i)_{\frac{d}{4}}\) for some \(i \in \{1, \dots, N\}\). Since \(\dist(x, \Gamma) < \frac{d}{4}\), there is a \(y \in \Gamma\) s.t. \(|x-y| < \frac{d}{4}\). By definition of \(\tilde d\) there is a \(i \in \{1, \dots, N\}\) s.t. \(\dist(y, U_i^c) \ge \tilde d \ge d\), i.e. for all \(z \in U_i^c\) we have \(|y-z| \ge d\). We get 
    \begin{align*}
      |x-z| \ge |\underbrace{|x-y|}_{< \frac{d}{4}}-\underbrace{|y-z|}_{\ge d}| \ge \frac{3d}{4} < \frac{d}{4}
    \end{align*}
    This implies \(\dist(x, U_i^c) > \frac{d}{4}\), so \(x \in (U_i)_{\frac{d}{4}}\) \(\lightning\).
    Define for all \(i \in \{0, \dots, N\}: \chi_i: \mathbb{R}^d \to [0, \infty)\) by 
    \begin{align*}
      \chi_i(x) &= \frac{\phi_i(x)}{\sum_{j=0}^N \phi_j(x)}
    \end{align*}
    \(\chi_i\) is well-defined by Claim 2 and \(\chi_i \in C^\infty(\mathbb{R}^d)\). Also note that \(\sum \chi_i = 1\), \(\chi_i \ge 0\), which implies 1. Furthermore, since \(\supp(\phi_i) \subseteq U_i\), we have \(\supp(\chi_i) \subseteq U_i\) for all \(i \in \{1, \dots, N\}\), which implies 2. Finally, since \(\supp(\phi_0) \subseteq \mathbb{R}^d \setminus \Gamma\), we get \(\supp(\chi_0) \subseteq \mathbb{R}^d \setminus \Gamma\). This implies 3.
  \end{proof}

  \section{Variational problem for weak solutions}
  \begin{align*}
    \begin{cases}
      -\Delta u = f &\text{in } \Omega \\
      u = 0 &\text{on } \partial \Omega
    \end{cases}
  \end{align*}
    ("formally") for all \(\phi \in C_c^\infty(\Omega)\), then
    \begin{align*}
      \int_\Omega \nabla u \nabla \phi &= \int_\Omega f \phi
    \end{align*}
    if \(\nabla u \in L^2\), \(f \in L^2\). By a density argument:
    \begin{align*}
      \int_\Omega \nabla u \nabla \phi &= \int_\Omega f \phi
    \end{align*}
    for all \(\phi \in \overline{C_c^\infty(\Omega)}^{H^1(\Omega)} = H^1_0(\Omega)\).

  \begin{thm}[Poincare inequality]\label{Poincare-inequality}
    There is a \(C > 0\) s.t. 
    \begin{align*}
      C \int_\Omega|\nabla v|^2 \ge \int_\Omega |v|^2
    \end{align*}
    for all \(v \in H_0^1(\Omega)\).
  \end{thm}
  
  \begin{rem}
    \(H^1(\Omega)\) with \(\|v\|_{H^1(\Omega)} = \left(\|v\|_{L^2}^2 + \|\nabla v\|_{L^2}^2\right)^{\frac{1}{2}}\) is a Hilbert-Space. This implies that \(H_0^1(\Omega) \overset{(\text{closed})}{\subseteq} H^1(\Omega)\) is also a Hilbert space.
    By the Poincare inequality (\ref{Poincare-inequality}) we have for all \(v \in H_0^1(\Omega)\):
    \begin{align*}
      \|v\|_{H^1(\Omega)} \ge \|\nabla v\|_{L^2} \ge \frac{1}{2c} \|v\|_{L^2} + \frac{1}{2} \|\nabla v\|_{L^2} \ge \frac{1}{C^1} \|v\|_{H^1(\Omega)}
    \end{align*}
    We can think of \(H_0^1(\Omega)\) as a Hilbert space with \(\|v\|_{H_0^1(\Omega)} \coloneqq \|\nabla v\|_{L^2(\Omega)}\).
  \end{rem}

  \begin{proof}(Of the Poincare inequality (\ref{Poincare-inequality}))
    We need to prove: 
    \begin{align*}
      \exists C > 0: \quad C \int_\Omega |\nabla v|^2 &\ge \int_\Omega |v|^2 \quad \forall v \in H_0^1(\Omega)\\
      \Leftrightarrow \quad \exists C > 0: \quad C \int_\Omega |\nabla v|^2 &\ge \int_\Omega |v|^2 \quad \forall v \in C_c^\infty(\Omega)
    \end{align*}
    Assume by contradiction that this does not hold, i.e. there is no \(C > 0\) s.t. the statement holds. Thus there is a sequence \(\{v_n\} \subseteq C_c^\infty(\Omega)\) s.t. 
    \begin{align*}
      \int_\Omega |v_n|^2 = 1, \quad
      \int_\Omega |\nabla v_n |^2 \xrightarrow{n \to \infty} 0
    \end{align*}
    Since \(v_n \in C_c^2(\Omega)\) we can extend \(v_n\) by \(0\) outside \(\Omega\), so \(v_n \in C_c^\infty(\mathbb{R}^d)\). Then:
    \begin{align*}
      \int_{\mathbb{R}^d}|v_n|^2 = 1, \quad \int_{\mathbb{R}^d} |\nabla v_n|^2 \to 0, \quad \supp v_n \subseteq \Omega
    \end{align*}
    By the Fourier transform:
    \begin{align*}
      \int_{\mathbb{R}^d} |\hat v_n(k)|^2 \, dk = 1, \quad 
      \int_{\mathbb{R}^d} |2 \pi k|^2 |\hat v_n(k)|^2 \, dk \to 0, \quad 
      \supp v_n \subseteq \Omega  
    \end{align*}
    We prove that
    \begin{align*}
      \int_{\mathbb{R}^d} |\hat v_n(k)|^2 \, dk \to 0
    \end{align*}
    We write
    \begin{align*}
      \int_{\mathbb{R}^d} |\hat v_n(k)| \, dk = \int_{|k| \le \epsilon} + \int_{|k| > \epsilon}
    \end{align*}
    First, for all \(\epsilon > 0\):
    \begin{align*}
      \int_{|k|> \epsilon} |\hat v_n(k)|^2 \le \int_{\mathbb{R}^d} \frac{|k|^2}{\epsilon^2} |\hat v_n(k)|^2 \, dk \xrightarrow{n \to \infty} 0
    \end{align*}
    Second:
    \begin{align*}
      \int_{|k|\le \epsilon} |\hat v_n(k)|^2 \, dk 
      &\le \left(\int_{|k| \le \epsilon} 1 \, dk\right)^{\frac{1}{q}} \left(\int_{|k|\le \epsilon} |\hat v_n(k)|^{2p} \, dk\right)^{\frac{1}{p}}, \quad 1 < p,q < \infty \\
      & \le C \epsilon^{\frac{d}{q}}\|\hat v_n\|_{L^{2p}}^2, \quad \frac{1}{p} + \frac{1}{q} = 1 \text{ and } 1 \le r \le 2
    \end{align*}
    Moreover, since \(\Omega\) is bounded, 
    \begin{align*}
      \|v_n\|_{L^r} &\le \left(\int_\Omega |v_n|^r\right)^{\frac{1}{r}} \le \|1_\Omega\|_{L^s} \|v_n\|_{L^2}^{1 - \theta} \le C_\Omega \quad \forall 1 \le r \le 2.
    \end{align*}
    Thus we can take \(r < 1\) but close to \(1\). Then \(p\) is sufficiently large, so \(q\) is close to \(1\). Then 
    \begin{align*}
      \int_{|k| \le \epsilon} |\hat v_n(k)|^2 \le C \epsilon^{\frac{d}{q}} \|\hat v_n\|_{L^{2p}}^2 \le C \epsilon^{\frac{d}{q}} \|v_n\|_{L^r}^2 \le C \epsilon^{\frac{d}{q}}
    \end{align*}
    Conclusion:
    \begin{align*}
      \int_{\mathbb{R}^d} |\hat v_n(k)|^2 
      = \int_{|k| \le \epsilon} + \int_{|k| > \epsilon} \le C \epsilon^{\frac{d}{q}} + \int_{|k| > \epsilon} \xrightarrow{n \to \infty} C \epsilon^{\frac{d}{q}} \xrightarrow{\epsilon \to 0} 0
    \end{align*}
    which contradicts to the assumtion \(\|\hat v\|_{L^2} = \|v\|_{L^2} = 1\).
  \end{proof}

  \begin{ex}
    Let \(\Omega \subseteq \mathbb{R}^d\) be open, bounded with \(C^1\)-boundary. Let \(u \in W^{1,p}(\Omega)\), for some \(1 \le p < \infty\). Then the following is equivalent:
    \begin{enumerate}[label=\alph*)]
      \item \(u \in W^{1,p}_0(\Omega)\)
      \item \(\tilde u(x) = \begin{cases}
        u(x) & x \in \Omega \\ 0 & x \in \mathbb{R}^d \setminus \Omega
      \end{cases} \in W^{1,p}(\mathbb)\)
    \end{enumerate}
  \end{ex}

  \begin{thm}[Dirichlet, Riemann, Poincare, Hilbert]\label{Dirichlet, Riemann, Poincare, Hilbert}
    Let \(\Omega \subseteq \mathbb{R}^d\) be open, bounded with \(C^1\)-boundary. Let \(f \in L^2(\Omega)\). Then there exists a unique solution \(u \in H_0^1(\Omega)\) of the variational problem 
    \begin{align*}
      \int_\Omega \nabla u \nabla \phi &= \int_\Omega f \phi
    \end{align*}
    for all \(\phi \in H_0^1(\Omega)\). (\(\Rightarrow - \Delta = f\) in \(D'(\Omega)\)). Moreover, \(u\) is the unique minimizer of 
    \begin{align*}
      \inf_{v \in H_0^1(\Omega)} \left(\frac{1}{2} \int_\Omega |\nabla v|^2 - \int_\Omega f v\right)
    \end{align*}
  \end{thm}

  \begin{proof}
    Let us prove that there is a solution \(u \in H_0^1(\Omega)\) for \(\inf_{v \in H_0^1(\Omega)} E(v)\), \(E(v) = \frac{1}{2} \int_\Omega|\nabla v|^2 - \int_\Omega f v\).
    \begin{itemize}
      \item[Step 1:] We prove \(E > - \infty\). Take \(v \in H_0^1(\Omega)\). By the Poincare and Hölder inequalites:
      \begin{align*}
        E(v) &= \frac{1}{2} \int_\Omega |\nabla v|^2 - \int_\Omega f v \\ 
        &\ge \frac{1}{2C} \|v\|_{L^2(\Omega)} - \|f\|_{L^2(\Omega)} \|v\|_{L^2(\Omega)} \\
        &\ge \frac{1}{2C} \|v\|^2_{L^2(\Omega)} - \left(\frac{1}{4C} \|v\|_{L^2(\Omega)}^2 + C \|f\|_{L^2(\Omega)}^2\right) \\
        &\ge - C \|f\|_{L^2(\Omega)}^2 > - \infty
      \end{align*}
      We can also bound:
      \begin{align*}
        E(v) &= \frac{1}{2} \int_\Omega|\nabla v|^2 - \int_\Omega f v \\
        &\ge \frac{1}{4} \int_\Omega |\nabla v|^2 - \frac{1}{4C} \int_\Omega |v|^2 - \|f\|_{L^2} \|v\|_{L^2} \\
        &\ge \frac{1}{4} \int_\Omega |\nabla v|^2 - C \|f\|_{L^2}^2
      \end{align*}
      \item[Step 2:] We can take a minimizing sequence \(\{v_n\} \subseteq H_0^1(\Omega)\) s.t. \(E(v_n) \xrightarrow{n \to \infty} E\). Then:
      \begin{align*}
        \frac{1}{4} \int_\Omega |\nabla v_n|^2 \le E(v_n) + C \|f\|_{L^2}^2 \longrightarrow const.
      \end{align*} 
      So \(|\nabla v_n|\) is bounded in \(L^2(\Omega)\). We know that \(H_0^1(\Omega)\) is a Hilbert space with norm \(\|v\|_{H_0^1(\Omega)} = \|\nabla v\|_{L^2(\Omega)}\) (and the norm is equivalent to the \(H^1\)-norm). Thus \(\{v_n\}\) is bounded in \(H_0^1(\Omega)\). 

      \begin{rem}[Reminder from functional analysis]
        Let \(H\) be a Hilbert space. We say that \(v_n \to v\) if \(\|v_n - v\| \to 0\) and \(v_n \to v\) weakly in \(H\) if \(\langle v_n, \phi \rangle \to \langle v, \phi \rangle\) for all \(\phi \in H\).
      \end{rem}

      \begin{thm}[Banach-Alaoglu]\label{banach-alagoglu}
        If \(H\) is a Hilbert space and \(\{v_n\}\) is a bounded sequence, then there is a subsequence \(\{v_{n_k}\}\) s.t. \(v_{n_k} \to v\) weakly in \(H\).
      \end{thm}

      \begin{rem}
        \begin{itemize}
          \item \(v_n \to v\) in \(H\) iff \(f(v_n) \to f(v)\) for all \(f \in H^\star = \mathcal{L}(H, \mathbb{R})\).
          \item If \(v_n \to v\) in \(H\), then: \(\liminf_{n \to \infty} \|v_n\| \ge \|v\|\) (Fatous Lemma)
        \end{itemize}
        In fact, for all \(\phi \in H\) \(\langle v_n, \phi \rangle \to \langle v, \phi \rangle\) and \(|\langle v_n, \phi\rangle| \le \|v_n\| \|\phi\|\). This implies 
        \[\frac{|\langle v, \phi \rangle|}{\|\phi\|} \le \liminf_{n \to \infty} \|v_n\|.\] So we get 
        \[\|v\| = \sup_{\phi \ne 0} \frac{\langle v, \phi\rangle|}{\|\phi\|} \le \liminf_{n \to \infty} \|v_n\|\] 
      \end{rem}

      By the Banach-Alaoglu theorem, up to a subsequence, \(v_n \to u\) weakly in \(H_0^1(\Omega)\). We prove that \(u\) is aminimizer for \(\mathcal{E}\)
      \[E \longleftarrow \mathcal{E}(v_n) = \frac{1}{2} \int |\nabla v_n|^2 - \int f v_n\]
      \begin{itemize}
        \item Since \(v_n \to u\) in \(H_0^1(\Omega)\) we have that 
        \begin{align*}
          \liminf_{n \to \infty} \|v_n\|^2_{H_0^1(\Omega)} \ge \|u\|^2_{H_0^1(\Omega)}
        \end{align*}
        So we have 
        \begin{align*}
          \liminf_{n \to \infty} \int_\Omega|\nabla v_n|^2 \ge \int_\Omega |\nabla u|^2.
        \end{align*}
        \item Consider the functional \(\mathcal{L}: \phi \in H_0^1(\Omega) \to \int_\Omega f \phi\). We claim that \(\mathcal{L}\) is continuous. In fact:
        \begin{align*}
          |\mathcal{L}| = \left| \int_\Omega f \phi \right| \le \|f\|_{L^2}\|\phi\|_{L^2} \le C\|f\|_{L^2}\|\nabla f\|_{L^2} = C \|f\|_{L^2}\|\phi\|_{H_0^1(\Omega)}
        \end{align*}
        Thus from \(v_n \to v\) in \(H_0^1(\Omega)\) we get \(\mathcal{L}(v_n) \to \mathcal{L}(u)\), thus \(\int_\Omega f v_n \to \int_\Omega f u\).
      \end{itemize}
      Conclusion: \(E = \liminf \mathcal{E}(v_n) \ge \mathcal{E}(u)\), so \(u\) is a minimizer for \(\mathcal{E}\).
      \item[Step 3:] Uniqueness. If \(E\) has 2 minimizers \(u_1, u_2\) we can prove that \(u_1 = u_2\). This is because of the convexity:
      \begin{align*}
        0 &\ge \frac{\mathcal{E}(u_1) + \mathcal{E}(u_2)}{2} - \mathcal{E}\left(\frac{u_1 + u_2}{2}\right) \\
        &= \frac{1}{8} \left[2 \int_\Omega |\nabla u_1|^2 + 2 \int_\Omega |\nabla u_2|^2 - \int_\Omega |\nabla (u_1 + u_2)|^2\right] \\
        &= \frac{1}{8} \int_\omega|\nabla u_1 - \nabla u_2|^2 \ge 0
      \end{align*} 
      This implies that \(\nabla(u_1 - u_2) = 0\), so \(u_1 - u_2 = const = c_0\). Since \(u_1, u_2 \in H_0^1(\Omega)\), we have that \(u_1 - u_2 \in H_0^1(\Omega)\) and \(c_0 \in C(\bar \Omega)\). Hence \(c_0 = 0\) on \(\partial \Omega\), so \(c_0 = 0\). \qedhere
    \end{itemize}
  \end{proof}

  \begin{rem}
    We can also prove directly that there is a unique \(u \in H_0^1(\Omega)\) s.t. 
    \begin{align*}
      \int_\Omega \nabla u \nabla \phi = \int_\Omega f \phi \quad \forall \phi \in H_0^1(\Omega)
    \end{align*}    
    by Riesz theorem. So we get \(\langle u, \phi \rangle_{H_0^1(\Omega)} = \mathcal{L}(\phi)\).
  \end{rem}

  Recall the corrector function for the unit ball:
  \[\phi_x(y) = G(|x||y-\tilde x|), \quad \tilde x = \frac{x}{|x|^2}\]
  This is ok if \(x \ne 0\). When \(x \to 0\):
  \begin{align*}
    G(|x|(y-\tilde x)) = G(\underbrace{|x|y - \frac{x}{|x|}}_{|\cdot|\to 1}) G(z), \quad |z| = 1
  \end{align*}
  is well-defined as \(G\) is radial. \\
  Question: If \(u \in H^1(\Omega)\), then how can we define \(u|_{\partial \Omega}\)?

  \section{Theory of Trace}
  \begin{thm}[Trace Operator]\label{Trace Operator}
    Let \(\Omega \subseteq \mathbb{R^d}\) be open, bounded with \(C^1\) boundary. Then there is a unique linear bounded operator \(T: H^1(\Omega) \to L^2(\partial \Omega)\) such that
    \begin{itemize}
      \item If \(u \in H^1(\Omega) \cap C(\bar \Omega)\), then \(Tu = u|_{\partial \Omega}\) in the usual restriction sense.
      \item There is a \(C > 0\) s.t. \(\|Tu\|_{L^2(\partial \Omega)} \le C \|u\|_{H^1(\Omega)}\) for all \(u \in H^1(\Omega)\)
    \end{itemize}
  \end{thm}

  \begin{thm}
    If \(u \in H^1(\Omega)\), then \(u \in H_0^1(\Omega)\) is equivalent to \(Tu = 0\) in \(L^2(\partial \Omega)\). (\(H_0^1(\Omega) = T^{-1}(\{0\})\)). The we can discuss about 
    \[\begin{cases}
      - \Delta u = f & \text{in } \Omega \\ u|_{\partial \Omega} = g &\text{on } \partial \Omega
    \end{cases}\]
  \end{thm}

  \begin{lem}[Trace inequality on \(\mathbb{R}_+^d\)] if \(u \in C_c^\infty(\mathbb{R}^d)\), then:
    \begin{align*}
      \|u|_{\partial \mathbb{R}_+^d}\|_{L^2(\partial \mathbb{R}_+^d)} \le C \|u\|_{H^1(\mathbb{R}^d)} \quad \text{with } C > 0 \text{ independent of \(u\).}
    \end{align*}
  \end{lem}

  \begin{proof}
    \(x = (x', x_d) \in \mathbb{R}^{d-1} \times \mathbb{R}\). 
    \begin{align*}
      |u(x', 0)|^2 &= - \int_0^\infty \partial_d(|u(x', x_d)|^2) \, dx_d \\
      &= - \int_0^\infty 2\partial_d u(x', x_d) u(x', x_d) \, dx_d \\
      &\le \int_0^\infty \left[|\partial_d u(x', x_d)|^2 + |u(x', x_d)|^2\right] \, dx_d
    \end{align*}
    This implies:
    \begin{align*}
      \int_{\mathbb{R}^{d-1}} |u(x',0)|^2 \, dx'
      &\le \int_{\mathbb{R}^{d-1}} \left(\int_0^\infty [\dots] \, dx_d\right) \, dx' \\
      &= \int_{\mathbb{R}_+^d} \left[|\partial_d u|^2 + |u|^2\right] = \|u\|^2_{H^1(\mathbb{R}_+^d)} \qedhere
    \end{align*}
  \end{proof}
  
  \begin{cor}
    If \(u \in H^1(Q)\) and \(u\) is compactly supported, then:
    \[\|u\|_{L^2(Q_0)} \le \|u\|_{H^1(Q_+)}\]
  \end{cor}
  Here \begin{align*}
    Q &= \{x = (x', x_d) \in \mathbb{R}^{d-1} \times \mathbb{R} \mid |x'| < 1, |x_d| < 1\} \\
    Q_+ &= \{x = (x', x_d) \in Q \mid x_d > 0\} \\
    Q_0 &= \{x = (x', x_d) \in Q \mid x_d = 0\}.
  \end{align*}

  \begin{proof}
    We extend \(u\) by \(0\) outside of \(Q\), so \(u \in H^1(\mathbb{R}^d)\).
  \end{proof}

  \begin{thm}[Extension]\label{extension-theorem} If \(\Omega \subseteq \mathbb{R}^d\) is open, bounded with \(C^1\)-boundary, then there is a bounded linear operator \(B: H^1(\Omega) \to H^1(\mathbb{R}^d)\) s.t. 
  \begin{itemize}
    \item \(Bu|_\Omega = u\) for all \(u \in H^1(\Omega)\)
    \item \(\|Bu\|_{H^1(\mathbb{R}^d)} \le C \|u\|_{H^1(\Omega)}\) and \(\|Bu\|_{L^2(\mathbb{R}^d)}\le C \|u\|_{L^2(\Omega)}\).
  \end{itemize}
  \end{thm}

  \begin{proof}[Proof of Theorem \ref{Trace Operator}] Since \(\partial \Omega\) is \(C^1\) there are open sets \(\{U_i\}_{i=1}^N \subseteq \mathbb{R}^d\) such that \(\partial \Omega \subseteq \bigcup_{i=1}^N U_i\) and for all \(i\) there is a \(C^1\)-diffeomorphism \(h_i: U_i \to Q\) s.t. \(h_i(U_i) = Q\), \(h_i(U_i \cap \Omega) = Q_+\), \(h_i(U_i \cap \partial \Omega) = Q_0\). Then there exists a partition of unity \(\{\theta_i\}_{i=0}^N \subseteq C^\infty(\mathbb{R}^d)\) s.t.
    \begin{enumerate}
      \item \(\sum_{i=0}^N \theta_i = 1\) for all \(x \in \mathbb{R}^d\)
      \item For all \(i=1, \dots, N\): \(\theta_i \in C_c^\infty(U_i)\)
      \item \(\supp \theta_0 \subseteq \mathbb{R}^d \setminus \partial \Omega\) (in particular \(\theta_0|_\Omega \in C_c^\infty(\Omega)\))
    \end{enumerate}
    Then given \(u \in H^1(\Omega)\), we can write \(u = \sum_{i=0}^N u_i\), where \(u_i = \theta_i u\). By the extension theorem (\ref{extension-theorem}), \(u \longrightarrow \) extended to \(Bu \in H^1(\mathbb{R}^d)\), thus
    \begin{align*}
      Bu &= \sum_{i=0}^N \theta_i (Bu) = \sum_{i=0}^N v_i, \quad v_i = \theta_i(Bu)
    \end{align*}
    Then \(v_i \in H^1(\mathbb{R}^d)\) and \(v_i\) is compactly supported in \(U_i\) for all \(i=1, 2, \dots, N\) and \(\supp v_0 \subseteq \mathbb{R}^d \setminus \partial \Omega\), \(v_i \in H^1(\mathbb{R}^d)\) and compactly supported inside \(U_i\). This implies \(\tilde v_i(y) = v_i(h^{-1}_i(y)) \in H^1(Q)\) and compactly supported inside \(Q\), \(y \in Q\). Thus \(\|\tilde v_i\|_{L^2(Q_0)} \le C \|\tilde v_i\|_{H^1(Q_+)}\). So we have \(\|v_i\|_{L^2(\partial \Omega)} \le C \|\tilde v_i\|_{L^2(Q_0)} \le C' \|\tilde v\|_{H^1(Q_+)} \le C'' \|v_i\|_{H^1(U_i \cap \Omega)}\). Thus:
    \begin{align*}
      \|u\|_{L^2(\partial \Omega)} &= \left\| \sum_{i=1}^N v_i\right\|_{L^2(\partial \Omega)} \le \sum_{i=1}^N \left\|v_i\right\|_{L^2(\partial \Omega)} \le \sum_{i=1}^N C'' \|v_i\|_{H^1(U_i \cap \Omega)} \\
      &= C'' \sum_{i=1}^N \|\theta_i u\|_{H^1(\Omega)} \le C'' \sum_{i=1}^N C \|u\|_{H^1(\Omega)}
    \end{align*}
    This proof works for \(u \in C(\bar \Omega)\). This implies
    \begin{align*}
      \|u\|_{L^2(\partial \Omega)} \le C \|u\|_{H^1(\Omega)} \quad \text{for all } u \in H^1(\Omega) \cap C(\bar \Omega).
    \end{align*}
    This allows us to define 
    \begin{align*}
      T:  H^1(\Omega) &\longrightarrow L^2(\partial \Omega) \\
      u &\longmapsto u|_{\partial \Omega}
    \end{align*}
    by continuity. I.e. for all \(u \in H^1(\Omega)\) there is \(\{u_n\} \subseteq H^1(\Omega) \cap C(\bar \Omega)\) s.t. \(u_n \to u\) in \(H_0^1\). Then \(Tu_n \to T u\) in \(L^2(\partial \Omega)\).
  \end{proof}

  \begin{lem}[Extension for \(Q\)] Let \(u \in H^1(Q_+)\). Then we define \(Bu: Q \to \mathbb{R}\) by
    \begin{align*}
      Bu(x) = \begin{cases}
        u(x) & x \in Q_+ \\ -u(x', -x_d) & x \in Q_-
      \end{cases},
    \end{align*}
    \(x = (x, x_d)\). Then \(Bu \in H^1(Q)\) and \(Bu|_{Q^+} = u\), \(\|Bu\|^2_{L^2(Q)} = 2 \|u\|_{L^2(Q_+)}^2\), \(\|\nabla (Bu)\|_{L^2(Q)}^2 = \|\nabla u\|_{L^2(Q_+)}^2\)
  \end{lem}

  \begin{proof}
    It is obvious \(Bu|_{Q^+} = u\) and
    \begin{align*}
      \int_Q|Bu|^2 
      &= \int_{Q_+} |Bu|^2  \int_{Q_-}|Bu|^2 \\
      &= \int_Q |u|^2 + \int_{Q_- = \{(x,-x_d) \mid (x,x_d) \in Q_+\}} |u(x,-x_d)|^2 \\
      &= 2 \int_{Q_+} |u|^2
    \end{align*}
    We prove:
    \begin{align*}
      \nabla (Bu)(x) &= \begin{cases}
        \nabla u(x) & u \in Q_+ \\ \nabla u(x', -x_d) & u \in Q_-
      \end{cases}
    \end{align*}
    First, \(\partial_d Bu(x) = \partial_d u(x', -x_d)\) if \(x \in Q_-\). Take \(\phi \in C_c^\infty(Q)\), then:
    \begin{align*}
      \int_Q (Bu(x)) (\partial_d \phi) (x) \, dx 
      &= \int_{Q_+} u \partial_d \phi + \int_{Q_-} - u(x', -x_d) \partial_d [\phi(x', x_d)] \, dx \\
      (x \to -x_d) \quad &= \int_{Q_+} u \partial_d \phi + \int_{Q_+}[u(x', x_d) (\partial_d \phi) (x', -x_d)]\, dx \\
      &\overset{\mathclap{(\phi \notin C_c^\infty(Q_+))}}\approx \quad - \int_{Q^+} (\partial_d u) \phi(x) + \int_{Q_+}(\partial_d u (x', x_d)) \phi(x', -x_d) \, dx \\
      &= - \int_{Q_+} (\partial_d u) \phi(x) + \int_{Q_-} \partial_d u(x', -x_d) \phi(x', x_d) \, dx \\
      &= - \int_Q f \phi, \quad \text{where \(f(x) = \begin{cases}
        \partial_d u & x \in Q_+ \\ - \partial_d u(x',-x_d) & x \in Q_-
      \end{cases}\)} 
    \end{align*}
    We prove \(\int_{Q_+} u \partial_d \tilde \phi = - \int_{Q_+} (\partial_d u) \tilde \phi\) where \(\tilde \phi(x,x_d) = \phi(x,x_d) - \phi(x, -x_d)\), \(\tilde \phi \notin C_c^\infty(Q_+)\). Define \(\eta_\epsilon = 0\) when \(|x_d| \le \epsilon\), \(\eta_\epsilon = 1\) if \(|x_d| \ge 2 \epsilon\), \(\eta_\epsilon \in C^\infty\), \(\eta_\epsilon(x', x_d) = \eta_0(x', \frac{x_d}{\epsilon}), \eta_0 = \begin{cases}
      1 & |x_d| \ge 2 \\ 0 & |x_d| \ge 1\end{cases}\). We have 
      \begin{align*}
        \int_{Q_+} u \partial_d(\eta_\epsilon \tilde \phi) = - \int_{Q_+} \partial_d u(\eta_\epsilon \tilde \phi)
      \end{align*}
      We take \(\epsilon \to 0\), \begin{align*}
        \int_{Q_+} (\partial_d u) (\eta_\epsilon \tilde \phi) \to \int_{Q_+} (\partial_d u) \tilde \phi 
      \end{align*}
      by dominated convergence.
      \begin{align*}
        \int_{Q_+} u \partial_d(\eta_\epsilon \tilde \phi) &= \int_{Q_+} u(\partial_d \eta_\epsilon) \tilde \phi + \int_{Q_+} u \eta_\epsilon \partial_d \tilde \phi \\
        \int_{Q_+} u \eta_\epsilon \partial_d \tilde \phi &\xrightarrow{\epsilon \to 0} u \partial_d \tilde \phi
      \end{align*}
      by dominated convergence.
      \begin{align*}
        \left|\int_{Q_+} u(\partial_d \eta_\epsilon) \tilde \phi\right| &= \left|\int_{Q} u \frac{1}{\epsilon} (\partial_d \eta_0)\left(x,\frac{x_d}{\epsilon}\right) \tilde \phi\right| \\
        \left(\begin{aligned}
          &|\tilde \phi(x', x_d)| \\
        &= |\phi(x, x_d) - \phi(x,x_d)| \\
        &\le \|\partial_d \phi\|_{L^\infty}|x_d|
        \end{aligned}\right) \quad &\le \frac{1}{\epsilon} \|\partial_d \eta_0\|_{L^\infty} \int_{Q_+ \cap \{x_d \le 2 \epsilon\}}|u|\underbrace{|\tilde \phi|}_{\le C|x_d| \le C \epsilon} \\
        \text{(Dominated cv \(u \in L^1(Q_+)\))} \quad &\le C \int_{Q_+ \cap \{0 < x_d \le 2 \epsilon\}}|u|\xrightarrow{\epsilon \to 0} 0
      \end{align*}
      where \(u \in L^2(Q_+)\) because \(u \in H^1(Q_+)\).
  \end{proof}

  \begin{ex}[E. 9.1]
    Let \(\Omega\) be open, bounded with \(C^1\)-boundary. Let \(u \in H_0^1(\Omega)\), \(f \in L^2(\Omega)\). Show that the following statements are equivalent:
    \begin{enumerate}[label=\arabic*)]
      \item \(- \Delta u = f\) in \(D'(\Omega)\)
      \item \(\int \nabla u \nabla \phi = \int f \phi\) for all \(\phi \in H_0^1\)
      \item \(E = \inf_v \left(\frac{1}{2} \int_\Omega |\nabla v|^2 - \int_\Omega f v\right)\) 
    \end{enumerate}
  \end{ex}

  \begin{proof}[Solution]\
    \begin{itemize}
      \item[1) \(\Rightarrow\) 2)] From \(- \Delta u = f\) in \(D'(\Omega)\) we get that 
      \begin{align*}
        \int_{\Omega} u(-\Delta \phi) = \int_\Omega f \phi
      \end{align*}
      for all \(\phi \in C_c^\infty(\Omega)\). Claim: If \(u \in H_0^1, \phi \in C_c^\infty\), then 
      \begin{align*}
        \int_\Omega (-\Delta \phi) = \int_\Omega \nabla u \nabla \phi
      \end{align*}
      Density argument: \(u \in H_0^1 = \overline{C_c^\infty(\Omega)}^{\|\cdot\|_{H}}\), so there is a sequence \(\{u_n\} \subseteq C_c^\infty(\Omega)\) s.t. \(u_n \to u\) in \(H^1(\Omega)\). Since \(u_n, \phi \in C_c^\infty(\Omega)\), then by the integration by parts:
      \begin{align*}
        \int_\Omega u_n(-\Delta \phi) = \int_\Omega (\nabla u_n) \nabla \phi \forall n
      \end{align*}
      Take \(n \to \infty\), then, 
      \begin{align*}
        \int_\Omega u (-\Delta \phi) = \int_\Omega (\nabla u) \nabla \phi 
      \end{align*}
      as \(u_n \to u\) and \(\nabla u_n \to \nabla u\) in \(L^2\).
      Claim: If \(\int_\Omega \nabla u \nabla \phi = \int_\Omega f \phi\)for all \(\phi \in C_c^\infty(\Omega)\), then 
      \begin{align*}
        \int_\Omega \nabla u \nabla \phi = \int_\Omega f \phi 
      \end{align*}
      for all \(\phi \in H_0^1\). (Given \(\nabla u, f \in L^2\)). With density argument: For all \(\phi \in H_0^1\) there is a sequence \(\{\phi_n\}\subseteq C_c^\infty(\Omega)\) s.t. \(\phi_n \to \phi\) in \(H^1\). Then:
      \begin{align*}
        \int_\Omega \nabla u \nabla \phi_n = \int_\Omega f \phi_n
      \end{align*}
      for all \(n\). Take \(n \to \infty\):
      \begin{align*}
        \int \nabla u \nabla \phi = \int f \phi
      \end{align*}
      as \(\phi_n \to \phi\), \(\nabla \phi_n \to \nabla \phi\) in \(L^2\).
      \item[2) \(\Rightarrow\) 3)] We show \(E(u) \le E(v)\) for all \(v \in H_0^1\), i.e. 
      \begin{align*}
        \frac{1}{2} \int_\Omega |\nabla u|^2 - \int_\Omega f u \le \frac{1}{2} \int_\Omega |\nabla v|^2 - \int_\Omega f v 
      \end{align*}  
      for all \(v \in H_0^1\). Write \(v = u + w\), then:
      \begin{align*}
        E(v) &= \frac{1}{2} \int |\nabla v|^2 - \int f v \\
        &= \frac{1}{2} \int_\Omega |\nabla (u + w)|^2 - \int_\Omega f (u + w) \\
        &= \frac{1}{2} \int_\Omega \left[|\nabla u|^2 + |\nabla w|^2 + 2 \nabla u \nabla w\right] \int_\Omega(fu + fw)\\
        &= E(u) + \frac{1}{2} \int_\Omega|\nabla w|^2 + \left(\underbrace{\nabla u \nabla w - \int f w}_{\smash{= 0}}\right)
      \end{align*}
      as \(w = v-u \in H_0^1\) (by (2))
      \item[3) \(\Rightarrow\) 1)] 
      \begin{align*}
        E(u) \le E(u + t\phi)
      \end{align*}  
      for all \(\phi \in H_0^1\) (or \(C_c^\infty\)) for all \(t \in \mathbb{R}\). This implies:
      \[\frac{d}{dt} E(u + t \phi)|_{t = 0} = 0\]
      Here 
      \begin{align*}
        E(u + t \phi) &= \frac{1}{2} \int \underbrace{|\nabla ( u + t \phi)|^2}_{\mathclap{|\nabla u|^2 + t^2 |\nabla \phi|^2 + 2 t \nabla u \nabla \phi}} - \int f(u + t \phi) \\
        &= E(u) + t \left[\int_\Omega \nabla u \nabla \phi \int_\Omega f \phi\right] + t^2 \int_\Omega |\nabla \phi|^2
      \end{align*}
      This implies
      \begin{align*}
        \frac{d}{dt} E(u + t \phi)|_{t = 0} &= \int_\Omega \nabla u \nabla \phi - \int_\Omega f \phi
      \end{align*}
      Conclude:
      \begin{align*}
        \int_\Omega \nabla u \nabla \phi &= \int_\Omega f \phi 
      \end{align*}
      for all \(\phi \in H_0^1\) or \(C_c^\infty\) So we get 
      \begin{align*}
        \int_\Omega u (- \Delta \phi) = \int_\Omega f \phi
      \end{align*}
      for all \(\phi \in C_c^\infty\). so we can conclude:
      \begin{align*}
        - \Delta u = f 
      \end{align*}
      in \(D'(\Omega)\) \(\Rightarrow\) 1)
    \end{itemize}
  \end{proof}

  \begin{ex}[E 9.2]
    \begin{align*}
      Q = \{(x', x_d) \mid |x'| < 1, |x_d| < 1\}
    \end{align*}
    Given \(u \in H^1(Q_+)\), define \(Bu: Q \to \mathbb{R}\) as 
    \begin{align*}
      Bu (x) = \begin{cases}
        u(x) & x \in Q_+ \\ u(\tilde x) & x \in Q_-
      \end{cases},
    \end{align*}
    \(x = (x', x_d) \Leftrightarrow \tilde x = (x', -x_d)\), \(x \in Q_- \Leftrightarrow \tilde x \in Q_+\). In the lectures:
    \begin{align*}
      \partial_d(Bu)(x) = \begin{cases}
        \partial_d u(x) & x \in Q_+ \\ -(\partial_d u)(\tilde x) & x \in Q_-
      \end{cases}
    \end{align*}
    This implies \(\partial_d (Bu) \in L^2(Q)\).
    \begin{enumerate}
      \item For all \(i=1,\dots,d-1\), then:
      \begin{align*}
        \partial_i (Bu)(x) = \begin{cases}
          \partial_i u(x) & x \in Q_+ \\ \partial_i u(\tilde x) & x \in Q_-
        \end{cases}
      \end{align*}
      \item Example \(u \in H^2(Q_+)\) but \(Bu \notin H^2(Q)\).
    \end{enumerate}
  \end{ex}

  \begin{proof}[Solution]
    \begin{enumerate}
      \item For all \(\phi \in C_c^\infty(Q)\):
        \begin{align*}
          \int_Q Bu(x) \partial_i \phi(x) &= \int_{Q_+} u(x) \partial_i \phi(x) + \int_{Q_-} u(\tilde x) \partial_i \phi(x)
        \end{align*}
        Write \(\vec{n} = (n_1, \dots, n_d)\). Here:
        \begin{align*}
          \int_{Q_+} u(x) \partial_i \phi(x) \, dx &= \int_{Q_+} - \partial_i u(x) \phi(x) \, dx + \int_{\partial Q_+} u(x) \phi(x) n_i \, dS \\
          \int_{Q_-}u(x', -x_d) \partial_i \phi(x', x_d) \, dx' dx_d &= -\int_{Q_+} u(x', x_d) \partial \phi(x', -x_d) \, dx' \, dx_d \\
          &= \int_{Q_+} \partial_i u(x) \phi(\tilde x) - \int_{\partial Q_+} u \phi n_i \, dS \\
          &= \int_{Q_-} - \partial_i u(\tilde x) \phi(x)- \int_{\partial Q_+} u \phi n_i \, dS
        \end{align*}
        with \(d(-x_d) = d(x_d)\). Conclude:
        \begin{align*}
          \int_Q (Bu)(x) \partial_i \phi(x) \, dx 
          &= \int_{Q_+}(-\partial_iu)(x) \phi(x) + \int_{Q_-} (-\partial_i u) (\tilde x) \phi(x) \\
          &= \int_Q - h(x) \phi(x) \, dx, \quad h(x) = \begin{cases}
            \partial_i u(x), & x \in Q_+ \\ \partial_i u(\tilde x), & x \in Q_-
          \end{cases}
        \end{align*}
        for all \(\phi \in C_c^\infty(Q)\), so \(\partial_i (Bu) \in L^2\) for all \(i = 1, 2, \dots, d-1\). Thus \(Bu \in H^1(Q)\).
      \item 1D: Take \(Q_+(0,1), Q_-=(-1,0), Q_0= \{0\}, Q=(-1,1)\), \(u(x) = x\) in \(Q_+ = (0,1)\), \(Bu(x) = u(x) = -x\) if \(x \in Q_- = (-1,0)\), i.e. \(Bu(x) = |x|\) if \(x \in Q = (-1,1)\). We know 
      \begin{align*}
        (Bu)'(x) = \begin{cases}
          1 &  x \in (0,1) \\ -1 & x \in (-1,0)
        \end{cases} \in L^2(-1,1)
      \end{align*}
      i.e. \(B u \in H^1(Q)\).
      \[(Bu)''(x) = 2 \delta_0(x)\]
      in \(D'(Q)\) but \(\notin L^2(-1,1)\), i.e. \(Bu \notin H^2(Q)\).
    Question: Given \(u \in H^2(Q_+)\), can we find an extension \(Bu \in H^2(Q)\) Yes! E.g. \(u(x) = x\) in \((0,1)\), so \(Bu(x) = x\) in \((-1,1)\). In general: \(u \in H^2(Q) \leadsto \tilde u \in H^2(Q)\) but \(\nabla u = 0\) on \(\partial Q_+\). \qedhere
  \end{enumerate}
  \end{proof}

  \begin{ex}[Bonus 8]
    Assume \(u \in H^2(Q_+)\) and \(\begin{cases}
      u = 0 \\ \nabla u = 0 
    \end{cases}\text{on } \partial Q_+\)
    Prove that \(Bu \in H^2(Q)\). (Reflection extension) (Ok in 1D)
  \end{ex}
  
  \begin{rem}
    If \(u \in H^2(Q_+)\), then \(\nabla u \in H^1(Q_+)\), so \(\nabla u|_{\partial Q_+}\) by trace theory. In general: \(\Omega \subseteq \mathbb{R}^d, C^2\)-boundary condition, then the same result holds.
  \end{rem}

  \begin{rem}
    In 1D: \(\begin{cases}
      u \in H^2(0,1) \\ u(0) = 0 \\ u'(0) = 0
    \end{cases},\) \(u|_{Q_0} \in L^2(Q_0)\), 1D: \(Q_0 = \{0\}\). In general: If \(u \in H^1(0,1)\), then \(u(0)\) is determined by trace theory. If \(u \in H^2(0,1)\), \(u'(0)\) is determined. Sobolev:
    \begin{align*}
      H^1(0,1) \subseteq C([0,1]) \\ H^2(0,1) \subseteq C^1([0,1])
    \end{align*}    
  \end{rem}
  
  \begin{lem}[Poincare inequality]\label{Poincare inequality for L2}
    Let \(\Omega\) be open, bounded connected with \(C^1\)-boundary. Then for all \(g \in L^2(\partial \Omega)\) s.t. \(g \ne constant\) there is a \(C > 0\) s.t. 
    \[\|u\|_{L^2(\Omega)} \le C \|\nabla u\|_{L^2(\Omega)}\]
    for all \(u \in M\), where 
    \[M = \{v \in H^1(\Omega) \mid v|_{\partial \Omega} = g\}.\]
  \end{lem}

  \begin{proof}
    We assume that the statement does not hold true. Then there is a sequence \(\{u_n\} \subseteq H^1(\Omega)\), \(u_n|_{\partial \Omega} = g\) s.t.
    \[\|\nabla u_n\|_{L^2(\Omega)} \to 0, \quad \|u_n\|_{L^2(\Omega)} = 1.\]
    Since \(\{u_n\}\) is bounded in \(H^1(\Omega)\), by the Banach-Alaoglu theorem (\ref{banach-alagoglu}), up to a subsequence 
    \[u_n \to u_0 \quad \text{ weakly in } H^1(\Omega)\]
    Since \(\nabla u_n \to 0\) strongly in \(L^2\) and \(\nabla u_n \to \nabla u_0\) weakly in \(L^2\), we have \(\nabla u_0 = 0\), so \(u_0|_{\partial \Omega} = const.\) (here we need \(\Omega\) to be connected), so \(u_0|_{\partial \Omega} = const.\) On the other hand, note that \(M\) is convex and closed in \(H^1(\Omega)\) since the trace operator \(T: H^1(\Omega) \to L^2(\partial \Omega)\) is continuous. Therefore, \(M\) is also weakly closed in \(H^1(\Omega)\) by the Hahn-Banach theorem. Thus from \(\{u_n\} \subseteq M\), \(u_n \to u_0\) wekly in \(H^1(\Omega)\) we get that \(u_0 \in M\), so \(u_0|_{\partial \Omega} = g\). We get a contradiction since \(g \ne const\) 
  \end{proof}

  \begin{thm}[Solution for Poisson Equation with inhomogeneous boundary condition]\label{Solution for Poisson Equation with inhomogeneous boundary condition}
    Let \(\Omega\) be open, bounded with \(C^1\)-boundary. Let \(f \in L^2(\Omega)\), \(g \in L^2(\partial \Omega)\). There there is a unique \(u \in H^1(\Omega)\) s.t. 
    \begin{align*}
      \begin{cases}
        - \Delta u = f &\text{in } D'(\Omega) \\ u|_{\partial \Omega} = g &\text{on } \partial \Omega
      \end{cases}
    \end{align*}
    Here \(u|_{\partial \Omega} = T(u) \in L^2(\partial \Omega)\) is defined by the trace operator. Moreover if \(\Omega\) is connected and \(g \ne constant\), then \(u\) is the unique minimizer for the variational problem
    \[E = \inf_{v \in M} \frac{1}{2} \int_\Omega |\nabla v|^2 - \int_\Omega f v,\]
    where \(M = \{v \in H^1(\Omega), v|_{\partial \Omega} = g \text{ on }\partial \Omega\}\)
  \end{thm}

  \begin{proof}
    First let us assume that \(\Omega\) is connected and \(g \ne const\).
    \begin{enumerate}[label=Step \arabic*:]
      \item We prove that \(E = \int_{v \in M} E(v)\) has a minimizer. By Poincares Inequality (\ref{Poincare inequality for L2}), for all \(v \in M\):
      \begin{align*}
        E(v) 
        &= \frac{1}{2} \int_\Omega|\nabla v|^2 - \int_\Omega fv \\
        \text{(Hölder)}\quad&\ge \frac{1}{2} \|\nabla v\|_{L^2(\Omega)}^2 - \|f\|_{L^2(\Omega)} \|v\|_{L^2(\Omega)} \\
        \text{(Poincaré \ref{Poincare inequality for L2})}\quad&\ge \frac{1}{2} \|\nabla v\|_{L^2(\Omega)}^2 - C \|f\|_{L^2(\Omega)} \|\nabla v\|_{L^2(\Omega)} \\
        &\ge \frac{1}{4} \| \nabla v\|_{L^2(\Omega)} - C\|f\|_{L^2(\Omega)}
      \end{align*}
      Thus \(E = \inf_{v \in M} E(v) > - \infty\). Moreover, taking a minimizing sequence \(\{v_n\} \subseteq M\), \(E(v_n) \to E\), we find that \(\|\nabla v_n\|_{L^2(\Omega)}\) is bounded, and hence \(\|v_n\|_{H^1(\Omega)}\) is bouned (by Poincare inequality) again. By Banach-Alaoglu (\ref{banach-alagoglu}), up to a subsequence we have \(v_n \to u\) weakly in \(H^1(\Omega)\). Hence \begin{align*}
        \begin{cases}
          \limsup_{n \to \infty} \int_\Omega |\nabla v_n|^2 \ge \int_\Omega |\nabla u|^2 & \text{as } \nabla v_n \to \nabla u \text{ in } L^2 \\
          \int_\Omega v_n f \to \int_\Omega u f &\text{as } v_n \to u \text{ in } L^2
        \end{cases}
      \end{align*}
      Note that \(\{v_n\} \subseteq M\), \(v_n \to u\) in \(H^1(\Omega)\) and \(M\) is weakly closed in \(H^1(\Omega)\) (as argued in the proof of Poincare inequality), therefore \(u \in M\). This means that \(u\) is a minimizer for \(E = \inf_{v \in M} E(v)\).
      \item Now we prove that if \(u\) is a minimizer for \(E\), then \(- \Delta u = f\) in \(D'(\Omega)\). In fact, for all \(\phi \in C_c^\infty(\Omega)\) we have 
      \[E(u) \le E(u + t \phi) \quad \forall t \in \mathbb{R}\]
      because \(u + t \phi \in M\). So we get that 
      \begin{align*}
        0 
        &= \frac{d}{dt} E(u + t \phi)|_{t=0}
        = \int_\Omega \nabla u \nabla \phi - \int_\Omega f \phi
      \end{align*}
      Thus
      \begin{align*}
        \int_\Omega u(-\Delta \phi) 
        &= \int_\Omega \nabla u \nabla \phi,
        = \int_\Omega f \phi \quad \forall \phi \in C_c^\infty(\Omega).
      \end{align*}
      So \(- \Delta u = f\) in \(D'(\Omega)\).
      \item We prove that Poissons equation has at most one solution. Assume that \(u_1\), \(u_2\) are 2 solutions. Then \(u = u_1 - u_2\) solves 
      \begin{align*}
        \begin{cases}
          - \Delta u = 0 \text{ in } \Omega \\ u|_{\partial \Omega} = 0 \text{ on } \Omega
        \end{cases}
      \end{align*}
      so \(u = 0\).
      \item If \(g = c_0\) is a constant, then Poissons equation can be rewritten with \(\tilde u = u - c_0\):
      \begin{align*}
        &\begin{cases}
          - \Delta u = f &\text{in } \Omega \\ u|_{\partial \Omega} = c_0 &\text{on } \Omega
        \end{cases}
        &&\Leftrightarrow
        &&\begin{cases}
          - \Delta \tilde u = f &\text{in } \Omega \\\tilde u = 0 &\text{on } \Omega
        \end{cases}
      \end{align*}
      If \(\Omega\) is not connected, then by considering connected components of \(\Omega\) we can prove that Poisson's equation always has a unique solution (for all \(f \in L^2(\Omega), g \in L^2(\partial \Omega)\)).
    \end{enumerate}
  \end{proof}

  \section{Final Remarks}
  We can describe \(H_0^1(\Omega)\) as the kernel of the trace operator \(T: H^1(\Omega) \to L^2(\partial \Omega)\)

  \begin{thm}
    Let \(\Omega \subseteq \mathbb{R}^d\) be open, bounded with \(C^1\)-boundary. Then:
    \[H_0^1(\Omega) = \{u \in H^1(\Omega) \mid T(u) = 0 \text{ on } \partial \Omega\}\]
    Recall that if \(u \in H^1(\Omega) \cap C(\bar \Omega)\), then \(T(u) = u|_{\partial \Omega}\) is the usual restriction. In this case we recover a result proved before.
  \end{thm}

  \begin{proof}
    % We start with the easy direction
    % \[H_0^1(\Omega) \subseteq \{u \in H^1(\Omega) \mid T(u) = 0\}.\]
    % This follows from the facts that \(C_c^\infty(\Omega)\) is dense in \(H_0^1(\Omega)\) and the trace operator is continuous.
    % Now, let us consider the difficult direction: Given \(u \in H^1(\Omega)\) and \(T(u) = 0\), we need to show that \(u \in H_0^1(\Omega)\). Let us think of the case \(\Omega = \mathbb{R}_+^d\) for simplicity. Then by a density argument, take \(\{u_n\} \subseteq C_c^\infty(\mathbb{R}^d)\) s.t. \(u_n|_\Omega \to u\) in \(H^1(\Omega)\). Then \(T(u_n) \to T(u) = 0\) in \(L^2(\partial \Omega)\). We have for \(x=(x', x_d) \in \mathbb{R}^{d-1} \times [0,\infty)\):
    % \begin{align*}
    %   u_n(x', x_d) &= u_n(x', 0) + \int_0^{x_d} \partial_d u_n(x', t) \, dt \\
    %   \Rightarrow \quad |u_n(x', x_d)|^2 &\le 2 |u_n(x', 0)|^2 + 2 x_d \int_0^{x_d} |\partial_d u_n(x', t)|^2 \, dt \\
    %   (n \to \infty) \quad \int_{\mathbb{R}^{d-1}} |u_n(x', x_d)|^2 \, dx' &\le 2 \|T(u_n)\|^2_{L^2(\mathbb{R}^{d-1})}
    % \end{align*}
    % \dots
  \end{proof}

  Recall that the varionational characterization of the Poisson equation 
  \begin{align*}
    \begin{cases}
      - \Delta u = f &\text{in } \Omega \\ u = g &\text{on } \partial \Omega
    \end{cases}
  \end{align*}
  is 
  \begin{align*}
    \int_\Omega \nabla u \nabla \phi = \int_\Omega f \phi \quad \forall \phi \in M
  \end{align*}
  where \(M = \{v \in H^1(\Omega) \mid v = g \text{ on } \partial \Omega\}\)
  In fact, if \(u \in H^2(\Omega)\) and 
  \[\int_\Omega \nabla u \nabla \phi = \int_\Omega f \phi \quad \forall \phi \in H^1(\Omega)\]
  Then \(u\) satisfies the Neumann condition:
  \begin{align*}
    \frac{\partial u}{\partial n} = \nabla u \cdot \vec{n} = 0 \text{ on } \partial \Omega
  \end{align*}

  (justification ...) \\
  
  For the exercises of sheet 10: Let \(\Omega = (a,b) \subseteq \mathbb{R}\) be an open bounded interval. For every \(u \in H^1(\Omega)\) the values \(u(a)\) and \(u(b)\) are determined uniquely by trace theory, or by Sobolev's embedding theorem.
  Recall:
  If \(u \in H^1((a,b)) \leadsto \partial \Omega = \{a,b\}\) counting measure iff \(g \in L^2(\partial \Omega)\) i.e. \(g(a) = g(b)\) are \emph{well-defined}.
  
  \begin{ex}[E 10.1]
    \begin{enumerate}[label=\alph*)]
      \item Prove \(H^1(\mathbb{R}) \subseteq (C(\mathbb{R}) \cap L^\infty(\mathbb{R}))\) \\ Hint: You can use Fourier Transform
      \item \(H^1(\Omega) \subseteq C(\Omega)\)
    \end{enumerate}
  \end{ex}

  \begin{proof}[Solution]
    \begin{enumerate}[label=\alph*)]
      \item Let \(u \in H^1(\mathbb{R})\). Then \(u, u' \in L^2(\mathbb{R})\) \(\Leftrightarrow\) \(\hat u(k) (1 + |2 \pi k|) \in L^2(\mathbb{R})\). Thus:
      \begin{align*}
        u(x) 
        &= \int_{\mathbb{R}} \hat u(k) e^{2\pi ik x} \, dk \in C(\mathbb{R}) \cap L^\infty(\mathbb{R})
      \end{align*}
      if \(\hat u \in L^1(\mathbb{R})\). So we have to show \(\hat u \in L^1(\mathbb{R})\).
      \begin{align*}
        \int_{\mathbb{R}} |\hat u(k)|
        &= \int_{\mathbb{R}} \frac{|g(k)|}{1 + |2 \pi k|} \\
        &\le \left(\int_{\mathbb{R}}|g(k)|^2 \, dk\right) \left[\int_{\mathbb{R}} \left(\frac{1}{1 + |2 \pi k|}\right)^2 \, dk\right]^{\frac{1}{2}} < \infty
      \end{align*}
      \item Given \(u \in H^1(\Omega)\), then there is an extension \(\tilde u \in H^1(\mathbb{R})\). By a) \(\tilde u \in C(\mathbb{R})\), so \(u = \tilde u|_{\tilde \Omega} \in C(\bar \Omega)\).
      Remak: We have \(\|u\|_{L^\infty(\Omega)} \le C \|u\|_{H^1(\Omega)}\), where \(\Omega = (a,b)\) or \(\mathbb{R}\) (but only in 1D) \qedhere
    \end{enumerate}
  \end{proof}

  Recall: If \(\Omega \subseteq \mathbb{R}^d (d \ge 1)\) open, bounded with \(C^1\)-boundary. Then 
  \[\|u\|_{L^2(\Omega)} \le C \|\nabla u\|_{L^2(\Omega)} \quad \forall u \in H_0^1(\Omega)\]
  Actually the same bound holds if \(u \in H^1(\Omega)\) and \(u|_\Gamma = 0\) for an open subset \(\Gamma \subseteq \partial \Omega\). In 1D we have:


  \begin{ex}[E 10.2 (Poincare inequality)]
    Let \(u \in H^1(\Omega)\), \(u(a) = 0\).
    Prove that there exists a constant \(C > 0\) such that 
    \[\|u\|_{L^2(\Omega)} \le C \|u'\|_{L^2(\Omega)}\]
  \end{ex}

  \begin{proof}[Solution]
    Let \(u \in C^1(\bar \Omega)\) and \(u(a) = 0\). Then:
    \begin{align*}
      u(x) &= u(a) + \int_0^x u'(t) \, dt \quad \forall x \in (a,b) \\ 
      \Rightarrow \quad |u(x)| &\le \int_a^x |u'(t)|\, dt \le \int_a^b|u'(t)| \, dt = \|u'\|_{L^1(\Omega)} \le C\|u'\|_{L^2(\Omega)}
    \end{align*}
    as \(\Omega\) is bounded. This implies:
    \begin{align*}
      \frac{1}{C} \|u\|_{L^2(\Omega)} \le \|u\|_{L^\infty(\Omega)} \le C \|u'\|_{L^2(\Omega)}
    \end{align*}
    To extend this for \(u \in H^1(\Omega)\), we can use a density argument. More precisely, for all \(u \in H^1(\Omega)\) there is a sequence \(\{u_n\} \subseteq C^1(\bar \Omega)\) s.t \(u_n \to u\) in \(H^1(\Omega)\). Then:
    \[\|u\|_{L^2(\Omega)} = \lim_{n \to \infty} \|u_n\|_{L^2(\Omega)} \le C \lim_{n \to \infty} \|u'_n\|_{L^2(\Omega)} = C\|u'\|_{L^2(\Omega)}\]
    Recall: For all \(f \in W_{loc}^{1,1}(O)\) with \(O\) in \(\mathbb{R}^d\) we have
    \[
      f(x) - f(y) = \int_0^1\nabla f(y + t(x-y))(x-y) \, dt 
    \]
    if \(x,y \in O\), \(y + t(x-y) \in O\) for all \(t \in [0,1]\). For 1D: If \(u \in H^1(a,b)\):
    \[
      u(x) - u(y) = \int_y^x u'(t) \, dt \quad \forall x,y \in (a,b) \qedhere
    \]
  \end{proof}

  \begin{ex}[E 10.3 (Poincare inequality)]
    Let \(u \in H^2(\Omega)\) and \(f \in L^2(\Omega)\). Prove that the following statements are equivalent:
    \begin{enumerate}[label=\alph*)]
      \item \(u\) solves the equation:
      \[\begin{cases}
        - u'' = f & \text{in } D'(\Omega) \\ u'(0) = u'(1) = 0
      \end{cases}\]
      \item \[\int_\Omega u' \phi' = \int_\Omega f \phi\] for all \(\phi \in H^1(\Omega)\).
    \end{enumerate}
    Here \(u \in H^2(\Omega)\) \(\Rightarrow\) \(u' \in H^1(\Omega)\) \(\Rightarrow\) \(u'(0), u'(1)\) determined uniquely by trace theorem / Sobolev inequality \(H^1(\Omega) \subseteq C(\bar \Omega)\) 
  \end{ex}

  \begin{proof}[Solution]\
    \begin{itemize}
      \item [b) \(\Rightarrow\) a)] For all \(\phi \in C_c^\infty(\Omega)\):\begin{align*}
        \int_{\Omega} f \phi = \int_\Omega u' \phi' = - \int_\Omega u \phi''
      \end{align*}
      This implies \(- u'' = f\) in \(D'(\Omega)\) a.e. Thus for all \(\phi \in H^1(\Omega)\):
      \begin{align*}
        \int_\Omega f \phi = \int_\Omega - u'' \phi = \int_\Omega u'\phi' - [u' \phi]_a^b
      \end{align*}
      By b) we conclude \(0 = [u' \phi]_a^b = u'(b) \phi(b) - u'(a)\phi(a)\) for all \(\phi \in H^1(\Omega)\). We can choose \(\phi \in H^1(\Omega)\) s.t. \(\phi(a) = 0\), \(\phi(b) = 1\). This implies \(\phi'(b) = 0\). Similarly, we can chose \(\phi \in H^1(\Omega)\) s.t. \(\phi(a) = 1\), \(\phi(b) = 0\). This implies \(u'(a) = 0\).
      \item [a) \(\Rightarrow\) b)] From a) and Integration by parts:
      \begin{align*}
        \int_\Omega f \phi = \int_\Omega - u'' \phi = \int_\Omega u' \phi' - \underbrace{[u'\phi]_a^b}_{\mathclap{= 0 \text{ as } u'(a) = u'(b) = 0}}
      \end{align*}
      This implies:
      \begin{align*}
        \int_\Omega f \phi = \int_\Omega u' \phi' \quad \forall \phi \in H^1(\Omega) 
      \end{align*}\qedhere
    \end{itemize}
  \end{proof}

  
  \begin{ex}[E 10.4 (Robin boundary condition)]
    Let \(f \in L^2(\Omega)\).
    \begin{enumerate}[label=\alph*)]
      \item Prove that there exists a unique \(u \in M \coloneqq \{\phi \in H^1(\Omega) , u(a) = 0\}\) s.t.
      \[\int_\Omega u' \phi' = \int_\Omega f \phi \quad \forall \phi \in M\]
      \item Prove that the above function \(u\) is the unique solution to the equation
      \[\begin{cases}
        -u'' = f &\text{in } D'(\Omega) \\ u(a) = 0 & u'(b) = 0
      \end{cases}\]
    \end{enumerate}
  \end{ex}

  \begin{proof}[Solution]
    \begin{enumerate}[label=\alph*)]
      \item By 10.2 we have
      \begin{align*}
        \|\phi\|_{L^2(\Omega)} \le C \|\phi'\|_{L^2(\Omega)} \quad \forall \phi \in M
      \end{align*}
      Thus: \((M, \|\phi\|_M \coloneqq \|\phi'\|_{L^2(\Omega)})\) is a Hilbert space. More precisely, we know \((M, \|\cdot \|_M)\) is a closed subspace of \(H^1 \leadsto\) a Hilbert space. And \(\|\cdot\|_M\) is comparable to \(\| \cdot \|_{H^1}\). By Riesz representation theorem there is a unique \(u \in M\) s.t. \(\langle \phi , u \rangle_M = F(\phi)\) for all \(\phi \in M\). We use this for 
      \begin{align*}
        F(\phi) = \int_\Omega f \phi \quad \forall \phi \in M
      \end{align*}
      Here \(|F(\phi)| \le \|f\|_{L^2} \|\phi\|_{L^2}\).
      \item Let \(u \in M\) be the solution in (a) i.e. 
      \begin{align*}
        \int_\Omega f \phi = \int_\Omega u'\phi' \quad \forall \phi \in M
      \end{align*}
      Then we prove that \(u\) solves 
      \[\begin{cases}
        -u'' = f &\text{in } D'(\Omega) \\ u(a) = u'(b) = 0
      \end{cases}\]
      Since \(u \in M\) we have \(u \in H^1(\Omega)\) and \(u(a) = 0\). From 
      \begin{align*}
        \int_\Omega f \phi = \int_\Omega u' \phi' \quad \forall \phi \in M
      \end{align*}
      we get for all \(\phi \in C_c^\infty(\Omega)\):
      \begin{align*}
        \int_\Omega f \phi = \int_\Omega u' \phi' = \int_\Omega - u \phi''
      \end{align*}
      So we get \(- u'' = f\) in \(D'(\Omega)\). Since \(f \in L^2(\Omega)\) \(\Rightarrow\) \(u'' \in L^2(\Omega)\) \(\Rightarrow\) \(u \in H^2(\Omega)\) \(\Rightarrow\) \(u' \in H^1(\Omega)\) \(\Rightarrow\) \(u'(b)\) is uniquely determined. For all \(\phi \in M\):
      \begin{align*}
        \int_\Omega f \phi = \int_\Omega- u'' \phi = \int_\Omega u' \phi' - \left(u'(b) \phi(b) - u'(a) \phi(a)\right) \quad \text{as } \phi \in M
      \end{align*}
      and \(\int_\Omega f \phi = \int_\Omega u' \phi'\). This implies:
      \begin{align*}
        u'(b) \phi(b) = 0 \quad \forall \phi \in M
      \end{align*}
      Take \(\phi(x) = \frac{x-a}{b-a} \in M\), \(\phi(b) = 1\).
      Uniqueness of the solution: Take \(u \) s.t. 
      \[\begin{cases}
        - u'' = f &\text{in } D'(\Omega) \\ u(a) = u'(b) = 0
      \end{cases}\]
      This implies \(u \in H^2(\Omega)\). By integration by parts: For all \(\phi \in H^1(\Omega)\), \(\phi(a) = 0\).
      \begin{align*}
        \int_\Omega f \phi = \int_\Omega-u''\phi = \int u' \phi' \quad \forall \phi \in M
      \end{align*}
      Thus \(u \in M\) and 
      \[\int_\Omega f \phi = \int_\Omega u' \phi' \quad \forall \phi \in M. \qedhere\]
    \end{enumerate}
  \end{proof}

  \begin{ex}[Bonus 9]
    Prove that the solution \(u\) in Problem E 10.4 is the unique minimizer for the minimization problem:
    \[E = \inf_{v \in M} \left(\int_\Omega |v'|^2 - \int_\Omega f v\right)\]
  \end{ex}

  \chapter{Heat Equation}
  \[
    \begin{cases}
      \partial_t u = \Delta u &(x,t) \in \mathbb{R}^d \times (0, \infty) \\
      u=g &(x,t) \in \mathbb{R}^d \times \{0\}
    \end{cases}
  \]
  The fundamential solution is:
  \[
    \Phi(x,t) = \frac{1}{(4 \pi t)^{\frac{d}{2}}} e^{-\frac{|x|^2}{4t}}, \quad x \in \mathbb{R}^d, t > 0
  \]
  We have:
  \[
    \begin{cases}
      \partial_t \Phi = \Delta \phi &(x,t) \in \mathbb{R}^d \times (0,\infty) \\
      \int_{\mathbb{R}^d} \Phi(x,t) \, dx = 1 & \forall t > 0 \\
      \lim_{t \to 0} \Phi(x,t) = \delta_0(x) & \text{in } D'(\mathbb{R}^d)
    \end{cases}
  \]

  \begin{thm}
    If \(g \in C(\mathbb{R}^d) \cap L^\infty(\mathbb{R}^d)\), then 
    \[u(x,t) \coloneqq \int_{\mathbb{R}^d} \Phi(x-y,t) g(y) \, dy\]
    satisfies
    
    \begin{enumerate}[label=(\roman*)]
      \item \(u \in C^\infty(\mathbb{R}^d \times (0,\infty))\)
      \item \(\partial_t u = \Delta u\) for all \((x,t) \in \mathbb{R}^d \times (0,\infty)\)
      \item \(\lim_{t \to 0} u(x,t) = g(x)\) for all \(x \in \mathbb{R}^d\)
    \end{enumerate}
  \end{thm}

  \begin{nota}
    For functions of \((x,t)\) we introduce the following notation for different regularity in \(x\) and \(t\).
    \[f \in C_1^2 \Leftrightarrow f, D_x f, D_x^2 f, \partial_t f \in C\]
  \end{nota}

  \begin{thm}[Nonhomogeneous problem]
    Let \(f \in C_1^2(\mathbb{R}^d, [0,\infty))\) be compactly supported. Define
    \[u(x,t) = \int_0^t \int_{\mathbb{R}^d} \Phi(x-y, t-s) f(y,s) \, dy\, ds\]
    Then 
    \begin{enumerate}[label=(\roman*)]
      \item \(u \in C_1^2(\mathbb{R}^d \times (0,\infty))\)
      \item \(\partial_t u = \Delta u+ f\) for all \(x \in \mathbb{R}^d, t > 0\)
      \item \(\lim_{t \to 0} u(x,t) = 0\) for all \(x \in \mathbb{R}^d\).
    \end{enumerate}
  \end{thm}

  \begin{proof}
    We write
    \[
      u(x,t) = \int_0^t \int_{\mathbb{R}^d} \Phi(y,s) f(x-y, t-s) \, dy \, ds
    \]
    With the Leibniz integral rule we get
    \[
      \partial_t u(x,t) = \int_0^t \int_{\mathbb{R}^d} \Phi(y,s) \partial_t f(x-y,t-s) \, dy \, ds + \int_{\mathbb{R}^d} \Phi(y,s) f(x-y, 0) \, dy
    \]
    and 
    \[\partial_{ij} u(x,t) = \int_0^t \int_{\mathbb{R}^d} \Phi(y,s) \partial_{ij} f(x-y, t-s) \, dy.\]
    This shows that \(\partial_t u\), \(\partial_{ij} u\) are in \(C(\mathbb{R}^d \times (0, \infty))\). Next we calculate:
    \begin{align*}
      \partial_t u - \Delta u
      &= \int_0^t \int_{\mathbb{R}^d} \Phi(y,s) (\partial_t - \Delta_x) f(x-y, t-s) \, dy \, ds + \int_{\mathbb{R}^d} \Phi(y,s) f(x-y, 0) \, dy \\
      &= \underbrace{\int_\epsilon^t \int_{\mathbb{R}^d} \Phi(y,s) (\partial_t - \Delta_x) f(x-y, t-s) \, dy \, ds}_{\eqqcolon I_\epsilon} \\
      &\quad+ \underbrace{\int_0^\epsilon \int_{\mathbb{R}^d} \Phi(y,s) (\partial_t - \Delta_x) f(x-y, t-s) \, dy \, ds}_{J_\epsilon} \\ 
      &\quad+ \underbrace{\int_{\mathbb{R}^d} \Phi(y,s) f(x-y,0) \, dy}_K
    \end{align*}
    Then
    \begin{align*}
      |J_\epsilon| &\le \|(\partial_t - \Delta_x) f\|_{L^\infty} \int_0^\epsilon \int_{\mathbb{R}^d} \Phi(y,s) \, dy \, ds \le C \epsilon \xrightarrow{\epsilon \to 0} 0 \\
      I_\epsilon &= \int_\epsilon^t \int_{\mathbb{R}^d} \Phi(y,s) (-\partial_s - \Delta_y) f(x-y,t-s) \, dy \, ds \\
      \text{(Green (\ref{green-identities}))}\quad &= \int_{\epsilon}^t \int_{\mathbb{R}^d} \underbrace{(\partial_s - \Delta_y) \Phi(y,s)}_{= 0} f(x-y, t-s) \, dy \, ds \\
      &\quad - \left[\int_{\mathbb{R}^d} \Phi(y,s) f(x-y, t-s)\right]_{s=\epsilon}^{s=t}
    \end{align*}
    This implies:
    \begin{align*}
      I_\epsilon + K 
      &= \int_{\mathbb{R}^d} \Phi(y,\epsilon) f(x-y, t-\epsilon) \, dy \\
      &\quad \xrightarrow{\epsilon \to 0} \int_{\mathbb{R}^d} \delta_0(y) f(x-y, t) \, dy = f(x,t)
    \end{align*}
    Thus
    \[\partial_t u - \Delta u = f(x,t) \quad \forall (x,t) \in \mathbb{R}^d \times (0,\infty)\]
    Finally:
    \begin{align*}
      \|u(\cdot, t)\|_{L^\infty} &\le \|f\|_{L^\infty} \int_0^t \int_{\mathbb{R}^d} \Phi(y,s) \, dy \, ds
      = \|f\|_{L^\infty} t \xrightarrow{t \to 0} 0
    \end{align*}
  \end{proof}

  \begin{ex}
    If \(f,g\) are given as above, then 
    \[u(x,t) = \int_{\mathbb{R}^d} \Phi(x-y,t) g(y) \, gy + \int_0^t \int_{\mathbb{R}^d} \Phi(x-y,t-s) f(y,s) \, ds\]
  solves
  \[\begin{cases}
    \partial_t - \Delta u = f \\ u(\cdot, t) = g
  \end{cases}\]
\end{ex}

\begin{rem}[Duhamel formula]
  Consider the ODE \(\partial_t w(t) = Aw(t)\) for all \(A \in \mathbb{R}\). Then the solution is
  \[w(t) = e^{tA} w(0).\]
  More generally: If \(\partial_t w(t) = Aw(t) + f(t)\), then
  \begin{align*}
    &&\partial_t(e^{-tA} w(t)) &= e^{-tA}(\partial_tw(t) - Aw(t)) = e^{-tA}f(t) = e^{-tA} f(t) \\
    &\Rightarrow &e^{-tA} w(t) &= w(0) + \int_0^t e^{-sA} f(s) \, ds \\
    &\Rightarrow &w(t) &= e^{tA} w(0) + \int_0^t e^{(t-s)A}f(s) \, ds
  \end{align*}
  More generally, if \(A\) is an operator (independend of time) then:
 \begin{align*}
   &&\partial_tw(t) &= Aw(t) + f(t) \\
   &\Rightarrow &w(t) &= e^{tA} w(0) + \int_0^t e^{(t-s)A} f(s) \, ds
 \end{align*}
 Application: If \(A = \Delta\), then the operator \(e^{t\Delta}\) has kernel
 \[e^{t\Delta}(x,y) = \Phi(x-y,t) = \frac{1}{(4 \pi t)^{\frac{d}{2}}}e^{-\frac{|x-y|^2}{4t}}.\]
 This is called the \emph{heat kernel}.
\end{rem}

\begin{thm}[\(L^2\)-data]
  For every \(g \in L^2(\mathbb{R}^d)\), define 
  \[u(t,x) = \int_{\mathbb{R}^d} \Phi(x-y,t)g(y) \, dy\]
  Then \(u \in C^\infty(\mathbb{R}^d \times (0,\infty))\) and it solves the heat equation
  \[\begin{cases}
    \partial_t u = \Delta_x u & \mathbb{R}^d \times (0,\infty) \\
    \lim_{t \to 0} u(\cdot, t) = g &\text{in } L^2(\mathbb{R}^d)
  \end{cases}\]
\end{thm}

\begin{proof}
  Recall the heuristic computation from the heat equation using the Fourier transform 
  \begin{align*}
    &&\partial_t u(x,t) &= \Delta_x u(x,t) \\
    &\Leftrightarrow&\partial_t \hat u (k,t) &= -|2\pi k|^2 \hat u (k,t) \\
    &\Leftrightarrow&\partial_t(e^{t|2\pi k|^2} \hat u(k,t)) &= 0 \\
    &\Leftrightarrow&e^{t|2\pi k|^2} \hat u(k,t) &= \hat u(k,0) = \hat g (k) \\
    &\Leftrightarrow& \hat u(k,t) &= e^{-t|2 \pi k|^2} \hat g(k) = \hat \Phi(k,t) \hat g(k) = \widehat{\Phi \star g} \qedhere \\
    &\Leftrightarrow& u(x,t) &= \Phi \star g = \int_{\mathbb{R}^d} \Phi(x-y,t) g(y) \, dy
  \end{align*}
  Here we only need the direction \(\Leftarrow\) which is rigorous if \(g \in L^2(\mathbb{R}^d)\). From the Fourier transform, it is also easy to check that \(u(\cdot, t) \to g\) in \(L^2\) as \(t \to 0\) (exercise). To see the smoothness, note that for all \(t>0,\) and for all \(m \in \mathbb{N}\):
  \begin{align*}
    (1 + |2 \pi k|^m) \hat u(k,t) = \underbrace{(1+|2 \pi k|^m) e^{-t|2 \pi k|^2}}_{\in L^\infty} \underbrace{\hat g(k)}_{\in L^2} \in L^2
  \end{align*}
  This implies \(u(\cdot, t) \in H^m(\mathbb{R}^d)\) for all \(m \ge 1\), so \(u(\cdot, t) \in C^\infty(\mathbb{R}^d)\) by Sobolev embedding (see below). This argument can also be used to show that \(u \in C^\infty(\mathbb{R}^d \times (0,\infty))\) (exercise)
\end{proof}


\begin{thm}[Sobolev embedding] 
  If \(m > \frac{d}{2}\), then \(H^m(\mathbb{R}^d) \subseteq (C(\mathbb{R}^d) \cap L^\infty(\mathbb{R}^d))\).
\end{thm}

\begin{proof}
  We write for all \(u \in H^m(\mathbb{R}^d)\):
  \[\hat u(k) = \underbrace{\hat u(k) (1 + |2\pi k|^m)}_{\in L^2 \text{ as } u \in H^m} \underbrace{\frac{1}{1 + |2 \pi k|^m}}_{\in L^2 \text{ as } m > \frac{d}{2}}\]
  This implies \(\hat u(k) \in L^1(\mathbb{R}^d)\) and finally \(u = (\hat u)^\lor \in (C(\mathbb{R}^d) \cap L^\infty(\mathbb{R}^d))\).
\end{proof}


\begin{ex}[E 11.1]
  Let \(g \in L^2(\mathbb{R}^d)\), 
  \begin{align*}
    u(x,t) &= \int_{\mathbb{R}^d} \Phi(x-y, t)g(y) \, dy,&
    \Phi(x,t) &= \frac{1}{(4\pi t)^{\frac{d}{2}}} e^{- \frac{|x|^2}{4t}}
  \end{align*}
  be the fundamential solution of the heat equation
  \[\begin{cases}
    \partial_t u - \Delta_x u = 0 &\forall (x,t) \in \mathbb{R}^d \times (0,\infty) \\ u(x,t) \to g(x) &\text{as } t \to 0.
  \end{cases}\]
  
  Prove that
  \begin{enumerate}[label=\alph*)]
    \item \(u \in C^\infty(\mathbb{R}^d \times (0, \infty))\).
    \item \(\|u(\cdot, t) - g\|_{L^2(\mathbb{R}^d)} \smash{\xrightarrow{t \to 0^+}} 0\)
    \item If \(g \in H^1(\mathbb{R}^d)\), then \(\|u(\cdot, t) -g\|_{L^2(\mathbb{R}^d)} \le C \sqrt{t}\) as \(t \to 0^+\).
  \end{enumerate}
\end{ex}

\begin{proof}[Solution]
  \begin{enumerate}[label=\alph*)]
    \item We prove for all \(t > 0\):
    \[u(x,t) \in \bigcap_{m \ge 1} H^m(\mathbb{R}^d) \subseteq C^\infty(\mathbb{R}^d)\]
    We use the Fourier transform:
    \begin{align*}
      \hat \Phi(k,t) &= e^{-t|2 \pi k|^2}
    \end{align*}
    Recall \(\widehat{e^{-\pi |x|^2}} = e^{- \pi |k|^2}\). From this we get \(\widehat{e^{- \pi \lambda |x|^2}} = \lambda^{-\frac{d}{2}} e^{-\frac{\pi |k|^2}{\lambda}}\). Then:
    \begin{align*}
      \widehat{e^{-\frac{|x|^2}{4t}}} = \widehat{e^{-\pi \frac{1}{4 \pi t} |x|^2}} = \left(\frac{1}{4 \pi t}\right)^{-\frac{d}{2}} e^{- \pi |k|^2 4 \pi t} = (4 \pi t)^{\frac{d}{2}} e^{-t |2 \pi k|^2}
    \end{align*}
    Hence:
    \begin{align*}
      \hat u(k,t) = \hat \Phi(k,t) \hat g(k) = e^{-t|2 \pi k|^2} \hat g(k) \in L^1(\mathbb{R}^d, dk) \quad \forall t > 0
    \end{align*}
    This implies:
    \begin{align*}
      u(x,t) &= \int_{\mathbb{R}^d} e^{-t|2 \pi k|^2} \hat g(k) e^{2 \pi i kx} \, dk \quad \forall (x,t) \in \mathbb{R}^d \times (0,\infty)
    \end{align*}
    Consequently:
    \begin{align*}
      D_x^\alpha u(x,t) &= \int_{\mathbb{R}^d} \underbrace{e^{-t|2 \pi k|^2} \hat g(k) (2 \pi ik)^\alpha}_{L^1(\mathbb{R}^d, dk)} e^{2 \pi i k x} \, dk \in C(\mathbb{R}^d, (0,\infty)) \\
      D_t^\alpha u(x,t) &= \int_{\mathbb{R}^d} (-|2\pi k|^2)^\alpha e^{-t|2\pi k|^2} \hat g(k) e^{2 \pi ikx} \, dk \in C(\mathbb{R}^d, (0,\infty))
    \end{align*}
    Also:
    \begin{align*}
      &\partial_t u - \Delta_x u \\&\quad= \int_{\mathbb{R}^d} -|2 \pi k|^2 e^{-t|2 \pi k|^2} \hat g(k) e^{2 \pi i k x} \, dk + \int_{\mathbb{R}^d} e^{-t |2 \pi k|^2} \hat g(k) |2 \pi i k|^2 e^{ 2 \pi i k x} \, dk \\&\quad= 0
    \end{align*}
    \item Finally:
    \begin{align*}
      \int_{\mathbb{R}^d} |u(x,t) - g(x)|^2 &= \int_{\mathbb{R}^d}|\hat u(k,t) - \hat g(k)|^2 \, dk \\
      &= \int_{\mathbb{R}^d} \underbrace{|e^{-t |2 \pi k|^2}- 1|^2}_{\in [0,1]}\underbrace{|\hat g(k)|^2}_{\in L^1(\mathbb{R}^d)} \, dk \xrightarrow{t \to 0} 0
    \end{align*}
    by dominated convergence. Now,
    \begin{align*}
      \int_{\mathbb{R}^d} |u(x,t)|^2 \, dx &= \int_{\mathbb{R}^d} |\hat u(k,t)|^2 \, dk \\
      &= \int_{\mathbb{R}^d} \underbrace{e^{-2t|2 \pi k|^2}}_{\in [0,1] \text{ and } \xrightarrow{t \to 0} 0} |\hat g (k)|^2 \, dk \xrightarrow{t \to \infty} 0
    \end{align*}
    \item Assume \(g \in H^1(\mathbb{R}^d)\) \(\Leftrightarrow\) \(\int_{\mathbb{R}^d} (1 + |2 \pi k|^2)|\hat g(k)|^2 \, dk < \infty\). We claim for all \(s \ge 0\) that \(\left|1 - e^{-s}\right| \le \min(1,Cs) \le C \sqrt{s}\): We have that \(s \mapsto \left|\frac{1-e^{-s}}{s}\right|\) is bounded and continuous in \([0,1]\) as \(\left|\frac{1 - e^{-s}}{s}\right| \to 1\), so \(\frac{1-e^{-s}}{s}\le C\) for all \(s \in [0,1]\).
    \begin{align*}
      \int_{\mathbb{R}^d} |u(x,t) - g(x)|^2 \, dx 
      &= \int_{\mathbb{R}^d} \underbrace{\left|1 - e^{-t |2 \pi k|^2}\right|^2|}_{\le C(t |2 \pi k|^2)}\hat g(k)|^2 \, dk \\
      &\le C \int_{\mathbb{R}^d} t |2 \pi k|^2 |\hat g(k)|^2 \, dk \\
      &\le Ct \|g\|_{H^1}^2 \quad \forall t > 0
    \end{align*}
  \end{enumerate}
\end{proof}


\begin{enumerate}[label=Step \arabic*:]
  \item Spectral problem:\[\begin{cases}
    -\Delta u_n = \lambda_n u_n &\text{in } \Omega \\ u_n|_{\partial \Omega} = 0
  \end{cases}\]
  \begin{lem}
    There is a \(\lambda_n > 0\), \(\lambda_n \xrightarrow{n \to \infty} \infty\) and an orthonormal family \(\{u_n\} \subseteq L^2(\Omega)\) s.t. \(u_n \in H_0^1(\Omega) \cap C^\infty(\Omega)\) solving this eigenvalue equation.
  \end{lem}
  \item \[\begin{cases}
    \partial_t - \Delta_x u = 0 \\ u(x,0) = g(x)
  \end{cases}
  \Rightarrow
  \begin{cases}
    \partial_t \langle u_n, u\rangle_{L^2(\Omega)} = \langle u_n, \Delta_x u\rangle = \langle \Delta_x u_n, u\rangle = -\lambda_n \langle u_n, u\rangle \\ \langle u_n, u\rangle_{t = 0} = \langle u_n, g \rangle
  \end{cases}\]
  \begin{align*}
    &\Rightarrow &\langle u_n, u\rangle &= e^{-t \lambda_n} \langle u_n, g\rangle &&\forall t > 0, \forall n = 1,2, \dots \\
    &\Rightarrow & u&= \sum_{n=0}^\infty \langle \rangle = - \sum e^{-t \lambda_n} \langle \rangle u
  \end{align*}
    
  \begin{eg}
    \(\Omega = (0,1)\), \[\begin{cases}
      - u_n'' = \lambda_n u_n &\text{in } (0,1) \\ u(0) = u(1) = 0
    \end{cases}\]
    has solution
    \[
      \begin{cases}
        u_n(x) = \sqrt{2}\sin(\pi n x)& n = 1,2, \dots \\ \lambda_n = (\pi_n)^2
      \end{cases}.
    \]
    has a solution:
    \begin{align*}
      u(x,t) &= \sum_{n=1}^\infty e^{-t\lambda_n} \underbrace{\langle u_n, g \rangle}_{g_n} u_n(x) = \sum_{n=1}^\infty e^{- t \pi^2 n^2} g_n \sin(\pi n x),
    \end{align*}
    \[\int_0^1 \sin(n \pi x)^2 \,dx = \frac{1}{2} \quad \forall n > 1\]
    \[g_n = \sqrt{2} \langle u_n, g\rangle = 2 \int_0^1 \sin(\pi n x) g(x)\, dx\]
  \end{eg}
\end{enumerate}


\begin{ex}[E 11.2]
  Consider the heat equation in a bounded domain
  \[
    \begin{cases}
      \partial_t u(x,t) = \Delta_x u(x,t) & \forall x \in \Omega, t > 0 \\
      u(x,t) = 0 & \forall x \in \partial \Omega, t > 0 \\
      u(x,0) = g(x) & \forall x \in \Omega
    \end{cases}\]
    Let us focus on the simlest case \(\Omega = (0,1)\). Prove that for every \(g \in C_c^1(0,1)\), the function 
    \begin{align*}
      u(x,t) &= \sum_{n=1}^\infty g_n e^{-t \pi^2 n^2} \sin(n \pi x), & g_n &= 2 \int_0^1 g(y) \sin(n\pi y) \, dy
    \end{align*}
    is a classical solution to the above heat equation.
\end{ex}

\begin{proof}[Solution]
  Direct proof of heat equation. \(g \in C_c^1(0,1) \subseteq H_0^1(0,1)\), \(\Rightarrow\) \(\sum_n \pi^2 n^2 |g_n|^2 = c \|g'\|_{L^2(0,1)}^2 < \infty\), so \(\sum_n |g_n| < \infty\).
  \begin{align*}
    u(x,0) &= \underbrace{\sum_{n=1}^\infty g_n \sin(\pi n x)}_{\in C[0,1]} = g(x) \quad \forall x \in [0,1]
  \end{align*}
  From \(u(x,t) = \sum_{n=1}^\infty e^{-tn^2\pi^2} g_n \sin(\pi n x)\) we get 
  \begin{align*}
    \begin{cases}
      \partial_t u(x,t) = \sum_{n=1}^\infty (-n^2 \pi^2) e^{-t\pi^2 n^2} g_n \sin(\pi n x) \quad \forall t > 0, \forall x \in (0,1) \\
      \Delta_x u(x,t) = \sum_{n=1}^\infty e^{-t\pi^2 n^2} g_n[-(\pi n)^2] \sin(\pi n x) \quad \forall t > 0, \forall x \in (0,1)
    \end{cases}
  \end{align*}
  So \(\partial_t u - \Delta_x u = 0\) for all \(t > 0, x \in (0,1)\)
\end{proof}


\begin{ex}[E 11.3]
  Let \(g(t) = e^{-\frac{1}{t^2}}\) and denote \(g^{(n)}(t)\) the \(n\)-th derivative of \(g\). Define 
  \[u(x,t) = \sum_{n=0}^\infty \frac{g^{(n)}(t)}{(2n)!} x^{2n}, \quad \forall x \in \mathbb{R}, t > 0\]
  Prove that \(u\) is a classical solution to the heat equation 
  \[\left\{\begin{aligned}
    \partial_t u(x,t) &= \Delta_x u(x,t) &\forall x &\in \mathbb{R}, t > 0 \\
    \lim_{t \to 0} u(x,t) &= 0 &\forall x &\in \mathbb{R}
  \end{aligned}\right.\]
\end{ex}

  \begin{proof}[Solution]
    Formally:
    \begin{align*}
      \left\{
        \begin{aligned}
          \partial_t u &= \sum_{n=0}^\infty \frac{g^{(n+1)}(t)}{(2n)!} x^{2n} \\
          -\Delta_x u &= \sum_{n=1}^\infty \frac{g^{(n)}}{(2n)!} (2n)(2n-1)x^{2n-2}
          = \sum_{n=1}^\infty \frac{g^{(n)}(t)}{(2n-2)!} x^{2n-2}
          = \sum_{m=0}^\infty \frac{g^{(m+1)(t)}}{(2m)!} x^{2m}
        \end{aligned}
      \right.
    \end{align*}
    This implies \(\partial_t u = \Delta_x u\) (if the series are convergent) \((x,t) \in B \times \left[\epsilon, \frac{1}{\epsilon}\right]\) for \(B \subset \mathbb{R}\) bounded, \(\epsilon > 0\). Also
    \begin{align*}
      g(t) &= e^{- \frac{1}{t^2}} \xrightarrow{t \to 0^+} e^{-\infty} = 0 \\
      g'(t) &= e^{- \frac{1}{t^2}} \left(\frac{2}{t^3}\right) \xrightarrow{t \to 0^+} 0 \\
      g''(t) &= e^{\frac{1}{t^2}} \left(- \frac{3!}{t^4} + \frac{2}{t^3}\right) \xrightarrow{t \to 0^+} 0 \\
      g'''(t) &= e^{- \frac{1}{t^2}} \left(\frac{4!}{t^5} - \frac{3!}{t^4} + \frac{2}{t^3}\right)
    \end{align*}
    Let's proof the convergence of the series:
    \begin{align*}
      u(x,t) = \sum_{n=0}^\infty \frac{g^{(n)}(t)}{(2n)!} x^{2n}
    \end{align*}
    converges absolutely for \(|x| \le C, t \in \left[\epsilon, \frac{1}{\epsilon}\right], \epsilon > 0\). By induction,
    \begin{align*}
      g^{(n)}(t) &= e^{-\frac{1}{t^2}} \underbrace{\left(\frac{(n+1)!}{t^{n+2}} - \frac{n!}{t^{n+1}} + \frac{(n+1)!}{t^n} - \dots\right)}_{\text{pol in } (\frac{1}{t}), \text{ all cos bounded by }(n+1)}(-1)^{n-1}
    \end{align*}
    This implies
    \begin{align*}
      |g^{(n)}(t)| \le e^{-\frac{1}{t^2}}[(n+2)!] \left(\frac{1}{t^{n+2}} + 1\right), \quad \frac{1}{t^s} \le \left(\frac{1}{t^{n+2}} + 1\right) \forall s =0,1,\dots, n+2
    \end{align*}
    Thus 
    \begin{align*}
      \sum_{n \ge 0} \left| \frac{g^{(n)}}{(2n)!}x^{2n}\right| \le \sum_{n \ge 0} e^{- \frac{1}{t^2}} \frac{(n+2)!}{(2n)!} \left(\frac{1}{t^{n+2}}+1\right)x^{2n}
    \end{align*}
    \begin{enumerate}[label=(\arabic*)]
      \item \begin{align*}
        \sum_{n \ge 0} \frac{(n+2)!}{(2n)!}x^{2n} &= \sum_{n \ge 0} \frac{1}{(n+3)(n+4)\cdots(2n)} \\
        &\le \sum_{n \ge 0} \frac{1}{n^{n-2}} x^{2n} \\
        &\le \sum_{n \ge M} + \sum_{n \ge M} \frac{1}{M^{n-2}} x^{2n} \\
        &M^2 \sum_n {\left(\frac{x^2}{M}\right)^n} \\
        &\le m^2 \frac{1}{1 - \left(\frac{x^2}{M}\right)}
      \end{align*}
      \item \(t \in [\epsilon, \frac{1}{\epsilon}]\), so \(\frac{1}{t} \le \frac{1}{\epsilon}\), so 
      \(\frac{1}{t^{n+2}} \le \frac{1}{\epsilon^{n+2}} \longrightarrow \sum_{n \ge 0} \frac{(n+2)!}{(2n)!} \frac{1}{t^{n+2}} x^{2n} \le \sum_{n \ge 0} \frac{1}{n^{n-2}} \frac{1}{\epsilon^{n-2}} x^{2n}\)
    \end{enumerate}
  \end{proof}

  \begin{rem}
    \(|u(x,t)| \le \exp \left(\frac{cx^2}{t}\right) \leadsto \) unphysical solution. Violates \(|u(x,t)| \le Ce^{C|x|^2}\) for all \(\forall (x,t) \in \mathbb{R} \times [0,T]\)
  \end{rem}

  \begin{ex}[Bonus 10]
    Consider 
    \[u(x,t) = \int_{\mathbb{R}^d} \Phi(x-y,t) g(y) \, dy\]
  \end{ex}
  where \(\Phi(x,t) = \frac{1}{(4 \pi t)^{\frac{d}{2}}}e^{-\frac{x^2}{4t}}\). Assume \(g \in C_c^\infty(\mathbb{R}^d)\). Prove or disprove that 
  \[\|u(\cdot, t)-g\|_{L^2(\mathbb{R}^d)} \le C_nt^n\]
  as \(t \to 0^+\) for all \(n = 1,2,\dots\)
  

  \section{Maximum Principle}

  Recall the Poisson eqation \(-\Delta u \le 0\) in \(\Omega \subseteq \mathbb{R}^d\) open, bounded. Then 
  \[\sup_{\bar \Omega} u(x) = \sup_{\partial \Omega} u(x).\]

  
  \begin{thm}[Maximum principle for bouneded sets]
    Let \(\Omega \subseteq \mathbb{R}^d\) be open and bouneded. Let \(T > 0\) and define 
    \begin{align*}
      \Omega_T &= \Omega \times (0,T), \\
      \partial^\star \Omega_T &= (\bar \Omega \times \{0\}) \cup (\partial \Omega \times [0,T])
    \end{align*}
    If \(u \in C_1^2(\Omega_T) \cap C(\bar \Omega_T)\) solves 
    \(\partial_t u - \Delta_x u \le 0\)
    in \(\Omega_T\), then \[\max_{\overline{\Omega_T}} u = \max_{\partial^\star \Omega_T} u.\]
  \end{thm}

  \begin{proof}
    We will use Hopf's argument which is simpler that the mean-value theorem (there exists a mean-value theorem for heat equation, but it is complicated and we will not discuss it).
    Firstly, to illustrate the principle, we proof the maximum princple for the Poisson Equation:
    Asumme \(u \in C^2(\Omega) \cap C(\bar \Omega)\)
    \begin{enumerate}[label=Step \arabic*)]
      \item Assume \(\Delta u > 0\) in \(\Omega\). Since \(\bar \Omega\) is compact, there is a \(x_0 \in \bar \Omega\) s.t. \(u(x_0) = \max_{x \in \bar \Omega} u(x)\). We prove that \(x_0 \in \partial \Omega\). In fact, if \(x_0 \in \Omega\), then since \(x_0\) is a (local) maximizer of \(u\) in \(\Omega\), we have \(\Delta u(x_0) \le 0\), which contradicts to the assumption that \(\Delta u > 0\) in \(\Omega\). Thus \(x_0 \in \partial \Omega\), and hence 
      \[\max_{x \in \bar \Omega} u(x) = u(x_0) \le \max_{x \in \partial \Omega} u(x).\]
      \item  Now assume \(\Delta u \ge 0\) in \(\Omega\). Define
      \[u_\epsilon(x) = u(x) + \epsilon |x|^2, \quad \epsilon > 0.\]
      Then, \(\Delta u_\epsilon > 0\) in \(\Omega\), hence by Step 1 and
      \[u \le u_\epsilon \le u + \epsilon \sup_{x \in \bar \Omega} |x|^2\]
      we have 
      \begin{align*}
        \max_{x \in \bar \Omega} u(x) 
        &\le \max_{x \in \bar \Omega} u_\epsilon(x) 
        \le \max_{x \in \partial \Omega} u_\epsilon(x) \\
        &\le \max_{x \in \partial \Omega} u(x) + \epsilon \left(\sup_{x \in \bar \Omega} |x|^2\right)
        \xrightarrow{\epsilon \to 0} \max_{x \in \partial \Omega} u(x)
      \end{align*}
    \end{enumerate}
    Proof for the heat equation:
    \begin{enumerate}[label=Step \arabic*)]
      \item Assume \(u \in C_1^2(\Omega \times (0,T]) \cap C(\bar \Omega \times [0,T])\) and \[\partial_t u - \Delta_x u < 0\]
      in \(\Omega \times (0,T]\). Then, because of compactness, there is \((x_0, t_0) \in \bar \Omega \times [0,T]\) s.t. 
      \begin{align*}
        u(x_0, t_0) &= \max_{(x,t) \in \bar \Omega \times [0,T]} u(x,t).
      \end{align*}
      We prove that \((x_0, t_0) \in \partial^\star \Omega_T\). Assume by contradiction that \((x_0, t_0) \notin \partial^\star \Omega_T\), then \(x_0 \in \Omega\) and \(t_0 \in (0,T]\). Since \(x \mapsto u(x,t_0)\) has a (local) maximizer \(x_0 \in \Omega\) we have that \(\Delta_x u(x_0, t_0) \le 0\). Since \(t \mapsto u(x_0, t)\) has a (local) maximizer \(t_0 \in (0,T]\) we have that \(\partial_t u(x_0, t_0) \ge 0\). This implies:
      \[(\partial_t u - \Delta_x u)(x_0, t_0) \ge 0\]
      which is a contradiction to the assumption. Thus \((x_0, t_0) \in \partial^\star \Omega_T\), i.e. \(\max_{\bar \Omega_T} u = \max_{\partial^\star \Omega_T} u.\)
      \item Assume \(u \in C_1^2(\Omega \times (0,T)) \cap C(\bar \Omega \times [0,T])\) and 
      \[\partial_t u - \Delta_x u \le 0 \quad \text{in } \Omega \times (0,T).\]
      Let \(\tilde T \in (0,T)\) and for \(\epsilon > 0:\)
      \[u_\epsilon(x,t) = u(x,t) + \epsilon |x|^2.\]
      Then: \(u_\epsilon \in C_1^2(\Omega \times (0,T']) \cap C(\bar \Omega \times [0, \tilde T])\) and \(\partial_t  u_\epsilon - \Delta_x  u_\epsilon < 0\) in \(\Omega \times (0,\tilde T]\). By Step 1:
      \begin{align*}
        &&\max_{\bar \Omega_{\tilde T}} u_\epsilon &\le \max_{\partial^\star \Omega_{\tilde T}} u_\epsilon \\
        &\overset{\epsilon \to 0}{\Rightarrow} &\max_{\bar \Omega_{\tilde T}} u &\le \max_{\partial^\star \Omega_{\tilde T}} u \\
        &\overset{\tilde T \to T}{\Rightarrow} &\max_{\bar \Omega_T} u &\le \max_{\partial^\star \Omega_T} u \qedhere
      \end{align*}
    \end{enumerate}
  \end{proof}

  \begin{thm}[Maximum principle for \(\Omega = \mathbb{R}^d\)]
    Let \(\Omega_T = \mathbb{R}^d \times (0,T)\), \(\bar \Omega_T = \mathbb{R}^d \times [0,T]\). Let \(u \in C_1^2(\Omega_T) \cap C(\bar \Omega_T)\) such that 
    \begin{itemize}
      \item \(\partial_t u - \Delta_x u \le 0\) in \(\Omega_T\)
      \item \(u(x,t) \le M e^{M|x|^2}\) for all \((x,t) \in \bar \Omega_T\)
    \end{itemize}
    Then \[\sup_{(x,t) \in \bar \Omega_T} u(x,t) = \sup_{x \in \mathbb{R}^d} u(x,0).\]
  \end{thm}

  \begin{proof}\
    \begin{enumerate}[label=Step \arabic*:]
      \item For all \(y \in \mathbb{R}^d\) and \(\epsilon > 0\) define 
      \[v(x,t) = u(x,t) - \frac{\epsilon}{(T+\epsilon-t)^{\frac{d}{2}}} \exp \left(\frac{|x-y|^2}{4(T+\epsilon-t)}\right)\]
      This implies 
      \[\partial_t v - \Delta_x v = \partial_t u - \Delta_x u \le 0\] in \(\Omega_T\). For \(U = B(y,r), U_T = U \times (0,T), \bar U_T = \bar U \times [0,T]\), \(
        \partial^\star U_T = (U \times \{0\}) \cup (\partial U \times [0,T])\), by the maximum principle for \(U\) bounded we have 
        \[\max_{\bar U_T} v \le \max_{\partial^\star U_T} v.\]
        Let us bound \(\max_{\partial^\star U_T} v\).
        \begin{itemize}
          \item On \(U \times \{0\}\) we use \(v \le u\) and hence 
          \[\max_{x \in \bar U} v(x,0) \le \max_{x \in \bar U} u(x,0) \le \max_{x \in \mathbb{R}^d} u(x,0).\]
          \item On \(\partial U \times [0,T]\) we use \(|x-y|=r \Rightarrow |x| \le |y| + r\).
          \begin{align*}
            v(x,t) &= u(x,t) - \frac{\epsilon}{(T+\epsilon-t)^{\frac{d}{2}}} \exp \left(\frac{|x-y|^2}{4(T+\epsilon-t)}\right) \\
            &\le Me^{M(|y|+r)^2}-\frac{\epsilon}{(T+\epsilon)^{\frac{d}{2}}} \exp \left(\frac{r^2}{4(T+\epsilon)}\right) \xrightarrow{r \to \infty} - \infty
          \end{align*}
          if  \(M < \frac{1}{4(T+\epsilon)}\). In particular, we can choose \(r\) large s.t.
          \[\max_{\substack{x \in \partial U\\t \in [0,T]}} v(x,t) \le \max_{x \in \mathbb{R}^d} u(x,0).\]
        \end{itemize}

		In summary, if \(M < \frac{1}{4(T+\epsilon)}\), then:
		\begin{align*}
		  u(y,t) - \frac{\epsilon}{(T+\epsilon-t)^{\frac{d}{2}}} = v(y,t) \le \max_{\bar U_T} v \le \max_{x \in \mathbb{R}^d} u(x,0)
		\end{align*}
		This holds for all \((y,t) \in \mathbb{R}^d \times [0,T]\). Thus,
		\[\max_{\mathbb{R}^d \times [0,T]} u \le \frac{\epsilon}{(T+\epsilon-t)^{\frac{d}{2}}} + \max_{x \in \mathbb{R}^d} u(x,0)\]
		Taking \(\epsilon \to 0\) we conclude that if \(M < \frac{1}{4T}\),
		\[\max_{\mathbb{R}^d \times [0,T]} u \le \max_{x \in \mathbb{R}^d} u(x,0)\]
		\item For general \(T\), we denote \(T_1 = \frac{T}{N}\), \(N \in \mathbb{N}\) s.t. \(M <\frac{4}{T_1}\). Then by step 1:
		\begin{align*}
			\max_{\mathbb{R}^d \times [0,T_1]} u &\le \max_{x \in \mathbb{R}^d} u(x,0) \\
			\max_{\mathbb{R}^d \times [T_1, 2 T_1]} u &\le \max_{x \in \mathbb{R}^d} u(x,T_1) \le \max_{x \in \mathbb{R}^d} u(x,0) \\
			&\vdots \\
			\max_{\mathbb{R}^d \times [(N-1) T_1, N T_1]} & \le \max_{x \in \mathbb{R}^d} u(x,(N-1) T_1) \le \max_{x \in \mathbb{R}^d} u(x,0) \\
			\leadsto \quad \max_{\mathbb{R}^d \times [0,T]} u &\le \max_{x \in \mathbb{R}^d} u(x,0) \qedhere
		\end{align*}
    \end{enumerate}
  \end{proof}

  \begin{rem}
    The condition \(u \le Me^{M|x|^2}\) is necessary, otherwise there are solutions \(u \ne 0\) s.t. \(u(x,0) = 0\)
  \end{rem}

  \begin{thm}[Uniqueness]
    If \(u \in C_1^2(\mathbb{R}^d \times (0,T)) \cap C(\mathbb{R}^d \times [0,T])\) and
    \begin{align*}
      u(x,t) &\le Me^{M|x|^2} &&\text{in } \mathbb{R}^d \times [0,T], \\
      \partial_t u - \Delta_x u &= 0 &&\text{in } \mathbb{R}^d \times (0,T), \\
      u(x,0) &= 0 &&\text{in } \mathbb{R}^d 
    \end{align*}
    Then \(u = 0\) in \(\mathbb{R}^d \times [0,T]\).
  \end{thm}

  \begin{proof}
	  Use the maximum principle for \(u\) and \(-u\).
  \end{proof}

  \begin{rem}
	If \(u(\cdot, t) \in L^2(\mathbb{R}^d)\), the proof of uniquness can be done without the maximum principle. Heuristically:
	\begin{align*}
		\frac{d}{dt} \int_{\mathbb{R}^d} |u(x,t)|^2 \, dt &= 2 \int_{\mathbb{R}^d} (\partial_t u)u \, dx
		= 2 \int_{\mathbb{R}^d} \Delta_x u u \, dx
		= -2 \int_{\mathbb{R}^d} |\nabla_x u|^2 \, dx \le 0
	\end{align*}
	This implies 
	\[e(t) \coloneqq \int_{\mathbb{R}^d} |u(x,t)|^2 \, dx\]
	is descreasing. Hence, if \(e(0) = 0\), then \(e(t) = 0\) for all \(t \ge 0\). This argument will be helpful below for the heat backward equation.
  \end{rem}

  \begin{rem}
    The heat equation 
    \[\begin{cases}
		\partial_t u - \Delta_x u = 0 \\ u(t=0) = g
	\end{cases}\] is a well-posed problem:
    \begin{itemize}
      \item Existence
      \item Uniqueness
      \item Stability (solution depends continuously on data)
    \end{itemize}    
    For the latter issue, by the maximum principle we have 
    \[\|u(\cdot, t)\|_{L^\infty} \le \|u(\cdot, 0)\|_{L^\infty} \quad \forall t\]
    or in the \(L^2\)-situation:
    \[\|u(\cdot, t)\|_{L^2} \le \|u(\cdot, 0)\|_{L^2} \quad \forall t\]
    On the other hand, the heat backward equation 
    \[\begin{cases}
      \partial_t u - \Delta_x u = 0 \\ u(t = T) = g
    \end{cases}\]
    is \emph{not} well-posed.
    \begin{itemize}
      \item Non-Existence: In general, the existence requires some special property on \(g\), e.g. \(g \) is very smooth (only \(g \in C(\mathbb{R}^d) \cap L^\infty(\mathbb{R}^d)\) or \(g \in L^2(\mathbb{R}^d)\) is not enough)
      \item Uniqueness: On the other hand, the uniqueness still holds.
    \end{itemize}
  \end{rem}

  \begin{lem}\label{e-becomes-zero}
		If \(e \in C^2(0,T)\), \(e(t) \ge 0\), \(e'(t) \le 0\), \(e''(t) \ge 0\) and \(|e'(t)|^2 \le e(t) e''(t)\) for \(t \in [0,T]\) and \(e(T) = 0\), then \(e \equiv 0\).
	\end{lem}

	\begin{proof}
		Since \(e\) is monotonly decreasing and \(e(T) = 0\) there is a \(t_0 \in [0,T]\) s.t. \(e(t_0) = 0\) and \(e(t) > 0\) if \(t \le t_0\). We need to prove that \(t_0 = 0\). Assume by contradiction \(0 < t_0 \le T\), then for \(t \in (0,t_0)\) define \(f(t) \coloneqq \log e(t)\). Then 
		\begin{align*}
			f'(t) &= \frac{e'(t)}{e(t)} \\
			\Rightarrow \quad f''(t) &= \frac{e''(t) e(t)-|e'(t)|^2}{e(t)^2} \ge 0
		\end{align*}
		This means that \(f\) is convex, so for all \(t_1, t_2 \in (0,t_0)\) and \(\tau \in (0,1)\):
		\begin{align*}
			f(\tau t_1 + (1-\tau)t_2) &\le \tau f(t_1) + (1-\tau)f(t_2) \\
			\Rightarrow \quad e(\tau t_1 + (1-\tau) t_2) &\le e(t_1)^\tau e(t_2)^{1-\tau}
		\end{align*}
		Now, \(e(\tau t_1 + (1-\tau) t_2) \xrightarrow{t_2 \to t_0} 0\) and \(\tau \to 1\) implies \(e(t_1) = 0\) for all \(t_1 \in (0,t_0)\) which is a contradiction.
	\end{proof}

  \begin{thm}
    If \(u \in C_1^2(\mathbb{R}^d \times [0,T]) \cap C^1(H^1(\mathbb{R}^d) \times [0,T])\) and 
    \[\begin{cases}
      \partial_t u - \Delta_x u = 0 & \text{in } \mathbb{R}^d \times (0,T) \\
      u(x,T) = 0
    \end{cases}\]
    Then \(u = 0\) in \(\mathbb{R}^d \times [0,T]\).
  \end{thm}

  \begin{proof}
    Recall 
	\[e(t) = \int_{\mathbb{R}^d} |u(x,t)|^2 \, dx.\]
	Then,
	\begin{align*}
		e'(t) &= 2 \int_{\mathbb{R}^d} u \partial_t u \, dx
		= 2 \int_{\mathbb{R}^d} u \Delta_x u \, dx
		= -2\int_{\mathbb{R}^d} |\nabla_x u|^2 \, dx \\
		e''(t) &= - 4 \int_{\mathbb{R}^d} \nabla_x u \nabla_x(\partial_t u) 
		= 4 \int_{\mathbb{R}^d} \Delta_x u \partial_t u \, dx
		= 4 \int_{\mathbb{R}^d}|\Delta_x u|^2 \, dx \ge 0
	\end{align*}
	and hence
	\begin{align*}
		|e'(t)|^2 &= 4 \left| \int_{\mathbb{R}^d} u \Delta_x u \, dx \right|^2 
		\le 4 \left(\int_{\mathbb{R}^d} |u|^2 \, dx\right) \left(\int_{\mathbb{R}^d} |\Delta_x u|^2 \, dx\right) = e(t) e''(t)
	\end{align*}
	Then the statement follows with lemma \ref{e-becomes-zero}.
  \end{proof}


  Some remarks about the eat equation in unbounded domains:
  \[\begin{cases}
    \partial_t u - \Delta_x u = 0 &\text{in } \mathbb{R}^d \times (0,\infty) \\
    u(x,0) = 0 &(\text{i.e. \(\lim_{t \to 0} u(x,t) = 0 \forall x \in \mathbb{R}^d\)})
  \end{cases}\]
  There is a classical solution \(0\ne u \in C^1(\mathbb{R}^d \times (0,\infty))\).
  An example is 
  \[u(x,t) = \sum_{n=0}^\infty \frac{g^{(n)}(t)}{(2n)!} x^{2n}, \quad g(t) = e^{-\frac{1}{t^2}}\]
  (s.t. \(g \to 0\) as \(t \to 0\)). Note
  \begin{align*}
    g(t) &= e^{- \frac{1}{t^2}}, \\
    g'(t) &= \frac{2}{t^3} g(t) \\
    g''(t) &= \left(\frac{2}{t^3}\right)'g(t) + \frac{2}{t^3} \frac{2}{t^3} g(t) \\
    g^{(n)}(t) &= P_n \left(\frac{1}{t}\right)g(t)
  \end{align*}
  where
  \[\begin{cases}
    P_0 = 1 \\ P_{n+1} \left(\frac{1}{t}\right) = \left(P_n\left(\frac{1}{t}\right)\right)' + \left(\frac{2}{t^3}\right)P_n \left(\frac{1}{t}\right) = A_1 P_n + A_2 P_n, \begin{cases}
      A_1 = \partial_t \\ A_2 = \frac{2}{t^3}
    \end{cases}\\
    P_{n+1} = (A_1 + A_2) P_n = (A_1 + A_2) (A_1 + A_2)P_{n-1} = 
  \end{cases}\]
  This implies:
  \begin{align}
    P_n = (A_1 + A_2)^n P_0 = \sum_{\sigma \in \{1,2\}^n} A_{\sigma(1)} A_{\sigma(2)} \cdots A_{\sigma(n)} P_0
  \end{align}
  \[A_1 \left(\frac{\alpha}{t^s}\right) = \frac{-s\alpha}{t^{s+1}} \to A_1\]
  Multiple coefficients by a factors and \(+\) power by 1
  \[A_2 \left(\frac{\alpha}{t^s}\right) = \frac{2\alpha}{t^{s+3}} \to A_2\]
  Mul Cof by a factor 2 and + power by 3
  \[|\underbrace{A_{\sigma(1)} \cdots A_{\sigma(n)},1}_{\text{\(k\) times \(A_2\), \(n-k\) times \(A_1\)}}| \le \frac{2^k}{t^{3k}} \le \frac{2^k}{t^{3k}} \frac{(3n)^{n-k}}{t^{n-k}} = \frac{2^k(3n)^{n-k}}{t^{n+2k}}\]
  This implies 
  \[|P_n \left(\frac{1}{t}\right)| \le \max_{0 \le k \le n} \frac{2^n 2^k (3n)^{n-k}}{t^{n+2k}}\]
  Thus:
  \begin{align*}
    \sum_{n} \left|\frac{g^{(n)}(t)}{(2n)!} x^{2n}\right| &
    \le \sum_n \max_{0 \le k \le n} \frac{2^n 2^k (3n)^{n-k}}{t^{n+2k}(2n)!} \frac{e^{-\frac{1}{t^2}}}{1} x^{2n} \\
    &\le \sum_{n} \max \frac{2^n 2^k (3n)^{n-k}}{t^{n+2k}(2n)!} (k!)(2t^2)^k e^{- \frac{1}{2t^2}} x^{2n} \\
    &= \sum_{n} \frac{2^n 2^k 2^k (3n)^{n-k}(k!)}{(2n)!t^n} e^{- \frac{1}{2t^2}} x^{2n} \\
    &\le \sum_{n} \frac{(c_n)^n}{(2n)!t^n} e^{- \frac{1}{2t^2}} x^{2n}\\
    &\le \sum_{n} \frac{c^n}{n!t^n} e^{- \frac{1}{2t^2}} x^{2n} \\
    &\le \sum_{n} e^{\frac{cx^2}{t}-\frac{1}{2t^2}}
  \end{align*}
  Where we used that
  \[e^s = \sum_k \frac{s^k}{k!} \ge \frac{s^k}{k!}\]
  for all \(s \ge 0\) implies 
  \[e^{- \frac{1}{2t^2}} = \frac{1}{e^{\frac{1}{2t^2}}} \le \frac{1}{\left(\frac{1}{2t^2}\right) \frac{1}{k!}} = k!(2t^2)^k.\]
  We conclude:
  \begin{itemize}
    \item \(u(x,t)\) is well-defined, \(x \in \mathbb{R}^d\), \(t > 0\) real? \(to\) heat equation. 
    \item \(u(x,t) \to 0\) as \(t \to 0\) for all \(x \in \mathbb{R}^d\).
  \end{itemize}

  \begin{ex}[E 12.1]
    Let \(\Omega \subseteq \mathbb{R}^d\) be open and \(u \in C^2(\Omega)\). Assume that \(x_0 \in \Omega\) is a local maximizer of \(u\), namely there exists some \(r > 0\) such that 
    \(u(x_0) \ge u(x)\)
    for all \(x \in B_r(x_0) \subseteq \Omega.\)
    \begin{enumerate}[label=(\alph*)]
      \item Prove that the Hessian matrix \(H = (D^\alpha u(x_0))_{|\alpha|=2}\) is negative semi-definite, namely \[y H y \le 0\]
      for all \(y \in \mathbb{R}^d\).
      \item Prove that \(\Delta u(x_0) \le 0\)
    \end{enumerate}
    Hint: Recall that we used (b) for the maximum principle by Hopf's method.
  \end{ex}

  \begin{proof}[Solution]
    \begin{enumerate}[label=(\alph*)]
      \item In 1D this is obvious. If \(x_0\) is a local minimizer of \(u\), then \(u'(x_0) = 0, u''(x_0) \le 0\) (Taylor expansion). \\
      In \(d\) dimensions:
      \[\phi(t) = u(x_0 + t \xi) \quad \xi \in \mathbb{R}^d, t \in \mathbb{R}, |t| \text{ small}\]
      So \(0\) is a local maximizer of \(\phi\). This implies
      \[0 = \phi'(0) = \nabla u (x_0) \xi \quad \forall \xi \in \mathbb{R}^d \Rightarrow H \le 0\]
      \begin{align*}
        \phi''(0) &= \lim_{t \to 0} \frac{\phi'(t) - \phi'(0)}{t}
        = \lim_{t \to 0}\frac{(\nabla u(x_0 + t \xi) - \nabla u(x_0))\xi}{t} \\
        &= \lim_{t \to 0} \sum_{i=1}^d \frac{(\partial_i u(x_0 + t\xi)-\partial_i u(x_0))\xi_i}{t} = \sum_{i=1}^d \sum_{j=1}^d \partial_j \partial_i u(x_0) \xi_j \xi_i = \langle \xi, H\xi\rangle,
      \end{align*}
      \(H = (\partial_i \partial_j u(x_0))_{i,j = 1}^d\). 
      \item Consequently
      \[\Delta u(x_0) = \sum_{i=1}^d \partial_i \partial_i u(x_0) = \operatorname{Tr}(H) \le 0 \qedhere\]
    \end{enumerate}
  \end{proof}

  \begin{ex}[E 12.2]
    Let \(\Omega \subseteq \mathbb{R}^d\) be open and bouned. We prove the maximum principle for a general elliptic operator
    \[Lu(x) = \sum_{i,j=1}^d a_{ij}(x) \partial_i \partial_j u(x) + \sum_{i=1}^d b_i(x) \partial_ju(x),\]
      \(a_{ij}, b_i \in C(\bar \Omega)\), \(A(x) = (a_{ij}(x))_{i,j = 1}^d \ge \mathbb{1}\) (as matrices). % \(A-\mathbb{1} \ge 0\)
      Prove that if \(Lu(x) \ge 0\) for all \(x \in \Omega\) and \(u \in C^2(\Omega) \cap C(\bar \Omega)\), then 
      \[\max_{x \in \bar \Omega}u(x) = \max_{x \in \partial \Omega} u(x).\]
  \end{ex}

  \begin{proof}[Solution]\
    \begin{enumerate}[label=Step \arabic*:]
      \item Assume \(Lu(x) > 0\) for all \(x \in \Omega\):
      Since \(u \in C(\bar \Omega)\) there is a \(x_0 \in \bar \Omega\) s.t. 
      \[u(x_0) = \max_{x \in \bar \Omega} u(x).\]
      We prove \(x_0 \in \partial \Omega.\) Assume by contradiction that \(x_0 \notin \partial \Omega\), so \(x_0 \in \Omega\) is a local maximizer. We prove \(Lu(x_0) \le 0\). Note:
      \begin{align*}
        Lu(x_0) &= \sum_{i,j=1}^d a_{ij}(x_0) \partial_i \partial_ju(x_0) + \sum_{i=1}^d b_i(x_0) \partial_i u(x_0) \\
        &= \operatorname{Tr}[A(x_0) H(x_0)] + B(x_0) \underbrace{\nabla u(x_0)}_{=0} \le 0 \quad \lightning
      \end{align*}
      \(A(x_0) = (a_{ij}(x_0))_{i,j=1}^d\), \(B(x_0) = (b_i(x_0))_{i=1}^d\),
      where \(\Tr[AH] = \sum_i(AH)_{ii} = \sum_i \sum_j A_{ij} H_{ij}\)
    \end{enumerate}
    General fact: If \(A \ge 0, B \ge 0\) (matrices), then \(Tr(AB) \ge 0\).
    \begin{itemize}
      \item \(A = (\sqrt{A})^2 \Rightarrow \Tr(AB) = \Tr((\sqrt{A})^2B) = \Tr(\underbrace{\sqrt{A}B\sqrt{A}}_{\ge 0}) \ge 0\)
      \item Spectral theorem: \(A \ge 0\), then there are eigenvectors \((\alpha_i)\) and eigenvalues \(\lambda_i \ge 0\) s.t. 
      \[\Tr(AB) = \sum_i \langle \alpha_i, AB\alpha_i\rangle = \sum_i \underbrace{\lambda_i}_{\ge 0} \underbrace{\langle \alpha_i, B \alpha_i\rangle}_{\ge 0} \ge 0\]
    \item General Case: \(L u(x) \ge 0\) for all \(x \in \Omega\). Assume that there is a \(v \in C^2(\Omega) \cap C(\bar \Omega)\) s.t. \(Lv(x) > 0\) for all \(x \in \Omega\). Define for all \(\epsilon > 0\) \(u_\epsilon = u + \epsilon v\). Then \(L u_\epsilon(x) = Lu(x) + \epsilon Lv(x) > 0\) for all \(x \in \Omega\). By Step 1,
    \begin{align*}
      \max_{x \in \bar \Omega} u_\epsilon(x) &\le \max_{x \in \partial \Omega} u_\epsilon(x) \\
      \xrightarrow{\epsilon \to 0} \quad \max_{x \in \bar \Omega} u(x) &\le \max_{x \in \partial \Omega} u(x)
    \end{align*}
    What \(v\)? First \(v(x) = x^2 = x_1^2 + \dots + x_d^2\),
    \begin{align*}
      Lv(x) &= \sum_{ij} a_{ij}(x) 2 \delta_{ij} + \sum_i b_i(x) 2x_i
    \end{align*}
    not clear to be \(\ge 0\).
    \begin{align*}
      v(x) &= x^{2n} \quad \text{\(n\) large} \\
      v(x) &= x_1^{2n} \longrightarrow Lv(x) = a_{11}(x) 2n(2n+1)x_1^{2n-2} + b_1(x) 2n x_1^{2n-1} \\
      &\ge 2n x_1^{2n-2}[\underbrace{(2n-1) + \underbrace{b_1(x)x_1}_{\text{b.d. in \(\bar \Omega\)}}}_{> 0}] \ge 0 \quad \forall x \in \bar \Omega
    \end{align*}
    if \(n\) is large enough.
    \[v(x) = (x_1 + R)^{2n}\]
    where \(R > 0\) large s.t. \(x_1 + R \ge 1\) for all \(\forall x \in \bar \Omega\). This implies
    \begin{align*}
      Lv(x) \ge 2n \underbrace{(x_1 + R)^{2n-2}}_{> 0}[\underbrace{2n-1 + b_1(x) (x_1 + R)}_{> 0}] > 0
    \end{align*}
    for all \(x \in \bar \Omega\) if \(n\) is large.
    \end{itemize}
  \end{proof}

  \begin{ex}[E 12.3]
    Consider the inhomogeneous heat equation
    \[\begin{cases}
      \partial_tu-\Delta_x u = f(x,t) &\text{in } \mathbb{R}^d \times (0,T)\\ u(t=0) = g &\text{in } \mathbb{R}^d
    \end{cases},\]
    \(f \in C_1^2(\mathbb{R}^d \times (0,T))\) and compactly supported and \(g \in C(\mathbb{R}^d \times [0,T]) \cap L^\infty(\mathbb{R}^d \times [0,T])\). Assume that there exists a solution \(u \in C_1^2(\mathbb{R}^d \times (0,T)) \cap C(\mathbb{R}^d \times [0,T])\) satisfying \[u(x,t) \le Me^{M|x|^2}, \quad (x,t) \in \mathbb{R}^d \times [0,T].\]
    Prove that 
    \[\max_{(x,t) \in \mathbb{R}^d \times [0,T]} |u(x,t)| \le \|g\|_{L^\infty} + T\|f\|_{L^\infty}.\]
  \end{ex}

  \begin{proof}[Solution]\
    \begin{enumerate}[label=Step \arabic*:]
      \item There is at most one solution \(u\).
      \item \[u(x,t) = \int_{\mathbb{R}^d} \phi(x-y,t) g(y) \, dy + \int_0^t \int_{\mathbb{R}^d} \phi(x-y,t-s)f(y,s) \, dy \, ds\]
    \end{enumerate}
    This implies:
    \begin{align*}
      \|u\|_{L^\infty} &\le \int_{\mathbb{R}^d} \phi(x-y,t) \|g\|_{L^\infty} \, dy + \int_0^t \int_{\mathbb{R}^d} \phi(x-y,t-s) \|f\|_{L^\infty} \, dy \, ds  \\
      \Rightarrow \quad \|u\|_{L_{x,t}^\infty} &\le \int_{\mathbb{R}^d} \phi(x-y,t) \|g\|_{L^\infty} \, dy + \int_0^T \int_{\mathbb{R}^d} \phi(x-y,t-s) \|f\|_{L^\infty} \, dy \, ds \\
      &= \|g\|_{L_x^\infty} + T \|f\|_{L_{x,t}^\infty}
    \end{align*}
    This is optimal! E.g. \(g = 0, f = 1\), \(u(x,t) = u(t)\).
    \[\begin{cases}
      u' = 1 \\ u(0) =0
    \end{cases}\Rightarrow u(t) = t\]
  \end{proof}

  \begin{ex}[Bonus 11]
    Denote for all \(u \in C^2(\Omega) \cap C(\bar \Omega)\):
    \[Lu(x) = \sum_{i,j=1}^d a_{ij}(x) \partial_i \partial_j u(x)\]
    where \(a_{ij} \in C(\bar \Omega)\) s.t. \(A(x) = (a_{ij}(x)) \ge 1\).
    Prove that if \(\Omega \subseteq \mathbb{R}^d\) is open and bounded, \(u \in C_1^2(\bar \Omega \times [0,T])\) and 
    \[\begin{cases}
      \partial_t u - Lu \le 0 &\text{in } \Omega \times (0,T) \\
      u(t=0) = 0 \\ u(x \in \partial \Omega) = 0
    \end{cases}\]
    Prove that \(u(x,t) \le 0\) for all \((x,t) \in \bar \Omega \times [0,T]\).
  \end{ex}
  %Hint: Apply half method (Hopf Method?). But with general operator L. 


  \section{Backward heat equation}
  \begin{thm}[Instability]\ \newline
    There exist functions \(u_\epsilon \in C_1^2(\mathbb{R}^d \times (0,T)) \cap C^1(H^1(\mathbb{R}^d) \times [0,T])\) s.t. \[\partial_t u - \Delta_x u = 0 \quad \text{in } \mathbb{R}^d \times [0,T]\]
		such that when \(\epsilon \to 0^+\):
		\[\|u_\epsilon(\bullet, T)\|_{L^2(\mathbb{R}^d)} \to 0, \quad \|u(\bullet, 0)\|_{L^2(\mathbb{R}^d)} \to \infty.\]
  \end{thm}

\end{document}